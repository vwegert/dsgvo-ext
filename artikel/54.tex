%!TEX root = ../DSGVO-ExtendedVersion.tex
\chapter{Errichtung der Aufsichtsbehörde}
\label{ch:54}

\crossrefArticleToReason{54}

\begin{enumerate}

  \item Jeder Mitgliedstaat sieht durch Rechtsvorschriften Folgendes vor:%
  \label{itm:54-1}

  \begin{enumerate}
  
    \item die Errichtung jeder \hyperref[itm:04-21]{Aufsichtsbehörde};%
    \label{itm:54-1a}

    \item die erforderlichen Qualifikationen und sonstigen Voraussetzungen für die Ernennung zum Mitglied jeder
     \hyperref[itm:04-21]{Aufsichtsbehörde};%
    \label{itm:54-1b}

    \item die Vorschriften und Verfahren für die Ernennung des Mitglieds oder der Mitglieder jeder \hyperref[itm:04-21]
     {Aufsichtsbehörde};%
    \label{itm:54-1c}

    \item die Amtszeit des Mitglieds oder der Mitglieder jeder \hyperref[itm:04-21]{Aufsichtsbehörde} von mindestens
     vier Jahren; dies gilt nicht für die erste Amtszeit nach 24. Mai 2016, die für einen Teil der Mitglieder kürzer
     sein kann, wenn eine zeitlich versetzte Ernennung zur Wahrung der Unabhängigkeit der \hyperref[itm:04-21]
     {Aufsichtsbehörde} notwendig ist;%
    \label{itm:54-1d}

    \item die Frage, ob und -- wenn ja -- wie oft das Mitglied oder die Mitglieder jeder \hyperref[itm:04-21]
     {Aufsichtsbehörde} wiederernannt werden können;%
    \label{itm:54-1e}

    \item die Bedingungen im Hinblick auf die Pflichten des Mitglieds oder der Mitglieder und der Bediensteten jeder
     \hyperref[itm:04-21]{Aufsichtsbehörde}, die Verbote von Handlungen, beruflichen Tätigkeiten und Vergütungen während
      und nach der Amtszeit, die mit diesen Pflichten unvereinbar sind, und die Regeln für die Beendigung des
      Beschäftigungsverhältnisses.%
   \label{itm:54-1f}

  \end{enumerate}

  \item Das Mitglied oder die Mitglieder und die Bediensteten jeder \hyperref[itm:04-21]{Aufsichtsbehörde} sind gemäß
   dem Unionsrecht oder dem Recht der Mitgliedstaaten sowohl während ihrer Amts- beziehungsweise Dienstzeit als auch
   nach deren Beendigung verpflichtet, über alle vertraulichen Informationen, die ihnen bei der Wahrnehmung ihrer
   Aufgaben oder der Ausübung ihrer Befugnisse bekannt geworden sind, Verschwiegenheit zu wahren. Während dieser Amts-
   beziehungsweise Dienstzeit gilt diese Verschwiegenheitspflicht insbesondere für die von natürlichen Personen
   gemeldeten Verstößen gegen diese Verordnung.%
  \label{itm:54-2}

\end{enumerate}

% \addsec{Ergänzende Hinweise}

