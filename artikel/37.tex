%!TEX root = ../DSGVO-Bearbeitung.tex
\chapter{Benennung eines Datenschutzbeauftragten}
\label{ch:37}

\addsec{Text der Verordnung}

\begin{enumerate}

  \item Der Verantwortliche und der Auftragsverarbeiter benennen auf jeden Fall einen Datenschutzbeauftragten, wenn
  \label{itm:37-1}

  \begin{enumerate}
  
    \item die \hyperref[itm:04-2]{Verarbeitung} von einer Behörde oder öffentlichen Stelle durchgeführt wird, mit Ausnahme von Gerichten, die
     im Rahmen ihrer justiziellen Tätigkeit handeln,
    \label{itm:37-1a}

    \item die Kerntätigkeit des Verantwortlichen oder des Auftragsverarbeiters in der Durchführung von
     Verarbeitungsvorgängen besteht, welche aufgrund ihrer Art, ihres Umfangs und/oder ihrer Zwecke eine umfangreiche
     regelmäßige und systematische Überwachung von \hyperref[itm:04-1]{betroffenen Personen} erforderlich machen, oder
    \label{itm:37-1b}

    \item die Kerntätigkeit des Verantwortlichen oder des Auftragsverarbeiters in der umfangreichen \hyperref[itm:04-2]{Verarbeitung}
     besonderer Kategorien von Daten gemäß \hyperref[ch:9]{Artikel 9} oder von \hyperref[itm:04-1]{personenbezogenen Daten} über
     strafrechtliche Verurteilungen und Straftaten gemäß \hyperref[ch:10]{Artikel 10} besteht.
    \label{itm:37-1c}

  \end{enumerate}

  \item Eine Unternehmensgruppe darf einen gemeinsamen Datenschutzbeauftragten ernennen, sofern von jeder Niederlassung
   aus der Datenschutzbeauftragte leicht erreicht werden kann.
  \label{itm:37-2}

  \item Falls es sich bei dem Verantwortlichen oder dem Auftragsverarbeiter um eine Behörde oder öffentliche Stelle
   handelt, kann für mehrere solcher Behörden oder Stellen unter Berücksichtigung ihrer Organisationsstruktur und ihrer
   Größe ein gemeinsamer Datenschutzbeauftragter benannt werden.
  \label{itm:37-3}

  \item In anderen als den in \hyperref[itm:37-1]{Absatz 1} genannten Fällen können der Verantwortliche oder der
   Auftragsverarbeiter oder Verbände und andere Vereinigungen, die Kategorien von Verantwortlichen oder
   Auftragsverarbeitern vertreten, einen Datenschutzbeauftragten benennen; falls dies nach dem Recht der Union oder der
   Mitgliedstaaten vorgeschrieben ist, müssen sie einen solchen benennen. Der Datenschutzbeauftragte kann für derartige
   Verbände und andere Vereinigungen, die Verantwortliche oder Auftragsverarbeiter vertreten, handeln.
  \label{itm:37-4}

  \item Der Datenschutzbeauftragte wird auf der Grundlage seiner beruflichen Qualifikation und insbesondere des
   Fachwissens benannt, das er auf dem Gebiet des Datenschutzrechts und der Datenschutzpraxis besitzt, sowie auf der
   Grundlage seiner Fähigkeit zur Erfüllung der in \hyperref[ch:39]{Artikel 39} genannten Aufgaben.
  \label{itm:37-5}

  \item Der Datenschutzbeauftragte kann Beschäftigter des Verantwortlichen oder des Auftragsverarbeiters sein oder seine
   Aufgaben auf der Grundlage eines Dienstleistungsvertrags erfüllen.
  \label{itm:37-6}

  \item Der Verantwortliche oder der Auftragsverarbeiter veröffentlicht die Kontaktdaten des Datenschutzbeauftragten und
   teilt diese Daten der Aufsichtsbehörde mit.
  \label{itm:37-7}

\end{enumerate}

\addsec{Eigene Notizen}

