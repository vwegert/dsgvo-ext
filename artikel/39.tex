%!TEX root = ../DSGVO-Bearbeitung.tex
\chapter{Aufgaben des Datenschutzbeauftragten}
\label{ch:39}

\addsec{Text der Verordnung}

\begin{enumerate}

  \item Dem Datenschutzbeauftragten obliegen zumindest folgende Aufgaben:
  \label{itm:39-1}

  \begin{enumerate}
  
    \item Unterrichtung und Beratung des Verantwortlichen oder des Auftragsverarbeiters und der Beschäftigten, die
     \hyperref[itm:04-2]{Verarbeitungen} durchführen, hinsichtlich ihrer Pflichten nach dieser Verordnung sowie nach sonstigen
     Datenschutzvorschriften der Union bzw. der Mitgliedstaaten;
    \label{itm:39-1a}

    \item Überwachung der Einhaltung dieser Verordnung, anderer Datenschutzvorschriften der Union bzw. der
     Mitgliedstaaten sowie der Strategien des Verantwortlichen oder des Auftragsverarbeiters für den Schutz
     \hyperref[itm:04-1]{personenbezogener Daten} einschließlich der Zuweisung von Zuständigkeiten, der Sensibilisierung und Schulung der an
     den Verarbeitungsvorgängen beteiligten Mitarbeiter und der diesbezüglichen Überprüfungen;
    \label{itm:39-1b}

    \item Beratung -- auf Anfrage -- im Zusammenhang mit der Datenschutz-Folgenabschätzung und Überwachung ihrer
     Durchführung gemäß \hyperref[ch:35]{Artikel 35};
    \label{itm:39-1c}

    \item Zusammenarbeit mit der \hyperref[itm:04-21]{Aufsichtsbehörde};
    \label{itm:39-1d}

    \item Tätigkeit als Anlaufstelle für die \hyperref[itm:04-21]{Aufsichtsbehörde} in mit der \hyperref[itm:04-2]{Verarbeitung} zusammenhängenden Fragen,
     einschließlich der vorherigen Konsultation gemäß \hyperref[ch:36]{Artikel 36}, und gegebenenfalls Beratung zu
     allen sonstigen Fragen.
    \label{itm:39-1e}

  \end{enumerate}

  \item Der Datenschutzbeauftragte trägt bei der Erfüllung seiner Aufgaben dem mit den Verarbeitungsvorgängen
   verbundenen Risiko gebührend Rechnung, wobei er die Art, den Umfang, die Umstände und die Zwecke der \hyperref[itm:04-2]{Verarbeitung}
   berücksichtigt.
  \label{itm:39-e}

\end{enumerate}

\addsec{Eigene Notizen}

