%!TEX root = ../DSGVO-Bearbeitung.tex
\chapter{Garantien und Ausnahmen in Bezug auf die Verarbeitung zu im öffentlichen Interesse liegenden Archivzwecken, zu
 wissenschaftlichen oder historischen Forschungszwecken und zu statistischen Zwecken}
\label{ch:89}

\addsec{Text der Verordnung}

\begin{enumerate}

  \item Die \hyperref[itm:04-2]{Verarbeitung} zu im öffentlichen Interesse liegenden Archivzwecken, zu wissenschaftlichen oder historischen
   Forschungszwecken oder zu statistischen Zwecken unterliegt geeigneten Garantien für die Rechte und Freiheiten der
   \hyperref[itm:04-1]{betroffenen Person} gemäß dieser Verordnung. Mit diesen Garantien wird sichergestellt, dass technische und
   organisatorische Maßnahmen bestehen, mit denen insbesondere die Achtung des Grundsatzes der Datenminimierung
   gewährleistet wird. Zu diesen Maßnahmen kann die Pseudonymisierung gehören, sofern es möglich ist, diese Zwecke auf
   diese Weise zu erfüllen. In allen Fällen, in denen diese Zwecke durch die Weiterverarbeitung, bei der die
   Identifizierung von \hyperref[itm:04-1]{betroffenen Personen} nicht oder nicht mehr möglich ist, erfüllt werden können, werden diese
   Zwecke auf diese Weise erfüllt.
  \label{itm:89-1}

  \item Werden \hyperref[itm:04-1]{personenbezogene Daten} zu wissenschaftlichen oder historischen Forschungszwecken oder zu statistischen
   Zwecken verarbeitet, können vorbehaltlich der Bedingungen und Garantien gemäß \hyperref[itm:89-1]{Absatz 1 des
   vorliegenden Artikels} im Unionsrecht oder im Recht der Mitgliedstaaten insoweit Ausnahmen von den Rechten gemäß der
   Artikel \hyperref[ch:15]{15}, \hyperref[ch:16]{16}, \hyperref[ch:18]{18} und \hyperref[ch:21]{21} vorgesehen werden,
   als diese Rechte voraussichtlich die Verwirklichung der spezifischen Zwecke unmöglich machen oder ernsthaft
   beeinträchtigen und solche Ausnahmen für die Erfüllung dieser Zwecke notwendig sind.
  \label{itm:89-2}

  \item Werden \hyperref[itm:04-1]{personenbezogene Daten} für im öffentlichen Interesse liegende Archivzwecke verarbeitet, können
   vorbehaltlich der Bedingungen und Garantien gemäß \hyperref[itm:89-1]{Absatz 1 des vorliegenden Artikels} im
   Unionsrecht oder im Recht der Mitgliedstaaten insoweit Ausnahmen von den Rechten gemäß der Artikel \hyperref[ch:15]
   {15}, \hyperref[ch:16]{16}, \hyperref[ch:18]{18}, \hyperref[ch:19]{19}, \hyperref[ch:20]{20} und \hyperref[ch:21]
   {21} vorgesehen werden, als diese Rechte voraussichtlich die Verwirklichung der spezifischen Zwecke unmöglich machen
   oder ernsthaft beeinträchtigen und solche Ausnahmen für die Erfüllung dieser Zwecke notwendig sind.
  \label{itm:89-3}

  \item Dient die in den Absätzen \hyperref[itm:89-2]{2} und \hyperref[itm:89-3]{3} genannte \hyperref[itm:04-2]{Verarbeitung} gleichzeitig
   einem anderen Zweck, gelten die Ausnahmen nur für die \hyperref[itm:04-2]{Verarbeitung} zu den in diesen Absätzen genannten Zwecken.
  \label{itm:89-4}

\end{enumerate}

\addsec{Eigene Notizen}

