%!TEX root = ../DSGVO-Bearbeitung.tex
\chapter{Sachlicher Anwendungsbereich}
\label{ch:02}

\addsec{Text der Verordnung}

\begin{enumerate}

  \item Diese Verordnung gilt für die ganz oder teilweise automatisierte Verarbeitung personenbezogener Daten sowie für
   die nichtautomatisierte Verarbeitung personenbezogener Daten, die in einem Dateisystem gespeichert sind oder
   gespeichert werden sollen.

  \item Diese Verordnung findet keine Anwendung auf die Verarbeitung personenbezogener Daten

  \begin{enumerate}
  
    \item im Rahmen einer Tätigkeit, die nicht in den Anwendungsbereich des Unionsrechts fällt,

    \item durch die Mitgliedstaaten im Rahmen von Tätigkeiten, die in den Anwendungsbereich von Titel V Kapitel 2 EUV
     fallen,

    \item durch natürliche Personen zur Ausübung ausschließlich persönlicher oder familiärer Tätigkeiten,

    \item durch die zuständigen Behörden zum Zwecke der Verhütung, Ermittlung, Aufdeckung oder Verfolgung von Straftaten
     oder der Strafvollstreckung, einschließlich des Schutzes vor und der Abwehr von Gefahren für die öffentliche
     Sicherheit.

  \end{enumerate}

  \item Für die Verarbeitung personenbezogener Daten durch die Organe, Einrichtungen, Ämter und Agenturen der Union gilt
   die Verordnung (EG) Nr. 45/2001. Die Verordnung (EG) Nr. 45/2001 und sonstige Rechtsakte der Union, die diese
   Verarbeitung personenbezogener Daten regeln, werden im Einklang mit Artikel 98 an die Grundsätze und Vorschriften
   der vorliegenden Verordnung angepasst.

  \item Die vorliegende Verordnung lässt die Anwendung der Richtlinie 2000/31/EG und speziell die Vorschriften der
   Artikel 12 bis 15 dieser Richtlinie zur Verantwortlichkeit der Vermittler unberührt.

\end{enumerate}

\addsec{Eigene Notizen}

