%!TEX root = ../DSGVO-Bearbeitung.tex
\chapter{Sachlicher Anwendungsbereich}
\label{ch:2}

\addsec{Text der Verordnung}

\begin{enumerate}

  \item Diese Verordnung gilt für die ganz oder teilweise automatisierte \hyperref[itm:04-2]{Verarbeitung} \hyperref
   [itm:04-1]{personenbezogener Daten} sowie für die nichtautomatisierte \hyperref[itm:04-2]{Verarbeitung} \hyperref
   [itm:04-1]{personenbezogener Daten}, die in einem \hyperref[itm:04-6]{Dateisystem} gespeichert sind oder gespeichert
   werden sollen.
  \label{itm:02-1}

  \item Diese Verordnung findet keine Anwendung auf die \hyperref[itm:04-2]{Verarbeitung} \hyperref[itm:04-1]
   {personenbezogener Daten}
  \label{itm:02-2}

  \begin{enumerate}
  
    \item im Rahmen einer Tätigkeit, die nicht in den Anwendungsbereich des Unionsrechts fällt,
    \label{itm:02-2a}

    \item durch die Mitgliedstaaten im Rahmen von Tätigkeiten, die in den Anwendungsbereich von Titel V Kapitel 2 EUV
     fallen,
    \label{itm:02-2b}

    \item durch natürliche Personen zur Ausübung ausschließlich persönlicher oder familiärer Tätigkeiten,
    \label{itm:02-2c}

    \item durch die zuständigen Behörden zum Zwecke der Verhütung, Ermittlung, Aufdeckung oder Verfolgung von Straftaten
     oder der Strafvollstreckung, einschließlich des Schutzes vor und der Abwehr von Gefahren für die öffentliche
     Sicherheit.
    \label{itm:02-2d}

  \end{enumerate}

  \item Für die \hyperref[itm:04-2]{Verarbeitung} \hyperref[itm:04-1]{personenbezogener Daten} durch die Organe,
   Einrichtungen, Ämter und Agenturen der Union gilt die Verordnung (EG) Nr. 45/2001. Die Verordnung (EG) Nr. 45/2001
   und sonstige Rechtsakte der Union, die diese \hyperref[itm:04-2]{Verarbeitung} \hyperref[itm:04-1]
   {personenbezogener Daten} regeln, werden im Einklang mit \hyperref[ch:98]{Artikel 98} an die Grundsätze und
   Vorschriften der vorliegenden Verordnung angepasst.
  \label{itm:02-3}

  \item Die vorliegende Verordnung lässt die Anwendung der Richtlinie 2000/31/EG und speziell die Vorschriften der
   Artikel 12 bis 15 dieser Richtlinie zur Verantwortlichkeit der Vermittler unberührt.
  \label{itm:02-4}

\end{enumerate}

\addsec{Eigene Notizen}

\begin{itemize}

  \item Der Verweis auf \href
   {https://eur-lex.europa.eu/legal-content/DE/TXT/HTML/?uri=CELEX:12016M/TXT&qid=1659176223545&from=DE#d1e1808-1-1}
   {Titel V Kapitel 2 EUV} in \hyperref[itm:02-2a]{Absatz 2a} bezieht sich auf die gemeinsame Außen- und
   Sicherheitspolitik der EU"=Mitgliedsstaaten.

  \item Die \href{https://eur-lex.europa.eu/legal-content/DE/TXT/HTML/?uri=CELEX:32001R0045&qid=1659176445961&from=DE}
   {Verordnung (EG) Nr. 45/2001} in \hyperref[itm:02-3]{Absatz 3} hat den Schutz natürlicher Personen bei der
   Verarbeitung \hyperref[itm:04-1]{personenbezogener Daten} durch die Organe und Einrichtungen der Gemeinschaft und
   den freien Datenverkehr zum Inhalt.

  \item Die in \hyperref[itm:02-4]{Absatz 4} referenzierte
   \href{https://eur-lex.europa.eu/legal-content/DE/TXT/HTML/?uri=CELEX:32000L0031&qid=1659176679229&from=DE}
    {Richtlinie 2000/31/EG} ("`Richtlinie über den elektronischen Geschäftsverkehr"') regelt bestimmte rechtliche
    Aspekte der Dienste der Informationsgesellschaft, insbesondere des elektronischen Geschäftsverkehrs, im
    Binnenmarkt. Die referenzierten Artikel betreffen die Themenkomplexe

   \begin{itemize}
     \item Reine Durchleitung (Artikel 12)
     \item Caching (Artikel 13)
     \item Hosting (Artikel 14)
     \item Keine allgemeine Überwachungspflicht (Artikel 15)
   \end{itemize}

\end{itemize}
