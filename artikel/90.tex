%!TEX root = ../DSGVO-Bearbeitung.tex
\chapter{Geheimhaltungspflichten}
\label{ch:90}

\addsec{Text der Verordnung}

\begin{enumerate}

  \item Die Mitgliedstaaten können die Befugnisse der Aufsichtsbehörden im Sinne des \hyperref[itm:58-1]{Artikels 58
   Absatz 1} Buchstaben \hyperref[itm:58-1e]{e} und \hyperref[itm:58-1f]{f} gegenüber den Verantwortlichen oder den
   Auftragsverarbeitern, die nach Unionsrecht oder dem Recht der Mitgliedstaaten oder nach einer von den zuständigen
   nationalen Stellen erlassenen Verpflichtung dem Berufsgeheimnis oder einer gleichwertigen Geheimhaltungspflicht
   unterliegen, regeln, soweit dies notwendig und verhältnismäßig ist, um das Recht auf Schutz der personenbezogenen
   Daten mit der Pflicht zur Geheimhaltung in Einklang zu bringen. Diese Vorschriften gelten nur in Bezug auf
   \hyperref[itm:04-1]{personenbezogene Daten}, die der Verantwortliche oder der Auftragsverarbeiter bei einer Tätigkeit erlangt oder
   erhoben hat, die einer solchen Geheimhaltungspflicht unterliegt.
  \label{itm:90-1}

  \item Jeder Mitgliedstaat teilt der Kommission bis zum 25. Mai 2018 die Vorschriften mit, die er aufgrund von
   \hyperref[itm:90-1]{Absatz 1} erlässt, und setzt sie unverzüglich von allen weiteren Änderungen dieser Vorschriften
    in Kenntnis.
  \label{itm:90-2}

\end{enumerate}

\addsec{Eigene Notizen}

