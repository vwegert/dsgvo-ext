%!TEX root = ../DSGVO-ExtendedVersion.tex
\chapter{Widerspruchsrecht}
\label{ch:21}

\addsec{Text der Verordnung}

\begin{enumerate}

  \item Die \hyperref[itm:04-1]{betroffene Person} hat das Recht, aus Gründen, die sich aus ihrer besonderen Situation
   ergeben, jederzeit gegen die \hyperref[itm:04-2]{Verarbeitung} sie betreffender \hyperref[itm:04-1]
   {personenbezogener Daten}, die aufgrund von \hyperref[itm:06-1]{Artikel 6 Absatz 1} Buchstaben \hyperref[itm:06-1e]
   {e} oder \hyperref[itm:06-1f]{f} erfolgt, Widerspruch einzulegen; dies gilt auch für ein auf diese Bestimmungen
   gestütztes \hyperref[itm:04-4]{Profiling}. Der \hyperref[itm:04-7]{Verantwortliche} verarbeitet die
   personenbezogenen Daten nicht mehr, es sei denn, er kann zwingende schutzwürdige Gründe für die \hyperref[itm:04-2]
   {Verarbeitung} nachweisen, die die Interessen, Rechte und Freiheiten der \hyperref[itm:04-1]{betroffenen Person}
   überwiegen, oder die \hyperref[itm:04-2]{Verarbeitung} dient der Geltendmachung, Ausübung oder Verteidigung von
   Rechtsansprüchen.%
  \label{itm:21-1}

  \item Werden \hyperref[itm:04-1]{personenbezogene Daten} verarbeitet, um Direktwerbung zu betreiben, so hat
   die \hyperref[itm:04-1]{betroffene Person} das Recht, jederzeit Widerspruch gegen die \hyperref[itm:04-2]
   {Verarbeitung} sie betreffender \hyperref[itm:04-1]{personenbezogener Daten} zum Zwecke derartiger Werbung
   einzulegen; dies gilt auch für das \hyperref[itm:04-4]{Profiling}, soweit es mit solcher Direktwerbung in Verbindung
   steht.%
  \label{itm:21-2}

  \item Widerspricht die \hyperref[itm:04-1]{betroffene Person} der \hyperref[itm:04-2]{Verarbeitung} für Zwecke der
   Direktwerbung, so werden die
   \hyperref[itm:04-1]{personenbezogenen Daten} nicht mehr für diese Zwecke verarbeitet.%
  \label{itm:21-3}

  \item Die \hyperref[itm:04-1]{betroffene Person} muss spätestens zum Zeitpunkt der ersten Kommunikation mit ihr
   ausdrücklich auf das in den Absätzen \hyperref[itm:21-1]{1} und \hyperref[itm:21-2]{2} genannte Recht hingewiesen
   werden; dieser Hinweis hat in einer verständlichen und von anderen Informationen getrennten Form zu erfolgen.%
  \label{itm:21-4}

  \item Im Zusammenhang mit der Nutzung von \hyperref[itm:04-28]{Diensten der Informationsgesellschaft} kann
   die \hyperref[itm:04-1]{betroffene Person} ungeachtet der Richtlinie 2002/58/EG ihr Widerspruchsrecht mittels
   automatisierter Verfahren ausüben, bei denen technische Spezifikationen verwendet werden.%
  \label{itm:21-5}

  \item Die \hyperref[itm:04-1]{betroffene Person} hat das Recht, aus Gründen, die sich aus ihrer besonderen Situation
   ergeben, gegen die sie betreffende \hyperref[itm:04-2]{Verarbeitung} sie betreffender \hyperref[itm:04-1]
   {personenbezogener Daten}, die zu wissenschaftlichen oder historischen Forschungszwecken oder zu statistischen
   Zwecken gemäß \hyperref[itm:89-1]{Artikel 89 Absatz 1} erfolgt, Widerspruch einzulegen, es sei denn, die \hyperref
   [itm:04-2]{Verarbeitung} ist zur Erfüllung einer im öffentlichen Interesse liegenden Aufgabe erforderlich.%
  \label{itm:21-6}

\end{enumerate}

\addsec{Eigene Notizen}

\begin{itemize}

  \item Die in \hyperref[itm:21-5]{Absatz 5} referenzierte
   \href{https://eur-lex.europa.eu/legal-content/DE/TXT/HTML/?uri=CELEX:32002L0058&qid=1659212895940&from=DE}
    {Richtlinie 2002/58/EG} regelt die Verarbeitung \hyperref[itm:04-1]{personenbezogener Daten} und den Schutz der
    Privatsphäre in der elektronischen Kommunikation.

\end{itemize}




