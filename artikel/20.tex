%!TEX root = ../DSGVO-Bearbeitung.tex
\chapter{Recht auf Datenübertragbarkeit}
\label{ch:20}

\addsec{Text der Verordnung}

\begin{enumerate}

  \item Die betroffene Person hat das Recht, die sie betreffenden \hyperref[itm:04-1]{personenbezogenen Daten}, die sie einem
   Verantwortlichen bereitgestellt hat, in einem strukturierten, gängigen und maschinenlesbaren Format zu erhalten, und
   sie hat das Recht, diese Daten einem anderen Verantwortlichen ohne Behinderung durch den Verantwortlichen, dem die
   \hyperref[itm:04-1]{personenbezogenen Daten} bereitgestellt wurden, zu übermitteln, sofern
  \label{itm:20-1}

  \begin{enumerate}
  
    \item die Verarbeitung auf einer Einwilligung gemäß \hyperref[itm:06-1a]{Artikel 6 Absatz 1 Buchstabe a} oder
    \hyperref[itm:09-2a]{Artikel 9 Absatz 2 Buchstabe a} oder auf einem Vertrag gemäß \hyperref[itm:06-1b]{Artikel 6
     Absatz 1 Buchstabe b} beruht und
  \label{itm:20-1a}

    \item die Verarbeitung mithilfe automatisierter Verfahren erfolgt.
  \label{itm:20-1b}

  \end{enumerate}

  \item Bei der Ausübung ihres Rechts auf Datenübertragbarkeit gemäß \hyperref[itm:20-1]{Absatz 1} hat die betroffene
   Person das Recht, zu erwirken, dass die \hyperref[itm:04-1]{personenbezogenen Daten} direkt von einem Verantwortlichen einem anderen
   Verantwortlichen übermittelt werden, soweit dies technisch machbar ist.
  \label{itm:20-2}

  \item Die Ausübung des Rechts nach \hyperref[itm:20-1]{Absatz 1} des vorliegenden Artikels lässt \hyperref[ch:17]
   {Artikel 17} unberührt. Dieses Recht gilt nicht für eine Verarbeitung, die für die Wahrnehmung einer Aufgabe
   erforderlich ist, die im öffentlichen Interesse liegt oder in Ausübung öffentlicher Gewalt erfolgt, die dem
   Verantwortlichen übertragen wurde.
  \label{itm:20-3}

  \item Das Recht gemäß \hyperref[itm:20-2]{Absatz 2} darf die Rechte und Freiheiten anderer Personen nicht
   beeinträchtigen.
  \label{itm:20-4}

\end{enumerate}

\addsec{Eigene Notizen}

