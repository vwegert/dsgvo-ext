%!TEX root = ../DSGVO-ExtendedVersion.tex
\chapter{Automatisierte Entscheidungen im Einzelfall einschließlich Profiling}
\label{ch:22}

\crossrefArticleToReason{22}

\begin{enumerate}

  \item Die \hyperref[itm:04-1]{betroffene Person} hat das Recht, nicht einer ausschließlich auf einer
   automatisierten \hyperref[itm:04-2]{Verarbeitung} -- einschließlich \hyperref[itm:04-4]{Profiling} -- beruhenden
   Entscheidung\comment{Benachrichtigungspflicht nach \hyperref[itm:13-2f]{Artikel 13 Absatz 2 Buchstabe f; Pflicht
   zur Datenschutz"=Folgenabschätzung nach \hyperref[itm:35-3a]{Artikel 35 Absatz 3 Buchstabe a}},
   \hyperref[itm:14-2g]{Artikel 14 Absatz 2 Buchstabe g} oder \hyperref[itm:15-1h]{Artikel 15 Absatz 1 Buchstabe h}
   beachten} unterworfen zu werden, die ihr gegenüber
   rechtliche Wirkung entfaltet oder sie in ähnlicher Weise erheblich beeinträchtigt.%
  \label{itm:22-1}

  \item \hyperref[itm:22-1]{Absatz 1} gilt nicht, wenn die Entscheidung\comment{Leitlinien, Empfehlungen
   und bewährte Verfahren des Ausschusses siehe \hyperref[itm:70-1f]{Artikel 70 Absatz 1 Buchstabe f}}%
  \label{itm:22-2}

  \begin{enumerate}
  
    \item für den Abschluss oder die Erfüllung eines Vertrags zwischen der \hyperref[itm:04-1]{betroffenen Person} und
     dem \hyperref[itm:04-7]{Verantwortlichen} erforderlich ist,%
    \label{itm:22-2a}

    \item aufgrund von Rechtsvorschriften der Union oder der Mitgliedstaaten, denen der \hyperref[itm:04-7]
     {Verantwortliche} unterliegt, zulässig ist und diese Rechtsvorschriften angemessene Maßnahmen zur Wahrung der
     Rechte und Freiheiten sowie der berechtigten Interessen der \hyperref[itm:04-1]{betroffenen Person} enthalten
     oder%
    \label{itm:22-2b}

    \item mit ausdrücklicher \hyperref[itm:04-11]{Einwilligung} der \hyperref[itm:04-1]{betroffenen Person} erfolgt.%
    \label{itm:22-2c}

  \end{enumerate}

  \item In den in \hyperref[itm:22-2]{Absatz 2} Buchstaben \hyperref[itm:22-2a]{a} und \hyperref[itm:22-2c]{c} genannten
   Fällen trifft der \hyperref[itm:04-7]{Verantwortliche} angemessene Maßnahmen, um die Rechte und Freiheiten sowie die
   berechtigten Interessen der \hyperref[itm:04-1]{betroffenen Person} zu wahren, wozu mindestens das Recht auf
   Erwirkung des Eingreifens einer Person seitens des \hyperref[itm:04-7]{Verantwortlichen}, auf Darlegung des eigenen
   Standpunkts und auf Anfechtung der Entscheidung gehört.%
  \label{itm:22-3}

  \item Entscheidungen nach \hyperref[itm:22-2]{Absatz 2} dürfen nicht auf besonderen Kategorien \hyperref[itm:04-1]
   {personenbezogener Daten} nach \hyperref[itm:09-1]{Artikel 9 Absatz 1} beruhen, sofern nicht \hyperref[itm:09-2]
   {Artikel 9 Absatz 2} Buchstabe
  \hyperref[itm:09-2a]{a} oder \hyperref[itm:09-2g]{g} gilt und angemessene Maßnahmen zum Schutz der Rechte und
   Freiheiten sowie der berechtigten Interessen der \hyperref[itm:04-1]{betroffenen Person} getroffen wurden.%
  \label{itm:22-4}

\end{enumerate}

% \addsec{Ergänzende Hinweise}

