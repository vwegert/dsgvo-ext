%!TEX root = ../DSGVO-Bearbeitung.tex
\chapter{Ausnahmen für bestimmte Fälle}
\label{ch:49}

\addsec{Text der Verordnung}

\begin{enumerate}

  \item Falls weder ein Angemessenheitsbeschluss nach \hyperref[itm:45-3]{Artikel 45 Absatz 3} vorliegt noch geeignete
   Garantien nach \hyperref[ch:46]{Artikel 46}, einschließlich \hyperref[itm:04-20]{verbindlicher interner
   Datenschutzvorschriften}, bestehen, ist eine Übermittlung oder eine Reihe von Übermittlungen \hyperref[itm:04-1]
   {personenbezogener Daten} an ein Drittland oder an eine \hyperref[itm:04-29]{internationale Organisation} nur unter
   einer der folgenden Bedingungen zulässig:%
  \label{itm:49-1-1}

  \begin{enumerate}
  
    \item die \hyperref[itm:04-1]{betroffene Person} hat in die vorgeschlagene Datenübermittlung ausdrücklich
     eingewilligt, nachdem sie über die für sie bestehenden möglichen Risiken derartiger Datenübermittlungen ohne
     Vorliegen eines Angemessenheitsbeschlusses und ohne geeignete Garantien unterrichtet wurde,%
    \label{itm:49-1-1a}

    \item die Übermittlung ist für die Erfüllung eines Vertrags zwischen der \hyperref[itm:04-1]{betroffenen Person} und
     dem \hyperref[itm:04-7]{Verantwortlichen} oder zur Durchführung von vorvertraglichen Maßnahmen auf Antrag
     der \hyperref[itm:04-1]{betroffenen Person} erforderlich,%
    \label{itm:49-1-1b}

    \item die Übermittlung ist zum Abschluss oder zur Erfüllung eines im Interesse der \hyperref[itm:04-1]
     {betroffenen Person} von dem
     \hyperref[itm:04-7]{Verantwortlichen} mit einer anderen natürlichen oder juristischen Person geschlossenen Vertrags
      erforderlich,%
    \label{itm:49-1-1c}

    \item die Übermittlung ist aus wichtigen Gründen des öffentlichen Interesses notwendig,%
    \label{itm:49-1-1d}

    \item die Übermittlung ist zur Geltendmachung, Ausübung oder Verteidigung von Rechtsansprüchen erforderlich,%
    \label{itm:49-1-1e}

    \item die Übermittlung ist zum Schutz lebenswichtiger Interessen der \hyperref[itm:04-1]{betroffenen Person} oder
     anderer Personen erforderlich, sofern die \hyperref[itm:04-1]{betroffene Person} aus physischen oder rechtlichen
     Gründen außerstande ist, ihre
     \hyperref[itm:04-11]{Einwilligung} zu geben,%
    \label{itm:49-1-1f}

    \item die Übermittlung erfolgt aus einem Register, das gemäß dem Recht der Union oder der Mitgliedstaaten zur
     Information der Öffentlichkeit bestimmt ist und entweder der gesamten Öffentlichkeit oder allen Personen, die ein
     berechtigtes Interesse nachweisen können, zur Einsichtnahme offensteht, aber nur soweit die im Recht der Union
     oder der Mitgliedstaaten festgelegten Voraussetzungen für die Einsichtnahme im Einzelfall gegeben sind.%
    \label{itm:49-1-1g}

  \end{enumerate}

  \phantomsection Falls die Übermittlung nicht auf eine Bestimmung der Artikel \hyperref[ch:45]{45} oder \hyperref
   [ch:46]{46} -- einschließlich der \hyperref[itm:04-20]{verbindlichen internen Datenschutzvorschriften} -- gestützt
   werden könnte und keine der Ausnahmen für einen bestimmten Fall gemäß dem ersten Unterabsatz anwendbar ist, darf
   eine Übermittlung an ein Drittland oder eine \hyperref[itm:04-29]{internationale Organisation} nur dann erfolgen,
   wenn die Übermittlung nicht wiederholt erfolgt, nur eine begrenzte Zahl von \hyperref[itm:04-1]
   {betroffenen Personen} betrifft, für die Wahrung der zwingenden berechtigten Interessen des
  \hyperref[itm:04-7]{Verantwortlichen} erforderlich ist, sofern die Interessen oder die Rechte und Freiheiten
   der \hyperref[itm:04-1]{betroffenen Person} nicht überwiegen, und der \hyperref[itm:04-7]{Verantwortliche} alle
   Umstände der Datenübermittlung beurteilt und auf der Grundlage dieser Beurteilung geeignete Garantien in Bezug auf
   den Schutz \hyperref[itm:04-1]{personenbezogener Daten} vorgesehen hat. Der \hyperref[itm:04-7]
   {Verantwortliche} setzt die \hyperref[itm:04-21]{Aufsichtsbehörde} von der Übermittlung in Kenntnis. Der \hyperref
   [itm:04-7]{Verantwortliche} unterrichtet die \hyperref[itm:04-1]{betroffene Person} über die Übermittlung und seine
   zwingenden berechtigten Interessen; dies erfolgt zusätzlich zu den der \hyperref[itm:04-1]{betroffenen Person} nach
   den Artikeln \hyperref[ch:13]{13} und \hyperref[ch:14]{14} mitgeteilten Informationen.%
  \label{itm:49-1-2}

  \item Datenübermittlungen gemäß \hyperref[itm:49-1-1g]{Absatz 1 Unterabsatz 1 Buchstabe g} dürfen nicht die Gesamtheit
   oder ganze Kategorien der im Register enthaltenen \hyperref[itm:04-1]{personenbezogenen Daten} umfassen. Wenn das
   Register der Einsichtnahme durch Personen mit berechtigtem Interesse dient, darf die Übermittlung nur auf Anfrage
   dieser Personen oder nur dann erfolgen, wenn diese Personen die Adressaten der Übermittlung sind.%
  \label{itm:49-2}

  \item Absatz 1 Unterabsatz 1 Buchstaben \hyperref[itm:49-1-1a]{a}, \hyperref[itm:49-1-1b]{b} und \hyperref
   [itm:49-1-1c]{c} und sowie \hyperref[itm:49-1-2]{Absatz 1 Unterabsatz 2} gelten nicht für Tätigkeiten, die Behörden
   in Ausübung ihrer hoheitlichen Befugnisse durchführen.%
  \label{itm:49-3}

  \item Das öffentliche Interesse im Sinne des \hyperref[itm:49-1-1d]{Absatzes 1 Unterabsatz 1 Buchstabe d} muss im
   Unionsrecht oder im Recht des Mitgliedstaats, dem der \hyperref[itm:04-7]{Verantwortliche} unterliegt, anerkannt
   sein.%
  \label{itm:49-4}

  \item Liegt kein Angemessenheitsbeschluss vor, so können im Unionsrecht oder im Recht der Mitgliedstaaten aus
   wichtigen Gründen des öffentlichen Interesses ausdrücklich Beschränkungen der Übermittlung bestimmter Kategorien
   von
   \hyperref[itm:04-1]{personenbezogenen Daten} an Drittländer oder \hyperref[itm:04-29]{internationale Organisationen}
    vorgesehen werden. Die Mitgliedstaaten teilen der Kommission derartige Bestimmungen mit.%
  \label{itm:49-5}

  \item Der \hyperref[itm:04-7]{Verantwortliche} oder der \hyperref[itm:04-8]{Auftragsverarbeiter} erfasst die von ihm
   vorgenommene Beurteilung sowie die angemessenen Garantien im Sinne des \hyperref[itm:49-1-2]{Absatzes 1 Unterabsatz
   2 des vorliegenden Artikels} in der Dokumentation gemäß \hyperref[ch:30]{Artikel 30}.%
  \label{itm:49-6}

\end{enumerate}

\addsec{Eigene Notizen}

