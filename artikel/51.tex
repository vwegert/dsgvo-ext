%!TEX root = ../DSGVO-ExtendedVersion.tex
\chapter{Aufsichtsbehörde}
\label{ch:51}

\begin{enumerate}

  \item Jeder Mitgliedstaat sieht vor, dass eine oder mehrere unabhängige Behörden für die Überwachung der Anwendung
   dieser Verordnung zuständig sind, damit die Grundrechte und Grundfreiheiten natürlicher Personen bei der
   \hyperref[itm:04-2]{Verarbeitung} geschützt werden und der freie Verkehr \hyperref[itm:04-1]{personenbezogener Daten}
    in der Union erleichtert wird(im Folgenden „\hyperref[itm:04-21]{Aufsichtsbehörde}“).%
  \label{itm:51-1}

  \item Jede \hyperref[itm:04-21]{Aufsichtsbehörde} leistet einen Beitrag zur einheitlichen Anwendung dieser Verordnung
   in der gesamten Union. Zu diesem Zweck arbeiten die \hyperref[itm:04-21]{Aufsichtsbehörden} untereinander sowie mit
   der Kommission gemäß \hyperref[part:7]{Kapitel VII} zusammen.%
  \label{itm:51-2}

  \item Gibt es in einem Mitgliedstaat mehr als eine \hyperref[itm:04-21]{Aufsichtsbehörde}, so bestimmt dieser
   Mitgliedstaat die
   \hyperref[itm:04-21]{Aufsichtsbehörde}, die diese Behörden im Ausschuss vertritt, und führt ein Verfahren ein, mit
    dem sichergestellt wird, dass die anderen Behörden die Regeln für das Kohärenzverfahren nach \hyperref[ch:63]
    {Artikel 63} einhalten.%
  \label{itm:51-3}

  \item Jeder Mitgliedstaat teilt der Kommission bis spätestens 25. Mai 2018 die Rechtsvorschriften, die er aufgrund
   dieses Kapitels erlässt, sowie unverzüglich alle folgenden Änderungen dieser Vorschriften mit.%
  \label{itm:51-4}

\end{enumerate}

% \addsec{Ergänzende Hinweise}

