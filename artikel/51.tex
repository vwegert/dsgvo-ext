%!TEX root = ../DSGVO-Bearbeitung.tex
\chapter{Aufsichtsbehörde}
\label{ch:51}

\addsec{Text der Verordnung}

\begin{enumerate}

  \item Jeder Mitgliedstaat sieht vor, dass eine oder mehrere unabhängige Behörden für die Überwachung der Anwendung
   dieser Verordnung zuständig sind, damit die Grundrechte und Grundfreiheiten natürlicher Personen bei der
   Verarbeitung geschützt werden und der freie Verkehr personenbezogener Daten in der Union erleichtert wird
   (im Folgenden „Aufsichtsbehörde“).
  \label{itm:51-1}

  \item Jede Aufsichtsbehörde leistet einen Beitrag zur einheitlichen Anwendung dieser Verordnung in der gesamten Union.
   Zu diesem Zweck arbeiten die Aufsichtsbehörden untereinander sowie mit der Kommission gemäß \hyperref[part7]
   {Kapitel VII} zusammen.
  \label{itm:51-2}

  \item Gibt es in einem Mitgliedstaat mehr als eine Aufsichtsbehörde, so bestimmt dieser Mitgliedstaat die
   Aufsichtsbehörde, die diese Behörden im Ausschuss vertritt, und führt ein Verfahren ein, mit dem sichergestellt
   wird, dass die anderen Behörden die Regeln für das Kohärenzverfahren nach \hyperref[ch:63]{Artikel 63} einhalten.
  \label{itm:51-3}

  \item Jeder Mitgliedstaat teilt der Kommission bis spätestens 25. Mai 2018 die Rechtsvorschriften, die er aufgrund
   dieses Kapitels erlässt, sowie unverzüglich alle folgenden Änderungen dieser Vorschriften mit.
  \label{itm:51-4}

\end{enumerate}

\addsec{Eigene Notizen}

