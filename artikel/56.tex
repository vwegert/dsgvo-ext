%!TEX root = ../DSGVO-Bearbeitung.tex
\chapter{Zuständigkeit der federführenden Aufsichtsbehörde}
\label{ch:56}

\addsec{Text der Verordnung}

\begin{enumerate}

  \item Unbeschadet des \hyperref[ch:55]{Artikels 55} ist die Aufsichtsbehörde der Hauptniederlassung oder der einzigen
   Niederlassung des Verantwortlichen oder des Auftragsverarbeiters gemäß dem Verfahren nach \hyperref[ch:60]
   {Artikel 60} die zuständige federführende Aufsichtsbehörde für die von diesem Verantwortlichen oder diesem
   Auftragsverarbeiter durchgeführte \hyperref[itm:04-23]{grenzüberschreitende Verarbeitung}.
  \label{itm:56-1}

  \item Abweichend von \hyperref[itm:56-1]{Absatz 1} ist jede Aufsichtsbehörde dafür zuständig, sich mit einer bei ihr
   eingereichten Beschwerde oder einem etwaigen Verstoß gegen diese Verordnung zu befassen, wenn der Gegenstand nur mit
   einer Niederlassung in ihrem Mitgliedstaat zusammenhängt oder \hyperref[itm:04-1]{betroffene Personen} nur ihres Mitgliedstaats erheblich
   beeinträchtigt.
  \label{itm:56-2}

  \item In den in \hyperref[itm:56-2]{Absatz 2 des vorliegenden Artikels} genannten Fällen unterrichtet die
   Aufsichtsbehörde unverzüglich die federführende Aufsichtsbehörde über diese Angelegenheit. Innerhalb einer Frist von
   drei Wochen nach der Unterrichtung entscheidet die federführende Aufsichtsbehörde, ob sie sich mit dem Fall gemäß
   dem Verfahren nach \hyperref[ch:60]{Artikel 60} befasst oder nicht, wobei sie berücksichtigt, ob der Verantwortliche
   oder der Auftragsverarbeiter in dem Mitgliedstaat, dessen Aufsichtsbehörde sie unterrichtet hat, eine Niederlassung
   hat oder nicht.
  \label{itm:56-3}

  \item Entscheidet die federführende Aufsichtsbehörde, sich mit dem Fall zu befassen, so findet das Verfahren nach
   \hyperref[ch:60]{Artikel 60} Anwendung. Die Aufsichtsbehörde, die die federführende Aufsichtsbehörde unterrichtet
    hat, kann dieser einen Beschlussentwurf vorlegen. Die federführende Aufsichtsbehörde trägt diesem Entwurf bei der
    Ausarbeitung des Beschlussentwurfs nach \hyperref[itm:60-3]{Artikel 60 Absatz 3} weitestgehend Rechnung.
  \label{itm:56-4}

  \item Entscheidet die federführende Aufsichtsbehörde, sich mit dem Fall nicht selbst zu befassen, so befasst die
   Aufsichtsbehörde, die die federführende Aufsichtsbehörde unterrichtet hat, sich mit dem Fall gemäß den Artikeln
   \hyperref[ch:61]{61} und \hyperref[ch:62]{62}.
  \label{itm:56-5}

  \item Die federführende Aufsichtsbehörde ist der einzige Ansprechpartner der Verantwortlichen oder der
   Auftragsverarbeiter für Fragen der von diesem Verantwortlichen oder diesem Auftragsverarbeiter durchgeführten
   \hyperref[itm:04-23]{grenzüberschreitenden Verarbeitung}.
  \label{itm:56-6}

\end{enumerate}

\addsec{Eigene Notizen}

