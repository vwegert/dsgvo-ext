%!TEX root = ../DSGVO-ExtendedVersion.tex
\chapter{Zuständigkeit der federführenden Aufsichtsbehörde}
\label{ch:56}

\addsec{Text der Verordnung}

\begin{enumerate}

  \item Unbeschadet des \hyperref[ch:55]{Artikels 55} ist die \hyperref[itm:04-21]{Aufsichtsbehörde} der \hyperref
   [itm:04-16]{Hauptniederlassung} oder der einzigen Niederlassung des \hyperref[itm:04-7]{Verantwortlichen} oder
   des \hyperref[itm:04-8]{Auftragsverarbeiters} gemäß dem Verfahren nach \hyperref[ch:60]{Artikel 60} die zuständige
   federführende \hyperref[itm:04-21]{Aufsichtsbehörde} für die von diesem \hyperref[itm:04-7]{Verantwortlichen} oder
   diesem
   \hyperref[itm:04-8]{Auftragsverarbeiter} durchgeführte \hyperref[itm:04-23]{grenzüberschreitende Verarbeitung}.%
  \label{itm:56-1}

  \item Abweichend von \hyperref[itm:56-1]{Absatz 1} ist jede \hyperref[itm:04-21]{Aufsichtsbehörde} dafür zuständig,
   sich mit einer bei ihr eingereichten Beschwerde oder einem etwaigen Verstoß gegen diese Verordnung zu befassen, wenn
   der Gegenstand nur mit einer Niederlassung in ihrem Mitgliedstaat zusammenhängt oder \hyperref[itm:04-1]
   {betroffene Personen} nur ihres Mitgliedstaats erheblich beeinträchtigt.%
  \label{itm:56-2}

  \item In den in \hyperref[itm:56-2]{Absatz 2 des vorliegenden Artikels} genannten Fällen unterrichtet die
   \hyperref[itm:04-21]{Aufsichtsbehörde} unverzüglich die federführende \hyperref[itm:04-21]{Aufsichtsbehörde} über
    diese Angelegenheit. Innerhalb einer Frist von drei Wochen nach der Unterrichtung entscheidet die
    federführende \hyperref[itm:04-21]{Aufsichtsbehörde}, ob sie sich mit dem Fall gemäß dem Verfahren nach \hyperref
    [ch:60]{Artikel 60} befasst oder nicht, wobei sie berücksichtigt, ob der \hyperref[itm:04-7]{Verantwortliche} oder
    der \hyperref[itm:04-8]{Auftragsverarbeiter} in dem Mitgliedstaat, dessen \hyperref[itm:04-21]
    {Aufsichtsbehörde} sie unterrichtet hat, eine Niederlassung hat oder nicht.%
  \label{itm:56-3}

  \item Entscheidet die federführende \hyperref[itm:04-21]{Aufsichtsbehörde}, sich mit dem Fall zu befassen, so findet
   das Verfahren nach
   \hyperref[ch:60]{Artikel 60} Anwendung. Die \hyperref[itm:04-21]{Aufsichtsbehörde}, die die federführende \hyperref
    [itm:04-21]{Aufsichtsbehörde} unterrichtet hat, kann dieser einen Beschlussentwurf vorlegen. Die
    federführende \hyperref[itm:04-21]{Aufsichtsbehörde} trägt diesem Entwurf bei der Ausarbeitung des
    Beschlussentwurfs nach \hyperref[itm:60-3]{Artikel 60 Absatz 3} weitestgehend Rechnung.%
  \label{itm:56-4}

  \item Entscheidet die federführende \hyperref[itm:04-21]{Aufsichtsbehörde}, sich mit dem Fall nicht selbst zu
   befassen, so befasst die
   \hyperref[itm:04-21]{Aufsichtsbehörde}, die die federführende \hyperref[itm:04-21]{Aufsichtsbehörde} unterrichtet
    hat, sich mit dem Fall gemäß den Artikeln
   \hyperref[ch:61]{61} und \hyperref[ch:62]{62}.%
  \label{itm:56-5}

  \item Die federführende \hyperref[itm:04-21]{Aufsichtsbehörde} ist der einzige Ansprechpartner der \hyperref[itm:04-7]
   {Verantwortlichen} oder der
   \hyperref[itm:04-8]{Auftragsverarbeiter} für Fragen der von diesem \hyperref[itm:04-7]{Verantwortlichen} oder
    diesem \hyperref[itm:04-8]{Auftragsverarbeiter} durchgeführten
   \hyperref[itm:04-23]{grenzüberschreitenden Verarbeitung}.%
  \label{itm:56-6}

\end{enumerate}

\addsec{Eigene Notizen}

