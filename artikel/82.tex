%!TEX root = ../DSGVO-Bearbeitung.tex
\chapter{Haftung und Recht auf Schadenersatz}
\label{ch:82}

\addsec{Text der Verordnung}

\begin{enumerate}

  \item Jede Person, der wegen eines Verstoßes gegen diese Verordnung ein materieller oder immaterieller Schaden
   entstanden ist, hat Anspruch auf Schadenersatz gegen den Verantwortlichen oder gegen den Auftragsverarbeiter.
  \label{itm:82-1}

  \item Jeder an einer Verarbeitung beteiligte Verantwortliche haftet für den Schaden, der durch eine nicht dieser
   Verordnung entsprechende Verarbeitung verursacht wurde. Ein Auftragsverarbeiter haftet für den durch eine
   Verarbeitung verursachten Schaden nur dann, wenn er seinen speziell den Auftragsverarbeitern auferlegten Pflichten
   aus dieser Verordnung nicht nachgekommen ist oder unter Nichtbeachtung der rechtmäßig erteilten Anweisungen des für
   die Datenverarbeitung Verantwortlichen oder gegen diese Anweisungen gehandelt hat.
  \label{itm:82-2}

  \item Der Verantwortliche oder der Auftragsverarbeiter wird von der Haftung gemäß \hyperref[itm:82-2]{Absatz 2}
   befreit, wenn er nachweist, dass er in keinerlei Hinsicht für den Umstand, durch den der Schaden eingetreten ist,
   verantwortlich ist.
  \label{itm:82-3}

  \item Ist mehr als ein Verantwortlicher oder mehr als ein Auftragsverarbeiter bzw. sowohl ein Verantwortlicher als
   auch ein Auftragsverarbeiter an derselben Verarbeitung beteiligt und sind sie gemäß den Absätzen \hyperref[itm:82-2]
   {2} und \hyperref[itm:82-3]{3} für einen durch die Verarbeitung verursachten Schaden verantwortlich, so haftet jeder
   Verantwortliche oder jeder Auftragsverarbeiter für den gesamten Schaden, damit ein wirksamer Schadensersatz für die
   betroffene Person sichergestellt ist.
  \label{itm:82-4}

  \item Hat ein Verantwortlicher oder Auftragsverarbeiter gemäß \hyperref[itm:82-4]{Absatz 4} vollständigen
   Schadenersatz für den erlittenen Schaden gezahlt, so ist dieser Verantwortliche oder Auftragsverarbeiter berechtigt,
   von den übrigen an derselben Verarbeitung beteiligten für die Datenverarbeitung Verantwortlichen oder
   Auftragsverarbeitern den Teil des Schadenersatzes zurückzufordern, der unter den in \hyperref[itm:82-2]{Absatz 2}
   festgelegten Bedingungen ihrem Anteil an der Verantwortung für den Schaden entspricht.
  \label{itm:82-5}

  \item Mit Gerichtsverfahren zur Inanspruchnahme des Rechts auf Schadenersatz sind die Gerichte zu befassen, die nach
   den in \hyperref[itm:79-2]{Artikel 79 Absatz 2} genannten Rechtsvorschriften des Mitgliedstaats zuständig sind.
  \label{itm:82-6}

\end{enumerate}

\addsec{Eigene Notizen}

