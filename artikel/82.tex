%!TEX root = ../DSGVO-Bearbeitung.tex
\chapter{Haftung und Recht auf Schadenersatz}
\label{ch:82}

\addsec{Text der Verordnung}

\begin{enumerate}

  \item Jede Person, der wegen eines Verstoßes gegen diese Verordnung ein materieller oder immaterieller Schaden
   entstanden ist, hat Anspruch auf Schadenersatz gegen den \hyperref[itm:04-7]{Verantwortlichen} oder gegen
   den \hyperref[itm:04-8]{Auftragsverarbeiter}.%
  \label{itm:82-1}

  \item Jeder an einer \hyperref[itm:04-2]{Verarbeitung} beteiligte \hyperref[itm:04-7]{Verantwortliche} haftet für den
   Schaden, der durch eine nicht dieser Verordnung entsprechende \hyperref[itm:04-2]{Verarbeitung} verursacht wurde.
   Ein \hyperref[itm:04-8]{Auftragsverarbeiter} haftet für den durch eine
   \hyperref[itm:04-2]{Verarbeitung} verursachten Schaden nur dann, wenn er seinen speziell den \hyperref[itm:04-8]
    {Auftragsverarbeitern} auferlegten Pflichten aus dieser Verordnung nicht nachgekommen ist oder unter Nichtbeachtung
    der rechtmäßig erteilten Anweisungen des für die Daten\hyperref[itm:04-2]{verarbeitung} \hyperref[itm:04-7]
    {Verantwortlichen} oder gegen diese Anweisungen gehandelt hat.%
  \label{itm:82-2}

  \item Der \hyperref[itm:04-7]{Verantwortliche} oder der \hyperref[itm:04-8]{Auftragsverarbeiter} wird von der Haftung
   gemäß \hyperref[itm:82-2]{Absatz 2} befreit, wenn er nachweist, dass er in keinerlei Hinsicht für den Umstand, durch
   den der Schaden eingetreten ist, verantwortlich ist.%
  \label{itm:82-3}

  \item Ist mehr als ein \hyperref[itm:04-7]{Verantwortlicher} oder mehr als ein \hyperref[itm:04-8]
   {Auftragsverarbeiter} bzw. sowohl ein \hyperref[itm:04-7]{Verantwortlicher} als auch ein \hyperref[itm:04-8]
   {Auftragsverarbeiter} an derselben \hyperref[itm:04-2]{Verarbeitung} beteiligt und sind sie gemäß den
   Absätzen \hyperref[itm:82-2]{2} und \hyperref[itm:82-3]{3} für einen durch die \hyperref[itm:04-2]
   {Verarbeitung} verursachten Schaden verantwortlich, so haftet jeder
   \hyperref[itm:04-7]{Verantwortliche} oder jeder \hyperref[itm:04-8]{Auftragsverarbeiter} für den gesamten Schaden,
    damit ein wirksamer Schadensersatz für die
   \hyperref[itm:04-1]{betroffene Person} sichergestellt ist.%
  \label{itm:82-4}

  \item Hat ein \hyperref[itm:04-7]{Verantwortlicher} oder \hyperref[itm:04-8]{Auftragsverarbeiter} gemäß \hyperref
   [itm:82-4]{Absatz 4} vollständigen Schadenersatz für den erlittenen Schaden gezahlt, so ist dieser \hyperref
   [itm:04-7]{Verantwortliche} oder \hyperref[itm:04-8]{Auftragsverarbeiter} berechtigt, von den übrigen an
   derselben \hyperref[itm:04-2]{Verarbeitung} beteiligten für die Datenverarbeitung \hyperref[itm:04-7]
   {Verantwortlichen} oder
   \hyperref[itm:04-8]{Auftragsverarbeitern} den Teil des Schadenersatzes zurückzufordern, der unter den in \hyperref
    [itm:82-2]{Absatz 2} festgelegten Bedingungen ihrem Anteil an der Verantwortung für den Schaden entspricht.%
  \label{itm:82-5}

  \item Mit Gerichtsverfahren zur Inanspruchnahme des Rechts auf Schadenersatz sind die Gerichte zu befassen, die nach
   den in \hyperref[itm:79-2]{Artikel 79 Absatz 2} genannten Rechtsvorschriften des Mitgliedstaats zuständig sind.%
  \label{itm:82-6}

\end{enumerate}

\addsec{Eigene Notizen}

