%!TEX root = ../DSGVO-Bearbeitung.tex
\chapter{Verbindliche interne Datenschutzvorschriften}
\label{ch:47}

\addsec{Text der Verordnung}

\begin{enumerate}

  \item Die zuständige Aufsichtsbehörde genehmigt gemäß dem Kohärenzverfahren nach \hyperref[ch:63]{Artikel 63}
   \hyperref[itm:04-20]{verbindliche interne Datenschutzvorschriften}, sofern diese
  \label{itm:47-1}

  \begin{enumerate}
  
    \item rechtlich bindend sind, für alle betreffenden Mitglieder der Unternehmensgruppe oder einer Gruppe von
     Unternehmen, die eine gemeinsame Wirtschaftstätigkeit ausüben, gelten und von diesen Mitgliedern durchgesetzt
     werden, und dies auch für ihre Beschäftigten gilt,
    \label{itm:47-1a}

    \item den \hyperref[itm:04-1]{betroffenen Personen} ausdrücklich durchsetzbare Rechte in Bezug auf die \hyperref[itm:04-2]{Verarbeitung} ihrer
     \hyperref[itm:04-1]{personenbezogenen Daten} übertragen und
    \label{itm:47-1b}

    \item die in \hyperref[itm:47-2]{Absatz 2} festgelegten Anforderungen erfüllen.
    \label{itm:47-1c}

  \end{enumerate}

  \item Die \hyperref[itm:04-20]{verbindlichen internen Datenschutzvorschriften} nach \hyperref[itm:47-1]{Absatz 1} enthalten mindestens
   folgende Angaben:
  \label{itm:47-2}

  \begin{enumerate}
  
    \item Struktur und Kontaktdaten der Unternehmensgruppe oder Gruppe von Unternehmen, die eine gemeinsame
     Wirtschaftstätigkeit ausüben, und jedes ihrer Mitglieder;
    \label{itm:4722a}

    \item die betreffenden Datenübermittlungen oder Reihen von Datenübermittlungen einschließlich der betreffenden Arten
     \hyperref[itm:04-1]{personenbezogener Daten}, Art und Zweck der Datenverarbeitung, Art der \hyperref[itm:04-1]{betroffenen Personen} und das betreffende
     Drittland beziehungsweise die betreffenden Drittländer;
    \label{itm:47-2b}

    \item interne und externe Rechtsverbindlichkeit der betreffenden internen Datenschutzvorschriften;
    \label{itm:47-2c}

    \item die Anwendung der allgemeinen Datenschutzgrundsätze, insbesondere Zweckbindung, Datenminimierung, begrenzte
     Speicherfristen, Datenqualität, Datenschutz durch Technikgestaltung und durch datenschutzfreundliche
     Voreinstellungen, Rechtsgrundlage für die \hyperref[itm:04-2]{Verarbeitung}, \hyperref[itm:04-2]{Verarbeitung} besonderer Kategorien von personenbezogenen
     Daten, Maßnahmen zur Sicherstellung der Datensicherheit und Anforderungen für die Weiterübermittlung an nicht an
     diese internen Datenschutzvorschriften gebundene Stellen;
    \label{itm:47-2d}

    \item die Rechte der \hyperref[itm:04-1]{betroffenen Personen} in Bezug auf die \hyperref[itm:04-2]{Verarbeitung} und die diesen offenstehenden Mittel zur
     Wahrnehmung dieser Rechte einschließlich des Rechts, nicht einer ausschließlich auf einer automatisierten
     \hyperref[itm:04-2]{Verarbeitung} -- einschließlich Profiling -- beruhenden Entscheidung nach \hyperref[ch:22]{Artikel 22} unterworfen
     zu werden sowie des in \hyperref[ch:79]{Artikel 79} niedergelegten Rechts auf Beschwerde bei der zuständigen
     Aufsichtsbehörde beziehungsweise auf Einlegung eines Rechtsbehelfs bei den zuständigen Gerichten der
     Mitgliedstaaten und im Falle einer Verletzung der \hyperref[itm:04-20]{verbindlichen internen Datenschutzvorschriften} Wiedergutmachung
     und gegebenenfalls Schadenersatz zu erhalten;
    \label{itm:47-2e}

    \item die von dem in einem Mitgliedstaat niedergelassenen Verantwortlichen oder Auftragsverarbeiter übernommene
     Haftung für etwaige Verstöße eines nicht in der Union niedergelassenen betreffenden Mitglieds der
     Unternehmensgruppe gegen die \hyperref[itm:04-20]{verbindlichen internen Datenschutzvorschriften}; der Verantwortliche oder der
     Auftragsverarbeiter ist nur dann teilweise oder vollständig von dieser Haftung befreit, wenn er nachweist, dass
     der Umstand, durch den der Schaden eingetreten ist, dem betreffenden Mitglied nicht zur Last gelegt werden kann;
    \label{itm:47-2f}

    \item die Art und Weise, wie die \hyperref[itm:04-1]{betroffenen Personen} über die Bestimmungen der Artikel \hyperref[ch:13]{13} und
     \hyperref[ch:14]{14} hinaus über die \hyperref[itm:04-20]{verbindlichen internen Datenschutzvorschriften} und insbesondere über die unter
     den Buchstaben \hyperref[itm:47-2d]{d}, \hyperref[itm:47-2e]{e} und \hyperref[itm:47-2f]{f} dieses Absatzes
     genannten Aspekte informiert werden;
    \label{itm:47-2g}

    \item die Aufgaben jedes gemäß \hyperref[ch:37]{Artikel 37} benannten Datenschutzbeauftragten oder jeder anderen
     Person oder Einrichtung, die mit der Überwachung der Einhaltung der \hyperref[itm:04-20]{verbindlichen internen Datenschutzvorschriften}
     in der Unternehmensgruppe oder Gruppe von Unternehmen, die eine gemeinsame Wirtschaftstätigkeit ausüben, sowie mit
     der Überwachung der Schulungsmaßnahmen und dem Umgang mit Beschwerden befasst ist;
    \label{itm:47-2h}

    \item die Beschwerdeverfahren;
    \label{itm:47-2i}

    \item die innerhalb der Unternehmensgruppe oder Gruppe von Unternehmen, die eine gemeinsame Wirtschaftstätigkeit
     ausüben, bestehenden Verfahren zur Überprüfung der Einhaltung der \hyperref[itm:04-20]{verbindlichen internen Datenschutzvorschriften}.
     Derartige Verfahren beinhalten Datenschutzüberprüfungen und Verfahren zur Gewährleistung von Abhilfemaßnahmen zum
     Schutz der Rechte der \hyperref[itm:04-1]{betroffenen Person}. Die Ergebnisse derartiger Überprüfungen sollten der in \hyperref
     [itm:47-2h]{Buchstabe h} genannten Person oder Einrichtung sowie dem Verwaltungsrat des herrschenden Unternehmens
     einer Unternehmensgruppe oder der Gruppe von Unternehmen, die eine gemeinsame Wirtschaftstätigkeit ausüben,
     mitgeteilt werden und sollten der zuständigen Aufsichtsbehörde auf Anfrage zur Verfügung gestellt werden;
    \label{itm:47-2j}

    \item die Verfahren für die Meldung und Erfassung von Änderungen der Vorschriften und ihre Meldung an die
     Aufsichtsbehörde;
    \label{itm:47-2k}

    \item die Verfahren für die Zusammenarbeit mit der Aufsichtsbehörde, die die Befolgung der Vorschriften durch
     sämtliche Mitglieder der Unternehmensgruppe oder Gruppe von Unternehmen, die eine gemeinsame Wirtschaftstätigkeit
     ausüben, gewährleisten, insbesondere durch Offenlegung der Ergebnisse von Überprüfungen der unter \hyperref
     [itm:47-2j]{Buchstabe j} genannten Maßnahmen gegenüber der Aufsichtsbehörde;
    \label{itm:47-2l}

    \item die Meldeverfahren zur Unterrichtung der zuständigen Aufsichtsbehörde über jegliche für ein Mitglied der
     Unternehmensgruppe oder Gruppe von Unternehmen, die eine gemeinsame Wirtschaftstätigkeit ausüben, in einem
     Drittland geltenden rechtlichen Bestimmungen, die sich nachteilig auf die Garantien auswirken könnten, die die
     \hyperref[itm:04-20]{verbindlichen internen Datenschutzvorschriften} bieten, und
    \label{itm:47-2m}

    \item geeignete Datenschutzschulungen für Personal mit ständigem oder regelmäßigem Zugang zu personenbezogenen
     Daten.
    \label{itm:47-2n}

  \end{enumerate}

  \item Die Kommission kann das Format und die Verfahren für den Informationsaustausch über \hyperref[itm:04-20]{verbindliche interne
   Datenschutzvorschriften} im Sinne des vorliegenden Artikels zwischen Verantwortlichen, Auftragsverarbeitern und
   Aufsichtsbehörden festlegen. Diese Durchführungsrechtsakte werden gemäß dem Prüfverfahren nach \hyperref[itm:93-2]
   {Artikel 93 Absatz 2} erlassen.
  \label{itm:47-3}

\end{enumerate}

\addsec{Eigene Notizen}

