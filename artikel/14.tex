%!TEX root = ../DSGVO-Bearbeitung.tex
\chapter{Informationspflicht, wenn die personenbezogenen Daten nicht bei der betroffenen Person erhoben wurden}
\label{ch:14}

\addsec{Text der Verordnung}

\begin{enumerate}

  \item Werden \hyperref[itm:04-1]{personenbezogene Daten} nicht bei der \hyperref[itm:04-1]{betroffenen Person} erhoben, so teilt der Verantwortliche der
   \hyperref[itm:04-1]{betroffenen Person} Folgendes mit:
  \label{itm:14-1}

  \begin{enumerate}
  
    \item den Namen und die Kontaktdaten des Verantwortlichen sowie gegebenenfalls seines Vertreters;
    \label{itm:14-1a}

    \item zusätzlich die Kontaktdaten des Datenschutzbeauftragten;
    \label{itm:14-1b}

    \item die Zwecke, für die die \hyperref[itm:04-1]{personenbezogenen Daten} verarbeitet werden sollen, sowie die Rechtsgrundlage für die
     \hyperref[itm:04-2]{Verarbeitung};
    \label{itm:14-1c}

    \item die Kategorien \hyperref[itm:04-1]{personenbezogener Daten}, die verarbeitet werden;
    \label{itm:14-1d}

    \item gegebenenfalls die Empfänger oder Kategorien von Empfängern der \hyperref[itm:04-1]{personenbezogenen Daten};
    \label{itm:14-1e}

    \item gegebenenfalls die Absicht des Verantwortlichen, die \hyperref[itm:04-1]{personenbezogenen Daten} an einen Empfänger in einem
     Drittland oder einer internationalen Organisation zu übermitteln, sowie das Vorhandensein oder das Fehlen eines
     Angemessenheitsbeschlusses der Kommission oder im Falle von Übermittlungen gemäß \hyperref[ch:46]{Artikel 46} oder
     \hyperref[ch:47]{Artikel 47} oder \hyperref[itm:49-1-2]{Artikel 49 Absatz 1 Unterabsatz 2} einen Verweis auf die
      geeigneten oder angemessenen Garantien und die Möglichkeit, eine Kopie von ihnen zu erhalten, oder wo sie
      verfügbar sind.
    \label{itm:14-1f}

  \end{enumerate}

  \item Zusätzlich zu den Informationen gemäß \hyperref[itm:14-1]{Absatz 1} stellt der Verantwortliche der \hyperref[itm:04-1]{betroffenen
   Person} die folgenden Informationen zur Verfügung, die erforderlich sind, um der \hyperref[itm:04-1]{betroffenen Person} gegenüber eine
   faire und transparente \hyperref[itm:04-2]{Verarbeitung} zu gewährleisten:
  \label{itm:14-2}

  \begin{enumerate}
  
    \item die Dauer, für die die \hyperref[itm:04-1]{personenbezogenen Daten} gespeichert werden oder, falls dies nicht möglich ist, die
     Kriterien für die Festlegung dieser Dauer;
    \label{itm:14-2a}

    \item wenn die \hyperref[itm:04-2]{Verarbeitung} auf \hyperref[itm:06-1f]{Artikel 6 Absatz 1 Buchstabe f} beruht, die berechtigten
     Interessen, die von dem Verantwortlichen oder einem Dritten verfolgt werden;
    \label{itm:14-2b}

    \item das Bestehen eines Rechts auf Auskunft seitens des Verantwortlichen über die betreffenden personenbezogenen
     Daten sowie auf Berichtigung oder Löschung oder auf \hyperref[itm:04-3]{Einschränkung der Verarbeitung} und eines Widerspruchsrechts
     gegen die \hyperref[itm:04-2]{Verarbeitung} sowie des Rechts auf Datenübertragbarkeit;
    \label{itm:14-2c}

    \item wenn die \hyperref[itm:04-2]{Verarbeitung} auf \hyperref[itm:06-1a]{Artikel 6 Absatz 1 Buchstabe a} oder \hyperref[itm:09-2a]
     {Artikel 9 Absatz 2 Buchstabe a} beruht, das Bestehen eines Rechts, die Einwilligung jederzeit zu widerrufen, ohne
     dass die Rechtmäßigkeit der aufgrund der Einwilligung bis zum Widerruf erfolgten \hyperref[itm:04-2]{Verarbeitung} berührt wird;
    \label{itm:14-2d}

    \item das Bestehen eines Beschwerderechts bei einer \hyperref[itm:04-21]{Aufsichtsbehörde};
    \label{itm:14-2e}

    \item aus welcher Quelle die \hyperref[itm:04-1]{personenbezogenen Daten} stammen und gegebenenfalls ob sie aus öffentlich zugänglichen
     Quellen stammen;
    \label{itm:14-2f}

    \item das Bestehen einer automatisierten Entscheidungsfindung einschließlich \hyperref[itm:04-4]{Profiling} gemäß \hyperref[ch:22]
     {Artikel 22} Absätze \hyperref[itm:22-1]{1} und \hyperref[itm:22-4]{4} und -- zumindest in diesen Fällen --
     aussagekräftige Informationen über die involvierte Logik sowie die Tragweite und die angestrebten Auswirkungen
     einer derartigen \hyperref[itm:04-2]{Verarbeitung} für die \hyperref[itm:04-1]{betroffene Person}.
    \label{itm:14-2g}

  \end{enumerate}

  \item Der Verantwortliche erteilt die Informationen gemäß den Absätzen \hyperref[itm:14-1]{1} und \hyperref[itm:14-2]
  {2}
  \label{itm:14-3}

  \begin{enumerate}
  
    \item unter Berücksichtigung der spezifischen Umstände der \hyperref[itm:04-2]{Verarbeitung} der \hyperref[itm:04-1]{personenbezogenen Daten} innerhalb einer
     angemessenen Frist nach Erlangung der \hyperref[itm:04-1]{personenbezogenen Daten}, längstens jedoch innerhalb eines Monats,
    \label{itm:14-3a}

    \item falls die \hyperref[itm:04-1]{personenbezogenen Daten} zur Kommunikation mit der \hyperref[itm:04-1]{betroffenen Person} verwendet werden sollen,
     spätestens zum Zeitpunkt der ersten Mitteilung an sie, oder,
    \label{itm:14-3b}

    \item falls die Offenlegung an einen anderen Empfänger beabsichtigt ist, spätestens zum Zeitpunkt der ersten
     Offenlegung.
    \label{itm:14-3c}

  \end{enumerate}

  \item Beabsichtigt der Verantwortliche, die \hyperref[itm:04-1]{personenbezogenen Daten} für einen anderen Zweck weiterzuverarbeiten als
   den, für den die \hyperref[itm:04-1]{personenbezogenen Daten} erlangt wurden, so stellt er der \hyperref[itm:04-1]{betroffenen Person} vor dieser
   Weiterverarbeitung Informationen über diesen anderen Zweck und alle anderen maßgeblichen Informationen gemäß
   \hyperref[itm:14-2]{Absatz 2} zur Verfügung.
  \label{itm:14-4}

  \item Die Absätze \hyperref[itm:14-1]{1} bis \hyperref[itm:14-4]{4} finden keine Anwendung, wenn und soweit
  \label{itm:14-5}

  \begin{enumerate}
  
    \item die \hyperref[itm:04-1]{betroffene Person} bereits über die Informationen verfügt,
    \label{itm:14-5a}

    \item die Erteilung dieser Informationen sich als unmöglich erweist oder einen unverhältnismäßigen Aufwand erfordern
     würde; dies gilt insbesondere für die \hyperref[itm:04-2]{Verarbeitung} für im öffentlichen Interesse liegende Archivzwecke, für
     wissenschaftliche oder historische Forschungszwecke oder für statistische Zwecke vorbehaltlich der in
     \hyperref[itm:89-1]{Artikel 89 Absatz 1} genannten Bedingungen und Garantien oder soweit die in \hyperref[itm:14-1]
      {Absatz 1 des vorliegenden Artikels} genannte Pflicht voraussichtlich die Verwirklichung der Ziele dieser
      \hyperref[itm:04-2]{Verarbeitung} unmöglich macht oder ernsthaft beeinträchtigt In diesen Fällen ergreift der Verantwortliche
      geeignete Maßnahmen zum Schutz der Rechte und Freiheiten sowie der berechtigten Interessen der \hyperref[itm:04-1]{betroffenen
      Person}, einschließlich der Bereitstellung dieser Informationen für die Öffentlichkeit,
    \label{itm:14-5b}

    \item die Erlangung oder Offenlegung durch Rechtsvorschriften der Union oder der Mitgliedstaaten, denen der
     Verantwortliche unterliegt und die geeignete Maßnahmen zum Schutz der berechtigten Interessen der \hyperref[itm:04-1]{betroffenen
     Person} vorsehen, ausdrücklich geregelt ist oder
    \label{itm:14-5c}

    \item die \hyperref[itm:04-1]{personenbezogenen Daten} gemäß dem Unionsrecht oder dem Recht der Mitgliedstaaten dem Berufsgeheimnis,
     einschließlich einer satzungsmäßigen Geheimhaltungspflicht, unterliegen und daher vertraulich behandelt werden
     müssen.
    \label{itm:14-5d}

  \end{enumerate}

\end{enumerate}

\addsec{Eigene Notizen}

