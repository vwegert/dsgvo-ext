%!TEX root = ../DSGVO-Bearbeitung.tex
\chapter{Informationspflicht bei Erhebung von personenbezogenen Daten bei der betroffenen Person}
\label{ch:13}

\addsec{Text der Verordnung}

\begin{enumerate}

  \item Werden \hyperref[itm:04-1]{personenbezogene Daten} bei der \hyperref[itm:04-1]{betroffenen Person} erhoben, so teilt der Verantwortliche der \hyperref[itm:04-1]{betroffenen
   Person} zum Zeitpunkt der Erhebung dieser Daten Folgendes mit:
  \label{itm:13-1}

  \begin{enumerate}
  
    \item den Namen und die Kontaktdaten des Verantwortlichen sowie gegebenenfalls seines Vertreters;
    \label{itm:13-1a}

    \item gegebenenfalls die Kontaktdaten des Datenschutzbeauftragten;
    \label{itm:13-1b}

    \item die Zwecke, für die die \hyperref[itm:04-1]{personenbezogenen Daten} verarbeitet werden sollen, sowie die Rechtsgrundlage für die
     Verarbeitung;
    \label{itm:13-1c}

    \item wenn die Verarbeitung auf \hyperref[itm:06-1f]{Artikel 6 Absatz 1 Buchstabe f} beruht, die berechtigten
     Interessen, die von dem Verantwortlichen oder einem Dritten verfolgt werden;
    \label{itm:13-1d}

    \item gegebenenfalls die Empfänger oder Kategorien von Empfängern der \hyperref[itm:04-1]{personenbezogenen Daten} und
    \label{itm:13-1e}

    \item gegebenenfalls die Absicht des Verantwortlichen, die \hyperref[itm:04-1]{personenbezogenen Daten} an ein Drittland oder eine
     \hyperref[itm:04-29]{internationale Organisation} zu übermitteln, sowie das Vorhandensein oder das Fehlen eines
     Angemessenheitsbeschlusses der Kommission oder im Falle von Übermittlungen gemäß \hyperref[ch:46]{Artikel 46} oder
     \hyperref[ch:47]{Artikel 47} oder \hyperref[itm:49-1-2]{Artikel 49 Absatz 1 Unterabsatz 2} einen Verweis auf die
      geeigneten oder angemessenen Garantien und die Möglichkeit, wie eine Kopie von ihnen zu erhalten ist, oder wo sie
      verfügbar sind.
    \label{itm:13-1f}

  \end{enumerate}

  \item Zusätzlich zu den Informationen gemäß \hyperref[itm:13-1]{Absatz 1} stellt der Verantwortliche der \hyperref[itm:04-1]{betroffenen
   Person} zum Zeitpunkt der Erhebung dieser Daten folgende weitere Informationen zur Verfügung, die notwendig sind, um
   eine faire und transparente Verarbeitung zu gewährleisten:
  \label{itm:13-2}

  \begin{enumerate}
  
    \item die Dauer, für die die \hyperref[itm:04-1]{personenbezogenen Daten} gespeichert werden oder, falls dies nicht möglich ist, die
     Kriterien für die Festlegung dieser Dauer;
    \label{itm:13-2a}

    \item das Bestehen eines Rechts auf Auskunft seitens des Verantwortlichen über die betreffenden personenbezogenen
     Daten sowie auf Berichtigung oder Löschung oder auf \hyperref[itm:04-3]{Einschränkung der Verarbeitung} oder eines Widerspruchsrechts
     gegen die Verarbeitung sowie des Rechts auf Datenübertragbarkeit;
    \label{itm:13-2b}

    \item wenn die Verarbeitung auf \hyperref[itm:06-1a]{Artikel 6 Absatz 1 Buchstabe a} oder \hyperref[itm:09-2a]
     {Artikel 9 Absatz 2 Buchstabe a} beruht, das Bestehen eines Rechts, die Einwilligung jederzeit zu widerrufen, ohne
     dass die Rechtmäßigkeit der aufgrund der Einwilligung bis zum Widerruf erfolgten Verarbeitung berührt wird;
    \label{itm:13-2c}

    \item das Bestehen eines Beschwerderechts bei einer Aufsichtsbehörde;
    \label{itm:13-2d}

    \item ob die Bereitstellung der \hyperref[itm:04-1]{personenbezogenen Daten} gesetzlich oder vertraglich vorgeschrieben oder für einen
     Vertragsabschluss erforderlich ist, ob die \hyperref[itm:04-1]{betroffene Person} verpflichtet ist, die \hyperref[itm:04-1]{personenbezogenen Daten}
     bereitzustellen, und welche mögliche Folgen die Nichtbereitstellung hätte und
    \label{itm:13-2e}

    \item das Bestehen einer automatisierten Entscheidungsfindung einschließlich Profiling gemäß \hyperref[ch:22]
     {Artikel 22} Absätze \hyperref[itm:22-1]{1} und \hyperref[itm:22-4]{4} und -- zumindest in diesen Fällen --
     aussagekräftige Informationen über die involvierte Logik sowie die Tragweite und die angestrebten Auswirkungen
     einer derartigen Verarbeitung für die \hyperref[itm:04-1]{betroffene Person}.
    \label{itm:13-2f}

  \end{enumerate}

  \item Beabsichtigt der Verantwortliche, die \hyperref[itm:04-1]{personenbezogenen Daten} für einen anderen Zweck weiterzuverarbeiten als
   den, für den die \hyperref[itm:04-1]{personenbezogenen Daten} erhoben wurden, so stellt er der \hyperref[itm:04-1]{betroffenen Person} vor dieser
   Weiterverarbeitung Informationen über diesen anderen Zweck und alle anderen maßgeblichen Informationen gemäß
   \hyperref[itm:13-2]{Absatz 2} zur Verfügung.
  \label{itm:13-3}  

  \item Die Absätze \hyperref[itm:13-1]{1}, \hyperref[itm:13-2]{2} und \hyperref[itm:13-3]{3} finden keine Anwendung,
   wenn und soweit die \hyperref[itm:04-1]{betroffene Person} bereits über die Informationen verfügt.
  \label{itm:13-4}  

\end{enumerate}

\addsec{Eigene Notizen}

