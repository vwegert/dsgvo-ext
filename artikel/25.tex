%!TEX root = ../DSGVO-Bearbeitung.tex
\chapter{Datenschutz durch Technikgestaltung und durch datenschutzfreundliche Voreinstellungen}
\label{ch:25}

\addsec{Text der Verordnung}

\begin{enumerate}

  \item Unter Berücksichtigung des Stands der Technik, der Implementierungskosten und der Art, des Umfangs, der Umstände
   und der Zwecke der \hyperref[itm:04-2]{Verarbeitung} sowie der unterschiedlichen Eintrittswahrscheinlichkeit und Schwere der mit der
   \hyperref[itm:04-2]{Verarbeitung} verbundenen Risiken für die Rechte und Freiheiten natürlicher Personen trifft der Verantwortliche
   sowohl zum Zeitpunkt der Festlegung der Mittel für die \hyperref[itm:04-2]{Verarbeitung} als auch zum Zeitpunkt der eigentlichen
   \hyperref[itm:04-2]{Verarbeitung} geeignete technische und organisatorische Maßnahmen -- wie z. B. \hyperref[itm:04-5]{Pseudonymisierung} -- trifft, die dafür
   ausgelegt sind, die Datenschutzgrundsätze wie etwa Datenminimierung wirksam umzusetzen und die notwendigen Garantien
   in die \hyperref[itm:04-2]{Verarbeitung} aufzunehmen, um den Anforderungen dieser Verordnung zu genügen und die Rechte der \hyperref[itm:04-1]{betroffenen
   Personen} zu schützen.
  \label{itm:25-1}

  \item Der Verantwortliche trifft geeignete technische und organisatorische Maßnahmen, die sicherstellen, dass durch
   Voreinstellung grundsätzlich nur \hyperref[itm:04-1]{personenbezogene Daten}, deren \hyperref[itm:04-2]{Verarbeitung} für den jeweiligen bestimmten
   Verarbeitungszweck erforderlich ist, verarbeitet werden. Diese Verpflichtung gilt für die Menge der erhobenen
   \hyperref[itm:04-1]{personenbezogenen Daten}, den Umfang ihrer \hyperref[itm:04-2]{Verarbeitung}, ihre Speicherfrist und ihre Zugänglichkeit. Solche Maßnahmen
   müssen insbesondere sicherstellen, dass \hyperref[itm:04-1]{personenbezogene Daten} durch Voreinstellungen nicht ohne Eingreifen der
   Person einer unbestimmten Zahl von natürlichen Personen zugänglich gemacht werden.
  \label{itm:25-2}

  \item Ein genehmigtes Zertifizierungsverfahren gemäß \hyperref[ch:42]{Artikel 42} kann als Faktor herangezogen werden,
   um die Erfüllung der in den Absätzen \hyperref[itm:25-1]{1} und \hyperref[itm:25-2]{2} des vorliegenden Artikels
   genannten Anforderungen nachzuweisen.
  \label{itm:25-3}

\end{enumerate}

\addsec{Eigene Notizen}

