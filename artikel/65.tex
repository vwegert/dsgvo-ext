%!TEX root = ../DSGVO-Bearbeitung.tex
\chapter{Streitbeilegung durch den Ausschuss}
\label{ch:65}

\addsec{Text der Verordnung}

\begin{enumerate}

  \item Um die ordnungsgemäße und einheitliche Anwendung dieser Verordnung in Einzelfällen sicherzustellen, erlässt der
   Ausschuss in den folgenden Fällen einen verbindlichen Beschluss:
  \label{itm:65-1}

  \begin{enumerate}
  
    \item wenn eine betroffene Aufsichtsbehörde in einem Fall nach \hyperref[itm:60-4]{Artikel 60 Absatz 4} einen
     maßgeblichen und begründeten Einspruch gegen einen Beschlussentwurf der federführenden Behörde eingelegt hat oder
     die federführende Behörde einen solchen Einspruch als nicht maßgeblich oder nicht begründet abgelehnt hat. Der
     verbindliche Beschluss betrifft alle Angelegenheiten, die Gegenstand des maßgeblichen und begründeten Einspruchs
     sind, insbesondere die Frage, ob ein Verstoß gegen diese Verordnung vorliegt;
    \label{itm:65-1a}

    \item wenn es widersprüchliche Standpunkte dazu gibt, welche der betroffenen Aufsichtsbehörden für die
     Hauptniederlassung zuständig ist,
    \label{itm:65-1b}

    \item wenn eine zuständige Aufsichtsbehörde in den in \hyperref[itm:64-1]{Artikel 64 Absatz 1} genannten Fällen
     keine Stellungnahme des Ausschusses einholt oder der Stellungnahme des Ausschusses gemäß \hyperref[ch:64]
     {Artikel 64} nicht folgt. In diesem Fall kann jede betroffene Aufsichtsbehörde oder die Kommission die
     Angelegenheit dem Ausschuss vorlegen.
    \label{itm:65-1c}

  \end{enumerate}

  \item Der in \hyperref[itm:65-1]{Absatz 1} genannte Beschluss wird innerhalb eines Monats nach der Befassung mit der
   Angelegenheit mit einer Mehrheit von zwei Dritteln der Mitglieder des Ausschusses angenommen. Diese Frist kann wegen
   der Komplexität der Angelegenheit um einen weiteren Monat verlängert werden. Der in \hyperref[itm:65-1]{Absatz 1}
   genannte Beschluss wird begründet und an die federführende Aufsichtsbehörde und alle betroffenen Aufsichtsbehörden
   übermittelt und ist für diese verbindlich.
  \label{itm:65-2}

  \item War der Ausschuss nicht in der Lage, innerhalb der in \hyperref[itm:65-2]{Absatz 2} genannten Fristen einen
   Beschluss anzunehmen, so nimmt er seinen Beschluss innerhalb von zwei Wochen nach Ablauf des in \hyperref[itm:65-2]
   {Absatz 2} genannten zweiten Monats mit einfacher Mehrheit der Mitglieder des Ausschusses an. Bei Stimmengleichheit
   zwischen den Mitgliedern des Ausschusses gibt die Stimme des Vorsitzes den Ausschlag.
  \label{itm:65-3}

  \item Die betroffenen Aufsichtsbehörden nehmen vor Ablauf der in den Absätzen \hyperref[itm:65-2]{2} und \hyperref
   [itm:65-3]{3} genannten Fristen keinen Beschluss über die dem Ausschuss vorgelegte Angelegenheit an.
  \label{itm:65-4}

  \item Der Vorsitz des Ausschusses unterrichtet die betroffenen Aufsichtsbehörden unverzüglich über den in \hyperref
   [itm:65-1]{Absatz 1} genannten Beschluss. Er setzt die Kommission hiervon in Kenntnis. Der Beschluss wird
   unverzüglich auf der Website des Ausschusses veröffentlicht, nachdem die Aufsichtsbehörde den in \hyperref[itm:65-6]
   {Absatz 6} genannten endgültigen Beschluss mitgeteilt hat.
  \label{itm:65-5}

  \item Die federführende Aufsichtsbehörde oder gegebenenfalls die Aufsichtsbehörde, bei der die Beschwerde eingereicht
   wurde, trifft den endgültigen Beschluss auf der Grundlage des in \hyperref[itm:65-1]{Absatz 1} des vorliegenden
   Artikels genannten Beschlusses unverzüglich und spätestens einen Monat, nachdem der Europäische Datenschutzausschuss
   seinen Beschluss mitgeteilt hat. Die federführende Aufsichtsbehörde oder gegebenenfalls die Aufsichtsbehörde, bei
   der die Beschwerde eingereicht wurde, setzt den Ausschuss von dem Zeitpunkt, zu dem ihr endgültiger Beschluss dem
   Verantwortlichen oder dem Auftragsverarbeiter bzw. der \hyperref[itm:04-1]{betroffenen Person} mitgeteilt wird, in Kenntnis. Der
   endgültige Beschluss der betroffenen Aufsichtsbehörden wird gemäß \hyperref[ch:60]{Artikel 60} Absätze \hyperref
   [itm:60-7]{7}, \hyperref[itm:60-8]{8} und \hyperref[itm:60-9]{9} angenommen. Im endgültigen Beschluss wird auf den
   in \hyperref[itm:65-1]{Absatz 1} genannten Beschluss verwiesen und festgelegt, dass der in \hyperref[itm:65-1]
   {Absatz 1 des vorliegenden Artikels} genannte Beschluss gemäß \hyperref[itm:65-5]{Absatz 5} auf der Website des
   Ausschusses veröffentlicht wird. Dem endgültigen Beschluss wird der in \hyperref[itm:65-1]{Absatz 1 des vorliegenden
   Artikels} genannte Beschluss beigefügt.
  \label{itm:65-6}

\end{enumerate}

\addsec{Eigene Notizen}

