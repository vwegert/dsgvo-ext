%!TEX root = ../DSGVO-Bearbeitung.tex
\chapter{Verarbeitung besonderer Kategorien personenbezogener Daten}
\label{ch:9}

\addsec{Text der Verordnung}

\begin{enumerate}

  \item Die Verarbeitung \hyperref[itm:04-1]{personenbezogener Daten}, aus denen die rassische und ethnische Herkunft, politische Meinungen,
   religiöse oder weltanschauliche Überzeugungen oder die Gewerkschaftszugehörigkeit hervorgehen, sowie die
   Verarbeitung von genetischen Daten, biometrischen Daten zur eindeutigen Identifizierung einer natürlichen Person,
   Gesundheitsdaten oder Daten zum Sexualleben oder der sexuellen Orientierung einer natürlichen Person ist untersagt.
  \label{itm:09-1}

  \item \hyperref[itm:09-1]{Absatz 1} gilt nicht in folgenden Fällen:
  \label{itm:09-2}

  \begin{enumerate}
  
    \item Die betroffene Person hat in die Verarbeitung der genannten \hyperref[itm:04-1]{personenbezogenen Daten} für einen oder mehrere
     festgelegte Zwecke ausdrücklich eingewilligt, es sei denn, nach Unionsrecht oder dem Recht der Mitgliedstaaten
     kann das Verbot nach \hyperref[itm:09-1]{Absatz 1} durch die Einwilligung der betroffenen Person nicht aufgehoben
     werden,
    \label{itm:09-2a}

    \item die Verarbeitung ist erforderlich, damit der Verantwortliche oder die betroffene Person die ihm bzw. ihr aus
     dem Arbeitsrecht und dem Recht der sozialen Sicherheit und des Sozialschutzes erwachsenden Rechte ausüben und
     seinen bzw. ihren diesbezüglichen Pflichten nachkommen kann, soweit dies nach Unionsrecht oder dem Recht der
     Mitgliedstaaten oder einer Kollektivvereinbarung nach dem Recht der Mitgliedstaaten, das geeignete Garantien für
     die Grundrechte und die Interessen der betroffenen Person vorsieht, zulässig ist,
    \label{itm:09-2b}

    \item die Verarbeitung ist zum Schutz lebenswichtiger Interessen der betroffenen Person oder einer anderen
     natürlichen Person erforderlich und die betroffene Person ist aus körperlichen oder rechtlichen Gründen
     außerstande, ihre Einwilligung zu geben,
    \label{itm:09-2c}

    \item die Verarbeitung erfolgt auf der Grundlage geeigneter Garantien durch eine politisch, weltanschaulich,
     religiös oder gewerkschaftlich ausgerichtete Stiftung, Vereinigung oder sonstige Organisation ohne
     Gewinnerzielungsabsicht im Rahmen ihrer rechtmäßigen Tätigkeiten und unter der Voraussetzung, dass sich die
     Verarbeitung ausschließlich auf die Mitglieder oder ehemalige Mitglieder der Organisation oder auf Personen, die
     im Zusammenhang mit deren Tätigkeitszweck regelmäßige Kontakte mit ihr unterhalten, bezieht und die
     \hyperref[itm:04-1]{personenbezogenen Daten} nicht ohne Einwilligung der betroffenen Personen nach außen offengelegt werden,
    \label{itm:09-2d}

    \item die Verarbeitung bezieht sich auf \hyperref[itm:04-1]{personenbezogene Daten}, die die betroffene Person offensichtlich öffentlich
     gemacht hat,
    \label{itm:09-2e}

    \item die Verarbeitung ist zur Geltendmachung, Ausübung oder Verteidigung von Rechtsansprüchen oder bei Handlungen
     der Gerichte im Rahmen ihrer justiziellen Tätigkeit erforderlich,
    \label{itm:09-2f}

    \item die Verarbeitung ist auf der Grundlage des Unionsrechts oder des Rechts eines Mitgliedstaats, das in
     angemessenem Verhältnis zu dem verfolgten Ziel steht, den Wesensgehalt des Rechts auf Datenschutz wahrt und
     angemessene und spezifische Maßnahmen zur Wahrung der Grundrechte und Interessen der betroffenen Person vorsieht,
     aus Gründen eines erheblichen öffentlichen Interesses erforderlich,
    \label{itm:09-2g}

    \item die Verarbeitung ist für Zwecke der Gesundheitsvorsorge oder der Arbeitsmedizin, für die Beurteilung der
     Arbeitsfähigkeit des Beschäftigten, für die medizinische Diagnostik, die Versorgung oder Behandlung im
     Gesundheits- oder Sozialbereich oder für die Verwaltung von Systemen und Diensten im Gesundheits- oder
     Sozialbereich auf der Grundlage des Unionsrechts oder des Rechts eines Mitgliedstaats oder aufgrund eines Vertrags
     mit einem Angehörigen eines Gesundheitsberufs und vorbehaltlich der in \hyperref[itm:09-3]{Absatz 3} genannten
     Bedingungen und Garantien erforderlich,
    \label{itm:09-2h}

    \item die Verarbeitung ist aus Gründen des öffentlichen Interesses im Bereich der öffentlichen Gesundheit, wie dem
     Schutz vor schwerwiegenden grenzüberschreitenden Gesundheitsgefahren oder zur Gewährleistung hoher Qualitäts- und
     Sicherheitsstandards bei der Gesundheitsversorgung und bei Arzneimitteln und Medizinprodukten, auf der Grundlage
     des Unionsrechts oder des Rechts eines Mitgliedstaats, das angemessene und spezifische Maßnahmen zur Wahrung der
     Rechte und Freiheiten der betroffenen Person, insbesondere des Berufsgeheimnisses, vorsieht, erforderlich, oder
    \label{itm:09-2i}

    \item die Verarbeitung ist auf der Grundlage des Unionsrechts oder des Rechts eines Mitgliedstaats, das in
     angemessenem Verhältnis zu dem verfolgten Ziel steht, den Wesensgehalt des Rechts auf Datenschutz wahrt und
     angemessene und spezifische Maßnahmen zur Wahrung der Grundrechte und Interessen der betroffenen Person vorsieht,
     für im öffentlichen Interesse liegende Archivzwecke, für wissenschaftliche oder historische Forschungszwecke oder
     für statistische Zwecke gemäß \hyperref[itm:89-1]{Artikel 89 Absatz 1} erforderlich.
    \label{itm:09-2j}

  \end{enumerate}

  \item Die in \hyperref[itm:09-1]{Absatz 1} genannten \hyperref[itm:04-1]{personenbezogenen Daten} dürfen zu den in \hyperref[itm:09-2h]
   {Absatz 2 Buchstabe h} genannten Zwecken verarbeitet werden, wenn diese Daten von Fachpersonal oder unter dessen
   Verantwortung verarbeitet werden und dieses Fachpersonal nach dem Unionsrecht oder dem Recht eines Mitgliedstaats
   oder den Vorschriften nationaler zuständiger Stellen dem Berufsgeheimnis unterliegt, oder wenn die Verarbeitung
   durch eine andere Person erfolgt, die ebenfalls nach dem Unionsrecht oder dem Recht eines Mitgliedstaats oder den
   Vorschriften nationaler zuständiger Stellen einer Geheimhaltungspflicht unterliegt.
  \label{itm:09-3}

  \item Die Mitgliedstaaten können zusätzliche Bedingungen, einschließlich Beschränkungen, einführen oder
   aufrechterhalten, soweit die Verarbeitung von genetischen, biometrischen oder Gesundheitsdaten betroffen ist.
  \label{itm:09-4}

\end{enumerate}

\addsec{Eigene Notizen}

