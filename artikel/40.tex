%!TEX root = ../DSGVO-ExtendedVersion.tex
\chapter{Verhaltensregeln}
\label{ch:40}

\crossrefArticleToReason{40}

\begin{enumerate}

  \item Die Mitgliedstaaten, die \hyperref[itm:04-21]{Aufsichtsbehörden}, der Ausschuss und die Kommission fördern die
   Ausarbeitung von Verhaltensregeln, die nach Maßgabe der Besonderheiten der einzelnen Verarbeitungsbereiche und der
   besonderen Bedürfnisse von Kleinstunternehmen sowie kleinen und mittleren \hyperref[itm:04-18]{Unternehmen} zur
   ordnungsgemäßen Anwendung dieser Verordnung beitragen sollen.%
  \label{itm:40-1}

  \item Verbände und andere Vereinigungen, die Kategorien von \hyperref[itm:04-7]{Verantwortlichen} oder \hyperref
   [itm:04-8]{Auftragsverarbeitern} vertreten, können Verhaltensregeln ausarbeiten oder ändern oder erweitern, mit
   denen die Anwendung dieser Verordnung beispielsweise zu dem Folgenden präzisiert wird:%
  \label{itm:40-2}

  \begin{enumerate}
  
    \item faire und transparente \hyperref[itm:04-2]{Verarbeitung};%
    \label{itm:40-2a}

    \item die berechtigten Interessen des \hyperref[itm:04-7]{Verantwortlichen} in bestimmten Zusammenhängen;%
    \label{itm:40-2b}

    \item Erhebung \hyperref[itm:04-1]{personenbezogener Daten};%
    \label{itm:40-2c}

    \item \hyperref[itm:04-5]{Pseudonymisierung} \hyperref[itm:04-1]{personenbezogener Daten};%
    \label{itm:40-2d}

    \item Unterrichtung der Öffentlichkeit und der \hyperref[itm:04-1]{betroffenen Personen};%
    \label{itm:40-2e}

    \item Ausübung der Rechte \hyperref[itm:04-1]{betroffener Personen};%
    \label{itm:40-2f}

    \item Unterrichtung und Schutz von Kindern und Art und Weise, in der die \hyperref[itm:04-11]{Einwilligung} des
     Trägers der elterlichen Verantwortung für das Kind einzuholen ist;%
    \label{itm:40-2g}

    \item die Maßnahmen und Verfahren gemäß den Artikeln \hyperref[ch:24]{24} und \hyperref[ch:25]{25} und die Maßnahmen
     für die Sicherheit der \hyperref[itm:04-2]{Verarbeitung} gemäß \hyperref[ch:32]{Artikel 32};%
    \label{itm:40-2h}

    \item die Meldung von \hyperref[itm:04-12]{Verletzungen des Schutzes personenbezogener Daten} an \hyperref
     [itm:04-21]{Aufsichtsbehörden} und die Benachrichtigung der \hyperref[itm:04-1]{betroffenen Person} von
     solchen \hyperref[itm:04-12]{Verletzungen des Schutzes personenbezogener Daten};%
    \label{itm:40-2i}

    \item die Übermittlung \hyperref[itm:04-1]{personenbezogener Daten} an Drittländer oder an \hyperref[itm:04-26]
     {internationale Organisationen} oder%
    \label{itm:40-2j}

    \item außergerichtliche Verfahren und sonstige Streitbeilegungsverfahren zur Beilegung von Streitigkeiten zwischen
     \hyperref[itm:04-7]{Verantwortlichen} und \hyperref[itm:04-1]{betroffenen Personen} im Zusammenhang mit
      der \hyperref[itm:04-2]{Verarbeitung}, unbeschadet der Rechte \hyperref[itm:04-1]{betroffener Personen} gemäß den
      Artikeln \hyperref[ch:77]{77} und \hyperref[ch:79]{79}.%
    \label{itm:40-2k}

  \end{enumerate}

  \item Zusätzlich zur Einhaltung durch die unter diese Verordnung fallenden \hyperref[itm:04-7]
   {Verantwortlichen} oder \hyperref[itm:04-8]{Auftragsverarbeiter} können Verhaltensregeln, die gemäß \hyperref
   [itm:40-5]{Absatz 5 des vorliegenden Artikels} genehmigt wurden und gemäß \hyperref[itm:40-9]{Absatz 9 des
   vorliegenden Artikels} allgemeine Gültigkeit besitzen, auch von \hyperref[itm:04-7]{Verantwortlichen} oder \hyperref
   [itm:04-8]{Auftragsverarbeitern}, die gemäß \hyperref[ch:3]{Artikel 3} nicht unter diese Verordnung fallen,
   eingehalten werden, um geeignete Garantien im Rahmen der Übermittlung \hyperref[itm:04-1]{personenbezogener Daten}
   an Drittländer oder \hyperref[itm:04-26]{internationale Organisationen} nach Maßgabe des \hyperref[itm:46-2e]
   {Artikels 46 Absatz 2 Buchstabe e} zu bieten. Diese \hyperref[itm:04-7]{Verantwortlichen} oder \hyperref[itm:04-8]
   {Auftragsverarbeiter} gehen mittels vertraglicher oder sonstiger rechtlich bindender Instrumente die verbindliche
   und durchsetzbare Verpflichtung ein, die geeigneten Garantien anzuwenden, auch im Hinblick auf die Rechte
   der \hyperref[itm:04-1]{betroffenen Personen}.%
  \label{itm:40-3}

  \item Die Verhaltensregeln gemäß \hyperref[itm:40-2]{Absatz 2 des vorliegenden Artikels} müssen Verfahren vorsehen,
   die es der in \hyperref[itm:41-2]{Artikel 41 Absatz 1} genannten Stelle ermöglichen, die obligatorische Überwachung
   der Einhaltung ihrer Bestimmungen durch die \hyperref[itm:04-7]{Verantwortlichen} oder die \hyperref[itm:04-8]
   {Auftragsverarbeiter}, die sich zur Anwendung der Verhaltensregeln verpflichten, vorzunehmen, unbeschadet der
   Aufgaben und Befugnisse der \hyperref[itm:04-21]{Aufsichtsbehörde}, die nach Artikel \hyperref[ch:55]
   {55} oder \hyperref[ch:56]{56} zuständig ist.%
  \label{itm:40-4}

  \item Verbände und andere Vereinigungen gemäß \hyperref[itm:40-2]{Absatz 2 des vorliegenden Artikels}, die
   beabsichtigen, Verhaltensregeln auszuarbeiten oder bestehende Verhaltensregeln zu ändern oder zu erweitern, legen
   den Entwurf der Verhaltensregeln bzw. den Entwurf zu deren Änderung oder Erweiterung der \hyperref[itm:04-21]
   {Aufsichtsbehörde} vor, die nach \hyperref[ch:55]{Artikel 55} zuständig ist. Die \hyperref[itm:04-21]
   {Aufsichtsbehörde} gibt eine Stellungnahme darüber ab, ob der Entwurf der Verhaltensregeln bzw. der Entwurf zu deren
   Änderung oder Erweiterung mit dieser Verordnung vereinbar ist und genehmigt diesen Entwurf der Verhaltensregeln bzw.
   den Entwurf zu deren Änderung oder Erweiterung, wenn sie der Auffassung ist, dass er ausreichende geeignete
   Garantien bietet.%
  \label{itm:40-5}

  \item Wird durch die Stellungnahme nach \hyperref[itm:40-5]{Absatz 5} der Entwurf der Verhaltensregeln bzw. der
   Entwurf zu deren Änderung oder Erweiterung genehmigt und beziehen sich die betreffenden Verhaltensregeln nicht auf
   Verarbeitungstätigkeiten in mehreren Mitgliedstaaten, so nimmt die \hyperref[itm:04-21]{Aufsichtsbehörde} die
   Verhaltensregeln in ein Verzeichnis auf und veröffentlicht sie.%
  \label{itm:40-6}

  \item Bezieht sich der Entwurf der Verhaltensregeln auf Verarbeitungstätigkeiten in mehreren Mitgliedstaaten, so legt
   die nach \hyperref[ch:55]{Artikel 55} zuständige \hyperref[itm:04-21]{Aufsichtsbehörde} -- bevor sie den Entwurf der
   Verhaltensregeln bzw. den Entwurf zu deren Änderung oder Erweiterung genehmigt -- ihn nach dem Verfahren
   gemäß \hyperref[ch:63]{Artikel 63} dem Ausschuss vor, der zu der Frage Stellung nimmt, ob der Entwurf der
   Verhaltensregeln bzw. der Entwurf zu deren Änderung oder Erweiterung mit dieser Verordnung vereinbar ist oder -- im
   Fall nach \hyperref[itm:40-3]{Absatz 3 dieses Artikels} -- geeignete Garantien vorsieht.%
  \label{itm:40-7}

  \item Wird durch die Stellungnahme nach \hyperref[itm:40-3]{Absatz 7} bestätigt, dass der Entwurf der Verhaltensregeln
   bzw. der Entwurf zu deren Änderung oder Erweiterung mit dieser Verordnung vereinbar ist oder -- im Fall
   nach \hyperref[itm:40-3]{Absatz 3} -- geeignete Garantien vorsieht, so übermittelt der Ausschuss seine Stellungnahme
   der Kommission.%
  \label{itm:40-8}

  \item Die Kommission kann im Wege von Durchführungsrechtsakten beschließen, dass die ihr gemäß \hyperref[itm:40-8]
   {Absatz 8} übermittelten genehmigten Verhaltensregeln bzw. deren genehmigte Änderung oder Erweiterung allgemeine
   Gültigkeit in der Union besitzen. Diese Durchführungsrechtsakte werden gemäß dem Prüfverfahren nach \hyperref
   [itm:93-2]{Artikel 93 Absatz 2} erlassen.%
  \label{itm:40-9}

  \item Die Kommission trägt dafür Sorge, dass die genehmigten Verhaltensregeln, denen gemäß \hyperref[itm:40-9]
   {Absatz 9} allgemeine Gültigkeit zuerkannt wurde, in geeigneter Weise veröffentlicht werden.%
  \label{itm:40-10}

  \item Der Ausschuss nimmt alle genehmigten Verhaltensregeln bzw. deren genehmigte Änderungen oder Erweiterungen in ein
   Register auf und veröffentlicht sie in geeigneter Weise.%
  \label{itm:40-11}

\end{enumerate}

% \addsec{Ergänzende Hinweise}

