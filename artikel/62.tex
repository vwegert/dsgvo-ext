%!TEX root = ../DSGVO-Bearbeitung.tex
\chapter{Gemeinsame Maßnahmen der Aufsichtsbehörden}
\label{ch:62}

\addsec{Text der Verordnung}

\begin{enumerate}

  \item Die Aufsichtsbehörden führen gegebenenfalls gemeinsame Maßnahmen einschließlich gemeinsamer Untersuchungen und
   gemeinsamer Durchsetzungsmaßnahmen durch, an denen Mitglieder oder Bedienstete der Aufsichtsbehörden anderer
   Mitgliedstaaten teilnehmen.
  \label{itm:62-1}

  \item Verfügt der Verantwortliche oder der Auftragsverarbeiter über Niederlassungen in mehreren Mitgliedstaaten oder
   werden die Verarbeitungsvorgänge voraussichtlich auf eine bedeutende Zahl \hyperref[itm:04-1]{betroffener Personen} in mehr als einem
   Mitgliedstaat erhebliche Auswirkungen haben, ist die Aufsichtsbehörde jedes dieser Mitgliedstaaten berechtigt, an
   den gemeinsamen Maßnahmen teilzunehmen. Die gemäß \hyperref[itm:56-1]{Artikel 56 Absatz 1} oder \hyperref[itm:56-4]
   {Absatz 4} zuständige Aufsichtsbehörde lädt die Aufsichtsbehörde jedes dieser Mitgliedstaaten zur Teilnahme an den
   gemeinsamen Maßnahmen ein und antwortet unverzüglich auf das Ersuchen einer Aufsichtsbehörde um Teilnahme.
  \label{itm:62-2}

  \item Eine Aufsichtsbehörde kann gemäß dem Recht des Mitgliedstaats und mit Genehmigung der unterstützenden
   Aufsichtsbehörde den an den gemeinsamen Maßnahmen beteiligten Mitgliedern oder Bediensteten der unterstützenden
   Aufsichtsbehörde Befugnisse einschließlich Untersuchungsbefugnisse übertragen oder, soweit dies nach dem Recht des
   Mitgliedstaats der einladenden Aufsichtsbehörde zulässig ist, den Mitgliedern oder Bediensteten der unterstützenden
   Aufsichtsbehörde gestatten, ihre Untersuchungsbefugnisse nach dem Recht des Mitgliedstaats der unterstützenden
   Aufsichtsbehörde auszuüben. Diese Untersuchungsbefugnisse können nur unter der Leitung und in Gegenwart der
   Mitglieder oder Bediensteten der einladenden Aufsichtsbehörde ausgeübt werden. Die Mitglieder oder Bediensteten der
   unterstützenden Aufsichtsbehörde unterliegen dem Recht des Mitgliedstaats der einladenden Aufsichtsbehörde.
  \label{itm:62-3}

  \item Sind gemäß \hyperref[itm:62-1]{Absatz 1} Bedienstete einer unterstützenden Aufsichtsbehörde in einem anderen
   Mitgliedstaat im Einsatz, so übernimmt der Mitgliedstaat der einladenden Aufsichtsbehörde nach Maßgabe des Rechts
   des Mitgliedstaats, in dessen Hoheitsgebiet der Einsatz erfolgt, die Verantwortung für ihr Handeln, einschließlich
   der Haftung für alle von ihnen bei ihrem Einsatz verursachten Schäden.
  \label{itm:62-4}

  \item Der Mitgliedstaat, in dessen Hoheitsgebiet der Schaden verursacht wurde, ersetzt diesen Schaden so, wie er ihn
   ersetzen müsste, wenn seine eigenen Bediensteten ihn verursacht hätten. Der Mitgliedstaat der unterstützenden
   Aufsichtsbehörde, deren Bedienstete im Hoheitsgebiet eines anderen Mitgliedstaats einer Person Schaden zugefügt
   haben, erstattet diesem anderen Mitgliedstaat den Gesamtbetrag des Schadenersatzes, den dieser an die Berechtigten
   geleistet hat.
  \label{itm:62-5}

  \item Unbeschadet der Ausübung seiner Rechte gegenüber Dritten und mit Ausnahme des \hyperref[itm:62-5]{Absatzes 5}
   verzichtet jeder Mitgliedstaat in dem Fall des \hyperref[itm:62-1]{Absatzes 1} darauf, den in \hyperref[itm:62-4]
   {Absatz 4} genannten Betrag des erlittenen Schadens anderen Mitgliedstaaten gegenüber geltend zu machen.
  \label{itm:62-6}

  \item Ist eine gemeinsame Maßnahme geplant und kommt eine Aufsichtsbehörde binnen eines Monats nicht der Verpflichtung
   nach \hyperref[itm:62-2]{Absatz 2} Satz 2 des vorliegenden Artikels nach, so können die anderen Aufsichtsbehörden
   eine einstweilige Maßnahme im Hoheitsgebiet ihres Mitgliedstaats gemäß \hyperref[ch:55]{Artikel 55} ergreifen. In
   diesem Fall wird von einem dringenden Handlungsbedarf gemäß \hyperref[itm:66-1]{Artikel 66 Absatz 1} ausgegangen,
   der eine im Dringlichkeitsverfahren angenommene Stellungnahme oder einen im Dringlichkeitsverfahren angenommenen
   verbindlichen Beschluss des Ausschusses gemäß \hyperref[itm:66-2]{Artikel 66 Absatz 2} erforderlich macht.
  \label{itm:62-7}

\end{enumerate}   

\addsec{Eigene Notizen}

