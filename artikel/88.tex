%!TEX root = ../DSGVO-ExtendedVersion.tex
\chapter{Datenverarbeitung im Beschäftigungskontext}
\label{ch:88}

\addsec{Text der Verordnung}

\begin{enumerate}

  \item Die Mitgliedstaaten können durch Rechtsvorschriften oder durch Kollektivvereinbarungen spezifischere
   Vorschriften zur Gewährleistung des Schutzes der Rechte und Freiheiten hinsichtlich der \hyperref[itm:04-2]
   {Verarbeitung} \hyperref[itm:04-1]{personenbezogener Beschäftigtendaten} im Beschäftigungskontext, insbesondere für
   Zwecke der Einstellung, der Erfüllung des Arbeitsvertrags einschließlich der Erfüllung von durch Rechtsvorschriften
   oder durch Kollektivvereinbarungen festgelegten Pflichten, des Managements, der Planung und der Organisation der
   Arbeit, der Gleichheit und Diversität am Arbeitsplatz, der Gesundheit und Sicherheit am Arbeitsplatz, des Schutzes
   des Eigentums der Arbeitgeber oder der Kunden sowie für Zwecke der Inanspruchnahme der mit der Beschäftigung
   zusammenhängenden individuellen oder kollektiven Rechte und Leistungen und für Zwecke der Beendigung des
   Beschäftigungsverhältnisses vorsehen.%
  \label{itm:88-1}

  \item Diese Vorschriften umfassen geeignete und besondere Maßnahmen zur Wahrung der menschlichen Würde, der
   berechtigten Interessen und der Grundrechte der \hyperref[itm:04-1]{betroffenen Person}, insbesondere im Hinblick
   auf die \hyperref[itm:05-1a]{Transparenz} der \hyperref[itm:04-2]{Verarbeitung}, die Übermittlung \hyperref
   [itm:04-1]{personenbezogener Daten} innerhalb einer \hyperref[itm:04-19]{Unternehmensgruppe} oder einer Gruppe von
   \hyperref[itm:04-18]{Unternehmen}, die eine gemeinsame Wirtschaftstätigkeit ausüben, und die Überwachungssysteme am
    Arbeitsplatz.%
  \label{itm:88-2}

  \item Jeder Mitgliedstaat teilt der Kommission bis zum 25. Mai 2018 die Rechtsvorschriften, die er aufgrund von
   \hyperref[itm:88-1]{Absatz 1} erlässt, sowie unverzüglich alle späteren Änderungen dieser Vorschriften mit.%
  \label{itm:88-3}

\end{enumerate}

\addsec{Eigene Notizen}

