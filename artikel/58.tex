%!TEX root = ../DSGVO-Bearbeitung.tex
\chapter{Befugnisse}
\label{ch:58}

\addsec{Text der Verordnung}

\begin{enumerate}

  \item Jede \hyperref[itm:04-21]{Aufsichtsbehörde} verfügt über sämtliche folgenden Untersuchungsbefugnisse, die es ihr
   gestatten,
  \label{itm:58-1}

  \begin{enumerate}
  
    \item den \hyperref[itm:04-7]{Verantwortlichen}, den \hyperref[itm:04-8]{Auftragsverarbeiter} und gegebenenfalls
     den \hyperref[itm:04-17]{Vertreter} des \hyperref[itm:04-7]{Verantwortlichen} oder des
     \hyperref[itm:04-8]{Auftragsverarbeiters} anzuweisen, alle Informationen bereitzustellen, die für die Erfüllung
      ihrer Aufgaben erforderlich sind,
    \label{itm:58-1a}

    \item Untersuchungen in Form von Datenschutzüberprüfungen durchzuführen,
    \label{itm:581b}

    \item eine Überprüfung der nach \hyperref[itm:42-7]{Artikel 42 Absatz 7} erteilten Zertifizierungen durchzuführen,
    \label{itm:58-1c}

    \item den \hyperref[itm:04-7]{Verantwortlichen} oder den \hyperref[itm:04-8]{Auftragsverarbeiter} auf einen
     vermeintlichen Verstoß gegen diese Verordnung hinzuweisen,
    \label{itm:58-1d}

    \item von dem \hyperref[itm:04-7]{Verantwortlichen} und dem \hyperref[itm:04-8]{Auftragsverarbeiter} Zugang zu
     allen \hyperref[itm:04-1]{personenbezogenen Daten} und Informationen, die zur Erfüllung ihrer Aufgaben notwendig
     sind, zu erhalten,
    \label{itm:58-1e}

    \item gemäß dem Verfahrensrecht der Union oder dem Verfahrensrecht des Mitgliedstaats Zugang zu den Räumlichkeiten,
     einschließlich aller Datenverarbeitungsanlagen und -geräte, des \hyperref[itm:04-7]{Verantwortlichen} und
     des \hyperref[itm:04-8]{Auftragsverarbeiters} zu erhalten.
    \label{itm:58-1f}

  \end{enumerate}

  \item Jede \hyperref[itm:04-21]{Aufsichtsbehörde} verfügt über sämtliche folgenden Abhilfebefugnisse, die es ihr
   gestatten,
  \label{itm:58-2}

  \begin{enumerate}
  
    \item einen \hyperref[itm:04-7]{Verantwortlichen} oder einen \hyperref[itm:04-8]{Auftragsverarbeiter} zu warnen,
     dass beabsichtigte Verarbeitungsvorgänge voraussichtlich gegen diese Verordnung verstoßen,
    \label{itm:58-2a}

    \item einen \hyperref[itm:04-7]{Verantwortlichen} oder einen \hyperref[itm:04-8]{Auftragsverarbeiter} zu verwarnen,
     wenn er mit Verarbeitungsvorgängen gegen diese Verordnung verstoßen hat,
    \label{itm:58-2b}

    \item den \hyperref[itm:04-7]{Verantwortlichen} oder den \hyperref[itm:04-8]{Auftragsverarbeiter} anzuweisen, den
     Anträgen der \hyperref[itm:04-1]{betroffenen Person} auf Ausübung der ihr nach dieser Verordnung zustehenden
     Rechte zu entsprechen,
    \label{itm:58-2c}

    \item den \hyperref[itm:04-7]{Verantwortlichen} oder den \hyperref[itm:04-8]{Auftragsverarbeiter} anzuweisen,
     Verarbeitungsvorgänge gegebenenfalls auf bestimmte Weise und innerhalb eines bestimmten Zeitraums in Einklang mit
     dieser Verordnung zu bringen,
    \label{itm:58-2d}

    \item den \hyperref[itm:04-7]{Verantwortlichen} anzuweisen, die von einer \hyperref[itm:04-12]{Verletzung des
     Schutzes personenbezogener Daten} \hyperref[itm:04-1]{betroffene Person} entsprechend zu benachrichtigen,
    \label{itm:58-2e}

    \item eine vorübergehende oder endgültige Beschränkung der \hyperref[itm:04-2]{Verarbeitung}, einschließlich eines
     Verbots, zu verhängen,
    \label{itm:58-2f}

    \item die Berichtigung oder Löschung von \hyperref[itm:04-1]{personenbezogenen Daten} oder die \hyperref[itm:04-3]
     {Einschränkung der Verarbeitung} gemäß den Artikeln \hyperref[ch:16]{16}, \hyperref[ch:17]{17} und \hyperref
     [ch:17]{18} und die Unterrichtung der \hyperref[itm:04-9]{Empfänger}, an die diese \hyperref[itm:04-1]
     {personenbezogenen Daten} gemäß \hyperref[itm:17-2]{Artikel 17 Absatz 2} und \hyperref[ch:19]{Artikel 19}
     offengelegt wurden, über solche Maßnahmen anzuordnen,
    \label{itm:58-2g}

    \item eine Zertifizierung zu widerrufen oder die Zertifizierungsstelle anzuweisen, eine gemäß den Artikel \hyperref
     [ch:42]{42} und \hyperref[ch:43]{43} erteilte Zertifizierung zu widerrufen, oder die Zertifizierungsstelle
     anzuweisen, keine Zertifizierung zu erteilen, wenn die Voraussetzungen für die Zertifizierung nicht oder nicht
     mehr erfüllt werden,
    \label{itm:58-2h}

    \item eine Geldbuße gemäß \hyperref[ch:83]{Artikel 83} zu verhängen, zusätzlich zu oder anstelle von in diesem
     Absatz genannten Maßnahmen, je nach den Umständen des Einzelfalls,
    \label{itm:58-2i}

    \item die Aussetzung der Übermittlung von Daten an einen \hyperref[itm:04-9]{Empfänger} in einem Drittland oder an
     eine \hyperref[itm:04-29]{internationale Organisation} anzuordnen.
    \label{itm:58-2j}

  \end{enumerate}

  \item Jede \hyperref[itm:04-21]{Aufsichtsbehörde} verfügt über sämtliche folgenden Genehmigungsbefugnisse und
   beratenden Befugnisse, die es ihr gestatten,
  \label{itm:58-3}

  \begin{enumerate}
  
    \item gemäß dem Verfahren der vorherigen Konsultation nach \hyperref[ch:36]{Artikel 36} den \hyperref[itm:04-7]
     {Verantwortlichen} zu beraten,
    \label{itm:58-3a}

    \item zu allen Fragen, die im Zusammenhang mit dem Schutz \hyperref[itm:04-1]{personenbezogener Daten} stehen, von
     sich aus oder auf Anfrage Stellungnahmen an das nationale Parlament, die Regierung des Mitgliedstaats oder im
     Einklang mit dem Recht des Mitgliedstaats an sonstige Einrichtungen und Stellen sowie an die Öffentlichkeit zu
     richten,
    \label{itm:58-3b}

    \item die \hyperref[itm:04-2]{Verarbeitung} gemäß \hyperref[itm:36-5]{Artikel 36 Absatz 5} zu genehmigen, falls im
     Recht des Mitgliedstaats eine derartige vorherige Genehmigung verlangt wird,
    \label{itm:58-3c}

    \item eine Stellungnahme abzugeben und Entwürfe von Verhaltensregeln gemäß \hyperref[itm:40-5]{Artikel 40 Absatz 5}
     zu billigen,
    \label{itm:58-3d}

    \item Zertifizierungsstellen gemäß \hyperref[ch:43]{Artikel 43} zu akkreditieren,
    \label{itm:58-3e}

    \item im Einklang mit \hyperref[itm:42-5]{Artikel 42 Absatz 5} Zertifizierungen zu erteilen und Kriterien für die
     Zertifizierung zu billigen,
    \label{itm:58-3f}

    \item Standarddatenschutzklauseln nach \hyperref[itm:28-8]{Artikel 28 Absatz 8} und \hyperref[itm:46-2d]{Artikel 46
     Absatz 2 Buchstabe d} festzulegen,
    \label{itm:58-3g}

    \item Vertragsklauseln gemäß \hyperref[itm:46-3a]{Artikel 46 Absatz 3 Buchstabe a} zu genehmigen,
    \label{itm:58-3h}

    \item Verwaltungsvereinbarungen gemäß \hyperref[itm:46-3b]{Artikel 46 Absatz 3 Buchstabe b} zu genehmigen
    \label{itm:58-3i}

    \item verbindliche interne Vorschriften gemäß \hyperref[ch:47]{Artikel 47} zu genehmigen.
    \label{itm:58-3j}

  \end{enumerate}

  \item Die Ausübung der der \hyperref[itm:04-21]{Aufsichtsbehörde} gemäß diesem Artikel übertragenen Befugnisse erfolgt
   vorbehaltlich geeigneter Garantien einschließlich wirksamer gerichtlicher Rechtsbehelfe und ordnungsgemäßer
   Verfahren gemäß dem Unionsrecht und dem Recht des Mitgliedstaats im Einklang mit der Charta.
  \label{itm:58-4}

  \item Jeder Mitgliedstaat sieht durch Rechtsvorschriften vor, dass seine \hyperref[itm:04-21]{Aufsichtsbehörde} befugt
   ist, Verstöße gegen diese Verordnung den Justizbehörden zur Kenntnis zu bringen und gegebenenfalls die Einleitung
   eines gerichtlichen Verfahrens zu betreiben oder sich sonst daran zu beteiligen, um die Bestimmungen dieser
   Verordnung durchzusetzen.
  \label{itm:58-5}

  \item Jeder Mitgliedstaat kann durch Rechtsvorschriften vorsehen, dass seine \hyperref[itm:04-21]
   {Aufsichtsbehörde} neben den in den Absätzen \hyperref[itm:58-1]{1}, \hyperref[itm:58-2]{2} und \hyperref[itm:58-3]
   {3} aufgeführten Befugnissen über zusätzliche Befugnisse verfügt. Die Ausübung dieser Befugnisse darf nicht die
   effektive Durchführung des \hyperref[part:7]{Kapitels VII} beeinträchtigen.
  \label{itm:58-6}

\end{enumerate}

\addsec{Eigene Notizen}

