%!TEX root = ../DSGVO-Bearbeitung.tex
\chapter{Dringlichkeitsverfahren}
\label{ch:66}

\addsec{Text der Verordnung}

\begin{enumerate}

  \item Unter außergewöhnlichen Umständen kann eine \hyperref[itm:04-22]{betroffene Aufsichtsbehörde} abweichend vom Kohärenzverfahren nach
   Artikel \hyperref[ch:63]{63}, \hyperref[ch:64]{64} und \hyperref[ch:65]{65} oder dem Verfahren nach \hyperref[ch:60]
   {Artikel 60} sofort einstweilige Maßnahmen mit festgelegter Geltungsdauer von höchstens drei Monaten treffen, die in
   ihrem Hoheitsgebiet rechtliche Wirkung entfalten sollen, wenn sie zu der Auffassung gelangt, dass dringender
   Handlungsbedarf besteht, um Rechte und Freiheiten von \hyperref[itm:04-1]{betroffenen Personen} zu schützen. Die Aufsichtsbehörde setzt
   die anderen \hyperref[itm:04-22]{betroffenen Aufsichtsbehörden}, den Ausschuss und die Kommission unverzüglich von diesen Maßnahmen und
   den Gründen für deren Erlass in Kenntnis.
  \label{itm:66-1}

  \item Hat eine Aufsichtsbehörde eine Maßnahme nach \hyperref[itm:66-1]{Absatz 1} ergriffen und ist sie der Auffassung,
   dass dringend endgültige Maßnahmen erlassen werden müssen, kann sie unter Angabe von Gründen im
   Dringlichkeitsverfahren um eine Stellungnahme oder einen verbindlichen Beschluss des Ausschusses ersuchen.
  \label{itm:66-2}

  \item Jede Aufsichtsbehörde kann unter Angabe von Gründen, auch für den dringenden Handlungsbedarf, im
   Dringlichkeitsverfahren um eine Stellungnahme oder gegebenenfalls einen verbindlichen Beschluss des Ausschusses
   ersuchen, wenn eine zuständige Aufsichtsbehörde trotz dringenden Handlungsbedarfs keine geeignete Maßnahme getroffen
   hat, um die Rechte und Freiheiten von \hyperref[itm:04-1]{betroffenen Personen} zu schützen.
  \label{itm:66-3}

  \item Abweichend von \hyperref[itm:64-3]{Artikel 64 Absatz 3} und \hyperref[itm:65-2]{Artikel 65 Absatz 2} wird eine
   Stellungnahme oder ein verbindlicher Beschluss im Dringlichkeitsverfahren nach den Absätzen \hyperref[itm:66-2]
   {2} und \hyperref[itm:66-3]{3} binnen zwei Wochen mit einfacher Mehrheit der Mitglieder des Ausschusses angenommen.
  \label{itm:66-4}

\end{enumerate}

\addsec{Eigene Notizen}

