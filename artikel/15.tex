%!TEX root = ../DSGVO-Bearbeitung.tex
\chapter{Auskunftsrecht der betroffenen Person}
\label{ch:15}

\addsec{Text der Verordnung}

\begin{enumerate}

  \item Die \hyperref[itm:04-1]{betroffene Person} hat das Recht, von dem Verantwortlichen eine Bestätigung darüber zu verlangen, ob sie
   betreffende \hyperref[itm:04-1]{personenbezogene Daten} verarbeitet werden; ist dies der Fall, so hat sie ein Recht auf Auskunft über
   diese \hyperref[itm:04-1]{personenbezogenen Daten} und auf folgende Informationen:
  \label{itm:15-1}

  \begin{enumerate}
  
    \item die Verarbeitungszwecke;
    \label{itm:15-1a}

    \item die Kategorien \hyperref[itm:04-1]{personenbezogener Daten}, die verarbeitet werden;
    \label{itm:15-1b}

    \item die Empfänger oder Kategorien von Empfängern, gegenüber denen die \hyperref[itm:04-1]{personenbezogenen Daten} offengelegt worden
     sind oder noch offengelegt werden, insbesondere bei Empfängern in Drittländern oder bei internationalen
     Organisationen;
    \label{itm:15-1c}

    \item falls möglich die geplante Dauer, für die die \hyperref[itm:04-1]{personenbezogenen Daten} gespeichert werden, oder, falls dies
     nicht möglich ist, die Kriterien für die Festlegung dieser Dauer;
    \label{itm:15-1d}

    \item das Bestehen eines Rechts auf Berichtigung oder Löschung der sie betreffenden \hyperref[itm:04-1]{personenbezogenen Daten} oder auf
     \hyperref[itm:04-3]{Einschränkung der Verarbeitung} durch den Verantwortlichen oder eines Widerspruchsrechts gegen diese \hyperref[itm:04-2]{Verarbeitung};
    \label{itm:15-1e}

    \item das Bestehen eines Beschwerderechts bei einer Aufsichtsbehörde;
    \label{itm:15-1f}

    \item wenn die \hyperref[itm:04-1]{personenbezogenen Daten} nicht bei der \hyperref[itm:04-1]{betroffenen Person} erhoben werden, alle verfügbaren
     Informationen über die Herkunft der Daten;
    \label{itm:15-1g}

    \item das Bestehen einer automatisierten Entscheidungsfindung einschließlich Profiling gemäß \hyperref[ch:22]
     {Artikel 22} Absätze \hyperref[itm:22-1]{1} und \hyperref[itm:22-4]{4} und -- zumindest in diesen Fällen --
     aussagekräftige Informationen über die involvierte Logik sowie die Tragweite und die angestrebten Auswirkungen
     einer derartigen \hyperref[itm:04-2]{Verarbeitung} für die \hyperref[itm:04-1]{betroffene Person}.
    \label{itm:15-1h}

  \end{enumerate}

  \item Werden \hyperref[itm:04-1]{personenbezogene Daten} an ein Drittland oder an eine \hyperref[itm:04-29]{internationale Organisation} übermittelt, so hat die
   \hyperref[itm:04-1]{betroffene Person} das Recht, über die geeigneten Garantien gemäß \hyperref[ch:46]{Artikel 46} im Zusammenhang mit der
   Übermittlung unterrichtet zu werden.
  \label{itm:15-2}

  \item Der Verantwortliche stellt eine Kopie der \hyperref[itm:04-1]{personenbezogenen Daten}, die Gegenstand der \hyperref[itm:04-2]{Verarbeitung} sind, zur
   Verfügung. Für alle weiteren Kopien, die die \hyperref[itm:04-1]{betroffene Person} beantragt, kann der Verantwortliche ein angemessenes
   Entgelt auf der Grundlage der Verwaltungskosten verlangen. Stellt die \hyperref[itm:04-1]{betroffene Person} den Antrag elektronisch, so
   sind die Informationen in einem gängigen elektronischen Format zur Verfügung zu stellen, sofern sie nichts anderes
   angibt.
  \label{itm:15-3}

  \item Das Recht auf Erhalt einer Kopie gemäß \sout{Absatz 1b} \hyperref[itm:15-3]{Absatz 3} darf die Rechte und
   Freiheiten anderer Personen nicht beeinträchtigen.
  \label{itm:15-4}

\end{enumerate}

\addsec{Eigene Notizen}

\begin{itemize}

  \item Der deutsche Originaltext verweist in \hyperref[itm:15-4]{Absatz 4} auf Absatz 1b, was inhaltlich nicht sinnvoll
   ist. Der englische und französische Text verweisen beide auf Absatz 3, was im Text oben korrigiert wurde.

\end{itemize}

