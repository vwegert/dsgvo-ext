%!TEX root = ../DSGVO-Bearbeitung.tex
\chapter{Grundsätze für die Verarbeitung personenbezogener Daten}
\label{ch:5}

\addsec{Text der Verordnung}

\begin{enumerate}

  \item \hyperref[itm:04-1]{Personenbezogene Daten} müssen
  \label{itm:05-1}

  \begin{enumerate}
  
    \item auf rechtmäßige Weise, nach Treu und Glauben und in einer für die \hyperref[itm:04-1]{betroffene Person} nachvollziehbaren Weise
     verarbeitet werden („Rechtmäßigkeit, \hyperref[itm:04-2]{Verarbeitung} nach Treu und Glauben, Transparenz“);
    \label{itm:05-1a}

    \item für festgelegte, eindeutige und legitime Zwecke erhoben werden und dürfen nicht in einer mit diesen Zwecken
     nicht zu vereinbarenden Weise weiterverarbeitet werden; eine Weiterverarbeitung für im öffentlichen Interesse
     liegende Archivzwecke, für wissenschaftliche oder historische Forschungszwecke oder für statistische Zwecke gilt
     gemäß \hyperref[itm:89-1]{Artikel 89 Absatz 1} nicht als unvereinbar mit den ursprünglichen
     Zwecken („Zweckbindung“);
    \label{itm:05-1b}

    \item dem Zweck angemessen und erheblich sowie auf das für die Zwecke der \hyperref[itm:04-2]{Verarbeitung} notwendige Maß beschränkt
     sein („Datenminimierung“);
    \label{itm:05-1c}

    \item sachlich richtig und erforderlichenfalls auf dem neuesten Stand sein; es sind alle angemessenen Maßnahmen zu
     treffen, damit \hyperref[itm:04-1]{personenbezogene Daten}, die im Hinblick auf die Zwecke ihrer \hyperref[itm:04-2]{Verarbeitung} unrichtig sind,
     unverzüglich gelöscht oder berichtigt werden („Richtigkeit“);
    \label{itm:05-1d}

    \item in einer Form gespeichert werden, die die Identifizierung der \hyperref[itm:04-1]{betroffenen Personen} nur so lange ermöglicht,
     wie es für die Zwecke, für die sie verarbeitet werden, erforderlich ist; \hyperref[itm:04-1]{personenbezogene Daten} dürfen länger
     gespeichert werden, soweit die \hyperref[itm:04-1]{personenbezogenen Daten} vorbehaltlich der Durchführung geeigneter technischer und
     organisatorischer Maßnahmen, die von dieser Verordnung zum Schutz der Rechte und Freiheiten der \hyperref[itm:04-1]{betroffenen Person}
     gefordert werden, ausschließlich für im öffentlichen Interesse liegende Archivzwecke oder für wissenschaftliche
     und historische Forschungszwecke oder für statistische Zwecke gemäß \hyperref[itm:89-1]{Artikel 89 Absatz 1}
     verarbeitet werden(„Speicherbegrenzung“);
    \label{itm:05-1e}

    \item in einer Weise verarbeitet werden, die eine angemessene Sicherheit der \hyperref[itm:04-1]{personenbezogenen Daten} gewährleistet,
     einschließlich Schutz vor unbefugter oder unrechtmäßiger \hyperref[itm:04-2]{Verarbeitung} und vor unbeabsichtigtem Verlust,
     unbeabsichtigter Zerstörung oder unbeabsichtigter Schädigung durch geeignete technische und organisatorische
     Maßnahmen („Integrität und Vertraulichkeit“);
    \label{itm:05-1f}

  \end{enumerate}

  \item Der \hyperref[itm:04-7]{Verantwortliche} ist für die Einhaltung des \hyperref[itm:05-1]{Absatzes 1} verantwortlich und muss dessen
   Einhaltung nachweisen können („Rechenschaftspflicht“).
  \label{itm:05-2}

\end{enumerate}

\addsec{Eigene Notizen}

