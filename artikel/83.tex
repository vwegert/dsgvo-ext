%!TEX root = ../DSGVO-Bearbeitung.tex
\chapter{Allgemeine Bedingungen für die Verhängung von Geldbußen}
\label{ch:83}

\addsec{Text der Verordnung}

\begin{enumerate}

  \item Jede Aufsichtsbehörde stellt sicher, dass die Verhängung von Geldbußen gemäß diesem Artikel für Verstöße gegen
   diese Verordnung gemäß den Absätzen \hyperref[itm:83-5]{5} und \hyperref[itm:83-6]{6} in jedem Einzelfall wirksam,
   verhältnismäßig und abschreckend ist.
  \label{itm:83-1}

  \item Geldbußen werden je nach den Umständen des Einzelfalls zusätzlich zu oder anstelle von Maßnahmen nach \hyperref
   [itm:58-2]{Artikel 58 Absatz 2} Buchstaben \hyperref[itm:58-2a]{a} bis \hyperref[itm:58-2h]{h} und \hyperref
   [itm:58-2i]{i} verhängt. Bei der Entscheidung über die Verhängung einer Geldbuße und über deren Betrag wird in jedem
   Einzelfall Folgendes gebührend berücksichtigt:
  \label{itm:83-2}

  \begin{enumerate}
  
    \item Art, Schwere und Dauer des Verstoßes unter Berücksichtigung der Art, des Umfangs oder des Zwecks der
     betreffenden Verarbeitung sowie der Zahl der von der Verarbeitung betroffenen Personen und des Ausmaßes des von
     ihnen erlittenen Schadens;
    \label{itm:83-2a}

    \item Vorsätzlichkeit oder Fahrlässigkeit des Verstoßes;
    \label{itm:83-2b}

    \item jegliche von dem Verantwortlichen oder dem Auftragsverarbeiter getroffenen Maßnahmen zur Minderung des den
     betroffenen Personen entstandenen Schadens;
    \label{itm:83-2c}

    \item Grad der Verantwortung des Verantwortlichen oder des Auftragsverarbeiters unter Berücksichtigung der von ihnen
     gemäß den Artikeln \hyperref[ch:25]{25} und \hyperref[ch:32]{32} getroffenen technischen und organisatorischen
     Maßnahmen;
    \label{itm:83-2d}

    \item etwaige einschlägige frühere Verstöße des Verantwortlichen oder des Auftragsverarbeiters;
    \label{itm:83-2e}

    \item Umfang der Zusammenarbeit mit der Aufsichtsbehörde, um dem Verstoß abzuhelfen und seine möglichen nachteiligen
     Auswirkungen zu mindern;
    \label{itm:83-2f}

    \item Kategorien \hyperref[itm:04-1]{personenbezogener Daten}, die von dem Verstoß betroffen sind;
    \label{itm:83-2g}

    \item Art und Weise, wie der Verstoß der Aufsichtsbehörde bekannt wurde, insbesondere ob und gegebenenfalls in
     welchem Umfang der Verantwortliche oder der Auftragsverarbeiter den Verstoß mitgeteilt hat;
    \label{itm:83-2h}

    \item Einhaltung der nach \hyperref[itm:58-2]{Artikel 58 Absatz 2} früher gegen den für den betreffenden
     Verantwortlichen oder Auftragsverarbeiter in Bezug auf denselben Gegenstand angeordneten Maßnahmen, wenn solche
     Maßnahmen angeordnet wurden;
    \label{itm:83-2i}

    \item Einhaltung von genehmigten Verhaltensregeln nach \hyperref[ch:40]{Artikel 40} oder genehmigten
     Zertifizierungsverfahren nach \hyperref[ch:42]{Artikel 42} und
    \label{itm:83-2j}

    \item jegliche anderen erschwerenden oder mildernden Umstände im jeweiligen Fall, wie unmittelbar oder mittelbar
     durch den Verstoß erlangte finanzielle Vorteile oder vermiedene Verluste.
    \label{itm:83-2k}

  \end{enumerate}

  \item Verstößt ein Verantwortlicher oder ein Auftragsverarbeiter bei gleichen oder miteinander verbundenen
   Verarbeitungsvorgängen vorsätzlich oder fahrlässig gegen mehrere Bestimmungen dieser Verordnung, so übersteigt der
   Gesamtbetrag der Geldbuße nicht den Betrag für den schwerwiegendsten Verstoß.
  \label{itm:83-3}

  \item Bei Verstößen gegen die folgenden Bestimmungen werden im Einklang mit \hyperref[itm:83-2]{Absatz 2} Geldbußen
   von bis zu 10~000~000~EUR oder im Fall eines Unternehmens von bis zu 2~\% seines gesamten weltweit erzielten
   Jahresumsatzes des vorangegangenen Geschäftsjahrs verhängt, je nachdem, welcher der Beträge höher ist:
  \label{itm:83-4}

  \begin{enumerate}
  
    \item die Pflichten der Verantwortlichen und der Auftragsverarbeiter gemäß den Artikeln \hyperref[ch:8]{8},
     \hyperref[ch:11]{11}, \hyperref[ch:25]{25} bis \hyperref[ch:39]{39}, \hyperref[ch:42]{42} und \hyperref[ch:43]{43};
    \label{itm:83-4a}

    \item die Pflichten der Zertifizierungsstelle gemäß den Artikeln \hyperref[ch:42]{42} und \hyperref[ch:43]{43};
    \label{itm:83-4b}

    \item die Pflichten der Überwachungsstelle gemäß \hyperref[itm:41-4]{Artikel 41 Absatz 4}.
    \label{itm:83-4c}

  \end{enumerate}

  \item Bei Verstößen gegen die folgenden Bestimmungen werden im Einklang mit \hyperref[itm:83-2]{Absatz 2} Geldbußen
   von bis zu 20~000~000~EUR oder im Fall eines Unternehmens von bis zu 4~\% seines gesamten weltweit erzielten
   Jahresumsatzes des vorangegangenen Geschäftsjahrs verhängt, je nachdem, welcher der Beträge höher ist:
  \label{itm:83-5}

  \begin{enumerate}
  
    \item die Grundsätze für die Verarbeitung, einschließlich der Bedingungen für die Einwilligung, gemäß den Artikeln
     \hyperref[ch:5]{5}, \hyperref[ch:6]{6}, \hyperref[ch:7]{7} und \hyperref[ch:9]{9};
    \label{itm:83-5a}

    \item die Rechte der betroffenen Person gemäß den Artikeln \hyperref[ch:12]{12} bis \hyperref[ch:22]{22};
    \label{itm:83-5b}

    \item die Übermittlung \hyperref[itm:04-1]{personenbezogener Daten} an einen Empfänger in einem Drittland oder an eine internationale
     Organisation gemäß den Artikeln \hyperref[ch:44]{44} bis \hyperref[ch:49]{49};
    \label{itm:83-5c}

    \item alle Pflichten gemäß den Rechtsvorschriften der Mitgliedstaaten, die im Rahmen des \hyperref[part:9]
     {Kapitels IX} erlassen wurden;
    \label{itm:83-5d}

    \item Nichtbefolgung einer Anweisung oder einer vorübergehenden oder endgültigen Beschränkung oder Aussetzung der
     Datenübermittlung durch die Aufsichtsbehörde gemäß \hyperref[itm:58-2]{Artikel 58 Absatz 2} oder Nichtgewährung
     des Zugangs unter Verstoß gegen \hyperref[itm:58-1]{Artikel 58 Absatz 1}.
    \label{itm:83-5e}

  \end{enumerate}

  \item Bei Nichtbefolgung einer Anweisung der Aufsichtsbehörde gemäß \hyperref[itm:58-2]{Artikel 58 Absatz 2} werden im
   Einklang mit \hyperref[itm:83-2]{Absatz 2 des vorliegenden Artikels} Geldbußen von bis zu 20~000~000~EUR oder im
   Fall eines Unternehmens von bis zu 4~\% seines gesamten weltweit erzielten Jahresumsatzes des vorangegangenen
   Geschäftsjahrs verhängt, je nachdem, welcher der Beträge höher ist.
  \label{itm:83-6}

  \item Unbeschadet der Abhilfebefugnisse der Aufsichtsbehörden gemäß \hyperref[itm:58-2]{Artikel 58 Absatz 2} kann
   jeder Mitgliedstaat Vorschriften dafür festlegen, ob und in welchem Umfang gegen Behörden und öffentliche Stellen,
   die in dem betreffenden Mitgliedstaat niedergelassen sind, Geldbußen verhängt werden können.
  \label{itm:83-7}

  \item Die Ausübung der eigenen Befugnisse durch eine Aufsichtsbehörde gemäß diesem Artikel muss angemessenen
   Verfahrensgarantien gemäß dem Unionsrecht und dem Recht der Mitgliedstaaten, einschließlich wirksamer gerichtlicher
   Rechtsbehelfe und ordnungsgemäßer Verfahren, unterliegen.
  \label{itm:83-8}

  \item Sieht die Rechtsordnung eines Mitgliedstaats keine Geldbußen vor, kann dieser Artikel so angewandt werden, dass
   die Geldbuße von der zuständigen Aufsichtsbehörde in die Wege geleitet und von den zuständigen nationalen Gerichten
   verhängt wird, wobei sicherzustellen ist, dass diese Rechtsbehelfe wirksam sind und die gleiche Wirkung wie die von
   Aufsichtsbehörden verhängten Geldbußen haben. In jeden Fall müssen die verhängten Geldbußen wirksam, verhältnismäßig
   und abschreckend sein. Die betreffenden Mitgliedstaaten teilen der Kommission bis zum 25. Mai 2018 die
   Rechtsvorschriften mit, die sie aufgrund dieses Absatzes erlassen, sowie unverzüglich alle späteren Änderungsgesetze
   oder Änderungen dieser Vorschriften.

\end{enumerate}

\addsec{Eigene Notizen}

