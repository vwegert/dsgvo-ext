%!TEX root = ../DSGVO-ExtendedVersion.tex
\chapter{Allgemeine Bedingungen für die Verhängung von Geldbußen}
\label{ch:83}

\begin{enumerate}

  \item Jede \hyperref[itm:04-21]{Aufsichtsbehörde} stellt sicher, dass die Verhängung von Geldbußen gemäß diesem
   Artikel für Verstöße gegen diese Verordnung gemäß den Absätzen \hyperref[itm:83-4]{4}, \hyperref[itm:83-5]{5} und
   \hyperref[itm:83-6]{6} in jedem Einzelfall wirksam, verhältnismäßig und abschreckend ist.%
  \label{itm:83-1}

  \item Geldbußen werden je nach den Umständen des Einzelfalls zusätzlich zu oder anstelle von Maßnahmen nach \hyperref
   [itm:58-2]{Artikel 58 Absatz 2} Buchstaben \hyperref[itm:58-2a]{a} bis \hyperref[itm:58-2h]{h} und \hyperref
   [itm:58-2j]{j} verhängt. Bei der Entscheidung über die Verhängung einer Geldbuße und über deren Betrag wird in jedem
   Einzelfall Folgendes gebührend berücksichtigt:%
  \label{itm:83-2}

  \begin{enumerate}
  
    \item Art, Schwere und Dauer des Verstoßes unter Berücksichtigung der Art, des Umfangs oder des Zwecks der
     betreffenden \hyperref[itm:04-2]{Verarbeitung} sowie der Zahl der von der \hyperref[itm:04-2]
     {Verarbeitung} \hyperref[itm:04-1]{betroffenen Personen} und des Ausmaßes des von ihnen erlittenen Schadens;%
    \label{itm:83-2a}

    \item Vorsätzlichkeit oder Fahrlässigkeit des Verstoßes;%
    \label{itm:83-2b}

    \item jegliche von dem \hyperref[itm:04-7]{Verantwortlichen} oder dem \hyperref[itm:04-8]
     {Auftragsverarbeiter} getroffenen Maßnahmen zur Minderung des den
     \hyperref[itm:04-1]{betroffenen Personen} entstandenen Schadens;%
    \label{itm:83-2c}

    \item Grad der Verantwortung des \hyperref[itm:04-7]{Verantwortlichen} oder des \hyperref[itm:04-8]
     {Auftragsverarbeiters} unter Berücksichtigung der von ihnen gemäß den Artikeln \hyperref[ch:25]{25} und \hyperref
     [ch:32]{32} getroffenen technischen und organisatorischen Maßnahmen;%
    \label{itm:83-2d}

    \item etwaige einschlägige frühere Verstöße des \hyperref[itm:04-7]{Verantwortlichen} oder des \hyperref[itm:04-8]
     {Auftragsverarbeiters};%
    \label{itm:83-2e}

    \item Umfang der Zusammenarbeit mit der \hyperref[itm:04-21]{Aufsichtsbehörde}, um dem Verstoß abzuhelfen und seine
     möglichen nachteiligen Auswirkungen zu mindern;%
    \label{itm:83-2f}

    \item Kategorien \hyperref[itm:04-1]{personenbezogener Daten}, die von dem Verstoß betroffen sind;%
    \label{itm:83-2g}

    \item Art und Weise, wie der Verstoß der \hyperref[itm:04-21]{Aufsichtsbehörde} bekannt wurde, insbesondere ob und
     gegebenenfalls in welchem Umfang der \hyperref[itm:04-7]{Verantwortliche} oder der \hyperref[itm:04-8]
     {Auftragsverarbeiter} den Verstoß mitgeteilt hat;%
    \label{itm:83-2h}

    \item Einhaltung der nach \hyperref[itm:58-2]{Artikel 58 Absatz 2} früher gegen den für den betreffenden
     \hyperref[itm:04-7]{Verantwortlichen} oder \hyperref[itm:04-8]{Auftragsverarbeiter} in Bezug auf denselben
      Gegenstand angeordneten Maßnahmen, wenn solche Maßnahmen angeordnet wurden;%
    \label{itm:83-2i}

    \item Einhaltung von genehmigten Verhaltensregeln nach \hyperref[ch:40]{Artikel 40} oder genehmigten
     Zertifizierungsverfahren nach \hyperref[ch:42]{Artikel 42} und%
    \label{itm:83-2j}

    \item jegliche anderen erschwerenden oder mildernden Umstände im jeweiligen Fall, wie unmittelbar oder mittelbar
     durch den Verstoß erlangte finanzielle Vorteile oder vermiedene Verluste.%
    \label{itm:83-2k}

  \end{enumerate}

  \item Verstößt ein \hyperref[itm:04-7]{Verantwortlicher} oder ein \hyperref[itm:04-8]{Auftragsverarbeiter} bei
   gleichen oder miteinander verbundenen Verarbeitungsvorgängen vorsätzlich oder fahrlässig gegen mehrere Bestimmungen
   dieser Verordnung, so übersteigt der Gesamtbetrag der Geldbuße nicht den Betrag für den schwerwiegendsten Verstoß.%
  \label{itm:83-3}

  \item Bei Verstößen gegen die folgenden Bestimmungen werden im Einklang mit \hyperref[itm:83-2]{Absatz 2} Geldbußen
   von bis zu 10~000~000~EUR oder im Fall eines Unternehmens von bis zu 2~\% seines gesamten weltweit erzielten
   Jahresumsatzes des vorangegangenen Geschäftsjahrs verhängt, je nachdem, welcher der Beträge höher ist:%
  \label{itm:83-4}

  \begin{enumerate}
  
    \item die Pflichten der \hyperref[itm:04-7]{Verantwortlichen} und der \hyperref[itm:04-8]{Auftragsverarbeiter} gemäß
     den Artikeln \hyperref[ch:8]{8},
     \hyperref[ch:11]{11}, \hyperref[ch:25]{25} bis \hyperref[ch:39]{39}, \hyperref[ch:42]{42} und \hyperref[ch:43]
      {43};%
    \label{itm:83-4a}

    \item die Pflichten der Zertifizierungsstelle gemäß den Artikeln \hyperref[ch:42]{42} und \hyperref[ch:43]{43};%
    \label{itm:83-4b}

    \item die Pflichten der Überwachungsstelle gemäß \hyperref[itm:41-4]{Artikel 41 Absatz 4}.%
    \label{itm:83-4c}

  \end{enumerate}

  \item Bei Verstößen gegen die folgenden Bestimmungen werden im Einklang mit \hyperref[itm:83-2]{Absatz 2} Geldbußen
   von bis zu 20~000~000~EUR oder im Fall eines Unternehmens von bis zu 4~\% seines gesamten weltweit erzielten
   Jahresumsatzes des vorangegangenen Geschäftsjahrs verhängt, je nachdem, welcher der Beträge höher ist:%
  \label{itm:83-5}

  \begin{enumerate}
  
    \item die Grundsätze für die \hyperref[itm:04-2]{Verarbeitung}, einschließlich der Bedingungen für die \hyperref
     [itm:04-11]{Einwilligung}, gemäß den Artikeln
     \hyperref[ch:5]{5}, \hyperref[ch:6]{6}, \hyperref[ch:7]{7} und \hyperref[ch:9]{9};%
    \label{itm:83-5a}

    \item die Rechte der \hyperref[itm:04-1]{betroffenen Person} gemäß den Artikeln \hyperref[ch:12]{12} bis \hyperref
     [ch:22]{22};%
    \label{itm:83-5b}

    \item die Übermittlung \hyperref[itm:04-1]{personenbezogener Daten} an einen \hyperref[itm:04-9]{Empfänger} in einem
     Drittland oder an eine \hyperref[itm:04-26]{internationale Organisation} gemäß den Artikeln \hyperref[ch:44]
     {44} bis \hyperref[ch:49]{49};%
    \label{itm:83-5c}

    \item alle Pflichten gemäß den Rechtsvorschriften der Mitgliedstaaten, die im Rahmen des \hyperref[part:9]
     {Kapitels IX} erlassen wurden;%
    \label{itm:83-5d}

    \item Nichtbefolgung einer Anweisung oder einer vorübergehenden oder endgültigen Beschränkung oder Aussetzung der
     Datenübermittlung durch die \hyperref[itm:04-21]{Aufsichtsbehörde} gemäß \hyperref[itm:58-2]{Artikel 58 Absatz 2}
     oder Nichtgewährung des Zugangs unter Verstoß gegen \hyperref[itm:58-1]{Artikel 58 Absatz 1}.%
    \label{itm:83-5e}

  \end{enumerate}

  \item Bei Nichtbefolgung einer Anweisung der \hyperref[itm:04-21]{Aufsichtsbehörde} gemäß \hyperref[itm:58-2]
   {Artikel 58 Absatz 2} werden im Einklang mit \hyperref[itm:83-2]{Absatz 2 des vorliegenden Artikels} Geldbußen von
   bis zu 20~000~000~EUR oder im Fall eines Unternehmens von bis zu 4~\% seines gesamten weltweit erzielten
   Jahresumsatzes des vorangegangenen Geschäftsjahrs verhängt, je nachdem, welcher der Beträge höher ist.%
  \label{itm:83-6}

  \item Unbeschadet der Abhilfebefugnisse der \hyperref[itm:04-21]{Aufsichtsbehörden} gemäß \hyperref[itm:58-2]
   {Artikel 58 Absatz 2} kann jeder Mitgliedstaat Vorschriften dafür festlegen, ob und in welchem Umfang gegen Behörden
   und öffentliche Stellen, die in dem betreffenden Mitgliedstaat niedergelassen sind, Geldbußen verhängt werden
   können.%
  \label{itm:83-7}

  \item Die Ausübung der eigenen Befugnisse durch eine \hyperref[itm:04-21]{Aufsichtsbehörde} gemäß diesem Artikel muss
   angemessenen Verfahrensgarantien gemäß dem Unionsrecht und dem Recht der Mitgliedstaaten, einschließlich wirksamer
   gerichtlicher Rechtsbehelfe und ordnungsgemäßer Verfahren, unterliegen.%
  \label{itm:83-8}

  \item Sieht die Rechtsordnung eines Mitgliedstaats keine Geldbußen vor, kann dieser Artikel so angewandt werden, dass
   die Geldbuße von der zuständigen \hyperref[itm:04-21]{Aufsichtsbehörde} in die Wege geleitet und von den zuständigen
   nationalen Gerichten verhängt wird, wobei sicherzustellen ist, dass diese Rechtsbehelfe wirksam sind und die gleiche
   Wirkung wie die von
   \hyperref[itm:04-21]{Aufsichtsbehörden} verhängten Geldbußen haben. In jeden Fall müssen die verhängten Geldbußen
    wirksam, verhältnismäßig und abschreckend sein. Die betreffenden Mitgliedstaaten teilen der Kommission bis zum 25.
    Mai 2018 die Rechtsvorschriften mit, die sie aufgrund dieses Absatzes erlassen, sowie unverzüglich alle späteren
    Änderungsgesetze oder Änderungen dieser Vorschriften.

\end{enumerate}

% \addsec{Ergänzende Hinweise}

