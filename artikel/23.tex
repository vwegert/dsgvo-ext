%!TEX root = ../DSGVO-ExtendedVersion.tex
\chapter{Beschränkungen}
\label{ch:23}

\crossrefArticleToReason{23}

\begin{enumerate}

  \item Durch Rechtsvorschriften der Union oder der Mitgliedstaaten, denen der \hyperref[itm:04-7]{Verantwortliche} oder
   der
   \hyperref[itm:04-8]{Auftragsverarbeiter} unterliegt, können die Pflichten und Rechte gemäß den Artikeln \hyperref
    [ch:12]{12} bis \hyperref[ch:22]{22} und \hyperref[ch:34]{Artikel 34} sowie \hyperref[ch:5]{Artikel 5}, insofern
    dessen Bestimmungen den in den Artikeln \hyperref[ch:12]{12} bis \hyperref[ch:22]{22} vorgesehenen Rechten und
    Pflichten entsprechen, im Wege von Gesetzgebungsmaßnahmen beschränkt werden, sofern eine solche Beschränkung den
    Wesensgehalt der Grundrechte und Grundfreiheiten achtet und in einer demokratischen Gesellschaft eine notwendige
    und verhältnismäßige Maßnahme darstellt, die Folgendes sicherstellt:%
  \label{itm:23-1}

  \begin{enumerate}
  
    \item die nationale Sicherheit;%
    \label{itm:23-1a}

    \item die Landesverteidigung;%
    \label{itm:23-1b}

    \item die öffentliche Sicherheit;%
    \label{itm:23-1c}

    \item die Verhütung, Ermittlung, Aufdeckung oder Verfolgung von Straftaten oder die Strafvollstreckung,
     einschließlich des Schutzes vor und der Abwehr von Gefahren für die öffentliche Sicherheit;%
    \label{itm:23-1d}

    \item den Schutz sonstiger wichtiger Ziele des allgemeinen öffentlichen Interesses der Union oder eines
     Mitgliedstaats, insbesondere eines wichtigen wirtschaftlichen oder finanziellen Interesses der Union oder eines
     Mitgliedstaats, etwa im Währungs-, Haushalts- und Steuerbereich sowie im Bereich der öffentlichen Gesundheit und
     der sozialen Sicherheit;%
    \label{itm:23-1e}

    \item den Schutz der Unabhängigkeit der Justiz und den Schutz von Gerichtsverfahren;%
    \label{itm:23-1f}

    \item die Verhütung, Aufdeckung, Ermittlung und Verfolgung von Verstößen gegen die berufsständischen Regeln
     reglementierter Berufe;%
    \label{itm:23-1g}

    \item Kontroll-, Überwachungs- und Ordnungsfunktionen, die dauernd oder zeitweise mit der Ausübung öffentlicher
     Gewalt für die unter den Buchstaben \hyperref[itm:23-1a]{a} bis \hyperref[itm:23-1e]{e} und \hyperref[itm:23-1g]
     {g} genannten Zwecke verbunden sind;%
    \label{itm:23-1h}

    \item den Schutz der \hyperref[itm:04-1]{betroffenen Person} oder der Rechte und Freiheiten anderer Personen;%
    \label{itm:23-1i}

    \item die Durchsetzung zivilrechtlicher Ansprüche.%
    \label{itm:23-1j}

  \end{enumerate}

  \item Jede Gesetzgebungsmaßnahme im Sinne des \hyperref[itm:23-1]{Absatzes 1} muss insbesondere gegebenenfalls
   spezifische Vorschriften enthalten zumindest in Bezug auf%
  \label{itm:23-2}

  \begin{enumerate}
  
    \item die Zwecke der \hyperref[itm:04-2]{Verarbeitung} oder die Verarbeitungskategorien,%
    \label{itm:23-2a}

    \item die Kategorien \hyperref[itm:04-1]{personenbezogener Daten},%
    \label{itm:23-2b}

    \item den Umfang der vorgenommenen Beschränkungen,%
    \label{itm:23-2c}

    \item die Garantien gegen Missbrauch oder unrechtmäßigen Zugang oder unrechtmäßige Übermittlung;%
    \label{itm:23-2d}

    \item die Angaben zu dem \hyperref[itm:04-7]{Verantwortlichen} oder den Kategorien von \hyperref[itm:04-7]
     {Verantwortlichen},%
    \label{itm:23-2e}

    \item die jeweiligen Speicherfristen sowie die geltenden Garantien unter Berücksichtigung von Art, Umfang und
     Zwecken der \hyperref[itm:04-2]{Verarbeitung} oder der Verarbeitungskategorien,%
    \label{itm:23-2f}

    \item die Risiken für die Rechte und Freiheiten der \hyperref[itm:04-1]{betroffenen Personen} und%
    \label{itm:23-2g}

    \item das Recht der \hyperref[itm:04-1]{betroffenen Personen} auf Unterrichtung über die Beschränkung, sofern dies
     nicht dem Zweck der Beschränkung abträglich ist.%
    \label{itm:23-2h}

  \end{enumerate}

\end{enumerate}

% \addsec{Ergänzende Hinweise}

