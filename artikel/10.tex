%!TEX root = ../DSGVO-ExtendedVersion.tex
\chapter{Verarbeitung von personenbezogenen Daten über strafrechtliche Verurteilungen und Straftaten}
\label{ch:10}

\crossrefArticleToReason{10}

Die \hyperref[itm:04-2]{Verarbeitung} \hyperref[itm:04-1]{personenbezogener Daten} über strafrechtliche Verurteilungen
und Straftaten oder damit zusammenhängende Sicherungsmaßregeln\comment{Pflicht zur Datenschutz"=Folgenabschätzung
nach \hyperref[itm:35-3b]{Artikel 35 Absatz 3 Buchstabe b}} aufgrund von \hyperref[itm:06-1]{Artikel 6 Absatz 1} darf
nur unter behördlicher Aufsicht vorgenommen werden oder wenn dies nach dem Unionsrecht oder dem Recht der
Mitgliedstaaten, das geeignete Garantien für die Rechte und Freiheiten der \hyperref[itm:04-1]{betroffenen Personen}
vorsieht, zulässig ist. Ein umfassendes Register der strafrechtlichen Verurteilungen darf nur unter behördlicher
Aufsicht geführt werden.

% \addsec{Ergänzende Hinweise}

