%!TEX root = ../DSGVO-Bearbeitung.tex
\chapter{Recht auf wirksamen gerichtlichen Rechtsbehelf gegen \hyperref[itm:04-7]{Verantwortliche} oder Auftragsverarbeiter}
\label{ch:79}

\addsec{Text der Verordnung}

\begin{enumerate}

  \item Jede \hyperref[itm:04-1]{betroffene Person} hat unbeschadet eines verfügbaren verwaltungsrechtlichen oder außergerichtlichen
   Rechtsbehelfs einschließlich des Rechts auf Beschwerde bei einer \hyperref[itm:04-21]{Aufsichtsbehörde} gemäß \hyperref[ch:77]{Artikel 77}
   das Recht auf einen wirksamen gerichtlichen Rechtsbehelf, wenn sie der Ansicht ist, dass die ihr aufgrund dieser
   Verordnung zustehenden Rechte infolge einer nicht im Einklang mit dieser Verordnung stehenden \hyperref[itm:04-2]{Verarbeitung} ihrer
   \hyperref[itm:04-1]{personenbezogenen Daten} verletzt wurden.
  \label{itm:79-1}

  \item Für Klagen gegen einen \hyperref[itm:04-7]{Verantwortlichen} oder gegen einen Auftragsverarbeiter sind die Gerichte des
   Mitgliedstaats zuständig, in dem der \hyperref[itm:04-7]{Verantwortliche} oder der Auftragsverarbeiter eine Niederlassung hat. Wahlweise
   können solche Klagen auch bei den Gerichten des Mitgliedstaats erhoben werden, in dem die \hyperref[itm:04-1]{betroffene Person} ihren
   Aufenthaltsort hat, es sei denn, es handelt sich bei dem \hyperref[itm:04-7]{Verantwortlichen} oder dem Auftragsverarbeiter um eine
   Behörde eines Mitgliedstaats, die in Ausübung ihrer hoheitlichen Befugnisse tätig geworden ist.
  \label{itm:79-2}

\end{enumerate}

\addsec{Eigene Notizen}

