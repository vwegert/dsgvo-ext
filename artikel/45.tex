%!TEX root = ../DSGVO-Bearbeitung.tex
\chapter{Datenübermittlung auf der Grundlage eines Angemessenheitsbeschlusses}
\label{ch:45}

\addsec{Text der Verordnung}

\begin{enumerate}

  \item Eine Übermittlung \hyperref[itm:04-1]{personenbezogener Daten} an ein Drittland oder eine \hyperref[itm:04-29]
   {internationale Organisation} darf vorgenommen werden, wenn die Kommission beschlossen hat, dass das betreffende
   Drittland, ein Gebiet oder ein oder mehrere spezifische Sektoren in diesem Drittland oder die betreffende \hyperref
   [itm:04-29]{internationale Organisation} ein angemessenes Schutzniveau bietet. Eine solche Datenübermittlung bedarf
   keiner besonderen Genehmigung.
  \label{itm:45-1}

  \item Bei der Prüfung der Angemessenheit des gebotenen Schutzniveaus berücksichtigt die Kommission insbesondere das
   Folgende:
  \label{itm:45-2}

  \begin{enumerate}
  
    \item die Rechtsstaatlichkeit, die Achtung der Menschenrechte und Grundfreiheiten, die in dem betreffenden Land bzw.
     bei der betreffenden internationalen Organisation geltenden einschlägigen Rechtsvorschriften sowohl allgemeiner
     als auch sektoraler Art -- auch in Bezug auf öffentliche Sicherheit, Verteidigung, nationale Sicherheit und
     Strafrecht sowie Zugang der Behörden zu \hyperref[itm:04-1]{personenbezogenen Daten} -- sowie die Anwendung dieser
     Rechtsvorschriften, Datenschutzvorschriften, Berufsregeln und Sicherheitsvorschriften einschließlich der
     Vorschriften für die Weiterübermittlung \hyperref[itm:04-1]{personenbezogener Daten} an ein anderes Drittland bzw.
     eine andere \hyperref[itm:04-29]{internationale Organisation}, die Rechtsprechung sowie wirksame und durchsetzbare
     Rechte der \hyperref[itm:04-1]{betroffenen Person} und wirksame verwaltungsrechtliche und gerichtliche
     Rechtsbehelfe für \hyperref[itm:04-1]{betroffene Personen}, deren \hyperref[itm:04-1]{personenbezogene Daten}
     übermittelt werden,
    \label{itm:45-2a}

    \item die Existenz und die wirksame Funktionsweise einer oder mehrerer unabhängiger \hyperref[itm:04-21]
     {Aufsichtsbehörden} in dem betreffenden Drittland oder denen eine \hyperref[itm:04-29]
     {internationale Organisation} untersteht und die für die Einhaltung und Durchsetzung der Datenschutzvorschriften,
     einschließlich angemessener Durchsetzungsbefugnisse, für die Unterstützung und Beratung der \hyperref[itm:04-1]
     {betroffenen Personen} bei der Ausübung ihrer Rechte und für die Zusammenarbeit mit den \hyperref[itm:04-21]
     {Aufsichtsbehörden} der Mitgliedstaaten zuständig sind, und
    \label{itm:45-2b}

    \item die von dem betreffenden Drittland bzw. der betreffenden internationalen Organisation eingegangenen
     internationalen Verpflichtungen oder andere Verpflichtungen, die sich aus rechtsverbindlichen Übereinkünften oder
     Instrumenten sowie aus der Teilnahme des Drittlands oder der internationalen Organisation an multilateralen oder
     regionalen Systemen insbesondere in Bezug auf den Schutz \hyperref[itm:04-1]{personenbezogener Daten} ergeben.
    \label{itm:45-2c}

  \end{enumerate}

  \item Nach der Beurteilung der Angemessenheit des Schutzniveaus kann die Kommission im Wege eines
   Durchführungsrechtsaktes beschließen, dass ein Drittland, ein Gebiet oder ein oder mehrere spezifische Sektoren in
   einem Drittland oder eine \hyperref[itm:04-29]{internationale Organisation} ein angemessenes Schutzniveau im Sinne
   des \hyperref[itm:45-2]{Absatzes 2 des vorliegenden Artikels} bieten. In dem Durchführungsrechtsakt ist ein
   Mechanismus für eine regelmäßige Überprüfung, die mindestens alle vier Jahre erfolgt, vorzusehen, bei der allen
   maßgeblichen Entwicklungen in dem Drittland oder bei der internationalen Organisation Rechnung getragen wird. Im
   Durchführungsrechtsakt werden der territoriale und der sektorale Anwendungsbereich sowie gegebenenfalls die in
   \hyperref[itm:45-2b]{Absatz 2 Buchstabe b des vorliegenden Artikels} genannte \hyperref[itm:04-21]
    {Aufsichtsbehörde} bzw. genannten
   \hyperref[itm:04-21]{Aufsichtsbehörden} angegeben. Der Durchführungsrechtsakt wird gemäß dem in \hyperref[itm:93-2]
    {Artikel 93 Absatz 2} genannten Prüfverfahren erlassen.
  \label{itm:45-3}

  \item Die Kommission überwacht fortlaufend die Entwicklungen in Drittländern und bei internationalen Organisationen,
   die die Wirkungsweise der nach \hyperref[itm:45-3]{Absatz 3 des vorliegenden Artikels} erlassenen Beschlüsse und der
   nach Artikel 25 Absatz 6 der Richtlinie 95/46/EG\todo{nachschlagen} erlassenen Feststellungen beeinträchtigen
   könnten.
  \label{itm:45-4}

  \item Die Kommission widerruft, ändert oder setzt die in \hyperref[itm:45-3]{Absatz 3 des vorliegenden Artikels}
   genannten Beschlüsse im Wege von Durchführungsrechtsakten aus, soweit dies nötig ist und ohne rückwirkende Kraft,
   soweit entsprechende Informationen -- insbesondere im Anschluss an die in \hyperref[itm:45-3]{Absatz 3 des
   vorliegenden Artikels} genannte Überprüfung -- dahingehend vorliegen, dass ein Drittland, ein Gebiet oder ein oder
   mehrere spezifischer Sektor in einem Drittland oder eine \hyperref[itm:04-29]{internationale Organisation} kein
   angemessenes Schutzniveau im Sinne des \hyperref[itm:45-2]{Absatzes 2 des vorliegenden Artikels} mehr gewährleistet.
   Diese Durchführungsrechtsakte werden gemäß dem Prüfverfahren nach \hyperref[itm:93-2]{Artikel 93 Absatz 2}
   erlassen. 
  \label{itm:45-5}

   In hinreichend begründeten Fällen äußerster Dringlichkeit erlässt die Kommission gemäß dem in \hyperref[itm:93-3]
   {Artikel 93 Absatz 3} genannten Verfahren sofort geltende Durchführungsrechtsakte.

  \item Die Kommission nimmt Beratungen mit dem betreffenden Drittland bzw. der betreffenden internationalen
   Organisation auf, um Abhilfe für die Situation zu schaffen, die zu dem gemäß \hyperref[itm:45-5]{Absatz 5}
   erlassenen Beschluss geführt hat.
  \label{itm:45-6}

  \item Übermittlungen \hyperref[itm:04-1]{personenbezogener Daten} an das betreffende Drittland, das Gebiet oder einen
   oder mehrere spezifische Sektoren in diesem Drittland oder an die betreffende \hyperref[itm:04-29]
   {internationale Organisation} gemäß den Artikeln
   \hyperref[ch:46]{46} bis \hyperref[ch:49]{49} werden durch einen Beschluss nach \hyperref[itm:45-5]{Absatz 5 des
    vorliegenden Artikels} nicht berührt.
  \label{itm:45-7}

  \item Die Kommission veröffentlicht im \emph{Amtsblatt der Europäischen Union} und auf ihrer Website eine Liste aller
   Drittländer beziehungsweise Gebiete und spezifischen Sektoren in einem Drittland und aller internationalen
   Organisationen, für die sie durch Beschluss festgestellt hat, dass sie ein angemessenes Schutzniveau gewährleisten
   bzw. nicht mehr gewährleisten.   
  \label{itm:45-8}

  \item Von der Kommission auf der Grundlage von Artikel 25 Absatz 6 der Richtlinie 95/46/EG\todo
   {nachschlagen} erlassene Feststellungen bleiben so lange in Kraft, bis sie durch einen nach dem Prüfverfahren gemäß
   den Absätzen \hyperref[itm:45-3]{3} oder \hyperref[itm:45-5]{5} des vorliegenden Artikels erlassenen Beschluss der
   Kommission geändert, ersetzt oder aufgehoben werden.
  \label{itm:45-9}

\end{enumerate}

\addsec{Eigene Notizen}

