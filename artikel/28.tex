%!TEX root = ../DSGVO-ExtendedVersion.tex
\chapter{Auftragsverarbeiter}
\label{ch:28}

\begin{enumerate}

  \item Erfolgt eine \hyperref[itm:04-2]{Verarbeitung} im Auftrag eines \hyperref[itm:04-7]{Verantwortlichen}, so
   arbeitet dieser nur mit \hyperref[itm:04-8]{Auftragsverarbeitern}, die hinreichend Garantien dafür bieten, dass
   geeignete technische und organisatorische Maßnahmen so durchgeführt werden, dass die \hyperref[itm:04-2]
   {Verarbeitung} im Einklang mit den Anforderungen dieser Verordnung erfolgt und den Schutz der Rechte der \hyperref
   [itm:04-1]{betroffenen Person} gewährleistet.%
  \label{itm:28-1}

  \item Der \hyperref[itm:04-8]{Auftragsverarbeiter} nimmt keinen weiteren \hyperref[itm:04-8]{Auftragsverarbeiter} ohne
   vorherige gesonderte oder allgemeine schriftliche Genehmigung des \hyperref[itm:04-7]{Verantwortlichen} in Anspruch.
   Im Fall einer allgemeinen schriftlichen Genehmigung informiert der \hyperref[itm:04-8]
   {Auftragsverarbeiter} den \hyperref[itm:04-7]{Verantwortlichen} immer über jede beabsichtigte Änderung in Bezug auf
   die Hinzuziehung oder die Ersetzung anderer \hyperref[itm:04-8]{Auftragsverarbeiter}, wodurch der \hyperref
   [itm:04-7]{Verantwortliche} die Möglichkeit erhält, gegen derartige Änderungen Einspruch zu erheben.%
  \label{itm:28-2}

  \item Die \hyperref[itm:04-2]{Verarbeitung} durch einen \hyperref[itm:04-8]{Auftragsverarbeiter} erfolgt auf der
   Grundlage eines Vertrags oder eines anderen Rechtsinstruments nach dem Unionsrecht oder dem Recht der
   Mitgliedstaaten, der bzw. das den \hyperref[itm:04-8]{Auftragsverarbeiter} in Bezug auf den \hyperref[itm:04-7]
   {Verantwortlichen} bindet und in dem Gegenstand und Dauer der \hyperref[itm:04-2]{Verarbeitung}, Art und Zweck der
   \hyperref[itm:04-2]{Verarbeitung}, die Art der \hyperref[itm:04-1]{personenbezogenen Daten}, die Kategorien \hyperref
    [itm:04-1]{betroffener Personen} und die Pflichten und Rechte des \hyperref[itm:04-7]{Verantwortlichen} festgelegt
    sind. Dieser Vertrag bzw. dieses andere Rechtsinstrument sieht insbesondere vor, dass der \hyperref[itm:04-8]
    {Auftragsverarbeiter}%
  \label{itm:28-3-1}

  \begin{enumerate}
  
    \item die \hyperref[itm:04-1]{personenbezogenen Daten} nur auf dokumentierte Weisung des \hyperref[itm:04-7]
     {Verantwortlichen} -- auch in Bezug auf die Übermittlung \hyperref[itm:04-1]{personenbezogener Daten} an ein
     Drittland oder eine \hyperref[itm:04-26]{internationale Organisation} -- verarbeitet, sofern er nicht durch das
     Recht der Union oder der Mitgliedstaaten, dem der \hyperref[itm:04-8]{Auftragsverarbeiter} unterliegt, hierzu
     verpflichtet ist; in einem solchen Fall teilt der \hyperref[itm:04-8]{Auftragsverarbeiter} dem \hyperref[itm:04-7]
     {Verantwortlichen} diese rechtlichen Anforderungen vor der \hyperref[itm:04-2]{Verarbeitung} mit, sofern das
     betreffende Recht eine solche Mitteilung nicht wegen eines wichtigen öffentlichen Interesses verbietet;%
    \label{itm:28-3-1a}

    \item gewährleistet, dass sich die zur \hyperref[itm:04-2]{Verarbeitung} der \hyperref[itm:04-1]
     {personenbezogenen Daten} befugten Personen zur Vertraulichkeit verpflichtet haben oder einer angemessenen
     gesetzlichen Verschwiegenheitspflicht unterliegen;%
    \label{itm:28-3-1b}

    \item alle gemäß \hyperref[ch:32]{Artikel 32} erforderlichen Maßnahmen ergreift;%
    \label{itm:28-3-1c}

    \item die in den Absätzen \hyperref[itm:28-2]{2} und \hyperref[itm:28-4]{4} genannten Bedingungen für die
     Inanspruchnahme der Dienste eines weiteren \hyperref[itm:04-8]{Auftragsverarbeiters} einhält;%
    \label{itm:28-3-1d}

    \item angesichts der Art der \hyperref[itm:04-2]{Verarbeitung} den \hyperref[itm:04-7]{Verantwortlichen} nach
     Möglichkeit mit geeigneten technischen und organisatorischen Maßnahmen dabei unterstützt, seiner Pflicht zur
     Beantwortung von Anträgen auf Wahrnehmung der in \hyperref[part:3]{Kapitel III} genannten Rechte der \hyperref
     [itm:04-1]{betroffenen Person} nachzukommen;%
    \label{itm:28-3-1e}

    \item unter Berücksichtigung der Art der \hyperref[itm:04-2]{Verarbeitung} und der ihm zur Verfügung stehenden
     Informationen den \hyperref[itm:04-7]{Verantwortlichen} bei der Einhaltung der in den Artikeln \hyperref[ch:32]
     {32} bis \hyperref[ch:36]{36} genannten Pflichten unterstützt;%
    \label{itm:28-3-1f}

    \item nach Abschluss der Erbringung der Verarbeitungsleistungen alle \hyperref[itm:04-1]{personenbezogenen Daten}
     nach Wahl des \hyperref[itm:04-7]{Verantwortlichen} entweder löscht oder zurückgibt und die vorhandenen Kopien
     löscht, sofern nicht nach dem Unionsrecht oder dem Recht der Mitgliedstaaten eine Verpflichtung zur Speicherung
     der \hyperref[itm:04-1]{personenbezogenen Daten} besteht;%
    \label{itm:28-3-1g}

    \item dem \hyperref[itm:04-7]{Verantwortlichen} alle erforderlichen Informationen zum Nachweis der Einhaltung der in
     diesem Artikel niedergelegten Pflichten zur Verfügung stellt und Überprüfungen -- einschließlich Inspektionen –,
     die vom \hyperref[itm:04-7]{Verantwortlichen} oder einem anderen von diesem beauftragten Prüfer durchgeführt
     werden, ermöglicht und dazu beiträgt.%
    \label{itm:28-3-1h}

  \end{enumerate}

  \phantomsection Mit Blick auf \hyperref[itm:28-3-1h]{Unterabsatz 1 Buchstabe h} informiert der \hyperref[itm:04-8]
   {Auftragsverarbeiter} den \hyperref[itm:04-7]{Verantwortlichen} unverzüglich, falls er der Auffassung ist, dass eine
   Weisung gegen diese Verordnung oder gegen andere Datenschutzbestimmungen der Union oder der Mitgliedstaaten
   verstößt.%
  \label{itm:28-3-2}

  \item Nimmt der \hyperref[itm:04-8]{Auftragsverarbeiter} die Dienste eines weiteren \hyperref[itm:04-8]
   {Auftragsverarbeiters} in Anspruch, um bestimmte Verarbeitungstätigkeiten im Namen des \hyperref[itm:04-7]
   {Verantwortlichen} auszuführen, so werden diesem weiteren \hyperref[itm:04-8]{Auftragsverarbeiter} im Wege eines
   Vertrags oder eines anderen Rechtsinstruments nach dem Unionsrecht oder dem Recht des betreffenden Mitgliedstaats
   dieselben Datenschutzpflichten auferlegt, die in dem Vertrag oder anderen Rechtsinstrument zwischen dem \hyperref
   [itm:04-7]{Verantwortlichen} und dem \hyperref[itm:04-8]{Auftragsverarbeiter} gemäß \hyperref[itm:28-3-1]{Absatz 3}
   festgelegt sind, wobei insbesondere hinreichende Garantien dafür geboten werden muss, dass die geeigneten
   technischen und organisatorischen Maßnahmen so durchgeführt werden, dass die \hyperref[itm:04-2]
   {Verarbeitung} entsprechend den Anforderungen dieser Verordnung erfolgt. Kommt der weitere \hyperref[itm:04-8]
   {Auftragsverarbeiter} seinen Datenschutzpflichten nicht nach, so haftet der erste
   \hyperref[itm:04-8]{Auftragsverarbeiter} gegenüber dem \hyperref[itm:04-7]{Verantwortlichen} für die Einhaltung der
    Pflichten jenes anderen
   \hyperref[itm:04-8]{Auftragsverarbeiters}.%
  \label{itm:28-4}

  \item Die Einhaltung genehmigter Verhaltensregeln gemäß \hyperref[ch:40]{Artikel 40} oder eines genehmigten
   Zertifizierungsverfahrens gemäß \hyperref[ch:42]{Artikel 42} durch einen \hyperref[itm:04-8]
   {Auftragsverarbeiter} kann als Faktor herangezogen werden, um hinreichende Garantien im Sinne der Absätze \hyperref
   [itm:28-1]{1} und \hyperref[itm:28-4]{4} des vorliegenden Artikels nachzuweisen.%
  \label{itm:28-5}

  \item Unbeschadet eines individuellen Vertrags zwischen dem \hyperref[itm:04-7]{Verantwortlichen} und dem \hyperref
   [itm:04-8]{Auftragsverarbeiter} kann der Vertrag oder das andere Rechtsinstrument im Sinne der Absätze \hyperref
   [itm:28-3-1]{3} und \hyperref[itm:28-4]{4} des vorliegenden Artikels ganz oder teilweise auf den in den
   Absätzen \hyperref[itm:28-7]{7} und \hyperref[itm:28-8]{8} des vorliegenden Artikels genannten
   Standardvertragsklauseln beruhen, auch wenn diese Bestandteil einer dem
   \hyperref[itm:04-7]{Verantwortlichen} oder dem \hyperref[itm:04-8]{Auftragsverarbeiter} gemäß den Artikeln \hyperref
    [ch:42]{42} und \hyperref[ch:43]{43} erteilten Zertifizierung sind.%
  \label{itm:28-6}

  \item Die Kommission kann im Einklang mit dem Prüfverfahren gemäß \hyperref[itm:93-2]{Artikel 93
   Absatz 2} Standardvertragsklauseln zur Regelung der in den Absätzen \hyperref[itm:28-3-1]{3} und \hyperref[itm:28-4]
   {4} des vorliegenden Artikels genannten Fragen festlegen.%
  \label{itm:28-7}

  \item Eine \hyperref[itm:04-21]{Aufsichtsbehörde} kann im Einklang mit dem Kohärenzverfahren gemäß \hyperref[ch:63]
   {Artikel 63} Standardvertragsklauseln zur Regelung der in den Absätzen \hyperref[itm:28-3-1]{3} und \hyperref
   [itm:28-4]{4} des vorliegenden Artikels genannten Fragen festlegen.%
  \label{itm:28-8}

  \item Der Vertrag oder das andere Rechtsinstrument im Sinne der Absätze \hyperref[itm:28-3-1]{3} und \hyperref
   [itm:28-4]{4} ist schriftlich abzufassen, was auch in einem elektronischen Format erfolgen kann.%
  \label{itm:28-9}

  \item Unbeschadet der Artikel \hyperref[ch:82]{82}, \hyperref[ch:83]{83} und \hyperref[ch:84]{84} gilt ein
   \hyperref[itm:04-8]{Auftragsverarbeiter}, der unter Verstoß gegen diese Verordnung die Zwecke und Mittel
    der \hyperref[itm:04-2]{Verarbeitung} bestimmt, in Bezug auf diese \hyperref[itm:04-2]{Verarbeitung} als \hyperref
    [itm:04-7]{Verantwortlicher}.%
  \label{itm:28-10}

\end{enumerate}

% \addsec{Ergänzende Hinweise}

