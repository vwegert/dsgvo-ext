%!TEX root = ../DSGVO-Bearbeitung.tex
\chapter{Auftragsverarbeiter}
\label{ch:28}

\addsec{Text der Verordnung}

\begin{enumerate}

  \item Erfolgt eine Verarbeitung im Auftrag eines Verantwortlichen, so arbeitet dieser nur mit Auftragsverarbeitern,
   die hinreichend Garantien dafür bieten, dass geeignete technische und organisatorische Maßnahmen so durchgeführt
   werden, dass die Verarbeitung im Einklang mit den Anforderungen dieser Verordnung erfolgt und den Schutz der Rechte
   der \hyperref[itm:04-1]{betroffenen Person} gewährleistet.
  \label{itm:28-1}

  \item Der Auftragsverarbeiter nimmt keinen weiteren Auftragsverarbeiter ohne vorherige gesonderte oder allgemeine
   schriftliche Genehmigung des Verantwortlichen in Anspruch. Im Fall einer allgemeinen schriftlichen Genehmigung
   informiert der Auftragsverarbeiter den Verantwortlichen immer über jede beabsichtigte Änderung in Bezug auf die
   Hinzuziehung oder die Ersetzung anderer Auftragsverarbeiter, wodurch der Verantwortliche die Möglichkeit erhält,
   gegen derartige Änderungen Einspruch zu erheben.
  \label{itm:28-2}

  \item Die Verarbeitung durch einen Auftragsverarbeiter erfolgt auf der Grundlage eines Vertrags oder eines anderen
   Rechtsinstruments nach dem Unionsrecht oder dem Recht der Mitgliedstaaten, der bzw. das den Auftragsverarbeiter in
   Bezug auf den Verantwortlichen bindet und in dem Gegenstand und Dauer der Verarbeitung, Art und Zweck der
   Verarbeitung, die Art der \hyperref[itm:04-1]{personenbezogenen Daten}, die Kategorien \hyperref[itm:04-1]{betroffener Personen} und die Pflichten und Rechte
   des Verantwortlichen festgelegt sind. Dieser Vertrag bzw. dieses andere Rechtsinstrument sieht insbesondere vor,
   dass der Auftragsverarbeiter
  \label{itm:28-3-1}

  \begin{enumerate}
  
    \item die \hyperref[itm:04-1]{personenbezogenen Daten} nur auf dokumentierte Weisung des Verantwortlichen -- auch in Bezug auf die
     Übermittlung \hyperref[itm:04-1]{personenbezogener Daten} an ein Drittland oder eine \hyperref[itm:04-29]{internationale Organisation} -- verarbeitet, sofern
     er nicht durch das Recht der Union oder der Mitgliedstaaten, dem der Auftragsverarbeiter unterliegt, hierzu
     verpflichtet ist; in einem solchen Fall teilt der Auftragsverarbeiter dem Verantwortlichen diese rechtlichen
     Anforderungen vor der Verarbeitung mit, sofern das betreffende Recht eine solche Mitteilung nicht wegen eines
     wichtigen öffentlichen Interesses verbietet;
    \label{itm:28-3-1a}

    \item gewährleistet, dass sich die zur Verarbeitung der \hyperref[itm:04-1]{personenbezogenen Daten} befugten Personen zur
     Vertraulichkeit verpflichtet haben oder einer angemessenen gesetzlichen Verschwiegenheitspflicht unterliegen;
    \label{itm:28-3-1b}

    \item alle gemäß \hyperref[ch:32]{Artikel 32} erforderlichen Maßnahmen ergreift;
    \label{itm:28-3-1c}

    \item die in den Absätzen \hyperref[itm:28-2]{2} und \hyperref[itm:28-4]{4} genannten Bedingungen für die
     Inanspruchnahme der Dienste eines weiteren Auftragsverarbeiters einhält;
    \label{itm:28-3-1d}

    \item angesichts der Art der Verarbeitung den Verantwortlichen nach Möglichkeit mit geeigneten technischen und
     organisatorischen Maßnahmen dabei unterstützt, seiner Pflicht zur Beantwortung von Anträgen auf Wahrnehmung der in
     \hyperref[part:3]{Kapitel III} genannten Rechte der \hyperref[itm:04-1]{betroffenen Person} nachzukommen;
    \label{itm:28-3-1e}

    \item unter Berücksichtigung der Art der Verarbeitung und der ihm zur Verfügung stehenden Informationen den
     Verantwortlichen bei der Einhaltung der in den Artikeln \hyperref[ch:32]{32} bis \hyperref[ch:36]{36} genannten
     Pflichten unterstützt;
    \label{itm:28-3-1f}

    \item nach Abschluss der Erbringung der Verarbeitungsleistungen alle \hyperref[itm:04-1]{personenbezogenen Daten} nach Wahl des
     Verantwortlichen entweder löscht oder zurückgibt, sofern nicht nach dem Unionsrecht oder dem Recht der
     Mitgliedstaaten eine Verpflichtung zur Speicherung der \hyperref[itm:04-1]{personenbezogenen Daten} besteht;
    \label{itm:28-3-1g}

    \item dem Verantwortlichen alle erforderlichen Informationen zum Nachweis der Einhaltung der in diesem Artikel
     niedergelegten Pflichten zur Verfügung stellt und Überprüfungen -- einschließlich Inspektionen –, die vom
     Verantwortlichen oder einem anderen von diesem beauftragten Prüfer durchgeführt werden, ermöglicht und dazu
     beiträgt.
    \label{itm:28-3-1h}

  \end{enumerate}

  \phantomsection
  Mit Blick auf \hyperref[itm:28-3-1h]{Unterabsatz 1 Buchstabe h} informiert der Auftragsverarbeiter den
  Verantwortlichen
  unverzüglich, falls er der Auffassung ist, dass eine Weisung gegen diese Verordnung oder gegen andere
  Datenschutzbestimmungen der Union oder der Mitgliedstaaten verstößt.
  \label{itm:28-3-2}

  \item Nimmt der Auftragsverarbeiter die Dienste eines weiteren Auftragsverarbeiters in Anspruch, um bestimmte
   Verarbeitungstätigkeiten im Namen des Verantwortlichen auszuführen, so werden diesem weiteren Auftragsverarbeiter im
   Wege eines Vertrags oder eines anderen Rechtsinstruments nach dem Unionsrecht oder dem Recht des betreffenden
   Mitgliedstaats dieselben Datenschutzpflichten auferlegt, die in dem Vertrag oder anderen Rechtsinstrument zwischen
   dem Verantwortlichen und dem Auftragsverarbeiter gemäß \hyperref[itm:28-3-1]{Absatz 3} festgelegt sind, wobei
   insbesondere hinreichende Garantien dafür geboten werden muss, dass die geeigneten technischen und organisatorischen
   Maßnahmen so durchgeführt werden, dass die Verarbeitung entsprechend den Anforderungen dieser Verordnung erfolgt.
   Kommt der weitere Auftragsverarbeiter seinen Datenschutzpflichten nicht nach, so haftet der erste
   Auftragsverarbeiter gegenüber dem Verantwortlichen für die Einhaltung der Pflichten jenes anderen
   Auftragsverarbeiters.
  \label{itm:28-4}

  \item Die Einhaltung genehmigter Verhaltensregeln gemäß \hyperref[ch:40]{Artikel 40} oder eines genehmigten
   Zertifizierungsverfahrens gemäß \hyperref[ch:42]{Artikel 42} durch einen Auftragsverarbeiter kann als Faktor
   herangezogen werden, um hinreichende Garantien im Sinne der Absätze \hyperref[itm:28-1]{1} und \hyperref[itm:28-4]
   {4} des vorliegenden Artikels nachzuweisen.
  \label{itm:28-5}

  \item Unbeschadet eines individuellen Vertrags zwischen dem Verantwortlichen und dem Auftragsverarbeiter kann der
   Vertrag oder das andere Rechtsinstrument im Sinne der Absätze \hyperref[itm:28-3-1]{3} und \hyperref[itm:28-4]{4} des
   vorliegenden Artikels ganz oder teilweise auf den in den Absätzen \hyperref[itm:28-7]{7} und \hyperref[itm:28-8]
   {8} des vorliegenden Artikels genannten Standardvertragsklauseln beruhen, auch wenn diese Bestandteil einer dem
   Verantwortlichen oder dem Auftragsverarbeiter gemäß den Artikeln \hyperref[ch:42]{42} und \hyperref[ch:43]{43}
   erteilten Zertifizierung sind.
  \label{itm:28-6}

  \item Die Kommission kann im Einklang mit dem Prüfverfahren gemäß \sout{Artikel 87} \hyperref[itm:93-2]{Artikel 93
   Absatz 2} Standardvertragsklauseln zur Regelung der in den Absätzen \hyperref[itm:28-3-1]{3} und \hyperref[itm:28-4]
   {4} des vorliegenden Artikels genannten Fragen festlegen.
  \label{itm:28-7}

  \item Eine Aufsichtsbehörde kann im Einklang mit dem Kohärenzverfahren gemäß \hyperref[ch:63]{Artikel 63}
   Standardvertragsklauseln zur Regelung der in den Absätzen \hyperref[itm:28-3-1]{3} und \hyperref[itm:28-4]{4} des
   vorliegenden Artikels genannten Fragen festlegen.
  \label{itm:28-8}

  \item Der Vertrag oder das andere Rechtsinstrument im Sinne der Absätze \hyperref[itm:28-3-1]{3} und \hyperref
  [itm:28-4]
   {4} ist schriftlich abzufassen, was auch in einem elektronischen Format erfolgen kann.
  \label{itm:28-9}

  \item Unbeschadet der Artikel \hyperref[ch:82]{82}, \hyperref[ch:83]{83} und \hyperref[ch:84]{84} gilt ein
   Auftragsverarbeiter, der unter Verstoß gegen diese Verordnung die Zwecke und Mittel der Verarbeitung bestimmt, in
   Bezug auf diese Verarbeitung als Verantwortlicher.
  \label{itm:28-10}

\end{enumerate}

\addsec{Eigene Notizen}

\begin{itemize}

  \item Der deutsche Originaltext verweist in \hyperref[itm:28-7]{Absatz 7} auf Artikel 87 Absatz 2. Dieser Artikel hat
  allerdings keine Absatzunterteilung. Der englische und französische Text verweisen beide auf Artikel 93, was im Text
  oben korrigiert wurde.

\end{itemize}

