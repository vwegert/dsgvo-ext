%!TEX root = ../DSGVO-Bearbeitung.tex
\chapter{Zusammenarbeit zwischen der federführenden Aufsichtsbehörde und den anderen betroffenen Aufsichtsbehörden}
\label{ch:60}

\addsec{Text der Verordnung}

\begin{enumerate}

  \item Die federführende Aufsichtsbehörde arbeitet mit den anderen betroffenen Aufsichtsbehörden im Einklang mit diesem
   Artikel zusammen und bemüht sich dabei, einen Konsens zu erzielen. Die federführende Aufsichtsbehörde und die
   betroffenen Aufsichtsbehörden tauschen untereinander alle zweckdienlichen Informationen aus.
  \label{itm:60-1}

  \item Die federführende Aufsichtsbehörde kann jederzeit andere betroffene Aufsichtsbehörden um Amtshilfe gemäß
  \hyperref[ch:61]{Artikel 61} ersuchen und gemeinsame Maßnahmen gemäß \hyperref[ch:62]{Artikel 62} durchführen,
   insbesondere zur Durchführung von Untersuchungen oder zur Überwachung der Umsetzung einer Maßnahme in Bezug auf
   einen Verantwortlichen oder einen Auftragsverarbeiter, der in einem anderen Mitgliedstaat niedergelassen ist.
  \label{itm:60-2}

  \item Die federführende Aufsichtsbehörde übermittelt den anderen betroffenen Aufsichtsbehörden unverzüglich die
   zweckdienlichen Informationen zu der Angelegenheit. Sie legt den anderen betroffenen Aufsichtsbehörden unverzüglich
   einen Beschlussentwurf zur Stellungnahme vor und trägt deren Standpunkten gebührend Rechnung.
  \label{itm:60-3}

  \item Legt eine der anderen betroffenen Aufsichtsbehörden innerhalb von vier Wochen, nachdem sie gemäß \hyperref
   [itm:60-3]{Absatz 3 des vorliegenden Artikels} konsultiert wurde, gegen diesen Beschlussentwurf einen maßgeblichen
   und begründeten Einspruch ein und schließt sich die federführende Aufsichtsbehörde dem maßgeblichen und begründeten
   Einspruch nicht an oder ist der Ansicht, dass der Einspruch nicht maßgeblich oder nicht begründet ist, so leitet die
   federführende Aufsichtsbehörde das Kohärenzverfahren gemäß \hyperref[ch:63]{Artikel 63} für die Angelegenheit ein.
  \label{itm:60-4}

  \item Beabsichtigt die federführende Aufsichtsbehörde, sich dem maßgeblichen und begründeten Einspruch anzuschließen,
   so legt sie den anderen betroffenen Aufsichtsbehörden einen überarbeiteten Beschlussentwurf zur Stellungnahme vor.
   Der überarbeitete Beschlussentwurf wird innerhalb von zwei Wochen dem Verfahren nach \hyperref[itm:60-4]{Absatz 4}
   unterzogen.
  \label{itm:60-5}

  \item Legt keine der anderen betroffenen Aufsichtsbehörden Einspruch gegen den Beschlussentwurf ein, der von der
   federführenden Aufsichtsbehörde innerhalb der in den Absätzen \hyperref[itm:60-4]{4} und \hyperref[itm:60-5]
   {5} festgelegten Frist vorgelegt wurde, so gelten die federführende Aufsichtsbehörde und die betroffenen
   Aufsichtsbehörden als mit dem Beschlussentwurf einverstanden und sind an ihn gebunden.
  \label{itm:60-6}

  \item Die federführende Aufsichtsbehörde erlässt den Beschluss und teilt ihn der Hauptniederlassung oder der einzigen
   Niederlassung des Verantwortlichen oder gegebenenfalls des Auftragsverarbeiters mit und setzt die anderen
   betroffenen Aufsichtsbehörden und den Ausschuss von dem betreffenden Beschluss einschließlich einer Zusammenfassung
   der maßgeblichen Fakten und Gründe in Kenntnis. Die Aufsichtsbehörde, bei der eine Beschwerde eingereicht worden
   ist, unterrichtet den Beschwerdeführer über den Beschluss.
  \label{itm:60-7}

  \item Wird eine Beschwerde abgelehnt oder abgewiesen, so erlässt die Aufsichtsbehörde, bei der die Beschwerde
   eingereicht wurde, abweichend von \hyperref[itm:60-7]{Absatz 7} den Beschluss, teilt ihn dem Beschwerdeführer mit
   und setzt den Verantwortlichen in Kenntnis.
  \label{itm:60-8}

  \item Sind sich die federführende Aufsichtsbehörde und die betreffenden Aufsichtsbehörden darüber einig, Teile der
   Beschwerde abzulehnen oder abzuweisen und bezüglich anderer Teile dieser Beschwerde tätig zu werden, so wird in
   dieser Angelegenheit für jeden dieser Teile ein eigener Beschluss erlassen. Die federführende Aufsichtsbehörde
   erlässt den Beschluss für den Teil, der das Tätigwerden in Bezug auf den Verantwortlichen betrifft, teilt ihn der
   Hauptniederlassung oder einzigen Niederlassung des Verantwortlichen oder des Auftragsverarbeiters im Hoheitsgebiet
   ihres Mitgliedstaats mit und setzt den Beschwerdeführer hiervon in Kenntnis, während die für den Beschwerdeführer
   zuständige Aufsichtsbehörde den Beschluss für den Teil erlässt, der die Ablehnung oder Abweisung dieser Beschwerde
   betrifft, und ihn diesem Beschwerdeführer mitteilt und den Verantwortlichen oder den Auftragsverarbeiter hiervon in
   Kenntnis setzt.
  \label{itm:60-9}

  \item Nach der Unterrichtung über den Beschluss der federführenden Aufsichtsbehörde gemäß den Absätzen \hyperref
   [itm:60-7]{7} und \hyperref[itm:60-9]{9} ergreift der Verantwortliche oder der Auftragsverarbeiter die
   erforderlichen Maßnahmen, um die Verarbeitungstätigkeiten all seiner Niederlassungen in der Union mit dem Beschluss
   in Einklang zu bringen. Der Verantwortliche oder der Auftragsverarbeiter teilt der federführenden Aufsichtsbehörde
   die Maßnahmen mit, die zur Einhaltung des Beschlusses ergriffen wurden; diese wiederum unterrichtet die anderen
   betroffenen Aufsichtsbehörden.
  \label{itm:60-10}

  \item Hat -- in Ausnahmefällen -- eine betroffene Aufsichtsbehörde Grund zu der Annahme, dass zum Schutz der
   Interessen \hyperref[itm:04-1]{betroffener Personen} dringender Handlungsbedarf besteht, so kommt das Dringlichkeitsverfahren nach
   \hyperref[ch:66]{Artikel 66} zur Anwendung.
  \label{itm:60-11}

  \item Die federführende Aufsichtsbehörde und die anderen betroffenen Aufsichtsbehörden übermitteln einander die nach
   diesem Artikel geforderten Informationen auf elektronischem Wege unter Verwendung eines standardisierten Formats.
  \label{itm:60-12}

\end{enumerate}

\addsec{Eigene Notizen}

