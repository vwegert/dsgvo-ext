%!TEX root = ../DSGVO-Bearbeitung.tex
\chapter{Zusammenarbeit zwischen der federführenden Aufsichtsbehörde und den anderen betroffenen Aufsichtsbehörden}
\label{ch:60}

\addsec{Text der Verordnung}

\begin{enumerate}

  \item Die federführende \hyperref[itm:04-21]{Aufsichtsbehörde} arbeitet mit den anderen \hyperref[itm:04-22]
   {betroffenen Aufsichtsbehörden} im Einklang mit diesem Artikel zusammen und bemüht sich dabei, einen Konsens zu
   erzielen. Die federführende \hyperref[itm:04-21]{Aufsichtsbehörde} und die
   \hyperref[itm:04-22]{betroffenen Aufsichtsbehörden} tauschen untereinander alle zweckdienlichen Informationen aus.
  \label{itm:60-1}

  \item Die federführende \hyperref[itm:04-21]{Aufsichtsbehörde} kann jederzeit andere \hyperref[itm:04-22]
   {betroffene Aufsichtsbehörden} um Amtshilfe gemäß
  \hyperref[ch:61]{Artikel 61} ersuchen und gemeinsame Maßnahmen gemäß \hyperref[ch:62]{Artikel 62} durchführen,
   insbesondere zur Durchführung von Untersuchungen oder zur Überwachung der Umsetzung einer Maßnahme in Bezug auf
   einen \hyperref[itm:04-7]{Verantwortlichen} oder einen \hyperref[itm:04-8]{Auftragsverarbeiter}, der in einem
   anderen Mitgliedstaat niedergelassen ist.
  \label{itm:60-2}

  \item Die federführende \hyperref[itm:04-21]{Aufsichtsbehörde} übermittelt den anderen \hyperref[itm:04-22]
   {betroffenen Aufsichtsbehörden} unverzüglich die zweckdienlichen Informationen zu der Angelegenheit. Sie legt den
   anderen \hyperref[itm:04-22]{betroffenen Aufsichtsbehörden} unverzüglich einen Beschlussentwurf zur Stellungnahme
   vor und trägt deren Standpunkten gebührend Rechnung.
  \label{itm:60-3}

  \item Legt eine der anderen \hyperref[itm:04-22]{betroffenen Aufsichtsbehörden} innerhalb von vier Wochen, nachdem sie
   gemäß \hyperref[itm:60-3]{Absatz 3 des vorliegenden Artikels} konsultiert wurde, gegen diesen Beschlussentwurf
   einen \hyperref[itm:04-27]{maßgeblichen und begründeten Einspruch} ein und schließt sich die federführende \hyperref
   [itm:04-21]{Aufsichtsbehörde} dem \hyperref[itm:04-27]{maßgeblichen und begründeten Einspruch} nicht an oder ist der
   Ansicht, dass der Einspruch nicht maßgeblich oder nicht begründet ist, so leitet die federführende \hyperref
   [itm:04-21]{Aufsichtsbehörde} das Kohärenzverfahren gemäß \hyperref[ch:63]{Artikel 63} für die Angelegenheit ein.
  \label{itm:60-4}

  \item Beabsichtigt die federführende \hyperref[itm:04-21]{Aufsichtsbehörde}, sich dem \hyperref[itm:04-27]
   {maßgeblichen und begründeten Einspruch} anzuschließen, so legt sie den anderen \hyperref[itm:04-22]
   {betroffenen Aufsichtsbehörden} einen überarbeiteten Beschlussentwurf zur Stellungnahme vor. Der überarbeitete
   Beschlussentwurf wird innerhalb von zwei Wochen dem Verfahren nach \hyperref[itm:60-4]{Absatz 4} unterzogen.
  \label{itm:60-5}

  \item Legt keine der anderen \hyperref[itm:04-22]{betroffenen Aufsichtsbehörden} Einspruch gegen den Beschlussentwurf
   ein, der von der federführenden \hyperref[itm:04-21]{Aufsichtsbehörde} innerhalb der in den Absätzen \hyperref
   [itm:60-4]{4} und \hyperref[itm:60-5]{5} festgelegten Frist vorgelegt wurde, so gelten die federführende \hyperref
   [itm:04-21]{Aufsichtsbehörde} und die \hyperref[itm:04-22]{betroffenen Aufsichtsbehörden} als mit dem
   Beschlussentwurf einverstanden und sind an ihn gebunden.
  \label{itm:60-6}

  \item Die federführende \hyperref[itm:04-21]{Aufsichtsbehörde} erlässt den Beschluss und teilt ihn der \hyperref
   [itm:04-16]{Hauptniederlassung} oder der einzigen Niederlassung des \hyperref[itm:04-7]{Verantwortlichen} oder
   gegebenenfalls des \hyperref[itm:04-8]{Auftragsverarbeiters} mit und setzt die anderen
   \hyperref[itm:04-22]{betroffenen Aufsichtsbehörden} und den Ausschuss von dem betreffenden Beschluss einschließlich
    einer Zusammenfassung der maßgeblichen Fakten und Gründe in Kenntnis. Die \hyperref[itm:04-21]
    {Aufsichtsbehörde}, bei der eine Beschwerde eingereicht worden ist, unterrichtet den Beschwerdeführer über den
    Beschluss.
  \label{itm:60-7}

  \item Wird eine Beschwerde abgelehnt oder abgewiesen, so erlässt die \hyperref[itm:04-21]{Aufsichtsbehörde}, bei der
   die Beschwerde eingereicht wurde, abweichend von \hyperref[itm:60-7]{Absatz 7} den Beschluss, teilt ihn dem
   Beschwerdeführer mit und setzt den \hyperref[itm:04-7]{Verantwortlichen} in Kenntnis.
  \label{itm:60-8}

  \item Sind sich die federführende \hyperref[itm:04-21]{Aufsichtsbehörde} und die betreffenden \hyperref[itm:04-21]
   {Aufsichtsbehörden} darüber einig, Teile der Beschwerde abzulehnen oder abzuweisen und bezüglich anderer Teile
   dieser Beschwerde tätig zu werden, so wird in dieser Angelegenheit für jeden dieser Teile ein eigener Beschluss
   erlassen. Die federführende \hyperref[itm:04-21]{Aufsichtsbehörde} erlässt den Beschluss für den Teil, der das
   Tätigwerden in Bezug auf den \hyperref[itm:04-7]{Verantwortlichen} betrifft, teilt ihn der
   \hyperref[itm:04-16]{Hauptniederlassung} oder einzigen Niederlassung des \hyperref[itm:04-7]{Verantwortlichen} oder
    des \hyperref[itm:04-8]{Auftragsverarbeiters} im Hoheitsgebiet ihres Mitgliedstaats mit und setzt den
    Beschwerdeführer hiervon in Kenntnis, während die für den Beschwerdeführer zuständige \hyperref[itm:04-21]
    {Aufsichtsbehörde} den Beschluss für den Teil erlässt, der die Ablehnung oder Abweisung dieser Beschwerde betrifft,
    und ihn diesem Beschwerdeführer mitteilt und den \hyperref[itm:04-7]{Verantwortlichen} oder den \hyperref[itm:04-8]
    {Auftragsverarbeiter} hiervon in Kenntnis setzt.
  \label{itm:60-9}

  \item Nach der Unterrichtung über den Beschluss der federführenden \hyperref[itm:04-21]{Aufsichtsbehörde} gemäß den
   Absätzen \hyperref[itm:60-7]{7} und \hyperref[itm:60-9]{9} ergreift der \hyperref[itm:04-7]{Verantwortliche} oder
   der \hyperref[itm:04-8]{Auftragsverarbeiter} die erforderlichen Maßnahmen, um die Verarbeitungstätigkeiten all
   seiner Niederlassungen in der Union mit dem Beschluss in Einklang zu bringen. Der \hyperref[itm:04-7]
   {Verantwortliche} oder der \hyperref[itm:04-8]{Auftragsverarbeiter} teilt der federführenden \hyperref[itm:04-21]
   {Aufsichtsbehörde} die Maßnahmen mit, die zur Einhaltung des Beschlusses ergriffen wurden; diese wiederum
   unterrichtet die anderen
   \hyperref[itm:04-22]{betroffenen Aufsichtsbehörden}.
  \label{itm:60-10}

  \item Hat -- in Ausnahmefällen -- eine \hyperref[itm:04-22]{betroffene Aufsichtsbehörde} Grund zu der Annahme, dass
   zum Schutz der Interessen \hyperref[itm:04-1]{betroffener Personen} dringender Handlungsbedarf besteht, so kommt das
   Dringlichkeitsverfahren nach
   \hyperref[ch:66]{Artikel 66} zur Anwendung.
  \label{itm:60-11}

  \item Die federführende \hyperref[itm:04-21]{Aufsichtsbehörde} und die anderen \hyperref[itm:04-22]
   {betroffenen Aufsichtsbehörden} übermitteln einander die nach diesem Artikel geforderten Informationen auf
   elektronischem Wege unter Verwendung eines standardisierten Formats.
  \label{itm:60-12}

\end{enumerate}

\addsec{Eigene Notizen}

