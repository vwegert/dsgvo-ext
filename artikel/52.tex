%!TEX root = ../DSGVO-Bearbeitung.tex
\chapter{Unabhängigkeit}
\label{ch:52}

\addsec{Text der Verordnung}

\begin{enumerate}

  \item Jede \hyperref[itm:04-21]{Aufsichtsbehörde} handelt bei der Erfüllung ihrer Aufgaben und bei der Ausübung ihrer Befugnisse gemäß
   dieser Verordnung völlig unabhängig.
  \label{itm:52-1}

  \item Das Mitglied oder die Mitglieder jeder \hyperref[itm:04-21]{Aufsichtsbehörde} unterliegen bei der Erfüllung ihrer Aufgaben und der
   Ausübung ihrer Befugnisse gemäß dieser Verordnung weder direkter noch indirekter Beeinflussung von außen und
   ersuchen weder um Weisung noch nehmen sie Weisungen entgegen.
  \label{itm:52-2}

  \item Das Mitglied oder die Mitglieder der \hyperref[itm:04-21]{Aufsichtsbehörde} sehen von allen mit den Aufgaben ihres Amtes nicht zu
   vereinbarenden Handlungen ab und üben während ihrer Amtszeit keine andere mit ihrem Amt nicht zu vereinbarende
   entgeltliche oder unentgeltliche Tätigkeit aus.
  \label{itm:52-3}

  \item Jeder Mitgliedstaat stellt sicher, dass jede \hyperref[itm:04-21]{Aufsichtsbehörde} mit den personellen, technischen und finanziellen
   Ressourcen, Räumlichkeiten und Infrastrukturen ausgestattet wird, die sie benötigt, um ihre Aufgaben und Befugnisse
   auch im Rahmen der Amtshilfe, Zusammenarbeit und Mitwirkung im Ausschuss effektiv wahrnehmen zu können.
  \label{itm:52-4}

  \item Jeder Mitgliedstaat stellt sicher, dass jede \hyperref[itm:04-21]{Aufsichtsbehörde} ihr eigenes Personal auswählt und hat, das
   ausschließlich der Leitung des Mitglieds oder der Mitglieder der betreffenden \hyperref[itm:04-21]{Aufsichtsbehörde} untersteht.
  \label{itm:52-5}

  \item Jeder Mitgliedstaat stellt sicher, dass jede \hyperref[itm:04-21]{Aufsichtsbehörde} einer Finanzkontrolle unterliegt, die ihre
   Unabhängigkeit nicht beeinträchtigt und dass sie über eigene, öffentliche, jährliche Haushaltspläne verfügt, die
   Teil des gesamten Staatshaushalts oder nationalen Haushalts sein können.
  \label{itm:52-6}

\end{enumerate}

\addsec{Eigene Notizen}

