%!TEX root = ../DSGVO-ExtendedVersion.tex
\chapter{Aufgaben}
\label{ch:57}

\crossrefArticleToReason{57}

\begin{enumerate}

  \item Unbeschadet anderer in dieser Verordnung dargelegter Aufgaben muss jede \hyperref[itm:04-21]
   {Aufsichtsbehörde} in ihrem Hoheitsgebiet
\label{itm:57-1}

  \begin{enumerate}
  
    \item die Anwendung dieser Verordnung überwachen und durchsetzen;%
    \label{itm:57-1a}

    \item die Öffentlichkeit für die Risiken, Vorschriften, Garantien und Rechte im Zusammenhang mit der \hyperref
     [itm:04-2]{Verarbeitung} sensibilisieren und sie darüber aufklären. Besondere Beachtung finden dabei spezifische
     Maßnahmen für Kinder;%
    \label{itm:57-1b}

    \item im Einklang mit dem Recht des Mitgliedsstaats das nationale Parlament, die Regierung und andere Einrichtungen
     und Gremien über legislative und administrative Maßnahmen zum Schutz der Rechte und Freiheiten natürlicher
     Personen in Bezug auf die \hyperref[itm:04-2]{Verarbeitung} beraten;%
    \label{itm:57-1c}

    \item die \hyperref[itm:04-7]{Verantwortlichen} und die \hyperref[itm:04-8]{Auftragsverarbeiter} für die ihnen aus
     dieser Verordnung entstehenden Pflichten sensibilisieren;%
    \label{itm:57-1d}

    \item auf Anfrage jeder \hyperref[itm:04-1]{betroffenen Person} Informationen über die Ausübung ihrer Rechte
     aufgrund dieser Verordnung zur Verfügung stellen und gegebenenfalls zu diesem Zweck mit den \hyperref[itm:04-21]
     {Aufsichtsbehörden} in anderen Mitgliedstaaten zusammenarbeiten;%
    \label{itm:57-1e}

    \item sich mit Beschwerden einer \hyperref[itm:04-1]{betroffenen Person} oder Beschwerden einer Stelle, einer
     Organisation oder eines Verbandes gemäß \hyperref[ch:80]{Artikel 80} befassen, den Gegenstand der Beschwerde in
     angemessenem Umfang untersuchen und den Beschwerdeführer innerhalb einer angemessenen Frist über den Fortgang und
     das Ergebnis der Untersuchung unterrichten, insbesondere, wenn eine weitere Untersuchung oder Koordinierung mit
     einer anderen
     \hyperref[itm:04-21]{Aufsichtsbehörde} notwendig ist;%
    \label{itm:57-1f}

    \item mit anderen \hyperref[itm:04-21]{Aufsichtsbehörden} zusammenarbeiten, auch durch Informationsaustausch, und
     ihnen Amtshilfe leisten, um die einheitliche Anwendung und Durchsetzung dieser Verordnung zu gewährleisten;%
    \label{itm:57-1g}

    \item Untersuchungen über die Anwendung dieser Verordnung durchführen, auch auf der Grundlage von Informationen
     einer anderen \hyperref[itm:04-21]{Aufsichtsbehörde} oder einer anderen Behörde;%
    \label{itm:57-1h}

    \item maßgebliche Entwicklungen verfolgen, soweit sie sich auf den Schutz \hyperref[itm:04-1]
     {personenbezogener Daten} auswirken, insbesondere die Entwicklung der Informations- und Kommunikationstechnologie
     und der Geschäftspraktiken;%
    \label{itm:57-1i}

    \item Standardvertragsklauseln im Sinne des \hyperref[itm:28-8]{Artikels 28 Absatz 8} und des \hyperref[itm:46-2d]
     {Artikels 46 Absatz 2 Buchstabe d} festlegen;%
    \label{itm:57-1j}

    \item eine Liste der Verarbeitungsarten erstellen und führen, für die gemäß \hyperref[itm:35-4]{Artikel 35 Absatz 4}
     eine Datenschutz"=Folgenabschätzung durchzuführen ist;%
    \label{itm:57-1k}

    \item Beratung in Bezug auf die in \hyperref[itm:36-1]{Artikel 36 Absatz 2} genannten Verarbeitungsvorgänge
     leisten;%
    \label{itm:57-1l}

    \item die Ausarbeitung von Verhaltensregeln gemäß \hyperref[itm:40-1]{Artikel 40 Absatz 1} fördern und zu diesen
     Verhaltensregeln, die ausreichende Garantien im Sinne des \hyperref[itm:40-5]{Artikels 40 Absatz 5} bieten müssen,
     Stellungnahmen abgeben und sie billigen;%
    \label{itm:57-1m}

    \item die Einführung von Datenschutzzertifizierungsmechanismen und von Datenschutzsiegeln und -prüfzeichen nach
     \hyperref[itm:42-1]{Artikel 42 Absatz 1} anregen und Zertifizierungskriterien nach \hyperref[itm:42-5]{Artikel 42
      Absatz 5} billigen;%
    \label{itm:57-1n}

    \item gegebenenfalls die nach \hyperref[itm:42-7]{Artikel 42 Absatz 7} erteilten Zertifizierungen regelmäßig
     überprüfen;%
    \label{itm:57-1o}

    \item die Anforderungen für die Akkreditierung einer Stelle für die Überwachung der Einhaltung der Verhaltensregeln
     gemäß \hyperref[ch:41]{Artikel 41} und einer Zertifizierungsstelle gemäß \hyperref[ch:43]{Artikel 43} abfassen und
     veröffentlichen;%
    \label{itm:57-1p}

    \item die Akkreditierung einer Stelle für die Überwachung der Einhaltung der Verhaltensregeln gemäß \hyperref[ch:41]
     {Artikel 41} und einer Zertifizierungsstelle gemäß \hyperref[ch:43]{Artikel 43} vornehmen;%
    \label{itm:57-1q}

    \item Vertragsklauseln und Bestimmungen im Sinne des \hyperref[itm:46-3]{Artikels 46 Absatz 3} genehmigen;%
    \label{itm:57-1r}

    \item verbindliche interne Vorschriften gemäß \hyperref[ch:47]{Artikel 47} genehmigen;%
    \label{itm:57-1s}

    \item Beiträge zur Tätigkeit des Ausschusses leisten;%
    \label{itm:57-1t}

    \item interne Verzeichnisse über Verstöße gegen diese Verordnung und gemäß \hyperref[itm:58-2]{Artikel 58 Absatz 2}
     ergriffene Maßnahmen und%
    \label{itm:57-1u}

    \item jede sonstige Aufgabe im Zusammenhang mit dem Schutz \hyperref[itm:04-1]{personenbezogener Daten} erfüllen.%
    \label{itm:57-1v}

  \end{enumerate}

  \item Jede \hyperref[itm:04-21]{Aufsichtsbehörde} erleichtert das Einreichen von in \hyperref[itm:57-1f]{Absatz 1
   Buchstabe f} genannten Beschwerden durch Maßnahmen wie etwa die Bereitstellung eines Beschwerdeformulars, das auch
   elektronisch ausgefüllt werden kann, ohne dass andere Kommunikationsmittel ausgeschlossen werden.%
  \label{itm:57-2}

  \item Die Erfüllung der Aufgaben jeder \hyperref[itm:04-21]{Aufsichtsbehörde} ist für die \hyperref[itm:04-1]
   {betroffene Person} und gegebenenfalls für den Datenschutzbeauftragten unentgeltlich.%
  \label{itm:57-3}

  \item Bei offenkundig unbegründeten oder -- insbesondere im Fall von häufiger Wiederholung -- exzessiven Anfragen kann
   die \hyperref[itm:04-21]{Aufsichtsbehörde} eine angemessene Gebühr auf der Grundlage der Verwaltungskosten verlangen
   oder sich weigern, aufgrund der Anfrage tätig zu werden. In diesem Fall trägt die \hyperref[itm:04-21]
   {Aufsichtsbehörde} die Beweislast für den offenkundig unbegründeten oder exzessiven Charakter der Anfrage.%
  \label{itm:57-4}

\end{enumerate}

% \addsec{Ergänzende Hinweise}

