%!TEX root = ../DSGVO-Bearbeitung.tex
\chapter{Benachrichtigung der von einer Verletzung des Schutzes personenbezogener Daten betroffenen Person}
\label{ch:34}

\addsec{Text der Verordnung}

\begin{enumerate}

  \item Hat die \hyperref[itm:04-12]{Verletzung des Schutzes personenbezogener Daten} voraussichtlich ein hohes Risiko
   für die persönlichen Rechte und Freiheiten natürlicher Personen zur Folge, so benachrichtigt der \hyperref[itm:04-7]
   {Verantwortliche} die \hyperref[itm:04-1]{betroffene Person} unverzüglich von der Verletzung.%
  \label{itm:34-1}

  \item Die in \hyperref[itm:34-1]{Absatz 1} genannte Benachrichtigung der \hyperref[itm:04-1]{betroffenen Person}
   beschreibt in klarer und einfacher Sprache die Art der \hyperref[itm:04-12]{Verletzung des Schutzes
   personenbezogener Daten} und enthält zumindest die in
   \hyperref[itm:33-3]{Artikel 33 Absatz 3} Buchstaben \hyperref[itm:33-3b]{b}, \hyperref[itm:33-3c]{c} und \hyperref
    [itm:33-3d]{d} genannten Informationen und Maßnahmen.%
  \label{itm:34-2}

  \item Die Benachrichtigung der \hyperref[itm:04-1]{betroffenen Person} gemäß \hyperref[itm:34-1]{Absatz 1} ist nicht
   erforderlich, wenn eine der folgenden Bedingungen erfüllt ist:%
  \label{itm:34-3}

  \begin{enumerate}
  
    \item der \hyperref[itm:04-7]{Verantwortliche} hat geeignete technische und organisatorische Sicherheitsvorkehrungen
     getroffen hat und diese Vorkehrungen wurden auf die von der Verletzung betroffenen \hyperref[itm:04-1]
     {personenbezogenen Daten} angewandt, insbesondere solche, durch die die \hyperref[itm:04-1]
     {personenbezogenen Daten} für alle Personen, die nicht zum Zugang zu den \hyperref[itm:04-1]
     {personenbezogenen Daten} befugt sind, unzugänglich gemacht werden, etwa durch Verschlüsselung;%
    \label{itm:34-3a}

    \item der \hyperref[itm:04-7]{Verantwortliche} hat durch nachfolgende Maßnahmen sichergestellt hat, dass das hohe
     Risiko für die Rechte und Freiheiten der \hyperref[itm:04-1]{betroffenen Personen} gemäß \hyperref[itm:34-1]
     {Absatz 1} aller Wahrscheinlichkeit nach nicht mehr besteht;%
    \label{itm:34-3b}

    \item die Benachrichtigung wäre mit einem unverhältnismäßigen Aufwand verbunden. In diesem Fall hat stattdessen eine
     öffentliche Bekanntmachung oder eine ähnliche Maßnahme zu erfolgen, durch die die \hyperref[itm:04-1]
     {betroffenen Personen} vergleichbar wirksam informiert werden.%
    \label{itm:34-3c}

  \end{enumerate}

  \item Wenn der \hyperref[itm:04-7]{Verantwortliche} die \hyperref[itm:04-1]{betroffene Person} nicht bereits über
   die \hyperref[itm:04-12]{Verletzung des Schutzes personenbezogener Daten} benachrichtigt hat, kann die \hyperref
   [itm:04-21]{Aufsichtsbehörde} unter Berücksichtigung der Wahrscheinlichkeit, mit der die
   \hyperref[itm:04-12]{Verletzung des Schutzes personenbezogener Daten} zu einem hohen Risiko führt, von dem \hyperref
    [itm:04-7]{Verantwortlichen} verlangen, dies nachzuholen, oder sie kann mit einem Beschluss feststellen, dass
    bestimmte der in \hyperref[itm:34-3]{Absatz 3} genannten Voraussetzungen erfüllt sind.%
  \label{itm:34-4}

\end{enumerate}

\addsec{Eigene Notizen}

