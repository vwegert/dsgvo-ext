%!TEX root = ../DSGVO-Bearbeitung.tex
\chapter{Vertretung von betroffenen Personen}
\label{ch:80}

\addsec{Text der Verordnung}

\begin{enumerate}

  \item Die \hyperref[itm:04-1]{betroffene Person} hat das Recht, eine Einrichtung, Organisationen oder Vereinigung ohne
   Gewinnerzielungsabsicht, die ordnungsgemäß nach dem Recht eines Mitgliedstaats gegründet ist, deren satzungsmäßige
   Ziele im öffentlichem Interesse liegen und die im Bereich des Schutzes der Rechte und Freiheiten von \hyperref[itm:04-1]{betroffenen
   Personen} in Bezug auf den Schutz ihrer \hyperref[itm:04-1]{personenbezogenen Daten} tätig ist, zu beauftragen, in ihrem Namen eine
   Beschwerde einzureichen, in ihrem Namen die in den Artikeln \hyperref[ch:77]{77}, \hyperref[ch:78]{78} und \hyperref
   [ch:79]{79} genannten Rechte wahrzunehmen und das Recht auf Schadensersatz gemäß \hyperref[ch:82]{Artikel 82} in
   Anspruch zu nehmen, sofern dieses im Recht der Mitgliedstaaten vorgesehen ist.
  \label{itm:80-1}

  \item Die Mitgliedstaaten können vorsehen, dass jede der in \hyperref[itm:80-1]{Absatz 1 des vorliegenden Artikels}
   genannten Einrichtungen, Organisationen oder Vereinigungen unabhängig von einem Auftrag der \hyperref[itm:04-1]{betroffenen Person} in
   diesem Mitgliedstaat das Recht hat, bei der gemäß \hyperref[ch:77]{Artikel 77} zuständigen \hyperref[itm:04-21]{Aufsichtsbehörde} eine
   Beschwerde einzulegen und die in den Artikeln \hyperref[ch:78]{78} und \hyperref[ch:79]{79} aufgeführten Rechte in
   Anspruch zu nehmen, wenn ihres Erachtens die Rechte einer \hyperref[itm:04-1]{betroffenen Person} gemäß dieser Verordnung infolge einer
   \hyperref[itm:04-2]{Verarbeitung} verletzt worden sind.
  \label{itm:80-2}

\end{enumerate}

\addsec{Eigene Notizen}

