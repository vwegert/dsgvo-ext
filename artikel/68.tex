%!TEX root = ../DSGVO-ExtendedVersion.tex
\chapter{Europäischer Datenschutzausschuss}
\label{ch:68}

\crossrefArticleToReason{68}

\begin{enumerate}

  \item Der Europäische Datenschutzausschuss (im Folgenden „Ausschuss“) wird als Einrichtung der Union mit eigener
   Rechtspersönlichkeit eingerichtet.%
  \label{itm:68-1}

  \item Der Ausschuss wird von seinem Vorsitz vertreten.%
  \label{itm:68-2}

  \item Der Ausschuss besteht aus dem Leiter einer \hyperref[itm:04-21]{Aufsichtsbehörde} jedes Mitgliedstaats und dem
   Europäischen Datenschutzbeauftragten oder ihren jeweiligen \hyperref[itm:04-17]{Vertretern}.%
  \label{itm:68-3}

  \item Ist in einem Mitgliedstaat mehr als eine \hyperref[itm:04-21]{Aufsichtsbehörde} für die Überwachung der
   Anwendung der nach Maßgabe dieser Verordnung erlassenen Vorschriften zuständig, so wird im Einklang mit den
   Rechtsvorschriften dieses Mitgliedstaats ein gemeinsamer \hyperref[itm:04-17]{Vertreter} benannt.%
  \label{itm:68-4}

  \item Die Kommission ist berechtigt, ohne Stimmrecht an den Tätigkeiten und Sitzungen des Ausschusses teilzunehmen.
   Die Kommission benennt einen \hyperref[itm:04-17]{Vertreter}. Der Vorsitz des Ausschusses unterrichtet die
   Kommission über die Tätigkeiten des Ausschusses.%
  \label{itm:68-5}

  \item In den in \hyperref[ch:65]{Artikel 65} genannten Fällen ist der Europäische Datenschutzbeauftragte nur bei
   Beschlüssen stimmberechtigt, die Grundsätze und Vorschriften betreffen, die für die Organe, Einrichtungen, Ämter und
   Agenturen der Union gelten und inhaltlich den Grundsätzen und Vorschriften dieser Verordnung entsprechen.%
  \label{itm:68-6}

\end{enumerate}

% \addsec{Ergänzende Hinweise}

