%!TEX root = ../DSGVO-Bearbeitung.tex
\chapter{Datenübermittlung vorbehaltlich geeigneter Garantien}
\label{ch:46}

\addsec{Text der Verordnung}

\begin{enumerate}

  \item Falls kein Beschluss nach \hyperref[itm:45-3]{Artikel 45 Absatz 3} vorliegt, darf ein \hyperref[itm:04-7]{Verantwortlicher} oder ein
   \hyperref[itm:04-8]{Auftragsverarbeiter} \hyperref[itm:04-1]{personenbezogene Daten} an ein Drittland oder eine \hyperref[itm:04-29]{internationale Organisation} nur übermitteln,
   sofern der \hyperref[itm:04-7]{Verantwortliche} oder der \hyperref[itm:04-8]{Auftragsverarbeiter} geeignete Garantien vorgesehen hat und sofern den
   \hyperref[itm:04-1]{betroffenen Personen} durchsetzbare Rechte und wirksame Rechtsbehelfe zur Verfügung stehen.
  \label{itm:46-1}

  \item Die in \hyperref[itm:46-1]{Absatz 1} genannten geeigneten Garantien können, ohne dass hierzu eine besondere
   Genehmigung einer \hyperref[itm:04-21]{Aufsichtsbehörde} erforderlich wäre, bestehen in
  \label{itm:46-2}

  \begin{enumerate}
  
    \item einem rechtlich bindenden und durchsetzbaren Dokument zwischen den Behörden oder öffentlichen Stellen,
    \label{itm:46-2a}

    \item \hyperref[itm:04-20]{verbindlichen internen Datenschutzvorschriften} gemäß \hyperref[ch:47]{Artikel 47},
    \label{itm:46-2b}

    \item Standarddatenschutzklauseln, die von der Kommission gemäß dem Prüfverfahren nach \hyperref[itm:93-2]
     {Artikel 93 Absatz 2} erlassen werden,
    \label{itm:46-2c}

    \item von einer \hyperref[itm:04-21]{Aufsichtsbehörde} angenommenen Standarddatenschutzklauseln, die von der Kommission gemäß dem
     Prüfverfahren nach \hyperref[itm:93-2]{Artikel 93 Absatz 2} genehmigt wurden,
    \label{itm:46-2d}

    \item genehmigten Verhaltensregeln gemäß \hyperref[ch:40]{Artikel 40} zusammen mit rechtsverbindlichen und
     durchsetzbaren Verpflichtungen des \hyperref[itm:04-7]{Verantwortlichen} oder des \hyperref[itm:04-8]{Auftragsverarbeiters} in dem Drittland zur Anwendung
     der geeigneten Garantien, einschließlich in Bezug auf die Rechte der \hyperref[itm:04-1]{betroffenen Personen}, oder
    \label{itm:46-2e}

    \item einem genehmigten Zertifizierungsmechanismus gemäß \hyperref[ch:42]{Artikel 42} zusammen mit
     rechtsverbindlichen und durchsetzbaren Verpflichtungen des \hyperref[itm:04-7]{Verantwortlichen} oder des \hyperref[itm:04-8]{Auftragsverarbeiters} in dem
     Drittland zur Anwendung der geeigneten Garantien, einschließlich in Bezug auf die Rechte der \hyperref[itm:04-1]{betroffenen
     Personen}.
    \label{itm:46-2f}

  \end{enumerate}

  \item Vorbehaltlich der Genehmigung durch die zuständige \hyperref[itm:04-21]{Aufsichtsbehörde} können die geeigneten Garantien gemäß
   \hyperref[itm:46-1]{Absatz 1} auch insbesondere bestehen in
  \label{itm:46-3}

  \begin{enumerate}
  
    \item Vertragsklauseln, die zwischen dem \hyperref[itm:04-7]{Verantwortlichen} oder dem \hyperref[itm:04-8]{Auftragsverarbeiter} und dem \hyperref[itm:04-7]{Verantwortlichen}, dem
     \hyperref[itm:04-8]{Auftragsverarbeiter} oder dem Empfänger der \hyperref[itm:04-1]{personenbezogenen Daten} im Drittland oder der internationalen
     Organisation vereinbart wurden, oder
    \label{itm:46-3a}

    \item Bestimmungen, die in Verwaltungsvereinbarungen zwischen Behörden oder öffentlichen Stellen aufzunehmen sind
     und durchsetzbare und wirksame Rechte für die \hyperref[itm:04-1]{betroffenen Personen} einschließen.
    \label{itm:46-3b}

  \end{enumerate}

  \item Die \hyperref[itm:04-21]{Aufsichtsbehörde} wendet das Kohärenzverfahren nach \hyperref[ch:63]{Artikel 63} an, wenn ein Fall gemäß
   \hyperref[itm:46-3]{Absatz 3 des vorliegenden Artikels} vorliegt.
  \label{itm:46-4}

  \item Von einem Mitgliedstaat oder einer \hyperref[itm:04-21]{Aufsichtsbehörde} auf der Grundlage von Artikel 26 Absatz 2 der Richtlinie
   95/46/EG\todo{nachschlagen} erteilte Genehmigungen bleiben so lange gültig, bis sie erforderlichenfalls von dieser
   \hyperref[itm:04-21]{Aufsichtsbehörde} geändert, ersetzt oder aufgehoben werden. Von der Kommission auf der Grundlage von Artikel 26
   Absatz 4 der Richtlinie 95/46/EG erlassene Feststellungen bleiben so lange in Kraft, bis sie erforderlichenfalls mit
   einem nach \hyperref[itm:46-2]{Absatz 2 des vorliegenden Artikels} erlassenen Beschluss der Kommission geändert,
   ersetzt oder aufgehoben werden.
  \label{itm:46-5}

\end{enumerate}

\addsec{Eigene Notizen}

