%!TEX root = ../DSGVO-Bearbeitung.tex
\chapter{Gemeinsam Verantwortliche}
\label{ch:26}

\addsec{Text der Verordnung}

\begin{enumerate}

  \item Legen zwei oder mehr \hyperref[itm:04-7]{Verantwortliche} gemeinsam die Zwecke der und die Mittel zur \hyperref
   [itm:04-2]{Verarbeitung} fest, so sind sie gemeinsam \hyperref[itm:04-7]{Verantwortliche}. Sie legen in einer
   Vereinbarung in transparenter Form fest, wer von ihnen welche Verpflichtung gemäß dieser Verordnung erfüllt,
   insbesondere was die Wahrnehmung der Rechte der \hyperref[itm:04-1]{betroffenen Person} angeht, und wer welchen
   Informationspflichten gemäß den Artikeln \hyperref[ch:13]{13} und \hyperref[ch:14]{14} nachkommt, sofern und soweit
   die jeweiligen Aufgaben der \hyperref[itm:04-7]{Verantwortlichen} nicht durch Rechtsvorschriften der Union oder der
   Mitgliedstaaten, denen die \hyperref[itm:04-7]{Verantwortlichen} unterliegen, festgelegt sind. In der Vereinbarung
   kann eine Anlaufstelle für die \hyperref[itm:04-1]{betroffenen Personen} angegeben werden.
  \label{itm:26-1}

  \item Die Vereinbarung gemäß \hyperref[itm:26-1]{Absatz 1} muss die jeweiligen tatsächlichen Funktionen und
   Beziehungen der gemeinsam \hyperref[itm:04-7]{Verantwortlichen} gegenüber \hyperref[itm:04-1]{betroffenen Personen}
   gebührend widerspiegeln. Das wesentliche der Vereinbarung wird der \hyperref[itm:04-1]{betroffenen Person} zur
   Verfügung gestellt.
  \label{itm:26-2}

  \item Ungeachtet der Einzelheiten der Vereinbarung gemäß \hyperref[itm:26-1]{Absatz 1} kann die \hyperref[itm:04-1]
   {betroffene Person} ihre Rechte im Rahmen dieser Verordnung bei und gegenüber jedem einzelnen der \hyperref
   [itm:04-7]{Verantwortlichen} geltend machen.
  \label{itm:26-3}

\end{enumerate}

\addsec{Eigene Notizen}

