%!TEX root = ../DSGVO-Bearbeitung.tex
\chapter{Verzeichnis von Verarbeitungstätigkeiten}
\label{ch:30}

\addsec{Text der Verordnung}

\begin{enumerate}

  \item Jeder Verantwortliche und gegebenenfalls sein Vertreter führen ein Verzeichnis aller Verarbeitungstätigkeiten,
   die ihrer Zuständigkeit unterliegen. Dieses Verzeichnis enthält sämtliche folgenden Angaben:
  \label{itm:30-1}

  \begin{enumerate}
  
    \item den Namen und die Kontaktdaten des Verantwortlichen und gegebenenfalls des gemeinsam mit ihm Verantwortlichen,
     des Vertreters des Verantwortlichen sowie eines etwaigen Datenschutzbeauftragten;
    \label{itm:30-1a}

    \item die Zwecke der Verarbeitung;
    \label{itm:30-1b}

    \item eine Beschreibung der Kategorien betroffener Personen und der Kategorien \hyperref[itm:04-1]{personenbezogener Daten};
    \label{itm:30-1c}

    \item die Kategorien von Empfängern, gegenüber denen die \hyperref[itm:04-1]{personenbezogenen Daten} offengelegt worden sind oder noch
     offengelegt werden, einschließlich Empfänger in Drittländern oder internationalen Organisationen;
    \label{itm:30-1d}

    \item gegebenenfalls Übermittlungen von \hyperref[itm:04-1]{personenbezogenen Daten} an ein Drittland oder an eine internationale
     Organisation, einschließlich der Angabe des betreffenden Drittlands oder der betreffenden internationalen
     Organisation, sowie bei den in \hyperref[itm:49-1-2]{Artikel 49 Absatz 1 Unterabsatz 2} genannten
     Datenübermittlungen die Dokumentierung geeigneter Garantien;
    \label{itm:30-1e}

    \item wenn möglich, die vorgesehenen Fristen für die Löschung der verschiedenen Datenkategorien;
    \label{itm:30-1f}

    \item wenn möglich, eine allgemeine Beschreibung der technischen und organisatorischen Maßnahmen gemäß
     \hyperref[itm:32-1]{Artikel 32 Absatz 1}.
    \label{itm:30-1g}

  \end{enumerate}

  \item Jeder Auftragsverarbeiter und gegebenenfalls sein Vertreter führen ein Verzeichnis zu allen Kategorien von im
   Auftrag eines Verantwortlichen durchgeführten Tätigkeiten der Verarbeitung, die Folgendes enthält:
  \label{itm:30-2}

  \begin{enumerate}
  
    \item den Namen und die Kontaktdaten des Auftragsverarbeiters oder der Auftragsverarbeiter und jedes
     Verantwortlichen, in dessen Auftrag der Auftragsverarbeiter tätig ist, sowie gegebenenfalls des Vertreters des
     Verantwortlichen oder des Auftragsverarbeiters und eines etwaigen Datenschutzbeauftragten;
    \label{itm:30-2a}

    \item die Kategorien von Verarbeitungen, die im Auftrag jedes Verantwortlichen durchgeführt werden;
    \label{itm:30-2b}

    \item gegebenenfalls Übermittlungen von \hyperref[itm:04-1]{personenbezogenen Daten} an ein Drittland oder an eine internationale
     Organisation, einschließlich der Angabe des betreffenden Drittlands oder der betreffenden internationalen
     Organisation, sowie bei den in \hyperref[itm:49-1-2]{Artikel 49 Absatz 1 Unterabsatz 2} genannten
     Datenübermittlungen die Dokumentierung geeigneter Garantien;
    \label{itm:30-2c}

    \item wenn möglich, eine allgemeine Beschreibung der technischen und organisatorischen Maßnahmen gemäß \hyperref
     [itm:32-1]{Artikel 32 Absatz 1}.
    \label{itm:30-2d}

  \end{enumerate}

  \item Das in den Absätzen \hyperref[itm:30-1]{1} und \hyperref[itm:30-2]{2} genannte Verzeichnis ist schriftlich zu
   führen, was auch in einem elektronischen Format erfolgen kann.
  \label{itm:30-3}

  \item Der Verantwortliche oder der Auftragsverarbeiter sowie gegebenenfalls der Vertreter des Verantwortlichen oder
   des Auftragsverarbeiters stellen der Aufsichtsbehörde das Verzeichnis auf Anfrage zur Verfügung.
  \label{itm:30-4}

  \item Die in den Absätzen \hyperref[itm:30-1]{1} und \hyperref[itm:30-2]{2} genannten Pflichten gelten nicht für
   Unternehmen oder Einrichtungen, die weniger als 250 Mitarbeiter beschäftigen, sofern die von ihnen vorgenommene
   Verarbeitung nicht ein Risiko für die Rechte und Freiheiten der betroffenen Personen birgt, die Verarbeitung nicht
   nur gelegentlich erfolgt oder nicht die Verarbeitung besonderer Datenkategorien gemäß \hyperref[itm:09-1]{Artikel 9
   Absatz 1} bzw. die  Verarbeitung von \hyperref[itm:04-1]{personenbezogenen Daten} über strafrechtliche Verurteilungen und Straftaten im
   Sinne des \hyperref[ch:10]{Artikels 10} einschließt.
  \label{itm:30-5}

\end{enumerate}

\addsec{Eigene Notizen}

