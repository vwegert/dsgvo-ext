%!TEX root = ../DSGVO-ExtendedVersion.tex
\chapter{Internationale Zusammenarbeit zum Schutz personenbezogener Daten}
\label{ch:50}

In Bezug auf Drittländer und \hyperref[itm:04-26]{internationale Organisationen} treffen die Kommission und
die \hyperref[itm:04-21]{Aufsichtsbehörden} geeignete Maßnahmen zur

\begin{enumerate}[label=\alph*)]
  
  \item Entwicklung von Mechanismen der internationalen Zusammenarbeit, durch die die wirksame Durchsetzung von
   Rechtsvorschriften zum Schutz \hyperref[itm:04-1]{personenbezogener Daten} erleichtert wird,%
  \label{itm:50-a}

  \item gegenseitigen Leistung internationaler Amtshilfe bei der Durchsetzung von Rechtsvorschriften zum Schutz
   \hyperref[itm:04-1]{personenbezogener Daten}, unter anderem durch Meldungen, Beschwerdeverweisungen, Amtshilfe bei
    Untersuchungen und Informationsaustausch, sofern geeignete Garantien für den Schutz \hyperref[itm:04-1]
    {personenbezogener Daten} und anderer Grundrechte und Grundfreiheiten bestehen,%
  \label{itm:50-b}

  \item Einbindung maßgeblicher Interessenträger in Diskussionen und Tätigkeiten, die zum Ausbau der internationalen
   Zusammenarbeit bei der Durchsetzung von Rechtsvorschriften zum Schutz \hyperref[itm:04-1]{personenbezogener Daten}
   dienen,%
  \label{itm:50-c}

  \item Förderung des Austauschs und der Dokumentation von Rechtsvorschriften und Praktiken zum Schutz personenbezogener
   Daten einschließlich Zuständigkeitskonflikten mit Drittländern.%
  \label{itm:50-d}

\end{enumerate}

% \addsec{Ergänzende Hinweise}

