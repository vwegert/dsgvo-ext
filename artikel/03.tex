%!TEX root = ../DSGVO-Bearbeitung.tex
\chapter{Räumlicher Anwendungsbereich}
\label{ch:3}

\addsec{Text der Verordnung}

\begin{enumerate}

  \item Diese Verordnung findet Anwendung auf die \hyperref[itm:04-2]{Verarbeitung} \hyperref[itm:04-1]{personenbezogener Daten}, soweit
   diese im Rahmen der Tätigkeiten einer Niederlassung eines Verantwortlichen oder eines Auftragsverarbeiters in der
   Union erfolgt, unabhängig davon, ob die \hyperref[itm:04-2]{Verarbeitung} in der Union stattfindet.
  \label{itm:03-1}

  \item Diese Verordnung findet Anwendung auf die \hyperref[itm:04-2]{Verarbeitung} \hyperref[itm:04-1]{personenbezogener Daten} von
   \hyperref[itm:04-1]{betroffenen Personen}, die sich in der Union befinden, durch einen nicht in der Union niedergelassenen
   Verantwortlichen oder Auftragsverarbeiter, wenn die Datenverarbeitung im Zusammenhang damit steht
  \label{itm:03-2}

  \begin{enumerate}
  
    \item \hyperref[itm:04-1]{betroffenen Personen} in der Union Waren oder Dienstleistungen anzubieten, unabhängig davon, ob von diesen
     \hyperref[itm:04-1]{betroffenen Personen} eine Zahlung zu leisten ist;
    \label{itm:03-2a}

    \item das Verhalten \hyperref[itm:04-1]{betroffener Personen} zu beobachten, soweit ihr Verhalten in der Union erfolgt.
    \label{itm:03-2b}

  \end{enumerate}

  \item Diese Verordnung findet Anwendung auf die \hyperref[itm:04-2]{Verarbeitung} \hyperref[itm:04-1]{personenbezogener Daten} durch einen
   nicht in der Union niedergelassenen Verantwortlichen an einem Ort, der aufgrund Völkerrechts dem Recht eines
   Mitgliedstaats unterliegt.
  \label{itm:03-3}

\end{enumerate}

\addsec{Eigene Notizen}

