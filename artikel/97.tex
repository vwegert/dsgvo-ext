%!TEX root = ../DSGVO-ExtendedVersion.tex
\chapter{Berichte der Kommission}
\label{ch:97}

\begin{enumerate}

  \item Bis zum 25. Mai 2020 und danach alle vier Jahre legt die Kommission dem Europäischen Parlament und dem Rat einen
   Bericht über die Bewertung und Überprüfung dieser Verordnung vor. Die Berichte werden öffentlich gemacht.%
  \label{itm:97-1}

  \item Im Rahmen der Bewertungen und Überprüfungen nach \hyperref[itm:97-1]{Absatz 1} prüft die Kommission insbesondere
   die Anwendung und die Wirkungsweise%
  \label{itm:97-2}

  \begin{enumerate}
  
    \item des \hyperref[part:5]{Kapitels V} über die Übermittlung \hyperref[itm:04-1]{personenbezogener Daten} an
     Drittländer oder an
     \hyperref[itm:04-26]{internationale Organisationen} insbesondere im Hinblick auf die gemäß \hyperref[itm:45-3]
      {Artikel 45 Absatz 3} der vorliegenden Verordnung erlassenen Beschlüsse sowie die gemäß Artikel 25 Absatz 6 der
      Richtlinie 95/46/EG
     \todo{Recherche}
     erlassenen Feststellungen,%
    \label{itm:97-2a}

    \item des \hyperref[part:7]{Kapitels VII} über Zusammenarbeit und Kohärenz.%
    \label{itm:97-2b}

  \end{enumerate}

  \item Für den in \hyperref[itm:97-1]{Absatz 1} genannten Zweck kann die Kommission Informationen von den
   Mitgliedstaaten und den \hyperref[itm:04-21]{Aufsichtsbehörden} anfordern.%
  \label{itm:97-3}

  \item Bei den in den Absätzen \hyperref[itm:97-1]{1} und \hyperref[itm:97-2]{2} genannten Bewertungen und
   Überprüfungen berücksichtigt die Kommission die Standpunkte und Feststellungen des Europäischen Parlaments, des
   Rates und anderer einschlägiger Stellen oder Quellen.%
  \label{itm:97-4}

  \item Die Kommission legt erforderlichenfalls geeignete Vorschläge zur Änderung dieser Verordnung vor und
   berücksichtigt dabei insbesondere die Entwicklungen in der Informationstechnologie und die Fortschritte in der
   Informationsgesellschaft.%
  \label{itm:97-5}

\end{enumerate}

% \addsec{Ergänzende Hinweise}

