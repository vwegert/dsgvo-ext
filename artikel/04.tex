%!TEX root = ../DSGVO-Bearbeitung.tex
\chapter{Begriffsbestimmungen}
\label{ch:04}

\addsec{Text der Verordnung}

Im Sinne dieser Verordnung bezeichnet der Ausdruck:

\begin{enumerate}

  \item „personenbezogene Daten“ alle Informationen, die sich auf eine identifizierte oder identifizierbare natürliche
   Person (im Folgenden „betroffene Person“) beziehen; als identifizierbar wird eine natürliche Person angesehen, die
   direkt oder indirekt, insbesondere mittels Zuordnung zu einer Kennung wie einem Namen, zu einer Kennnummer, zu
   Standortdaten, zu einer Online-Kennung oder zu einem oder mehreren besonderen Merkmalen, die Ausdruck der
   physischen, physiologischen, genetischen, psychischen, wirtschaftlichen, kulturellen oder sozialen Identität dieser
   natürlichen Person sind, identifiziert werden kann;

  \item „Verarbeitung“ jeden mit oder ohne Hilfe automatisierter Verfahren ausgeführten Vorgang oder jede solche
   Vorgangsreihe im Zusammenhang mit personenbezogenen Daten wie das Erheben, das Erfassen, die Organisation, das
   Ordnen, die Speicherung, die Anpassung oder Veränderung, das Auslesen, das Abfragen, die Verwendung, die Offenlegung
   durch Übermittlung, Verbreitung oder eine andere Form der Bereitstellung, den Abgleich oder die Verknüpfung, die
   Einschränkung, das Löschen oder die Vernichtung;

  \item „Einschränkung der Verarbeitung“ die Markierung gespeicherter personenbezogener Daten mit dem Ziel, ihre
   künftige Verarbeitung einzuschränken;

  \item „Profiling“ jede Art der automatisierten Verarbeitung personenbezogener Daten, die darin besteht, dass diese
   personenbezogenen Daten verwendet werden, um bestimmte persönliche Aspekte, die sich auf eine natürliche Person
   beziehen, zu bewerten, insbesondere um Aspekte bezüglich Arbeitsleistung, wirtschaftliche Lage, Gesundheit,
   persönliche Vorlieben, Interessen, Zuverlässigkeit, Verhalten, Aufenthaltsort oder Ortswechsel dieser natürlichen
   Person zu analysieren oder vorherzusagen;

  \item „Pseudonymisierung“ die Verarbeitung personenbezogener Daten in einer Weise, dass die personenbezogenen Daten
   ohne Hinzuziehung zusätzlicher Informationen nicht mehr einer spezifischen betroffenen Person zugeordnet werden
   können, sofern diese zusätzlichen Informationen gesondert aufbewahrt werden und technischen und organisatorischen
   Maßnahmen unterliegen, die gewährleisten, dass die personenbezogenen Daten nicht einer identifizierten oder
   identifizierbaren natürlichen Person zugewiesen werden;

  \item „Dateisystem“ jede strukturierte Sammlung personenbezogener Daten, die nach bestimmten Kriterien zugänglich
   sind, unabhängig davon, ob diese Sammlung zentral, dezentral oder nach funktionalen oder geografischen
   Gesichtspunkten geordnet geführt wird;

  \item „Verantwortlicher“ die natürliche oder juristische Person, Behörde, Einrichtung oder andere Stelle, die allein
   oder gemeinsam mit anderen über die Zwecke und Mittel der Verarbeitung von personenbezogenen Daten entscheidet; sind
   die Zwecke und Mittel dieser Verarbeitung durch das Unionsrecht oder das Recht der Mitgliedstaaten vorgegeben, so
   kann der Verantwortliche beziehungsweise können die bestimmten Kriterien seiner Benennung nach dem Unionsrecht oder
   dem Recht der Mitgliedstaaten vorgesehen werden;

  \item „Auftragsverarbeiter“ eine natürliche oder juristische Person, Behörde, Einrichtung oder andere Stelle, die
   personenbezogene Daten im Auftrag des Verantwortlichen verarbeitet;

  \item „Empfänger“ eine natürliche oder juristische Person, Behörde, Einrichtung oder andere Stelle, der
   personenbezogene Daten offengelegt werden, unabhängig davon, ob es sich bei ihr um einen Dritten handelt oder nicht.
   Behörden, die im Rahmen eines bestimmten Untersuchungsauftrags nach dem Unionsrecht oder dem Recht der
   Mitgliedstaaten möglicherweise personenbezogene Daten erhalten, gelten jedoch nicht als Empfänger; die Verarbeitung
   dieser Daten durch die genannten Behörden erfolgt im Einklang mit den geltenden Datenschutzvorschriften gemäß den
   Zwecken der Verarbeitung;

  \item „Dritter“ eine natürliche oder juristische Person, Behörde, Einrichtung oder andere Stelle, außer der
   betroffenen Person, dem Verantwortlichen, dem Auftragsverarbeiter und den Personen, die unter der unmittelbaren
   Verantwortung des Verantwortlichen oder des Auftragsverarbeiters befugt sind, die personenbezogenen Daten zu
   verarbeiten;

  \item „Einwilligung“ der betroffenen Person jede freiwillig für den bestimmten Fall, in informierter Weise und
   unmissverständlich abgegebene Willensbekundung in Form einer Erklärung oder einer sonstigen eindeutigen
   bestätigenden Handlung, mit der die betroffene Person zu verstehen gibt, dass sie mit der Verarbeitung der sie
   betreffenden personenbezogenen Daten einverstanden ist;

  \item „Verletzung des Schutzes personenbezogener Daten“ eine Verletzung der Sicherheit, die, ob unbeabsichtigt oder
   unrechtmäßig, zur Vernichtung, zum Verlust, zur Veränderung, oder zur unbefugten Offenlegung von beziehungsweise zum
   unbefugten Zugang zu personenbezogenen Daten führt, die übermittelt, gespeichert oder auf sonstige Weise verarbeitet
   wurden;

  \item „genetische Daten“ personenbezogene Daten zu den ererbten oder erworbenen genetischen Eigenschaften einer
   natürlichen Person, die eindeutige Informationen über die Physiologie oder die Gesundheit dieser natürlichen Person
   liefern und insbesondere aus der Analyse einer biologischen Probe der betreffenden natürlichen Person gewonnen
   wurden;

  \item „biometrische Daten“ mit speziellen technischen Verfahren gewonnene personenbezogene Daten zu den physischen,
   physiologischen oder verhaltenstypischen Merkmalen einer natürlichen Person, die die eindeutige Identifizierung
   dieser natürlichen Person ermöglichen oder bestätigen, wie Gesichtsbilder oder daktyloskopische Daten;

  \item „Gesundheitsdaten“ personenbezogene Daten, die sich auf die körperliche oder geistige Gesundheit einer
   natürlichen Person, einschließlich der Erbringung von Gesundheitsdienstleistungen, beziehen und aus denen
   Informationen über deren Gesundheitszustand hervorgehen;

  \item „Hauptniederlassung“

  \begin{enumerate}
  
    \item im Falle eines Verantwortlichen mit Niederlassungen in mehr als einem Mitgliedstaat den Ort seiner
     Hauptverwaltung in der Union, es sei denn, die Entscheidungen hinsichtlich der Zwecke und Mittel der Verarbeitung
     personenbezogener Daten werden in einer anderen Niederlassung des Verantwortlichen in der Union getroffen und
     diese Niederlassung ist befugt, diese Entscheidungen umsetzen zu lassen; in diesem Fall gilt die Niederlassung,
     die derartige Entscheidungen trifft, als Hauptniederlassung;

    \item im Falle eines Auftragsverarbeiters mit Niederlassungen in mehr als einem Mitgliedstaat den Ort seiner
     Hauptverwaltung in der Union oder, sofern der Auftragsverarbeiter keine Hauptverwaltung in der Union hat, die
     Niederlassung des Auftragsverarbeiters in der Union, in der die Verarbeitungstätigkeiten im Rahmen der Tätigkeiten
     einer Niederlassung eines Auftragsverarbeiters hauptsächlich stattfinden, soweit der Auftragsverarbeiter
     spezifischen Pflichten aus dieser Verordnung unterliegt;

  \end{enumerate}

  \item „Vertreter“ eine in der Union niedergelassene natürliche oder juristische Person, die von dem Verantwortlichen
   oder Auftragsverarbeiter schriftlich gemäß Artikel 27\todo{Querverweis} bestellt wurde und den Verantwortlichen oder
   Auftragsverarbeiter in Bezug auf die ihnen jeweils nach dieser Verordnung obliegenden Pflichten vertritt;

  \item „Unternehmen“ eine natürliche und juristische Person, die eine wirtschaftliche Tätigkeit ausübt, unabhängig von
   ihrer Rechtsform, einschließlich Personengesellschaften oder Vereinigungen, die regelmäßig einer wirtschaftlichen
   Tätigkeit nachgehen;

  \item „Unternehmensgruppe“ eine Gruppe, die aus einem herrschenden Unternehmen und den von diesem abhängigen
   Unternehmen besteht;

  \item „verbindliche interne Datenschutzvorschriften“ Maßnahmen zum Schutz personenbezogener Daten, zu deren Einhaltung
   sich ein im Hoheitsgebiet eines Mitgliedstaats niedergelassener Verantwortlicher oder Auftragsverarbeiter
   verpflichtet im Hinblick auf Datenübermittlungen oder eine Kategorie von Datenübermittlungen personenbezogener Daten
   an einen Verantwortlichen oder Auftragsverarbeiter derselben Unternehmensgruppe oder derselben Gruppe von
   Unternehmen, die eine gemeinsame Wirtschaftstätigkeit ausüben, in einem oder mehreren Drittländern;

  \item „Aufsichtsbehörde“ eine von einem Mitgliedstaat gemäß Artikel 51\todo{Querverweis} eingerichtete unabhängige
  staatliche Stelle;

  \item „betroffene Aufsichtsbehörde“ eine Aufsichtsbehörde, die von der Verarbeitung personenbezogener Daten betroffen ist, weil

  \begin{enumerate}

    \item der Verantwortliche oder der Auftragsverarbeiter im Hoheitsgebiet des Mitgliedstaats dieser Aufsichtsbehörde
     niedergelassen ist,

    \item diese Verarbeitung erhebliche Auswirkungen auf betroffene Personen mit Wohnsitz im Mitgliedstaat dieser
     Aufsichtsbehörde hat oder haben kann oder

    \item eine Beschwerde bei dieser Aufsichtsbehörde eingereicht wurde;

  \end{enumerate}

  \item „grenzüberschreitende Verarbeitung“ entweder

  \begin{enumerate}

    \item eine Verarbeitung personenbezogener Daten, die im Rahmen der Tätigkeiten von Niederlassungen eines
     Verantwortlichen oder eines Auftragsverarbeiters in der Union in mehr als einem Mitgliedstaat erfolgt, wenn der
     Verantwortliche oder Auftragsverarbeiter in mehr als einem Mitgliedstaat niedergelassen ist, oder

    \item eine Verarbeitung personenbezogener Daten, die im Rahmen der Tätigkeiten einer einzelnen Niederlassung eines
     Verantwortlichen oder eines Auftragsverarbeiters in der Union erfolgt, die jedoch erhebliche Auswirkungen auf
     betroffene Personen in mehr als einem Mitgliedstaat hat oder haben kann;

  \end{enumerate}


  \item „maßgeblicher und begründeter Einspruch“ einen Einspruch gegen einen Beschlussentwurf im Hinblick darauf, ob ein
   Verstoß gegen diese Verordnung vorliegt oder ob beabsichtigte Maßnahmen gegen den Verantwortlichen oder den
   Auftragsverarbeiter im Einklang mit dieser Verordnung steht, wobei aus diesem Einspruch die Tragweite der Risiken
   klar hervorgeht, die von dem Beschlussentwurf in Bezug auf die Grundrechte und Grundfreiheiten der betroffenen
   Personen und gegebenenfalls den freien Verkehr personenbezogener Daten in der Union ausgehen;

  \item „Dienst der Informationsgesellschaft“ eine Dienstleistung im Sinne des Artikels 1 Nummer 1 Buchstabe b der
   Richtlinie (EU) 2015/1535 des Europäischen Parlaments und des Rates;

  \item „internationale Organisation“ eine völkerrechtliche Organisation und ihre nachgeordneten Stellen oder jede
   sonstige Einrichtung, die durch eine zwischen zwei oder mehr Ländern geschlossene Übereinkunft oder auf der
   Grundlage einer solchen Übereinkunft geschaffen wurde.

\end{enumerate}

\addsec{Eigene Notizen}

\begin{itemize}

  \item zur Ziffer 12 -- klären: Liegt eine Verletzung des Schutzes nur dann vor, wenn ein tatsächlicher Vorfall
  eingetroffen ist?

  \item zur Ziffer 14 -- klären: Fallen schon die reinen Gesichtsbilder darunter oder müssen es biometrisch
  aufgeschlüsselte Daten sein?

  \item Die in Ziffer 25 referenzierte 
   \href{https://eur-lex.europa.eu/legal-content/DE/TXT/HTML/?uri=CELEX:32015L1535&qid=1659177673712&from=DE}{Richtlinie
    (EU) 2015/1535} hat  ein Informationsverfahren auf dem Gebiet der technischen Vorschriften und der Vorschriften für
    die Dienste der Informationsgesellschaft zum Gegenstand. Die referenzierte Begriffsbestimmung
    für "`Dienstleistung"' lautet "`jede in der Regel gegen Entgelt elektronisch im Fernabsatz und auf individuellen
    Abruf eines Empfängers erbrachte Dienstleistung"'. 
   \href{https://eur-lex.europa.eu/legal-content/DE/TXT/HTML/?uri=CELEX:32015L1535&qid=1659177673712&from=DE#d1e32-10-1}
    {Anhang 1} der Richtlinie enthält eine Beispielsammlung nicht unter die Richtlinie fallender Dienste.

\end{itemize}