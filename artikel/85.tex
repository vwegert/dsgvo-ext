%!TEX root = ../DSGVO-ExtendedVersion.tex
\chapter{Verarbeitung und Freiheit der Meinungsäußerung und Informationsfreiheit}
\label{ch:85}

\addsec{Text der Verordnung}

\begin{enumerate}

  \item Die Mitgliedstaaten bringen durch Rechtsvorschriften das Recht auf den Schutz \hyperref[itm:04-1]
   {personenbezogener Daten} gemäß dieser Verordnung mit dem Recht auf freie Meinungsäußerung und Informationsfreiheit,
   einschließlich der \hyperref[itm:04-2]{Verarbeitung} zu journalistischen Zwecken und zu wissenschaftlichen,
   künstlerischen oder literarischen Zwecken, in Einklang.%
  \label{itm:85-1}

  \item Für die \hyperref[itm:04-2]{Verarbeitung}, die zu journalistischen Zwecken oder zu wissenschaftlichen,
   künstlerischen oder literarischen Zwecken erfolgt, sehen die Mitgliedstaaten Abweichungen oder Ausnahmen
   von \hyperref[part:2]{Kapitel II} (Grundsätze), \hyperref[part:3]{Kapitel III} (Rechte der \hyperref[itm:04-1]
   {betroffenen Person}), \hyperref[part:4]{Kapitel IV}(\hyperref[itm:04-7]{Verantwortlicher} und \hyperref[itm:04-8]
   {Auftragsverarbeiter}), \hyperref[part:5]{Kapitel V} (Übermittlung personenbezogener Daten an Drittländer oder
   an \hyperref[itm:04-29]{internationale Organisationen}), \hyperref[part:6]{Kapitel VI}(Unabhängige \hyperref
   [itm:04-21]{Aufsichtsbehörden}), \hyperref[part:7]{Kapitel VII}(Zusammenarbeit und Kohärenz) und \hyperref[part:9]
   {Kapitel IX} (Vorschriften für besondere Verarbeitungssituationen) vor, wenn dies erforderlich ist, um das Recht auf
   Schutz der \hyperref[itm:04-1]{personenbezogenen Daten} mit der Freiheit der Meinungsäußerung und der
   Informationsfreiheit in Einklang zu bringen.%
  \label{itm:85-2}

  \item Jeder Mitgliedstaat teilt der Kommission die Rechtsvorschriften, die er aufgrund von \hyperref[itm:85-2]
   {Absatz 2} erlassen hat, sowie unverzüglich alle späteren Änderungsgesetze oder Änderungen dieser Vorschriften mit.%
  \label{itm:85-3}

\end{enumerate}

\addsec{Eigene Notizen}

