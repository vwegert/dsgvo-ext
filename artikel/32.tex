%!TEX root = ../DSGVO-Bearbeitung.tex
\chapter{Sicherheit der Verarbeitung}
\label{ch:32}

\addsec{Text der Verordnung}

\begin{enumerate}

  \item Unter Berücksichtigung des Stands der Technik, der Implementierungskosten und der Art, des Umfangs, der Umstände
   und der Zwecke der Verarbeitung sowie der unterschiedlichen Eintrittswahrscheinlichkeit und Schwere des Risikos für
   die Rechte und Freiheiten natürlicher Personen treffen der Verantwortliche und der Auftragsverarbeiter geeignete
   technische und organisatorische Maßnahmen, um ein dem Risiko angemessenes Schutzniveau zu gewährleisten; diese
   Maßnahmen schließen unter anderem Folgendes ein:
  \label{itm:32-1}

  \begin{enumerate}
  
    \item die Pseudonymisierung und Verschlüsselung personenbezogener Daten;
    \label{itm:32-1a}

    \item die Fähigkeit, die Vertraulichkeit, Integrität, Verfügbarkeit und Belastbarkeit der Systeme und Dienste im
     Zusammenhang mit der Verarbeitung auf Dauer sicherzustellen;
    \label{itm:32-1b}

    \item die Fähigkeit, die Verfügbarkeit der personenbezogenen Daten und den Zugang zu ihnen bei einem physischen oder
     technischen Zwischenfall rasch wiederherzustellen;
    \label{itm:32-1c}

    \item ein Verfahren zur regelmäßigen Überprüfung, Bewertung und Evaluierung der Wirksamkeit der technischen und
     organisatorischen Maßnahmen zur Gewährleistung der Sicherheit der Verarbeitung.
    \label{itm:32-1d}

  \end{enumerate}

  \item Bei der Beurteilung des angemessenen Schutzniveaus sind insbesondere die Risiken zu berücksichtigen, die mit der
   Verarbeitung verbunden sind, insbesondere durch -- ob unbeabsichtigt oder unrechtmäßig -- Vernichtung, Verlust,
   Veränderung oder unbefugte Offenlegung von beziehungsweise unbefugten Zugang zu personenbezogenen Daten, die
   übermittelt, gespeichert oder auf andere Weise verarbeitet wurden.
  \label{itm:32-2}

  \item Die Einhaltung genehmigter Verhaltensregeln gemäß \hyperref[ch:40]{Artikel 40} oder eines genehmigten
   Zertifizierungsverfahrens gemäß \hyperref[ch:42]{Artikel 42} kann als Faktor herangezogen werden, um die Erfüllung
   der in \hyperref[itm:32-1]{Absatz 1} des vorliegenden Artikels genannten Anforderungen nachzuweisen.
  \label{itm:32-3}

  \item Der Verantwortliche und der Auftragsverarbeiter unternehmen Schritte, um sicherzustellen, dass ihnen
   unterstellte natürliche Personen, die Zugang zu personenbezogenen Daten haben, diese nur auf Anweisung des
   Verantwortlichen verarbeiten, es sei denn, sie sind nach dem Recht der Union oder der Mitgliedstaaten zur
   Verarbeitung verpflichtet.
  \label{itm:32-4}

\end{enumerate}

\addsec{Eigene Notizen}

