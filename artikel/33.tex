%!TEX root = ../DSGVO-Bearbeitung.tex
\chapter{Meldung von Verletzungen des Schutzes personenbezogener Daten an die Aufsichtsbehörde}
\label{ch:33}

\addsec{Text der Verordnung}

\begin{enumerate}

  \item Im Falle einer \hyperref[itm:04-12]{Verletzung des Schutzes personenbezogener Daten} meldet der Verantwortliche unverzüglich und
   möglichst binnen 72 Stunden, nachdem ihm die Verletzung bekannt wurde, diese der gemäß Artikel 51 zuständigen
   Aufsichtsbehörde, es sei denn, dass die \hyperref[itm:04-12]{Verletzung des Schutzes personenbezogener Daten} voraussichtlich nicht zu
   einem Risiko für die Rechte und Freiheiten natürlicher Personen führt. Erfolgt die Meldung an die Aufsichtsbehörde
   nicht binnen 72 Stunden, so ist ihr eine Begründung für die Verzögerung beizufügen.
  \label{itm:33-1}

  \item Wenn dem Auftragsverarbeiter eine \hyperref[itm:04-12]{Verletzung des Schutzes personenbezogener Daten} bekannt wird, meldet er diese
   dem Verantwortlichen unverzüglich.
  \label{itm:33-2}

  \item Die Meldung gemäß \hyperref[itm:33-1]{Absatz 1} enthält zumindest folgende Informationen:
  \label{itm:33-3}

  \begin{enumerate}
  
    \item eine Beschreibung der Art der \hyperref[itm:04-12]{Verletzung des Schutzes personenbezogener Daten}, soweit möglich mit Angabe der
     Kategorien und der ungefähren Zahl der \hyperref[itm:04-1]{betroffenen Personen}, der betroffenen Kategorien und der ungefähren Zahl
     der betroffenen \hyperref[itm:04-1]{personenbezogenen Daten}sätze;
    \label{itm:33-3a}

    \item den Namen und die Kontaktdaten des Datenschutzbeauftragten oder einer sonstigen Anlaufstelle für weitere
     Informationen;
    \label{itm:33-3b}

    \item eine Beschreibung der wahrscheinlichen Folgen der \hyperref[itm:04-12]{Verletzung des Schutzes personenbezogener Daten};
    \label{itm:33-3c}

    \item eine Beschreibung der von dem Verantwortlichen ergriffenen oder vorgeschlagenen Maßnahmen zur Behebung der
     \hyperref[itm:04-12]{Verletzung des Schutzes personenbezogener Daten} und gegebenenfalls Maßnahmen zur Abmilderung ihrer möglichen
     nachteiligen Auswirkungen.
    \label{itm:33-3d}

  \end{enumerate}

  \item Wenn und soweit die Informationen nicht zur gleichen Zeit bereitgestellt werden können, kann der Verantwortliche
   diese Informationen ohne unangemessene weitere Verzögerung schrittweise zur Verfügung stellen.
  \label{itm:33-4}

  \item Der Verantwortliche dokumentiert \hyperref[itm:04-12]{Verletzungen des Schutzes personenbezogener Daten} einschließlich aller im
   Zusammenhang mit der \hyperref[itm:04-12]{Verletzung des Schutzes personenbezogener Daten} stehenden Fakten, von deren Auswirkungen und
   der ergriffenen Abhilfemaßnahmen. Diese Dokumentation muss der Aufsichtsbehörde die Überprüfung der Einhaltung der
   Bestimmungen dieses Artikels ermöglichen.
  \label{itm:33-5}

\end{enumerate}

\addsec{Eigene Notizen}

