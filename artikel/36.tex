%!TEX root = ../DSGVO-Bearbeitung.tex
\chapter{Vorherige Konsultation}
\label{ch:36}

\addsec{Text der Verordnung}

\begin{enumerate}

  \item Der Verantwortliche konsultiert vor der Verarbeitung die Aufsichtsbehörde, wenn aus einer
   Datenschutz-Folgenabschätzung gemäß \hyperref[ch:35]{Artikel 35} hervorgeht, dass die Verarbeitung ein hohes Risiko
   zur Folge hätte, sofern der Verantwortliche keine Maßnahmen zur Eindämmung des Risikos trifft.
  \label{itm:36-1}

  \item Falls die Aufsichtsbehörde der Auffassung ist, dass die geplante Verarbeitung gemäß \hyperref[itm:36-1]
   {Absatz 1} nicht im Einklang mit dieser Verordnung stünde, insbesondere weil der Verantwortliche das Risiko nicht
   ausreichend ermittelt oder nicht ausreichend eingedämmt hat, unterbreitet sie dem Verantwortlichen und
   gegebenenfalls dem Auftragsverarbeiter innerhalb eines Zeitraums von bis zu acht Wochen nach Erhalt des Ersuchens um
   Konsultation entsprechende schriftliche Empfehlungen und kann ihre in \hyperref[ch:58]{Artikel 58} genannten
   Befugnisse ausüben. Diese Frist kann unter Berücksichtigung der Komplexität der geplanten Verarbeitung um sechs
   Wochen verlängert werden. Die Aufsichtsbehörde unterrichtet den Verantwortlichen oder gegebenenfalls den
   Auftragsverarbeiter über eine solche Fristverlängerung innerhalb eines Monats nach Eingang des Antrags auf
   Konsultation zusammen mit den Gründen für die Verzögerung. Diese Fristen können ausgesetzt werden, bis die
   Aufsichtsbehörde die für die Zwecke der Konsultation angeforderten Informationen erhalten hat.
  \label{itm:36-2}

  \item Der Verantwortliche stellt der Aufsichtsbehörde bei einer Konsultation gemäß \hyperref[itm:36-1]{Absatz 1}
   folgende Informationen zur Verfügung:
  \label{itm:36-3}

  \begin{enumerate}
  
    \item gegebenenfalls Angaben zu den jeweiligen Zuständigkeiten des Verantwortlichen, der gemeinsam Verantwortlichen
     und der an der Verarbeitung beteiligten Auftragsverarbeiter, insbesondere bei einer Verarbeitung innerhalb einer
     Gruppe von Unternehmen;
    \label{itm:36-3a}

    \item die Zwecke und die Mittel der beabsichtigten Verarbeitung;
    \label{itm:36-3b}

    \item die zum Schutz der Rechte und Freiheiten der \hyperref[itm:04-1]{betroffenen Personen} gemäß dieser Verordnung vorgesehenen
     Maßnahmen und Garantien;
    \label{itm:36-3c}

    \item gegebenenfalls die Kontaktdaten des Datenschutzbeauftragten;
    \label{itm:36-3d}

    \item die Datenschutz-Folgenabschätzung gemäß \hyperref[ch:35]{Artikel 35} und
    \label{itm:36-3e}

    \item alle sonstigen von der Aufsichtsbehörde angeforderten Informationen.
    \label{itm:36-3f}

  \end{enumerate}

  \item Die Mitgliedstaaten konsultieren die Aufsichtsbehörde bei der Ausarbeitung eines Vorschlags für von einem
   nationalen Parlament zu erlassende Gesetzgebungsmaßnahmen oder von auf solchen Gesetzgebungsmaßnahmen basierenden
   Regelungsmaßnahmen, die die Verarbeitung betreffen.
  \label{itm:36-4}

  \item Ungeachtet des \hyperref[itm:36-1]{Absatzes 1} können Verantwortliche durch das Recht der Mitgliedstaaten
   verpflichtet werden, bei der Verarbeitung zur Erfüllung einer im öffentlichen Interesse liegenden Aufgabe,
   einschließlich der Verarbeitung zu Zwecken der sozialen Sicherheit und der öffentlichen Gesundheit, die
   Aufsichtsbehörde zu konsultieren und deren vorherige Genehmigung einzuholen.
  \label{itm:36-5}
   
\end{enumerate}

\addsec{Eigene Notizen}

