%!TEX root = ../DSGVO-ExtendedVersion.tex
\chapter{Verarbeitung, für die eine Identifizierung der betroffenen Person nicht erforderlich ist}
\label{ch:11}

\begin{enumerate}

  \item Ist für die Zwecke, für die ein \hyperref[itm:04-7]{Verantwortlicher} \hyperref[itm:04-1]
   {personenbezogene Daten} verarbeitet, die Identifizierung der
   \hyperref[itm:04-1]{betroffenen Person} durch den \hyperref[itm:04-7]{Verantwortlichen} nicht oder nicht mehr
    erforderlich, so ist dieser nicht verpflichtet, zur bloßen Einhaltung dieser Verordnung zusätzliche Informationen
    aufzubewahren, einzuholen oder zu verarbeiten, um die \hyperref[itm:04-1]{betroffene Person} zu identifizieren.%
  \label{itm:11-1}

  \item Kann der \hyperref[itm:04-7]{Verantwortliche} in Fällen gemäß \hyperref[itm:11-1]{Absatz 1} des vorliegenden
   Artikels nachweisen, dass er nicht in der Lage ist, die \hyperref[itm:04-1]{betroffene Person} zu identifizieren, so
   unterrichtet er die \hyperref[itm:04-1]{betroffene Person} hierüber, sofern möglich. In diesen Fällen finden die
   Artikel \hyperref[ch:15]{15} bis \hyperref[ch:20]{20} keine Anwendung, es sei denn, die \hyperref[itm:04-1]
   {betroffene Person} stellt zur Ausübung ihrer in diesen Artikeln niedergelegten Rechte zusätzliche Informationen
   bereit, die ihre Identifizierung ermöglichen.%
  \label{itm:11-2}

\end{enumerate}

% \addsec{Ergänzende Hinweise}

