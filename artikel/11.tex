%!TEX root = ../DSGVO-Bearbeitung.tex
\chapter{Verarbeitung, für die eine Identifizierung der betroffenen Person nicht erforderlich ist}
\label{ch:11}

\addsec{Text der Verordnung}

\begin{enumerate}

  \item Ist für die Zwecke, für die ein Verantwortlicher personenbezogene Daten verarbeitet, die Identifizierung der
   betroffenen Person durch den Verantwortlichen nicht oder nicht mehr erforderlich, so ist dieser nicht verpflichtet,
   zur bloßen Einhaltung dieser Verordnung zusätzliche Informationen aufzubewahren, einzuholen oder zu verarbeiten, um
   die betroffene Person zu identifizieren.
  \label{itm:11-1}

  \item Kann der Verantwortliche in Fällen gemäß \hyperref[itm:11-1]{Absatz 1} des vorliegenden Artikels nachweisen,
   dass er nicht in der Lage ist, die betroffene Person zu identifizieren, so unterrichtet er die betroffene Person
   hierüber, sofern möglich. In diesen Fällen finden die \hyperref[ch:15]{Artikel 15} bis \hyperref[ch:20]{20} keine
   Anwendung, es sei denn, die betroffene Person stellt zur Ausübung ihrer in diesen Artikeln niedergelegten Rechte
   zusätzliche Informationen bereit, die ihre Identifizierung ermöglichen.
  \label{itm:11-2}

\end{enumerate}

\addsec{Eigene Notizen}

