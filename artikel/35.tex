%!TEX root = ../DSGVO-Bearbeitung.tex
\chapter{Datenschutz-Folgenabschätzung}
\label{ch:35}

\addsec{Text der Verordnung}

\begin{enumerate}

  \item Hat eine Form der \hyperref[itm:04-2]{Verarbeitung}, insbesondere bei Verwendung neuer Technologien, aufgrund der Art, des Umfangs,
   der Umstände und der Zwecke der \hyperref[itm:04-2]{Verarbeitung} voraussichtlich ein hohes Risiko für die Rechte und Freiheiten
   natürlicher Personen zur Folge, so führt der Verantwortliche vorab eine Abschätzung der Folgen der vorgesehenen
   Verarbeitungsvorgänge für den Schutz \hyperref[itm:04-1]{personenbezogener Daten} durch. Für die Untersuchung mehrerer ähnlicher
   Verarbeitungsvorgänge mit ähnlich hohen Risiken kann eine einzige Abschätzung vorgenommen werden.
  \label{itm:35-1}

  \item Der Verantwortliche holt bei der Durchführung einer Datenschutz-Folgenabschätzung den Rat des
   Datenschutzbeauftragten, sofern ein solcher benannt wurde, ein.
  \label{itm:35-2}

  \item Eine Datenschutz-Folgenabschätzung gemäß \hyperref[itm:35-1]{Absatz 1} ist insbesondere in folgenden Fällen
   erforderlich:
  \label{itm:35-3}

  \begin{enumerate}
  
    \item systematische und umfassende Bewertung persönlicher Aspekte natürlicher Personen, die sich auf automatisierte
     \hyperref[itm:04-2]{Verarbeitung} einschließlich \hyperref[itm:04-4]{Profiling} gründet und die ihrerseits als Grundlage für Entscheidungen dient, die
     Rechtswirkung gegenüber natürlichen Personen entfalten oder diese in ähnlich erheblicher Weise beeinträchtigen;
    \label{itm:35-3a}

    \item umfangreiche \hyperref[itm:04-2]{Verarbeitung} besonderer Kategorien von \hyperref[itm:04-1]{personenbezogenen Daten} gemäß \hyperref[itm:09-1]
     {Artikel 9 Absatz 1} oder von \hyperref[itm:04-1]{personenbezogenen Daten} über strafrechtliche Verurteilungen und Straftaten
     gemäß \hyperref[ch:10]{Artikel 10} oder
    \label{itm:35-3b}

    \item systematische umfangreiche Überwachung öffentlich zugänglicher Bereiche.
    \label{itm:35-3c}

  \end{enumerate}

  \item Die Aufsichtsbehörde erstellt eine Liste der Verarbeitungsvorgänge, für die gemäß \hyperref[itm:35-1]{Absatz 1}
   eine Datenschutz-Folgenabschätzung durchzuführen ist, und veröffentlicht diese. Die Aufsichtsbehörde übermittelt
   diese Listen dem in \hyperref[ch:68]{Artikel 68} genannten Ausschuss.
  \label{itm:35-4}

  \item Die Aufsichtsbehörde kann des Weiteren eine Liste der Arten von Verarbeitungsvorgängen erstellen und
   veröffentlichen, für die keine Datenschutz-Folgenabschätzung erforderlich ist. Die Aufsichtsbehörde übermittelt
   diese Listen dem Ausschuss.
  \label{itm:35-5}

  \item Vor Festlegung der in den Absätzen \hyperref[itm:35-4]{4} und \hyperref[itm:35-5]{5} genannten Listen wendet die
   zuständige Aufsichtsbehörde das Kohärenzverfahren gemäß \hyperref[ch:63]{Artikel 63} an, wenn solche Listen
   Verarbeitungstätigkeiten umfassen, die mit dem Angebot von Waren oder Dienstleistungen für \hyperref[itm:04-1]{betroffene Personen} oder
   der Beobachtung des Verhaltens dieser Personen in mehreren Mitgliedstaaten im Zusammenhang stehen oder die den
   freien Verkehr \hyperref[itm:04-1]{personenbezogener Daten} innerhalb der Union erheblich beeinträchtigen könnten.
  \label{itm:35-6}

  \item Die Folgenabschätzung enthält zumindest Folgendes:
  \label{itm:35-7}

  \begin{enumerate}
  
    \item eine systematische Beschreibung der geplanten Verarbeitungsvorgänge und der Zwecke der \hyperref[itm:04-2]{Verarbeitung},
     gegebenenfalls einschließlich der von dem Verantwortlichen verfolgten berechtigten Interessen;
    \label{itm:35-7a}

    \item eine Bewertung der Notwendigkeit und Verhältnismäßigkeit der Verarbeitungsvorgänge in Bezug auf den Zweck;
    \label{itm:35-7b}

    \item eine Bewertung der Risiken für die Rechte und Freiheiten der \hyperref[itm:04-1]{betroffenen Personen} gemäß \hyperref[itm:35-1]
     {Absatz 1} und
    \label{itm:35-7c}

    \item die zur Bewältigung der Risiken geplanten Abhilfemaßnahmen, einschließlich Garantien, Sicherheitsvorkehrungen
     und Verfahren, durch die der Schutz \hyperref[itm:04-1]{personenbezogener Daten} sichergestellt und der Nachweis dafür erbracht wird,
     dass diese Verordnung eingehalten wird, wobei den Rechten und berechtigten Interessen der \hyperref[itm:04-1]{betroffenen Personen} und
     sonstiger Betroffener Rechnung getragen wird.
    \label{itm:35-7d}

  \end{enumerate}

  \item Die Einhaltung genehmigter Verhaltensregeln gemäß \hyperref[ch:40]{Artikel 40} durch die zuständigen
   Verantwortlichen oder die zuständigen Auftragsverarbeiter ist bei der Beurteilung der Auswirkungen der von diesen
   durchgeführten Verarbeitungsvorgänge, insbesondere für die Zwecke einer Datenschutz-Folgenabschätzung, gebührend zu
   berücksichtigen.
  \label{itm:35-8}

  \item Der Verantwortliche holt gegebenenfalls den Standpunkt der \hyperref[itm:04-1]{betroffenen Personen} oder ihrer Vertreter zu der
   beabsichtigten \hyperref[itm:04-2]{Verarbeitung} unbeschadet des Schutzes gewerblicher oder öffentlicher Interessen oder der Sicherheit
   der Verarbeitungsvorgänge ein.
  \label{itm:35-9}

  \item Falls die \hyperref[itm:04-2]{Verarbeitung} gemäß \hyperref[itm:06-1]{Artikel 6 Absatz 1} Buchstabe \hyperref[itm:06-1c]
   {c} oder \hyperref[itm:06-1e]{e} auf einer Rechtsgrundlage im Unionsrecht oder im Recht des Mitgliedstaats, dem der
   Verantwortliche unterliegt, beruht und falls diese Rechtsvorschriften den konkreten Verarbeitungsvorgang oder die
   konkreten Verarbeitungsvorgänge regeln und bereits im Rahmen der allgemeinen Folgenabschätzung im Zusammenhang mit
   dem Erlass dieser Rechtsgrundlage eine Datenschutz-Folgenabschätzung erfolgte, gelten die Absätze \hyperref
   [itm:35-1]{1} bis \hyperref[itm:35-7]{7} nur, wenn es nach dem Ermessen der Mitgliedstaaten erforderlich ist, vor
   den betreffenden Verarbeitungstätigkeiten eine solche Folgenabschätzung durchzuführen.
  \label{itm:35-10}

  \item Erforderlichenfalls führt der Verantwortliche eine Überprüfung durch, um zu bewerten, ob die \hyperref[itm:04-2]{Verarbeitung} gemäß
   der Datenschutz-Folgenabschätzung durchgeführt wird; dies gilt zumindest, wenn hinsichtlich des mit den
   Verarbeitungsvorgängen verbundenen Risikos Änderungen eingetreten sind.
  \label{itm:35-11}

\end{enumerate}

\addsec{Eigene Notizen}

