%!TEX root = ../DSGVO-ExtendedVersion.tex
\chapter{Aussetzung des Verfahrens}
\label{ch:81}

\begin{enumerate}

  \item Erhält ein zuständiges Gericht in einem Mitgliedstaat Kenntnis von einem Verfahren zu demselben Gegenstand in
   Bezug auf die \hyperref[itm:04-2]{Verarbeitung} durch denselben \hyperref[itm:04-7]{Verantwortlichen} oder \hyperref
   [itm:04-8]{Auftragsverarbeiter}, das vor einem Gericht in einem anderen Mitgliedstaat anhängig ist, so nimmt es mit
   diesem Gericht Kontakt auf, um sich zu vergewissern, dass ein solches Verfahren existiert.%
  \label{itm:81-1}

  \item Ist ein Verfahren zu demselben Gegenstand in Bezug auf die \hyperref[itm:04-2]{Verarbeitung} durch
   denselben \hyperref[itm:04-7]{Verantwortlichen} oder
   \hyperref[itm:04-8]{Auftragsverarbeiter} vor einem Gericht in einem anderen Mitgliedstaat anhängig, so kann jedes
    später angerufene zuständige Gericht das bei ihm anhängige Verfahren aussetzen.%
  \label{itm:81-2}

  \item Sind diese Verfahren in erster Instanz anhängig, so kann sich jedes später angerufene Gericht auf Antrag einer
   Partei auch für unzuständig erklären, wenn das zuerst angerufene Gericht für die betreffenden Klagen zuständig ist
   und die Verbindung der Klagen nach seinem Recht zulässig ist.%
  \label{itm:81-3}

\end{enumerate}

% \addsec{Ergänzende Hinweise}

