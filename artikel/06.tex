%!TEX root = ../DSGVO-Bearbeitung.tex
\chapter{Rechtmäßigkeit der Verarbeitung}
\label{ch:6}

\addsec{Text der Verordnung}

\begin{enumerate}

  \item Die \hyperref[itm:04-2]{Verarbeitung} ist nur rechtmäßig, wenn mindestens eine der nachstehenden Bedingungen erfüllt ist:
  \label{itm:06-1}

  \begin{enumerate}
  
    \item Die \hyperref[itm:04-1]{betroffene Person} hat ihre \hyperref[itm:04-11]{Einwilligung} zu der \hyperref[itm:04-2]{Verarbeitung} der sie betreffenden \hyperref[itm:04-1]{personenbezogenen Daten}
     für einen oder mehrere bestimmte Zwecke gegeben;
    \label{itm:06-1a}

    \item die \hyperref[itm:04-2]{Verarbeitung} ist für die Erfüllung eines Vertrags, dessen Vertragspartei die \hyperref[itm:04-1]{betroffene Person} ist, oder
     zur Durchführung vorvertraglicher Maßnahmen erforderlich, die auf Anfrage der \hyperref[itm:04-1]{betroffenen Person} erfolgen;
    \label{itm:06-1b}

    \item die \hyperref[itm:04-2]{Verarbeitung} ist zur Erfüllung einer rechtlichen Verpflichtung erforderlich, der der \hyperref[itm:04-7]{Verantwortliche}
     unterliegt;
    \label{itm:06-1c}

    \item die \hyperref[itm:04-2]{Verarbeitung} ist erforderlich, um lebenswichtige Interessen der \hyperref[itm:04-1]{betroffenen Person} oder einer anderen
     natürlichen Person zu schützen;
    \label{itm:06-1d}

    \item die \hyperref[itm:04-2]{Verarbeitung} ist für die Wahrnehmung einer Aufgabe erforderlich, die im öffentlichen Interesse liegt oder
     in Ausübung öffentlicher Gewalt erfolgt, die dem \hyperref[itm:04-7]{Verantwortlichen} übertragen wurde;
    \label{itm:06-1e}

    \item die \hyperref[itm:04-2]{Verarbeitung} ist zur Wahrung der berechtigten Interessen des \hyperref[itm:04-7]{Verantwortlichen} oder eines \hyperref[itm:04-10]{Dritten}
     erforderlich, sofern nicht die Interessen oder Grundrechte und Grundfreiheiten der \hyperref[itm:04-1]{betroffenen Person}, die den
     Schutz \hyperref[itm:04-1]{personenbezogener Daten} erfordern, überwiegen, insbesondere dann, wenn es sich bei der \hyperref[itm:04-1]{betroffenen Person}
     um ein Kind handelt.
    \label{itm:06-1f}

  \end{enumerate}

  Unterabsatz 1 Buchstabe f gilt nicht für die von Behörden in Erfüllung ihrer Aufgaben vorgenommene \hyperref[itm:04-2]{Verarbeitung}.

  \item Die Mitgliedstaaten können spezifischere Bestimmungen zur Anpassung der Anwendung der Vorschriften dieser
   Verordnung in Bezug auf die \hyperref[itm:04-2]{Verarbeitung} zur Erfüllung von \hyperref[itm:06-1c]{Absatz 1 Buchstaben c} und \hyperref
   [itm:06-1e]{e} beibehalten oder einführen, indem sie spezifische Anforderungen für die \hyperref[itm:04-2]{Verarbeitung} sowie sonstige
   Maßnahmen präziser bestimmen, um eine rechtmäßig und nach Treu und Glauben erfolgende \hyperref[itm:04-2]{Verarbeitung} zu gewährleisten,
   einschließlich für andere besondere Verarbeitungssituationen gemäß \hyperref[part:9]{Kapitel IX}.
  \label{itm:06-2}

  \item Die Rechtsgrundlage für die \hyperref[itm:04-2]{Verarbeitungen} gemäß \hyperref[itm:06-1c]{Absatz 1 Buchstaben c} und \hyperref
   [itm:06-1e]{e} wird festgelegt durch
  \label{itm:06-3}

  \begin{enumerate}
  
    \item Unionsrecht oder
    \label{itm:06-3a}

    \item das Recht der Mitgliedstaaten, dem der \hyperref[itm:04-7]{Verantwortliche} unterliegt.
    \label{itm:06-3b}

  \end{enumerate}

  Der Zweck der \hyperref[itm:04-2]{Verarbeitung} muss in dieser Rechtsgrundlage festgelegt oder hinsichtlich der \hyperref[itm:04-2]{Verarbeitung} gemäß
  \hyperref[itm:06-1e]{Absatz 1 Buchstaben e} für die Erfüllung einer Aufgabe erforderlich sein, die im öffentlichen
   Interesse liegt oder in Ausübung öffentlicher Gewalt erfolgt, die dem \hyperref[itm:04-7]{Verantwortlichen} übertragen wurde. Diese
   Rechtsgrundlage kann spezifische Bestimmungen zur Anpassung der Anwendung der Vorschriften dieser Verordnung
   enthalten, unter anderem Bestimmungen darüber, welche allgemeinen Bedingungen für die Regelung der Rechtmäßigkeit
   der \hyperref[itm:04-2]{Verarbeitung} durch den \hyperref[itm:04-7]{Verantwortlichen} gelten, welche Arten von Daten verarbeitet werden, welche Personen
   betroffen sind, an welche Einrichtungen und für welche Zwecke die \hyperref[itm:04-1]{personenbezogenen Daten} offengelegt werden dürfen,
   welcher Zweckbindung sie unterliegen, wie lange sie gespeichert werden dürfen und welche Verarbeitungsvorgänge
   und -verfahren angewandt werden dürfen, einschließlich Maßnahmen zur Gewährleistung einer rechtmäßig und nach Treu
   und Glauben erfolgenden Verarbeitung, wie solche für sonstige besondere Verarbeitungssituationen gemäß \hyperref
   [part:9]{Kapitel IX}. Das Unionsrecht oder das Recht der Mitgliedstaaten müssen ein im öffentlichen Interesse
   liegendes Ziel verfolgen und in einem angemessenen Verhältnis zu dem verfolgten legitimen Zweck stehen.

  \item Beruht die \hyperref[itm:04-2]{Verarbeitung} zu einem anderen Zweck als zu demjenigen, zu dem die \hyperref[itm:04-1]{personenbezogenen Daten} erhoben
   wurden, nicht auf der \hyperref[itm:04-11]{Einwilligung} der \hyperref[itm:04-1]{betroffenen Person} oder auf einer Rechtsvorschrift der Union oder der
   Mitgliedstaaten, die in einer demokratischen Gesellschaft eine notwendige und verhältnismäßige Maßnahme zum Schutz
   der in \hyperref[itm:23-1]{Artikel 23 Absatz 1} genannten Ziele darstellt, so berücksichtigt der \hyperref[itm:04-7]{Verantwortliche} --
   um festzustellen, ob die \hyperref[itm:04-2]{Verarbeitung} zu einem anderen Zweck mit demjenigen, zu dem die \hyperref[itm:04-1]{personenbezogenen Daten}
   ursprünglich erhoben wurden, vereinbar ist -- unter anderem
  \label{itm:06-4}

  \begin{enumerate}
  
    \item jede Verbindung zwischen den Zwecken, für die die \hyperref[itm:04-1]{personenbezogenen Daten} erhoben wurden, und den Zwecken der
     beabsichtigten Weiterverarbeitung,
    \label{itm:06-4a}

    \item den Zusammenhang, in dem die \hyperref[itm:04-1]{personenbezogenen Daten} erhoben wurden, insbesondere hinsichtlich des
     Verhältnisses zwischen den \hyperref[itm:04-1]{betroffenen Personen} und dem \hyperref[itm:04-7]{Verantwortlichen},
    \label{itm:06-4b}

    \item die Art der \hyperref[itm:04-1]{personenbezogenen Daten}, insbesondere ob besondere Kategorien \hyperref[itm:04-1]{personenbezogener Daten} gemäß
     \hyperref[ch:9]{Artikel 9} verarbeitet werden oder ob \hyperref[itm:04-1]{personenbezogene Daten} über strafrechtliche Verurteilungen
      und Straftaten gemäß \hyperref[ch:10]{Artikel 10} verarbeitet werden,
    \label{itm:06-4c}

    \item die möglichen Folgen der beabsichtigten Weiterverarbeitung für die \hyperref[itm:04-1]{betroffenen Personen},
    \label{itm:06-4d}

    \item das Vorhandensein geeigneter Garantien, wozu Verschlüsselung oder \hyperref[itm:04-5]{Pseudonymisierung} gehören kann.
    \label{itm:06-4e}

  \end{enumerate}

\end{enumerate}

\addsec{Eigene Notizen}

