%!TEX root = ../DSGVO-Bearbeitung.tex
\chapter{Recht auf wirksamen gerichtlichen Rechtsbehelf gegen eine Aufsichtsbehörde}
\label{ch:78}

\addsec{Text der Verordnung}

\begin{enumerate}

  \item Jede natürliche oder juristische Person hat unbeschadet eines anderweitigen verwaltungsrechtlichen oder
   außergerichtlichen Rechtsbehelfs das Recht auf einen wirksamen gerichtlichen Rechtsbehelf gegen einen sie
   betreffenden rechtsverbindlichen Beschluss einer \hyperref[itm:04-21]{Aufsichtsbehörde}.
  \label{itm:78-1}

  \item Jede \hyperref[itm:04-1]{betroffene Person} hat unbeschadet eines anderweitigen verwaltungsrechtlichen oder
   außergerichtlichen Rechtbehelfs das Recht auf einen wirksamen gerichtlichen Rechtsbehelf, wenn die nach den
   Artikeln \hyperref[ch:55]{55} und \hyperref[ch:56]{56} zuständige \hyperref[itm:04-21]{Aufsichtsbehörde} sich nicht
   mit einer Beschwerde befasst oder die
   \hyperref[itm:04-1]{betroffene Person} nicht innerhalb von drei Monaten über den Stand oder das Ergebnis der
    gemäß \hyperref[ch:77]{Artikel 77} erhobenen Beschwerde in Kenntnis gesetzt hat.
  \label{itm:78-2}

  \item Für Verfahren gegen eine \hyperref[itm:04-21]{Aufsichtsbehörde} sind die Gerichte des Mitgliedstaats zuständig,
   in dem die
   \hyperref[itm:04-21]{Aufsichtsbehörde} ihren Sitz hat.
  \label{itm:78-3}

  \item Kommt es zu einem Verfahren gegen den Beschluss einer \hyperref[itm:04-21]{Aufsichtsbehörde}, dem eine
   Stellungnahme oder ein Beschluss des Ausschusses im Rahmen des Kohärenzverfahrens vorangegangen ist, so leitet
   die \hyperref[itm:04-21]{Aufsichtsbehörde} diese Stellungnahme oder diesen Beschluss dem Gericht zu.
  \label{itm:78-4}

\end{enumerate}

\addsec{Eigene Notizen}

