%!TEX root = ../DSGVO-Bearbeitung.tex
\chapter{Bedingungen für die Einwilligung eines Kindes in Bezug auf Dienste der Informationsgesellschaft}
\label{ch:8}

\addsec{Text der Verordnung}

\begin{enumerate}

  \item Gilt \hyperref[itm:06-1a]{Artikel 6 Absatz 1 Buchstabe a} bei einem Angebot von Diensten der
   Informationsgesellschaft, das einem Kind direkt gemacht wird, so ist die Verarbeitung der \hyperref[itm:04-1]{personenbezogenen Daten}
   des Kindes rechtmäßig, wenn das Kind das sechzehnte Lebensjahr vollendet hat. Hat das Kind noch nicht das sechzehnte
   Lebensjahr vollendet, so ist diese Verarbeitung nur rechtmäßig, sofern und soweit diese Einwilligung durch den
   Träger der elterlichen Verantwortung für das Kind oder mit dessen Zustimmung erteilt wird. 
  \label{itm:08-1}

   Die Mitgliedstaaten können durch Rechtsvorschriften zu diesen Zwecken eine niedrigere Altersgrenze vorsehen, die
   jedoch nicht unter dem vollendeten dreizehnten Lebensjahr liegen darf.

  \item Der Verantwortliche unternimmt unter Berücksichtigung der verfügbaren Technik angemessene Anstrengungen, um sich
   in solchen Fällen zu vergewissern, dass die Einwilligung durch den Träger der elterlichen Verantwortung für das Kind
   oder mit dessen Zustimmung erteilt wurde.
  \label{itm:08-2}

  \item \hyperref[itm:08-1]{Absatz 1} lässt das allgemeine Vertragsrecht der Mitgliedstaaten, wie etwa die Vorschriften
   zur Gültigkeit, zum Zustandekommen oder zu den Rechtsfolgen eines Vertrags in Bezug auf ein Kind, unberührt.
  \label{itm:08-3}

\end{enumerate}

\addsec{Eigene Notizen}

