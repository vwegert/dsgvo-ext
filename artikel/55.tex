%!TEX root = ../DSGVO-ExtendedVersion.tex
\chapter{Zuständigkeit}
\label{ch:55}

\addsec{Text der Verordnung}

\begin{enumerate}

  \item Jede \hyperref[itm:04-21]{Aufsichtsbehörde} ist für die Erfüllung der Aufgaben und die Ausübung der Befugnisse,
   die ihr mit dieser Verordnung übertragen wurden, im Hoheitsgebiet ihres eigenen Mitgliedstaats zuständig.%
  \label{itm:55-1}

  \item Erfolgt die \hyperref[itm:04-2]{Verarbeitung} durch Behörden oder private Stellen auf der Grundlage
   von \hyperref[itm:06-1]{Artikel 6 Absatz 1} Buchstabe \hyperref[itm:06-1c]{c} oder \hyperref[itm:06-1e]{e}, so ist
   die \hyperref[itm:04-21]{Aufsichtsbehörde} des betroffenen Mitgliedstaats zuständig. In diesem Fall findet \hyperref
   [ch:56]{Artikel 56} keine Anwendung.%
  \label{itm:55-2}

  \item Die \hyperref[itm:04-21]{Aufsichtsbehörden} sind nicht zuständig für die Aufsicht über die von Gerichten im
   Rahmen ihrer justiziellen Tätigkeit vorgenommenen \hyperref[itm:04-2]{Verarbeitungen}.%
  \label{itm:55-3}

\end{enumerate}

\addsec{Eigene Notizen}

