%!TEX root = ../DSGVO-ExtendedVersion.tex
\chapter{Transparente Information, Kommunikation und Modalitäten für die Ausübung der Rechte der betroffenen Person}
\label{ch:12}

\begin{enumerate}

  \item Der \hyperref[itm:04-7]{Verantwortliche} trifft geeignete Maßnahmen, um der \hyperref[itm:04-1]
   {betroffenen Person} alle Informationen gemäß den Artikeln
   \hyperref[ch:13]{13} und \hyperref[ch:14]{14} und alle Mitteilungen gemäß den Artikeln \hyperref[ch:15]{15} bis
   \hyperref[ch:22]{22} und \hyperref[ch:34]{Artikel 34}, die sich auf die \hyperref[itm:04-2]{Verarbeitung} beziehen,
    in präziser, transparenter, verständlicher und leicht zugänglicher Form in einer klaren und einfachen Sprache zu
    übermitteln; dies gilt insbesondere für Informationen, die sich speziell an Kinder richten. Die Übermittlung der
    Informationen erfolgt schriftlich oder in anderer Form, gegebenenfalls auch elektronisch. Falls von der \hyperref
    [itm:04-1]{betroffenen Person} verlangt, kann die Information mündlich erteilt werden, sofern die Identität
    der \hyperref[itm:04-1]{betroffenen Person} in anderer Form nachgewiesen wurde.%
  \label{itm:12-1}

  \item Der \hyperref[itm:04-7]{Verantwortliche} erleichtert der \hyperref[itm:04-1]{betroffenen Person} die Ausübung
   ihrer Rechte gemäß den Artikeln \hyperref[ch:15]{15} bis \hyperref[ch:22]{22}. In den in \hyperref[itm:11-2]
   {Artikel 11 Absatz 2} genannten Fällen darf sich der \hyperref[itm:04-7]{Verantwortliche} nur dann weigern, aufgrund
   des Antrags der \hyperref[itm:04-1]{betroffenen Person} auf Wahrnehmung ihrer Rechte gemäß den Artikeln \hyperref
   [ch:15]{15} bis \hyperref[ch:22]{22} tätig zu werden, wenn er glaubhaft macht, dass er nicht in der Lage ist,
   die \hyperref[itm:04-1]{betroffene Person} zu identifizieren.%
  \label{itm:12-2}

  \item Der \hyperref[itm:04-7]{Verantwortliche} stellt der \hyperref[itm:04-1]{betroffenen Person} Informationen über
   die auf Antrag gemäß den Artikeln \hyperref[ch:15]{15} bis \hyperref[ch:22]{22} ergriffenen Maßnahmen unverzüglich,
   in jedem Fall aber innerhalb eines Monats nach Eingang des Antrags zur Verfügung. Diese Frist kann um weitere zwei
   Monate verlängert werden, wenn dies unter Berücksichtigung der Komplexität und der Anzahl von Anträgen erforderlich
   ist. Der \hyperref[itm:04-7]{Verantwortliche} unterrichtet die
   \hyperref[itm:04-1]{betroffene Person} innerhalb eines Monats nach Eingang des Antrags über eine Fristverlängerung,
    zusammen mit den Gründen für die Verzögerung. Stellt die \hyperref[itm:04-1]{betroffene Person} den Antrag
    elektronisch, so ist sie nach Möglichkeit auf elektronischem Weg zu unterrichten, sofern sie nichts anderes
    angibt.%
  \label{itm:12-3}

  \item Wird der \hyperref[itm:04-7]{Verantwortliche} auf den Antrag der \hyperref[itm:04-1]{betroffenen Person} hin
   nicht tätig, so unterrichtet er die
   \hyperref[itm:04-1]{betroffene Person} ohne Verzögerung, spätestens aber innerhalb eines Monats nach Eingang des
    Antrags über die Gründe hierfür und über die Möglichkeit, bei einer \hyperref[itm:04-21]
    {Aufsichtsbehörde} Beschwerde einzulegen oder einen gerichtlichen Rechtsbehelf einzulegen.%
  \label{itm:12-4}

  \item Informationen gemäß den Artikeln \hyperref[ch:13]{13} und \hyperref[ch:14]{14} sowie alle Mitteilungen und
   Maßnahmen gemäß den Artikeln \hyperref[ch:15]{15} bis \hyperref[ch:22]{22} und \hyperref[ch:34]{Artikel 34} werden
   unentgeltlich zur Verfügung gestellt. Bei offenkundig unbegründeten oder -- insbesondere im Fall von häufiger
   Wiederholung -- exzessiven Anträgen einer \hyperref[itm:04-1]{betroffenen Person} kann der \hyperref[itm:04-7]
   {Verantwortliche} entweder%
  \label{itm:12-5}

  \begin{enumerate}
  
    \item ein angemessenes Entgelt verlangen, bei dem die Verwaltungskosten für die Unterrichtung oder die Mitteilung
     oder die Durchführung der beantragten Maßnahme berücksichtigt werden, oder%
    \label{itm:12-5a}

    \item sich weigern, aufgrund des Antrags tätig zu werden.%
    \label{itm:12-5b}

  \end{enumerate}

  Der \hyperref[itm:04-7]{Verantwortliche} hat den Nachweis für den offenkundig unbegründeten oder exzessiven Charakter
  des Antrags zu erbringen.

  \item Hat der \hyperref[itm:04-7]{Verantwortliche} begründete Zweifel an der Identität der natürlichen Person, die den
   Antrag gemäß den Artikeln \hyperref[ch:15]{15} bis \hyperref[ch:21]{21} stellt, so kann er unbeschadet des \hyperref
   [ch:11]{Artikels 11} zusätzliche Informationen anfordern, die zur Bestätigung der Identität der \hyperref[itm:04-1]
   {betroffenen Person} erforderlich sind.%
  \label{itm:12-6}

  \item Die Informationen, die den \hyperref[itm:04-1]{betroffenen Personen} gemäß den Artikeln \hyperref[ch:13]
   {13} und \hyperref[ch:14]{14} bereitzustellen sind, können in Kombination mit standardisierten Bildsymbolen
   bereitgestellt werden, um in leicht wahrnehmbarer, verständlicher und klar nachvollziehbarer Form einen
   aussagekräftigen Überblick über die beabsichtigte \hyperref[itm:04-2]{Verarbeitung} zu vermitteln. Werden die
   Bildsymbole in elektronischer Form dargestellt, müssen sie maschinenlesbar sein.%
  \label{itm:12-7}

  \item Der Kommission wird die Befugnis übertragen, gemäß \hyperref[ch:92]{Artikel 92} delegierte Rechtsakte zur
   Bestimmung der Informationen, die durch Bildsymbole darzustellen sind, und der Verfahren für die Bereitstellung
   standardisierter Bildsymbole zu erlassen.%
  \label{itm:12-8}

\end{enumerate}

% \addsec{Ergänzende Hinweise}

