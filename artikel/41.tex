%!TEX root = ../DSGVO-Bearbeitung.tex
\chapter{Überwachung der genehmigten Verhaltensregeln}
\label{ch:41}

\addsec{Text der Verordnung}

\begin{enumerate}

  \item Unbeschadet der Aufgaben und Befugnisse der zuständigen Aufsichtsbehörde gemäß den Artikeln \hyperref[ch:57]
   {57} und \hyperref[ch:58]{58} kann die Überwachung der Einhaltung von Verhaltensregeln gemäß \hyperref[ch:40]
   {Artikel 40} von einer Stelle durchgeführt werden, die über das geeignete Fachwissen hinsichtlich des Gegenstands
   der Verhaltensregeln verfügt und die von der zuständigen Aufsichtsbehörde zu diesem Zweck akkreditiert wurde.
  \label{itm:41-1}

  \item Eine Stelle gemäß \hyperref[itm:41-1]{Absatz 1} kann zum Zwecke der Überwachung der Einhaltung von
   Verhaltensregeln akkreditiert werden, wenn sie
  \label{itm:41-2}

  \begin{enumerate}
  
    \item ihre Unabhängigkeit und ihr Fachwissen hinsichtlich des Gegenstands der Verhaltensregeln zur Zufriedenheit der
     zuständigen Aufsichtsbehörde nachgewiesen hat;
    \label{itm:41-2a}

    \item Verfahren festgelegt hat, die es ihr ermöglichen, zu bewerten, ob Verantwortliche und Auftragsverarbeiter die
     Verhaltensregeln anwenden können, die Einhaltung der Verhaltensregeln durch die Verantwortlichen und
     Auftragsverarbeiter zu überwachen und die Anwendung der Verhaltensregeln regelmäßig zu überprüfen;
    \label{itm:41-2b}

    \item Verfahren und Strukturen festgelegt hat, mit denen sie Beschwerden über Verletzungen der Verhaltensregeln oder
     über die Art und Weise, in der die Verhaltensregeln von dem Verantwortlichen oder dem Auftragsverarbeiter
     angewendet werden oder wurden, nachgeht und diese Verfahren und Strukturen für \hyperref[itm:04-1]{betroffene Personen} und die
     Öffentlichkeit transparent macht, und
    \label{itm:41-2c}

    \item zur Zufriedenheit der zuständigen Aufsichtsbehörde nachgewiesen hat, dass ihre Aufgaben und Pflichten nicht zu
     einem Interessenkonflikt führen.
    \label{itm:41-2d}

  \end{enumerate}

  \item Die zuständige Aufsichtsbehörde übermittelt den Entwurf der Kriterien für die Akkreditierung einer Stelle nach
   \hyperref[itm:41-1]{Absatz 1} gemäß dem Kohärenzverfahren nach \hyperref[ch:63]{Artikel 63} an den Ausschuss.
  \label{itm:41-3}

  \item Unbeschadet der Aufgaben und Befugnisse der zuständigen Aufsichtsbehörde und der Bestimmungen des \hyperref
   [part:8]{Kapitels VIII} ergreift eine Stelle gemäß \hyperref[itm:41-1]{Absatz 1} vorbehaltlich geeigneter Garantien
   im Falle einer Verletzung der Verhaltensregeln durch einen Verantwortlichen oder einen Auftragsverarbeiter geeignete
   Maßnahmen, einschließlich eines vorläufigen oder endgültigen Ausschlusses des Verantwortlichen oder
   Auftragsverarbeiters von den Verhaltensregeln. Sie unterrichtet die zuständige Aufsichtsbehörde über solche
   Maßnahmen und deren Begründung.
  \label{itm:41-4}

  \item Die zuständige Aufsichtsbehörde widerruft die Akkreditierung einer Stelle gemäß \hyperref[itm:41-1]{Absatz 1},
   wenn die Voraussetzungen für ihre Akkreditierung nicht oder nicht mehr erfüllt sind oder wenn die Stelle Maßnahmen
   ergreift, die nicht mit dieser Verordnung vereinbar sind.
  \label{itm:41-5}

  \item Dieser Artikel gilt nicht für die Verarbeitung durch Behörden oder öffentliche Stellen.
  \label{itm:41-6}

\end{enumerate}

\addsec{Eigene Notizen}

