%!TEX root = ../DSGVO-Bearbeitung.tex
\chapter{Bedingungen für die Einwilligung}
\label{ch:7}

\addsec{Text der Verordnung}

\begin{enumerate}

  \item Beruht die Verarbeitung auf einer Einwilligung, muss der Verantwortliche nachweisen können, dass die betroffene
   Person in die Verarbeitung ihrer personenbezogenen Daten eingewilligt hat.
  \label{itm:07-1}

  \item Erfolgt die Einwilligung der betroffenen Person durch eine schriftliche Erklärung, die noch andere Sachverhalte
   betrifft, so muss das Ersuchen um Einwilligung in verständlicher und leicht zugänglicher Form in einer klaren und
   einfachen Sprache so erfolgen, dass es von den anderen Sachverhalten klar zu unterscheiden ist. Teile der Erklärung
   sind dann nicht verbindlich, wenn sie einen Verstoß gegen diese Verordnung darstellen.
  \label{itm:07-2}

  \item Die betroffene Person hat das Recht, ihre Einwilligung jederzeit zu widerrufen. Durch den Widerruf der
   Einwilligung wird die Rechtmäßigkeit der aufgrund der Einwilligung bis zum Widerruf erfolgten Verarbeitung nicht
   berührt. Die betroffene Person wird vor Abgabe der Einwilligung hiervon in Kenntnis gesetzt. Der Widerruf der
   Einwilligung muss so einfach wie die Erteilung der Einwilligung sein.
  \label{itm:07-3}

  \item Bei der Beurteilung, ob die Einwilligung freiwillig erteilt wurde, muss dem Umstand in größtmöglichem Umfang
   Rechnung getragen werden, ob unter anderem die Erfüllung eines Vertrags, einschließlich der Erbringung einer
   Dienstleistung, von der Einwilligung zu einer Verarbeitung von personenbezogenen Daten abhängig ist, die für die
   Erfüllung des Vertrags nicht erforderlich sind.
  \label{itm:07-4}

\end{enumerate}

\addsec{Eigene Notizen}

