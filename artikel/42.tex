%!TEX root = ../DSGVO-Bearbeitung.tex
\chapter{Zertifizierung}
\label{ch:42}

\addsec{Text der Verordnung}

\begin{enumerate}

  \item Die Mitgliedstaaten, die Aufsichtsbehörden, der Ausschuss und die Kommission fördern insbesondere auf
   Unionsebene die Einführung von datenschutzspezifischen Zertifizierungsverfahren sowie von Datenschutzsiegeln
   und -prüfzeichen, die dazu dienen, nachzuweisen, dass diese Verordnung bei Verarbeitungsvorgängen von
   Verantwortlichen oder Auftragsverarbeitern eingehalten wird. Den besonderen Bedürfnissen von Kleinstunternehmen
   sowie kleinen und mittleren Unternehmen wird Rechnung getragen.
  \label{itm:42-1}

  \item Zusätzlich zur Einhaltung durch die unter diese Verordnung fallenden Verantwortlichen oder Auftragsverarbeiter
   können auch datenschutzspezifische Zertifizierungsverfahren, Siegel oder Prüfzeichen, die gemäß \hyperref[itm:42-5]
   {Absatz 5 des vorliegenden Artikels} genehmigt worden sind, vorgesehen werden, um nachzuweisen, dass die
   Verantwortlichen oder Auftragsverarbeiter, die gemäß \hyperref[ch:3]{Artikel 3} nicht unter diese Verordnung
   fallen, im Rahmen der Übermittlung \hyperref[itm:04-1]{personenbezogener Daten} an Drittländer oder internationale Organisationen nach
   Maßgabe von \hyperref[itm:46-2f]{Artikel 46 Absatz 2 Buchstabe f} geeignete Garantien bieten. Diese Verantwortlichen
   oder Auftragsverarbeiter gehen mittels vertraglicher oder sonstiger rechtlich bindender Instrumente die verbindliche
   und durchsetzbare Verpflichtung ein, diese geeigneten Garantien anzuwenden, auch im Hinblick auf die Rechte der
   betroffenen Personen.
  \label{itm:42-2}

  \item Die Zertifizierung muss freiwillig und über ein transparentes Verfahren zugänglich sein.
  \label{itm:42-3}

  \item Eine Zertifizierung gemäß diesem Artikel mindert nicht die Verantwortung des Verantwortlichen oder des
   Auftragsverarbeiters für die Einhaltung dieser Verordnung und berührt nicht die Aufgaben und Befugnisse der
   Aufsichtsbehörden, die gemäß Artikel \hyperref[ch:55]{55} oder \hyperref[ch:56]{56} zuständig sind.
  \label{itm:42-4}

  \item Eine Zertifizierung nach diesem Artikel wird durch die Zertifizierungsstellen nach \hyperref[ch:43]{Artikel 43}
   oder durch die zuständige Aufsichtsbehörde anhand der von dieser zuständigen Aufsichtsbehörde gemäß \hyperref
   [itm:58-3]{Artikel 58 Absatz 3} oder -- gemäß \hyperref[ch:63]{Artikel 63} -- durch den Ausschuss genehmigten
   Kriterien erteilt. Werden die Kriterien vom Ausschuss genehmigt, kann dies zu einer gemeinsamen Zertifizierung, dem
   Europäischen Datenschutzsiegel, führen.
  \label{itm:42-5}

  \item Der Verantwortliche oder der Auftragsverarbeiter, der die von ihm durchgeführte Verarbeitung dem
   Zertifizierungsverfahren unterwirft, stellt der Zertifizierungsstelle nach \hyperref[ch:43]{Artikel 43} oder
   gegebenenfalls der zuständigen Aufsichtsbehörde alle für die Durchführung des Zertifizierungsverfahrens
   erforderlichen Informationen zur Verfügung und gewährt ihr den in diesem Zusammenhang erforderlichen Zugang zu
   seinen Verarbeitungstätigkeiten.
  \label{itm:42-6}

  \item Die Zertifizierung wird einem Verantwortlichen oder einem Auftragsverarbeiter für eine Höchstdauer von drei
   Jahren erteilt und kann unter denselben Bedingungen verlängert werden, sofern die einschlägigen Voraussetzungen
   weiterhin erfüllt werden. Die Zertifizierung wird gegebenenfalls durch die Zertifizierungsstellen nach \hyperref
   [ch:43]{Artikel 43} oder durch die zuständige Aufsichtsbehörde widerrufen, wenn die Voraussetzungen für die
   Zertifizierung nicht oder nicht mehr erfüllt werden.
  \label{itm:42-7}

  \item Der Ausschuss nimmt alle Zertifizierungsverfahren und Datenschutzsiegel und -prüfzeichen in ein Register auf und
   veröffentlicht sie in geeigneter Weise.
  \label{itm:42-8}

\end{enumerate}

\addsec{Eigene Notizen}

