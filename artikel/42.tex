%!TEX root = ../DSGVO-ExtendedVersion.tex
\chapter{Zertifizierung}
\label{ch:42}

\crossrefArticleToReason{42}

\begin{enumerate}

  \item Die Mitgliedstaaten, die \hyperref[itm:04-21]{Aufsichtsbehörden}, der Ausschuss und die Kommission fördern
   insbesondere auf Unionsebene die Einführung von datenschutzspezifischen Zertifizierungsverfahren sowie von
   Datenschutzsiegeln und -prüfzeichen, die dazu dienen, nachzuweisen, dass diese Verordnung bei Verarbeitungsvorgängen
   von
   \hyperref[itm:04-7]{Verantwortlichen} oder \hyperref[itm:04-8]{Auftragsverarbeitern} eingehalten wird. Den besonderen
    Bedürfnissen von Kleinstunternehmen sowie kleinen und mittleren \hyperref[itm:04-18]{Unternehmen} wird Rechnung
    getragen.%
  \label{itm:42-1}

  \item Zusätzlich zur Einhaltung durch die unter diese Verordnung fallenden \hyperref[itm:04-7]
   {Verantwortlichen} oder \hyperref[itm:04-8]{Auftragsverarbeiter} können auch datenschutzspezifische
   Zertifizierungsverfahren, Siegel oder Prüfzeichen, die gemäß \hyperref[itm:42-5]{Absatz 5 des vorliegenden Artikels}
   genehmigt worden sind, vorgesehen werden, um nachzuweisen, dass die
   \hyperref[itm:04-7]{Verantwortlichen} oder \hyperref[itm:04-8]{Auftragsverarbeiter}, die gemäß \hyperref[ch:3]
    {Artikel 3} nicht unter diese Verordnung fallen, im Rahmen der Übermittlung \hyperref[itm:04-1]
    {personenbezogener Daten} an Drittländer oder \hyperref[itm:04-26]{internationale Organisationen} nach Maßgabe
    von \hyperref[itm:46-2f]{Artikel 46 Absatz 2 Buchstabe f} geeignete Garantien bieten. Diese \hyperref[itm:04-7]
    {Verantwortlichen} oder \hyperref[itm:04-8]{Auftragsverarbeiter} gehen mittels vertraglicher oder sonstiger
    rechtlich bindender Instrumente die verbindliche und durchsetzbare Verpflichtung ein, diese geeigneten Garantien
    anzuwenden, auch im Hinblick auf die Rechte der
   \hyperref[itm:04-1]{betroffenen Personen}.%
  \label{itm:42-2}

  \item Die Zertifizierung muss freiwillig und über ein transparentes Verfahren zugänglich sein.%
  \label{itm:42-3}

  \item Eine Zertifizierung gemäß diesem Artikel mindert nicht die Verantwortung des \hyperref[itm:04-7]
   {Verantwortlichen} oder des
   \hyperref[itm:04-8]{Auftragsverarbeiters} für die Einhaltung dieser Verordnung und berührt nicht die Aufgaben und
    Befugnisse der
   \hyperref[itm:04-21]{Aufsichtsbehörden}, die gemäß Artikel \hyperref[ch:55]{55} oder \hyperref[ch:56]{56} zuständig
    sind.%
  \label{itm:42-4}

  \item Eine Zertifizierung nach diesem Artikel wird durch die Zertifizierungsstellen nach \hyperref[ch:43]{Artikel 43}
   oder durch die zuständige \hyperref[itm:04-21]{Aufsichtsbehörde} anhand der von dieser zuständigen \hyperref
   [itm:04-21]{Aufsichtsbehörde} gemäß \hyperref[itm:58-3]{Artikel 58 Absatz 3} oder -- gemäß \hyperref[ch:63]
   {Artikel 63} -- durch den Ausschuss genehmigten Kriterien erteilt. Werden die Kriterien vom Ausschuss genehmigt,
   kann dies zu einer gemeinsamen Zertifizierung, dem Europäischen Datenschutzsiegel, führen.%
  \label{itm:42-5}

  \item Der \hyperref[itm:04-7]{Verantwortliche} oder der \hyperref[itm:04-8]{Auftragsverarbeiter}, der die von ihm
   durchgeführte \hyperref[itm:04-2]{Verarbeitung} dem Zertifizierungsverfahren unterwirft, stellt der
   Zertifizierungsstelle nach \hyperref[ch:43]{Artikel 43} oder gegebenenfalls der zuständigen \hyperref[itm:04-21]
   {Aufsichtsbehörde} alle für die Durchführung des Zertifizierungsverfahrens erforderlichen Informationen zur
   Verfügung und gewährt ihr den in diesem Zusammenhang erforderlichen Zugang zu seinen Verarbeitungstätigkeiten.%
  \label{itm:42-6}

  \item Die Zertifizierung wird einem \hyperref[itm:04-7]{Verantwortlichen} oder einem \hyperref[itm:04-8]
   {Auftragsverarbeiter} für eine Höchstdauer von drei Jahren erteilt und kann unter denselben Bedingungen verlängert
   werden, sofern die einschlägigen Kriterien weiterhin erfüllt werden. Die Zertifizierung wird gegebenenfalls
   durch die Zertifizierungsstellen nach \hyperref[ch:43]{Artikel 43} oder durch die zuständige \hyperref[itm:04-21]
   {Aufsichtsbehörde} widerrufen, wenn die Kriterien für die Zertifizierung nicht oder nicht mehr erfüllt
   werden.%
  \label{itm:42-7}

  \item Der Ausschuss nimmt alle Zertifizierungsverfahren und Datenschutzsiegel und -prüfzeichen in ein Register auf und
   veröffentlicht sie in geeigneter Weise.%
  \label{itm:42-8}

\end{enumerate}

% \addsec{Ergänzende Hinweise}

