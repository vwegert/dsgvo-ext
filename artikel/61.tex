%!TEX root = ../DSGVO-ExtendedVersion.tex
\chapter{Gegenseitige Amtshilfe}
\label{ch:61}

\begin{enumerate}

  \item Die \hyperref[itm:04-21]{Aufsichtsbehörden} übermitteln einander maßgebliche Informationen und gewähren einander
   Amtshilfe, um diese Verordnung einheitlich durchzuführen und anzuwenden, und treffen Vorkehrungen für eine wirksame
   Zusammenarbeit. Die Amtshilfe bezieht sich insbesondere auf Auskunftsersuchen und aufsichtsbezogene Maßnahmen,
   beispielsweise Ersuchen um vorherige Genehmigungen und eine vorherige Konsultation, um Vornahme von Nachprüfungen
   und Untersuchungen.%
  \label{itm:61-1}

  \item Jede \hyperref[itm:04-21]{Aufsichtsbehörde} ergreift alle geeigneten Maßnahmen, um einem Ersuchen einer
   anderen \hyperref[itm:04-21]{Aufsichtsbehörde} unverzüglich und spätestens innerhalb eines Monats nach Eingang des
   Ersuchens nachzukommen. Dazu kann insbesondere auch die Übermittlung maßgeblicher Informationen über die
   Durchführung einer Untersuchung gehören.%
  \label{itm:61-2}

  \item Amtshilfeersuchen enthalten alle erforderlichen Informationen, einschließlich Zweck und Begründung des
   Ersuchens. Die übermittelten Informationen werden ausschließlich für den Zweck verwendet, für den sie angefordert
   wurden.%
  \label{itm:61-3}

  \item Die ersuchte \hyperref[itm:04-21]{Aufsichtsbehörde} lehnt das Ersuchen nur ab, wenn%
  \label{itm:61-4}

  \begin{enumerate}
  
    \item sie für den Gegenstand des Ersuchens oder für die Maßnahmen, die sie durchführen soll, nicht zuständig ist
     oder%
    \label{itm:61-4a}

    \item ein Eingehen auf das Ersuchen gegen diese Verordnung verstoßen würde oder gegen das Unionsrecht oder das Recht
     der Mitgliedstaaten, dem die \hyperref[itm:04-21]{Aufsichtsbehörde}, bei der das Ersuchen eingeht, unterliegt.%
    \label{itm:61-4b}

  \end{enumerate}

  \item Die ersuchte \hyperref[itm:04-21]{Aufsichtsbehörde} informiert die ersuchende \hyperref[itm:04-21]
   {Aufsichtsbehörde} über die Ergebnisse oder gegebenenfalls über den Fortgang der Maßnahmen, die getroffen wurden, um
   dem Ersuchen nachzukommen. Die ersuchte \hyperref[itm:04-21]{Aufsichtsbehörde} erläutert gemäß \hyperref[itm:61-4]
   {Absatz 4} die Gründe für die Ablehnung des Ersuchens.%
  \label{itm:61-5}

  \item Die ersuchten \hyperref[itm:04-21]{Aufsichtsbehörden} übermitteln die Informationen, um die von einer
   anderen \hyperref[itm:04-21]{Aufsichtsbehörde} ersucht wurde, in der Regel auf elektronischem Wege unter Verwendung
   eines standardisierten Formats.%
  \label{itm:61-6}

  \item Ersuchte \hyperref[itm:04-21]{Aufsichtsbehörden} verlangen für Maßnahmen, die sie aufgrund eines
   Amtshilfeersuchens getroffen haben, keine Gebühren. Die \hyperref[itm:04-21]{Aufsichtsbehörden} können untereinander
   Regeln vereinbaren, um einander in Ausnahmefällen besondere aufgrund der Amtshilfe entstandene Ausgaben zu
   erstatten.%
  \label{itm:61-7}

  \item Erteilt eine ersuchte \hyperref[itm:04-21]{Aufsichtsbehörde} nicht binnen eines Monats nach Eingang des
   Ersuchens einer anderen
   \hyperref[itm:04-21]{Aufsichtsbehörde} die Informationen gemäß \hyperref[itm:61-5]{Absatz 5}, so kann die
    ersuchende \hyperref[itm:04-21]{Aufsichtsbehörde} eine einstweilige Maßnahme im Hoheitsgebiet ihres Mitgliedstaats
    gemäß \hyperref[itm:55-1]{Artikel 55 Absatz 1} ergreifen. In diesem Fall wird von einem dringenden Handlungsbedarf
    gemäß \hyperref[itm:66-1]{Artikel 66 Absatz 1} ausgegangen, der einen im Dringlichkeitsverfahren angenommenen
    verbindlichen Beschluss des Ausschuss gemäß \hyperref[itm:66-2]{Artikel 66 Absatz 2} erforderlich macht.%
  \label{itm:61-8}

  \item Die Kommission kann im Wege von Durchführungsrechtsakten Form und Verfahren der Amtshilfe nach diesem Artikel
   und die Ausgestaltung des elektronischen Informationsaustauschs zwischen den \hyperref[itm:04-21]
   {Aufsichtsbehörden} sowie zwischen den
   \hyperref[itm:04-21]{Aufsichtsbehörden} und dem Ausschuss, insbesondere das in \hyperref[itm:61-1]{Absatz 6 des
    vorliegenden Artikels} genannte standardisierte Format, festlegen. Diese Durchführungsrechtsakte werden gemäß dem
    in \hyperref[itm:93-2]{Artikel 93 Absatz 2} genannten Prüfverfahren erlassen.%
  \label{itm:61-9}

\end{enumerate}

% \addsec{Ergänzende Hinweise}

