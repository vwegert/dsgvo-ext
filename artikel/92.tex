%!TEX root = ../DSGVO-Bearbeitung.tex
\chapter{Ausübung der Befugnisübertragung}
\label{ch:92}

\addsec{Text der Verordnung}

\begin{enumerate}

  \item Die Befugnis zum Erlass delegierter Rechtsakte wird der Kommission unter den in diesem Artikel festgelegten
   Bedingungen übertragen.
  \label{itm:92-1}

  \item Die Befugnis zum Erlass delegierter Rechtsakte gemäß \hyperref[itm:12-8]{Artikel 12 Absatz 8} und \hyperref
   [itm:43-8]{Artikel 43 Absatz 8} wird der Kommission auf unbestimmte Zeit ab dem 24. Mai 2016 übertragen.
  \label{itm:92-2}

  \item Die Befugnisübertragung gemäß \hyperref[itm:12-8]{Artikel 12 Absatz 8} und \hyperref[itm:43-8]{Artikel 43 Absatz
   8} kann vom Europäischen Parlament oder vom Rat jederzeit widerrufen werden. Der Beschluss über den Widerruf beendet
   die Übertragung der in diesem Beschluss angegebenen Befugnis. Er wird am Tag nach seiner Veröffentlichung im
   Amtsblatt der Europäischen Union oder zu einem im Beschluss über den Widerruf angegebenen späteren Zeitpunkt
   wirksam. Die Gültigkeit von delegierten Rechtsakten, die bereits in Kraft sind, wird von dem Beschluss über den
   Widerruf nicht berührt.
  \label{itm:92-3}

  \item Sobald die Kommission einen delegierten Rechtsakt erlässt, übermittelt sie ihn gleichzeitig dem Europäischen
   Parlament und dem Rat.
  \label{itm:92-4}

  \item Ein delegierter Rechtsakt, der gemäß \hyperref[itm:12-8]{Artikel 12 Absatz 8} und \hyperref[itm:43-8]{Artikel 43
   Absatz 8} erlassen wurde, tritt nur in Kraft, wenn weder das Europäische Parlament noch der Rat innerhalb einer
   Frist von drei Monaten nach Übermittlung dieses Rechtsakts an das Europäische Parlament und den Rat Einwände erhoben
   haben oder wenn vor Ablauf dieser Frist das Europäische Parlament und der Rat beide der Kommission mitgeteilt haben,
   dass sie keine Einwände erheben werden. Auf Veranlassung des Europäischen Parlaments oder des Rates wird diese Frist
   um drei Monate verlängert.
  \label{itm:92-5}

\end{enumerate}

\addsec{Eigene Notizen}

