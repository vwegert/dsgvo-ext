%!TEX root = ../DSGVO-ExtendedVersion.tex
\chapter{Stellungnahme des Ausschusses}
\label{ch:64}

\begin{enumerate}

  \item Der Ausschuss gibt eine Stellungnahme ab, wenn die zuständige \hyperref[itm:04-21]
   {Aufsichtsbehörde} beabsichtigt, eine der nachstehenden Maßnahmen zu erlassen. Zu diesem Zweck übermittelt die
   zuständige \hyperref[itm:04-21]{Aufsichtsbehörde} dem Ausschuss den Entwurf des Beschlusses, wenn dieser%
  \label{itm:64-1}

  \begin{enumerate}
  
    \item der Annahme einer Liste der Verarbeitungsvorgänge dient, die der Anforderung einer
     Datenschutz"=Folgenabschätzung gemäß \hyperref[itm:35-4]{Artikel 35 Absatz 4} unterliegen,%
    \label{itm:64-1a}

    \item eine Angelegenheit gemäß \hyperref[itm:40-7]{Artikel 40 Absatz 7} und damit die Frage betrifft, ob ein Entwurf
     von Verhaltensregeln oder eine Änderung oder Ergänzung von Verhaltensregeln mit dieser Verordnung in Einklang
     steht,%
    \label{itm:64-1b}

    \item der Billigung der Anforderungen für die Akkreditierung einer Stelle nach \hyperref[itm:41-3]{Artikel 41 Absatz
     3}, einer Zertifizierungsstelle nach \hyperref[itm:43-3]{Artikel 43 Absatz 3} oder der Kriterien für die
     Zertifizierung gemäß \hyperref[itm:42-5]{Artikel 42 Absatz 5} dient,%
    \label{itm:64-1c}

    \item der Festlegung von Standard"=Datenschutzklauseln gemäß \hyperref[itm:46-2d]{Artikel 46 Absatz 2 Buchstabe d}
     und \hyperref[itm:28-8]{Artikel 28 Absatz 8} dient,%
    \label{itm:64-1d}

    \item der Genehmigung von Vertragsklauseln gemäß \hyperref[itm:46-3a]{Artikels 46 Absatz 3 Buchstabe a} dient, oder%
    \label{itm:64-1e}

    \item der Annahme verbindlicher interner Vorschriften im Sinne von \hyperref[ch:47]{Artikel 47} dient.%
    \label{itm:64-1f}

  \end{enumerate}

  \item Jede \hyperref[itm:04-21]{Aufsichtsbehörde}, der Vorsitz des Ausschuss oder die Kommission können beantragen,
   dass eine Angelegenheit mit allgemeiner Geltung oder mit Auswirkungen in mehr als einem Mitgliedstaat vom Ausschuss
   geprüft wird, um eine Stellungnahme zu erhalten, insbesondere wenn eine zuständige \hyperref[itm:04-21]
   {Aufsichtsbehörde} den Verpflichtungen zur Amtshilfe gemäß \hyperref[ch:61]{Artikel 61} oder zu gemeinsamen
   Maßnahmen gemäß \hyperref[ch:62]{Artikel 62} nicht nachkommt.%
  \label{itm:64-2}

  \item In den in den Absätzen \hyperref[itm:64-1]{1} und \hyperref[itm:64-2]{2} genannten Fällen gibt der Ausschuss
   eine Stellungnahme zu der Angelegenheit ab, die ihm vorgelegt wurde, sofern er nicht bereits eine Stellungnahme zu
   derselben Angelegenheit abgegeben hat. Diese Stellungnahme wird binnen acht Wochen mit der einfachen Mehrheit der
   Mitglieder des Ausschusses angenommen. Diese Frist kann unter Berücksichtigung der Komplexität der Angelegenheit um
   weitere sechs Wochen verlängert werden. Was den in \hyperref[itm:64-1]{Absatz 1} genannten Beschlussentwurf angeht,
   der gemäß \hyperref[itm:64-5]{Absatz 5} den Mitgliedern des Ausschusses übermittelt wird, so wird angenommen, dass
   ein Mitglied, das innerhalb einer vom Vorsitz angegebenen angemessenen Frist keine Einwände erhoben hat, dem
   Beschlussentwurf zustimmt.%
  \label{itm:64-3}

  \item Die \hyperref[itm:04-21]{Aufsichtsbehörden} und die Kommission übermitteln unverzüglich dem Ausschuss auf
   elektronischem Wege unter Verwendung eines standardisierten Formats alle zweckdienlichen Informationen,
   einschließlich -- je nach Fall -- einer kurzen Darstellung des Sachverhalts, des Beschlussentwurfs, der Gründe,
   warum eine solche Maßnahme ergriffen werden muss, und der Standpunkte anderer \hyperref[itm:04-22]
   {betroffener Aufsichtsbehörden}.%
  \label{itm:64-4}

  \item Der Vorsitz des Ausschusses unterrichtet unverzüglich auf elektronischem Wege%
  \label{itm:64-5}

  \begin{enumerate}
  
    \item unter Verwendung eines standardisierten Formats die Mitglieder des Ausschusses und die Kommission über alle
     zweckdienlichen Informationen, die ihm zugegangen sind. Soweit erforderlich stellt das Sekretariat des Ausschusses
     Übersetzungen der zweckdienlichen Informationen zur Verfügung und%
    \label{itm:64-5a}

    \item je nach Fall die in den Absätzen \hyperref[itm:64-1]{1} und \hyperref[itm:64-2]{2} genannte \hyperref
     [itm:04-21]{Aufsichtsbehörde} und die Kommission über die Stellungnahme und veröffentlicht sie.%
    \label{itm:64-5b}

  \end{enumerate}

  \item Die in \hyperref[itm:64-1]{Absatz 1} genannte zuständige \hyperref[itm:04-21]{Aufsichtsbehörde} nimmt den in
   \hyperref[itm:64-1]{Absatz 1} genannten Beschlussentwurf nicht vor Ablauf der in \hyperref[itm:64-3]{Absatz 3}
    genannten Frist an.%
  \label{itm:64-6}

  \item Die in \hyperref[itm:64-1]{Absatz 1} genannte \hyperref[itm:04-21]{Aufsichtsbehörde} trägt der Stellungnahme des
   Ausschusses weitestgehend Rechnung und teilt dessen Vorsitz binnen zwei Wochen nach Eingang der Stellungnahme auf
   elektronischem Wege unter Verwendung eines standardisierten Formats mit, ob sie den Beschlussentwurf beibehalten
   oder ändern wird; gegebenenfalls übermittelt sie den geänderten Beschlussentwurf.%
  \label{itm:64-7}

  \item Teilt in \hyperref[itm:64-1]{Absatz 1} genannte die zuständige Aufsichtsbehörde dem Vorsitz des Ausschusses
   innerhalb der Frist nach \hyperref[itm:64-7]{Absatz 7} des vorliegenden Artikels unter Angabe der maßgeblichen
   Gründe mit, dass sie beabsichtigt, der Stellungnahme des Ausschusses insgesamt oder teilweise nicht zu folgen, so
   gilt \hyperref[itm:65-2]{Artikel 65 Absatz 1}.%
  \label{itm:64-8}

\end{enumerate}

% \addsec{Ergänzende Hinweise}

