%!TEX root = ../DSGVO-Bearbeitung.tex
\chapter{Aufgaben des Ausschusses}
\label{ch:70}

\addsec{Text der Verordnung}

\begin{enumerate}

  \item Der Ausschuss stellt die einheitliche Anwendung dieser Verordnung sicher. Hierzu nimmt der Ausschuss von sich
   aus oder gegebenenfalls auf Ersuchen der Kommission insbesondere folgende Tätigkeiten wahr:
  \label{itm:70-1}

  \begin{enumerate}
  
    \item Überwachung und Sicherstellung der ordnungsgemäßen Anwendung dieser Verordnung in den in den Artikeln
     \hyperref[ch:64]{64} und \hyperref[ch:65]{65} genannten Fällen unbeschadet der Aufgaben der nationalen
      \hyperref[itm:04-21]{Aufsichtsbehörden};
    \label{itm:70-1a}

    \item Beratung der Kommission in allen Fragen, die im Zusammenhang mit dem Schutz \hyperref[itm:04-1]{personenbezogener Daten} in der
     Union stehen, einschließlich etwaiger Vorschläge zur Änderung dieser Verordnung;
    \label{itm:70-1b}

    \item Beratung der Kommission über das Format und die Verfahren für den Austausch von Informationen zwischen den
     Verantwortlichen, den Auftragsverarbeitern und den \hyperref[itm:04-21]{Aufsichtsbehörden} in Bezug auf \hyperref[itm:04-20]{verbindliche interne
     Datenschutzvorschriften};
    \label{itm:70-1c}

    \item Bereitstellung von Leitlinien, Empfehlungen und bewährten Verfahren zu Verfahren für die Löschung gemäß
     \hyperref[itm:17-2]{Artikel 17 Absatz 2} von Links zu \hyperref[itm:04-1]{personenbezogenen Daten} oder Kopien oder Replikationen dieser
     Daten aus öffentlich zugänglichen Kommunikationsdiensten;
    \label{itm:70-1d}

    \item Prüfung -- von sich aus, auf Antrag eines seiner Mitglieder oder auf Ersuchen der Kommission -- von die
     Anwendung dieser Verordnung betreffenden Fragen und Bereitstellung von Leitlinien, Empfehlungen und bewährten
     Verfahren zwecks Sicherstellung einer einheitlichen Anwendung dieser Verordnung;
    \label{itm:70-1e}

    \item Bereitstellung von Leitlinien, Empfehlungen und bewährten Verfahren gemäß \hyperref[itm:70-1e]{Buchstabe e des
     vorliegenden Absatzes} zur näheren Bestimmung der Kriterien und Bedingungen für die auf \hyperref[itm:04-4]{Profiling} beruhenden
     Entscheidungen gemäß \hyperref[itm:22-2]{Artikel 22 Absatz 2};
    \label{itm:70-1f}

    \item Bereitstellung von Leitlinien, Empfehlungen und bewährten Verfahren gemäß \hyperref[itm:70-1e]{Buchstabe e des
     vorliegenden Absatzes} für die Feststellung von \hyperref[itm:04-12]{Verletzungen des Schutzes personenbezogener Daten} und die
     Festlegung der Unverzüglichkeit im Sinne des \hyperref[ch:33]{Artikels 33} Absätze \hyperref[itm:33-1]{1} und
     \hyperref[itm:33-2]{2}, und zu den spezifischen Umständen, unter denen der Verantwortliche oder der
     Auftragsverarbeiter die \hyperref[itm:04-12]{Verletzung des Schutzes personenbezogener Daten} zu melden hat;
    \label{itm:70-1g}

    \item Bereitstellung von Leitlinien, Empfehlungen und bewährten Verfahren gemäß \hyperref[itm:70-1e]{Buchstabe e des
     vorliegenden Absatzes} zu den Umständen, unter denen eine \hyperref[itm:04-12]{Verletzung des Schutzes personenbezogener Daten}
     voraussichtlich ein hohes Risiko für die Rechte und Freiheiten natürlicher Personen im Sinne des \hyperref
     [itm:34-1]{Artikels 34 Absatz 1} zur Folge hat;
    \label{itm:70-1h}

    \item Bereitstellung von Leitlinien, Empfehlungen und bewährten Verfahren gemäß \hyperref[itm:70-1e]{Buchstabe e des
     vorliegenden Absatzes} zur näheren Bestimmung der in \hyperref[ch:47]{Artikel 47} aufgeführten Kriterien und
     Anforderungen für die Übermittlungen \hyperref[itm:04-1]{personenbezogener Daten}, die auf \hyperref[itm:04-20]{verbindlichen internen
     Datenschutzvorschriften} von Verantwortlichen oder Auftragsverarbeitern beruhen, und der dort aufgeführten weiteren
     erforderlichen Anforderungen zum Schutz \hyperref[itm:04-1]{personenbezogener Daten} der \hyperref[itm:04-1]{betroffenen Personen};
    \label{itm:70-1i}

    \item Bereitstellung von Leitlinien, Empfehlungen und bewährten Verfahren gemäß \hyperref[itm:70-1e]{Buchstabe e des
     vorliegenden Absatzes} zur näheren Bestimmung der Kriterien und Bedingungen für die Übermittlungen
     \hyperref[itm:04-1]{personenbezogener Daten} gemäß \hyperref[itm:49-1-1]{Artikel 49 Absatz 1};
    \label{itm:70-1j}

    \item Ausarbeitung von Leitlinien für die \hyperref[itm:04-21]{Aufsichtsbehörden} in Bezug auf die Anwendung von Maßnahmen nach \hyperref
     [ch:58]{Artikel 58} Absätze \hyperref[itm:58-1]{1}, \hyperref[itm:58-2]{2} und \hyperref[itm:58-3]{3} und die
     Festsetzung von Geldbußen gemäß \hyperref[ch:83]{Artikel 83};
    \label{itm:70-1k}

    \item Überprüfung der praktischen Anwendung der unter den Buchstaben \hyperref[itm:70-1e]{e} und \hyperref
     [itm:70-1f]{f} genannten Leitlinien, Empfehlungen und bewährten Verfahren;
    \label{itm:70-1l}

    \item Bereitstellung von Leitlinien, Empfehlungen und bewährten Verfahren gemäß \hyperref[itm:70-1e]{Buchstabe e des
     vorliegenden Absatzes} zur Festlegung gemeinsamer Verfahren für die von natürlichen Personen vorgenommene Meldung
     von Verstößen gegen diese Verordnung gemäß \hyperref[itm:54-2]{Artikel 54 Absatz 2};
    \label{itm:70-1m}

    \item Förderung der Ausarbeitung von Verhaltensregeln und der Einrichtung von datenschutzspezifischen
     Zertifizierungsverfahren sowie Datenschutzsiegeln und -prüfzeichen gemäß den Artikeln \hyperref[ch:40]{40} und
     \hyperref[ch:42]{42};
    \label{itm:70-1n}

    \item Akkreditierung von Zertifizierungsstellen und deren regelmäßige Überprüfung gemäß \hyperref[ch:43]{Artikel 43}
     und Führung eines öffentlichen Registers der akkreditierten Einrichtungen gemäß \hyperref[itm:43-6]{Artikel 43
     Absatz 6} und der in Drittländern niedergelassenen akkreditierten Verantwortlichen oder Auftragsverarbeiter
     gemäß \hyperref[itm:42-7]{Artikel 42 Absatz 7};
    \label{itm:70-1o}

    \item Präzisierung der in \hyperref[itm:43-3]{Artikel 43 Absatz 3} genannten Anforderungen im Hinblick auf die
     Akkreditierung von Zertifizierungsstellen gemäß \hyperref[ch:42]{Artikel 42};
    \label{itm:70-1p}

    \item Abgabe einer Stellungnahme für die Kommission zu den Zertifizierungsanforderungen gemäß \hyperref[itm:43-8]
     {Artikel 43 Absatz 8};
    \label{itm:70-1q}

    \item Abgabe einer Stellungnahme für die Kommission zu den Bildsymbolen gemäß \hyperref[itm:12-7]{Artikel 12 Absatz
     7};
    \label{itm:70-1r}

    \item Abgabe einer Stellungnahme für die Kommission zur Beurteilung der Angemessenheit des in einem Drittland oder
     einer internationalen Organisation gebotenen Schutzniveaus einschließlich zur Beurteilung der Frage, ob das
     Drittland, das Gebiet, ein oder mehrere spezifische Sektoren in diesem Drittland oder eine \hyperref[itm:04-29]{internationale
     Organisation} kein angemessenes Schutzniveau mehr gewährleistet. Zu diesem Zweck gibt die Kommission dem Ausschuss
     alle erforderlichen Unterlagen, darunter den Schriftwechsel mit der Regierung des Drittlands, dem Gebiet oder
     spezifischen Sektor oder der internationalen Organisation;
    \label{itm:70-1s}

    \item Abgabe von Stellungnahmen im Kohärenzverfahren gemäß \hyperref[itm:64-1]{Artikel 64 Absatz 1} zu
     Beschlussentwürfen von \hyperref[itm:04-21]{Aufsichtsbehörden}, zu Angelegenheiten, die nach \hyperref[itm:64-2]{Artikel 64 Absatz 2}
     vorgelegt wurden und um Erlass verbindlicher Beschlüsse gemäß \hyperref[ch:65]{Artikel 65}, einschließlich der
     in \hyperref[ch:66]{Artikel 66} genannten Fälle;
    \label{itm:70-1t}

    \item Förderung der Zusammenarbeit und eines wirksamen bilateralen und multilateralen Austauschs von Informationen
     und bewährten Verfahren zwischen den \hyperref[itm:04-21]{Aufsichtsbehörden};
    \label{itm:70-1u}

    \item Förderung von Schulungsprogrammen und Erleichterung des Personalaustausches zwischen \hyperref[itm:04-21]{Aufsichtsbehörden} sowie
     gegebenenfalls mit \hyperref[itm:04-21]{Aufsichtsbehörden} von Drittländern oder mit internationalen Organisationen;
    \label{itm:70-1v}

    \item Förderung des Austausches von Fachwissen und von Dokumentationen über Datenschutzvorschriften und -praxis mit
     Datenschutz\hyperref[itm:04-21]{aufsichtsbehörden} in aller Welt;
    \label{itm:70-1w}

    \item Abgabe von Stellungnahmen zu den auf Unionsebene erarbeiteten Verhaltensregeln gemäß \hyperref[itm:40-9]
     {Artikel 40 Absatz 9} und
    \label{itm:70-1x}

    \item Führung eines öffentlich zugänglichen elektronischen Registers der Beschlüsse der \hyperref[itm:04-21]{Aufsichtsbehörden} und
     Gerichte in Bezug auf Fragen, die im Rahmen des Kohärenzverfahrens behandelt wurden.
    \label{itm:70-1y}

  \end{enumerate}

  \item Die Kommission kann, wenn sie den Ausschuss um Rat ersucht, unter Berücksichtigung der Dringlichkeit des
   Sachverhalts eine Frist angeben.
  \label{itm:70-2}

  \item Der Ausschuss leitet seine Stellungnahmen, Leitlinien, Empfehlungen und bewährten Verfahren an die Kommission
   und an den in \hyperref[ch:93]{Artikel 93} genannten Ausschuss weiter und veröffentlicht sie.
  \label{itm:70-3}

  \item Der Ausschuss konsultiert gegebenenfalls interessierte Kreise und gibt ihnen Gelegenheit, innerhalb einer
   angemessenen Frist Stellung zu nehmen. Unbeschadet des \hyperref[ch:76]{Artikels 76} macht der Ausschuss die
   Ergebnisse der Konsultation der Öffentlichkeit zugänglich.
  \label{itm:70-4}

\end{enumerate}

\addsec{Eigene Notizen}

