%!TEX root = ../DSGVO-Bearbeitung.tex
\chapter{Zertifizierungsstellen}
\label{ch:43}

\addsec{Text der Verordnung}

\begin{enumerate}

  \item Unbeschadet der Aufgaben und Befugnisse der zuständigen \hyperref[itm:04-21]{Aufsichtsbehörde} gemäß den
   Artikeln \hyperref[ch:57]{57} und \hyperref[ch:58]{58} erteilen oder verlängern Zertifizierungsstellen, die über das
   geeignete Fachwissen hinsichtlich des Datenschutzes verfügen, nach Unterrichtung der \hyperref[itm:04-21]
   {Aufsichtsbehörde} -- damit diese erforderlichenfalls von ihren Befugnissen gemäß \hyperref[itm:58-2h]{Artikel 58
   Absatz 2 Buchstabe h} Gebrauch machen kann -- die Zertifizierung. Die Mitgliedstaaten stellen sicher, dass diese
   Zertifizierungsstellen von einer oder beiden der folgenden Stellen akkreditiert werden:
  \label{itm:43-1}

  \begin{enumerate}
  
    \item der gemäß Artikel \hyperref[ch:55]{55} oder \hyperref[ch:56]{56} zuständigen \hyperref[itm:04-21]
     {Aufsichtsbehörde};
    \label{itm:43-1a}

    \item der nationalen Akkreditierungsstelle, die gemäß der Verordnung (EG) Nr. 765/2008 des Europäischen Parlaments
     und des Rate im Einklang mit EN-ISO/IEC 17065/2012 und mit den zusätzlichen von der gemäß Artikel \hyperref[ch:55]
     {55} oder \hyperref[ch:56]{56} zuständigen \hyperref[itm:04-21]{Aufsichtsbehörde} festgelegten Anforderungen
     benannt wurde.\todo{Verordnungen nachschlagen}
    \label{itm:43-1b}

  \end{enumerate}

  \item Zertifizierungsstellen nach \hyperref[itm:43-1]{Absatz 1} dürfen nur dann gemäß dem genannten Absatz
   akkreditiert werden, wenn sie
  \label{itm:43-2}

  \begin{enumerate}
  
    \item ihre Unabhängigkeit und ihr Fachwissen hinsichtlich des Gegenstands der Zertifizierung zur Zufriedenheit der
     zuständigen \hyperref[itm:04-21]{Aufsichtsbehörde} nachgewiesen haben;
    \label{itm:43-2a}

    \item sich verpflichtet haben, die Kriterien nach \hyperref[itm:42-5]{Artikel 42 Absatz 5}, die von der gemäß
     Artikel \hyperref[ch:55]{55} oder \hyperref[ch:56]{56} zuständigen \hyperref[itm:04-21]{Aufsichtsbehörde} oder --
     gemäß \hyperref[ch:63]{Artikel 63} -- von dem Ausschuss genehmigt wurden, einzuhalten;
    \label{itm:43-2b}

    \item Verfahren für die Erteilung, die regelmäßige Überprüfung und den Widerruf der Datenschutzzertifizierung sowie
     der Datenschutzsiegel und -prüfzeichen festgelegt haben;
    \label{itm:43-2c}

    \item Verfahren und Strukturen festgelegt haben, mit denen sie Beschwerden über Verletzungen der Zertifizierung oder
     die Art und Weise, in der die Zertifizierung von dem \hyperref[itm:04-7]{Verantwortlichen} oder dem \hyperref
     [itm:04-8]{Auftragsverarbeiter} umgesetzt wird oder wurde, nachgehen und diese Verfahren und Strukturen
     für \hyperref[itm:04-1]{betroffene Personen} und die Öffentlichkeit transparent machen, und
    \label{itm:43-2d}

    \item zur Zufriedenheit der zuständigen \hyperref[itm:04-21]{Aufsichtsbehörde} nachgewiesen haben, dass ihre
     Aufgaben und Pflichten nicht zu einem Interessenkonflikt führen.
    \label{itm:43-2e}

  \end{enumerate}

  \item Die Akkreditierung von Zertifizierungsstellen nach den Absätzen \hyperref[itm:43-1]{1} und \hyperref[itm:43-2]
   {2} erfolgt anhand der Kriterien, die von der gemäß Artikel \hyperref[ch:55]{55} oder \hyperref[ch:56]
   {56} zuständigen \hyperref[itm:04-21]{Aufsichtsbehörde} oder -- gemäß \hyperref[ch:63]{Artikel 63} -- von dem
   Ausschuss genehmigt wurden. Im Fall einer Akkreditierung nach \hyperref[itm:43-1b]{Absatz 1 Buchstabe b des
   vorliegenden Artikels} ergänzen diese Anforderungen diejenigen, die in der Verordnung (EG) Nr. 765/2008\todo
   {Verordnung nachschlagen} und in den technischen Vorschriften, in denen die Methoden und Verfahren der
   Zertifizierungsstellen beschrieben werden, vorgesehen sind.
  \label{itm:43-3}

  \item Die Zertifizierungsstellen nach \hyperref[itm:43-1]{Absatz 1} sind unbeschadet der Verantwortung, die der
   \hyperref[itm:04-7]{Verantwortliche} oder der \hyperref[itm:04-8]{Auftragsverarbeiter} für die Einhaltung dieser
    Verordnung hat, für die angemessene Bewertung, die der Zertifizierung oder dem Widerruf einer Zertifizierung
    zugrunde liegt, verantwortlich. Die Akkreditierung wird für eine Höchstdauer von fünf Jahren erteilt und kann unter
    denselben Bedingungen verlängert werden, sofern die Zertifizierungsstelle die Anforderungen dieses Artikels
    erfüllt.
  \label{itm:43-4}

  \item Die Zertifizierungsstellen nach \hyperref[itm:43-1]{Absatz 1} teilen den zuständigen \hyperref[itm:04-21]
   {Aufsichtsbehörden} die Gründe für die Erteilung oder den Widerruf der beantragten Zertifizierung mit.
  \label{itm:43-5}

  \item Die Anforderungen nach \hyperref[itm:43-3]{Absatz 3 des vorliegenden Artikels} und die Kriterien nach A\hyperref
   [itm:42-5]{rtikel 42 Absatz 5} werden von der \hyperref[itm:04-21]{Aufsichtsbehörde} in leicht zugänglicher Form
   veröffentlicht. Die
   \hyperref[itm:04-21]{Aufsichtsbehörden} übermitteln diese Anforderungen und Kriterien auch dem Ausschuss. Der
    Ausschuss nimmt alle Zertifizierungsverfahren und Datenschutzsiegel in ein Register auf und veröffentlicht sie in
    geeigneter Weise.
  \label{itm:43-6}

  \item Unbeschadet des \hyperref[part:8]{Kapitels VIII} widerruft die zuständige \hyperref[itm:04-21]
   {Aufsichtsbehörde} oder die nationale Akkreditierungsstelle die Akkreditierung einer Zertifizierungsstelle
   nach \hyperref[itm:43-1]{Absatz 1}, wenn die Voraussetzungen für die Akkreditierung nicht oder nicht mehr erfüllt
   sind oder wenn eine Zertifizierungsstelle Maßnahmen ergreift, die nicht mit dieser Verordnung vereinbar sind.
  \label{itm:43-7}

  \item Der Kommission wird die Befugnis übertragen, gemäß \hyperref[ch:92]{Artikel 92} delegierte Rechtsakte zu
   erlassen, um die Anforderungen festzulegen, die für die in \hyperref[itm:42-1]{Artikel 42 Absatz 1} genannten
   datenschutzspezifischen Zertifizierungsverfahren zu berücksichtigen sind.
  \label{itm:43-8}

  \item Die Kommission kann Durchführungsrechtsakte erlassen, mit denen technische Standards für
   Zertifizierungsverfahren und Datenschutzsiegel und -prüfzeichen sowie Mechanismen zur Förderung und Anerkennung
   dieser Zertifizierungsverfahren und Datenschutzsiegel und -prüfzeichen festgelegt werden. Diese
   Durchführungsrechtsakte werden gemäß dem in \hyperref[itm:93-2]{Artikel 93 Absatz 2} genannten Prüfverfahren
   erlassen.
  \label{itm:43-9}

\end{enumerate}

\addsec{Eigene Notizen}

