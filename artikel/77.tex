%!TEX root = ../DSGVO-Bearbeitung.tex
\chapter{Recht auf Beschwerde bei einer Aufsichtsbehörde}
\label{ch:77}

\addsec{Text der Verordnung}

\begin{enumerate}

  \item Jede \hyperref[itm:04-1]{betroffene Person} hat unbeschadet eines anderweitigen verwaltungsrechtlichen oder
   gerichtlichen Rechtsbehelfs das Recht auf Beschwerde bei einer \hyperref[itm:04-21]{Aufsichtsbehörde}, insbesondere
   in dem Mitgliedstaat ihres Aufenthaltsorts, ihres Arbeitsplatzes oder des Orts des mutmaßlichen Verstoßes, wenn
   die \hyperref[itm:04-1]{betroffene Person} der Ansicht ist, dass die \hyperref[itm:04-2]{Verarbeitung} der sie
   betreffenden \hyperref[itm:04-1]{personenbezogenen Daten} gegen diese Verordnung verstößt.
  \label{itm:77-1}

  \item Die \hyperref[itm:04-21]{Aufsichtsbehörde}, bei der die Beschwerde eingereicht wurde, unterrichtet den
   Beschwerdeführer über den Stand und die Ergebnisse der Beschwerde einschließlich der Möglichkeit eines gerichtlichen
   Rechtsbehelfs nach \hyperref[ch:78]{Artikel 78}.
  \label{itm:77-2}

\end{enumerate}

\addsec{Eigene Notizen}

