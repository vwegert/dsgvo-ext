%!TEX root = ../DSGVO-Bearbeitung.tex

% ---------------------------------------------------------------------------------------------------------------------
% Dokumentenklasse
\documentclass[
  oneside,
  a4paper,
  10pt,
  DIV=11,
  parskip=half,
  numbers=noenddot,
  chapterprefix=false,
  headings=small
]{scrbook}


% ---------------------------------------------------------------------------------------------------------------------
% Makro-Hilfen
\usepackage{xspace}
\xspaceaddexceptions{"=}


% ---------------------------------------------------------------------------------------------------------------------
% Eingabekodierung
\usepackage[utf8]{inputenc}
\usepackage[T1]{fontenc}


% ---------------------------------------------------------------------------------------------------------------------
% Spracheinstellungen
\usepackage[ngerman]{babel}
\usepackage[german=german-x-latest]{hyphsubst} % bessere Silbentrennung
\usepackage{csquotes}


% ---------------------------------------------------------------------------------------------------------------------
% Strukturierung
\renewcaptionname{ngerman}{\chaptername}{Artikel}
\renewcommand*{\chapterformat}{%
  \mbox{\chapapp{\nobreakspace}\thechapter{} -- }}
\RedeclareSectionCommand[style=section]{chapter}


% ---------------------------------------------------------------------------------------------------------------------
% Aufzählungen
\usepackage{enumitem}
\setlist[enumerate, 1]{label = (\arabic*)}


% ---------------------------------------------------------------------------------------------------------------------
% Typographie
\usepackage{microtype}
\hbadness=10001
\vbadness=10001
\usepackage{ulem}


% ---------------------------------------------------------------------------------------------------------------------
% Hyperlinks und Querverweise
\usepackage[  
  colorlinks=true,
  linkcolor=blue,
  bookmarksnumbered=true,
  bookmarksdepth=9
]{hyperref}
\def\UrlBreaks{\do\/\do-}% Zeilenumbrüche in langen URLs

\usepackage[german]{fancyref} 

% weiterhin zu fancyref: Seitenzahl nur angeben, wenn mindestens 5 Seiten Abstand sind
% siehe http://tex.stackexchange.com/questions/136725/how-to-change-fancyrefs-page-numbering-behaviour
\renewcommand*{\reftextfaceafter}{\unskip}
\renewcommand*{\reftextafter}{\unskip}
\renewcommand*{\reftextfacebefore}{\unskip}
\renewcommand*{\reftextbefore}{\unskip}
\makeatletter
\let\saved@reftextfaraway\reftextfaraway
\renewcommand*{\reftextfaraway}[1]{%
  \begingroup
    \def\ref@unknown@value{??}%
    \ifx\@tempa\ref@unknown@value
      \count@=0 %
    \else
      \count@\thevpagerefnum\relax
      \advance\count@ by -\@tempa\relax
      \ifnum\count@<0 \count@=-\count@\fi
    \fi
    \ifnum\count@<5 %
      \unskip
    \else
      \saved@reftextfaraway{#1}%
    \fi
  \endgroup
}
\makeatother


% ---------------------------------------------------------------------------------------------------------------------
% Verzeichnisse, Glossar und Abkürzungsverzeichnis
\usepackage{imakeidx}                  % Indexgenerierung + Eintrag in TOC
%\setcounter{secnumdepth}{4}       % Wie "tief" sollen Nummern vergeben werden?
\setcounter{tocdepth}{0} % only parts and chapters


% ---------------------------------------------------------------------------------------------------------------------
% Arbeitshilfen
\usepackage[german]{todonotes}


% ---------------------------------------------------------------------------------------------------------------------
% eigene logische Textauszeichnungen


