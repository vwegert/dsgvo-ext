%!TEX root = ../DSGVO-ExtendedVersion.tex

\begin{enumerate}

   \item Der Schutz natürlicher Personen bei der Verarbeitung personenbezogener Daten ist ein Grundrecht. Gemäß Artikel
    8 Absatz 1 der Charta der Grundrechte der Europäischen Union (im Folgenden „Charta“) sowie Artikel 16 Absatz 1 des
    Vertrags über die Arbeitsweise der Europäischen Union (AEUV) hat jede Person das Recht auf Schutz der sie
    betreffenden personenbezogenen Daten.%
   \label{itm:eg-1}
   
   \crossrefReasonToArticle{1}

   \item Die Grundsätze und Vorschriften zum Schutz natürlicher Personen bei der Verarbeitung ihrer personenbezogenen
    Daten sollten gewährleisten, dass ihre Grundrechte und Grundfreiheiten und insbesondere ihr Recht auf Schutz
    personenbezogener Daten ungeachtet ihrer Staatsangehörigkeit oder ihres Aufenthaltsorts gewahrt bleiben. Diese
    Verordnung soll zur Vollendung eines Raums der Freiheit, der Sicherheit und des Rechts und einer Wirtschaftsunion,
    zum wirtschaftlichen und sozialen Fortschritt, zur Stärkung und zum Zusammenwachsen der Volkswirtschaften innerhalb
    des Binnenmarkts sowie zum Wohlergehen natürlicher Personen beitragen.%
   \label{itm:eg-2}
   
   \crossrefReasonToArticle{2}

   \item Zweck der Richtlinie 95/46/EG des Europäischen Parlaments und des Rates\comment{Richtlinie 95/46/EG siehe des
    Europäischen Parlaments und des Rates vom 24. Oktober 1995 zum Schutz natürlicher Personen bei der Verarbeitung
    personenbezogener Daten und zum freien Datenverkehr, konsolidierte Fassung siehe \cite{ril-ds-alt}} ist die
    Harmonisierung der Vorschriften zum Schutz der Grundrechte und Grundfreiheiten natürlicher Personen bei der
    Datenverarbeitung sowie die Gewährleistung des freien Verkehrs personenbezogener Daten zwischen den
    Mitgliedstaaten.%
   \label{itm:eg-3}
   
   \crossrefReasonToArticle{3}

   \item Die Verarbeitung personenbezogener Daten sollte im Dienste der Menschheit stehen. Das Recht auf Schutz der
    personenbezogenen Daten ist kein uneingeschränktes Recht; es muss im Hinblick auf seine gesellschaftliche Funktion
    gesehen und unter Wahrung des Verhältnismäßigkeitsprinzips gegen andere Grundrechte abgewogen werden. Diese
    Verordnung steht im Einklang mit allen Grundrechten und achtet alle Freiheiten und Grundsätze, die mit der Charta
    anerkannt wurden und in den Europäischen Verträgen verankert sind, insbesondere Achtung des Privat- und
    Familienlebens, der Wohnung und der Kommunikation, Schutz personenbezogener Daten, Gedanken-, Gewissens- und
    Religionsfreiheit, Freiheit der Meinungsäußerung und Informationsfreiheit, unternehmerische Freiheit, Recht auf
    einen wirksamen Rechtsbehelf und ein faires Verfahren und Vielfalt der Kulturen, Religionen und Sprachen.%
   \label{itm:eg-4}
   
   \crossrefReasonToArticle{4}

   \item Die wirtschaftliche und soziale Integration als Folge eines funktionierenden Binnenmarkts hat zu einem
    deutlichen Anstieg des grenzüberschreitenden Verkehrs personenbezogener Daten geführt. Der unionsweite Austausch
    personenbezogener Daten zwischen öffentlichen und privaten Akteuren einschließlich natürlichen Personen,
    Vereinigungen und Unternehmen hat zugenommen. Das Unionsrecht verpflichtet die Verwaltungen der Mitgliedstaaten,
    zusammenzuarbeiten und personenbezogene Daten auszutauschen, damit sie ihren Pflichten nachkommen oder für eine
    Behörde eines anderen Mitgliedstaats Aufgaben durchführen können.%
   \label{itm:eg-5}
   
   \crossrefReasonToArticle{5}

   \item Rasche technologische Entwicklungen und die Globalisierung haben den Datenschutz vor neue Herausforderungen
    gestellt. Das Ausmaß der Erhebung und des Austauschs personenbezogener Daten hat eindrucksvoll zugenommen. Die
    Technik macht es möglich, dass private Unternehmen und Behörden im Rahmen ihrer Tätigkeiten in einem noch nie
    dagewesenen Umfang auf personenbezogene Daten zurückgreifen. Zunehmend machen auch natürliche Personen
    Informationen öffentlich weltweit zugänglich. Die Technik hat das wirtschaftliche und gesellschaftliche Leben
    verändert und dürfte den Verkehr personenbezogener Daten innerhalb der Union sowie die Datenübermittlung an
    Drittländer und internationale Organisationen noch weiter erleichtern, wobei ein hohes Datenschutzniveau zu
    gewährleisten ist.%
   \label{itm:eg-6}
   
   \crossrefReasonToArticle{6}

   \item Diese Entwicklungen erfordern einen soliden, kohärenteren und klar durchsetzbaren Rechtsrahmen im Bereich des
    Datenschutzes in der Union, da es von großer Wichtigkeit ist, eine Vertrauensbasis zu schaffen, die die digitale
    Wirtschaft dringend benötigt, um im Binnenmarkt weiter wachsen zu können. Natürliche Personen sollten die Kontrolle
    über ihre eigenen Daten besitzen. Natürliche Personen, Wirtschaft und Staat sollten in rechtlicher und praktischer
    Hinsicht über mehr Sicherheit verfügen.%
   \label{itm:eg-7}
   
   \crossrefReasonToArticle{7}

   \item Wenn in dieser Verordnung Präzisierungen oder Einschränkungen ihrer Vorschriften durch das Recht der
    Mitgliedstaaten vorgesehen sind, können die Mitgliedstaaten Teile dieser Verordnung in ihr nationales Recht
    aufnehmen, soweit dies erforderlich ist, um die Kohärenz zu wahren und die nationalen Rechtsvorschriften für die
    Personen, für die sie gelten, verständlicher zu machen.%
   \label{itm:eg-8}
   
   \crossrefReasonToArticle{8}

   \item Die Ziele und Grundsätze der Richtlinie 95/46/EG besitzen nach wie vor Gültigkeit, doch hat die Richtlinie
    nicht verhindern können, dass der Datenschutz in der Union unterschiedlich gehandhabt wird, Rechtsunsicherheit
    besteht oder in der Öffentlichkeit die Meinung weit verbreitet ist, dass erhebliche Risiken für den Schutz
    natürlicher Personen bestehen, insbesondere im Zusammenhang mit der Benutzung des Internets. Unterschiede beim
    Schutzniveau für die Rechte und Freiheiten von natürlichen Personen im Zusammenhang mit der Verarbeitung
    personenbezogener Daten in den Mitgliedstaaten, vor allem beim Recht auf Schutz dieser Daten, können den
    unionsweiten freien Verkehr solcher Daten behindern. Diese Unterschiede im Schutzniveau können daher ein Hemmnis
    für die unionsweite Ausübung von Wirtschaftstätigkeiten darstellen, den Wettbewerb verzerren und die Behörden an
    der Erfüllung der ihnen nach dem Unionsrecht obliegenden Pflichten hindern. Sie erklären sich aus den Unterschieden
    bei der Umsetzung und Anwendung der Richtlinie 95/46/EG.%
   \label{itm:eg-9}
   
   \crossrefReasonToArticle{9}

   \item Um ein gleichmäßiges und hohes Datenschutzniveau für natürliche Personen zu gewährleisten und die Hemmnisse für
    den Verkehr personenbezogener Daten in der Union zu beseitigen, sollte das Schutzniveau für die Rechte und
    Freiheiten von natürlichen Personen bei der Verarbeitung dieser Daten in allen Mitgliedstaaten gleichwertig sein.
    Die Vorschriften zum Schutz der Grundrechte und Grundfreiheiten von natürlichen Personen bei der Verarbeitung
    personenbezogener Daten sollten unionsweit gleichmäßig und einheitlich angewandt werden. Hinsichtlich der
    Verarbeitung personenbezogener Daten zur Erfüllung einer rechtlichen Verpflichtung oder zur Wahrnehmung einer
    Aufgabe, die im öffentlichen Interesse liegt oder in Ausübung öffentlicher Gewalt erfolgt, die dem Verantwortlichen
    übertragen wurde, sollten die Mitgliedstaaten die Möglichkeit haben, nationale Bestimmungen, mit denen die
    Anwendung der Vorschriften dieser Verordnung genauer festgelegt wird, beizubehalten oder einzuführen. In Verbindung
    mit den allgemeinen und horizontalen Rechtsvorschriften über den Datenschutz zur Umsetzung der Richtlinie 95/46/EG
    gibt es in den Mitgliedstaaten mehrere sektorspezifische Rechtsvorschriften in Bereichen, die spezifischere
    Bestimmungen erfordern. Diese Verordnung bietet den Mitgliedstaaten zudem einen Spielraum für die Spezifizierung
    ihrer Vorschriften, auch für die Verarbeitung besonderer Kategorien von personenbezogenen Daten
    (im Folgenden „sensible Daten“). Diesbezüglich schließt diese Verordnung nicht Rechtsvorschriften der
    Mitgliedstaaten aus, in denen die Umstände besonderer Verarbeitungssituationen festgelegt werden, einschließlich
    einer genaueren Bestimmung der Voraussetzungen, unter denen die Verarbeitung personenbezogener Daten rechtmäßig
    ist.%
   \label{itm:eg-10}
   
   \crossrefReasonToArticle{10}

   \item Ein unionsweiter wirksamer Schutz personenbezogener Daten erfordert die Stärkung und präzise Festlegung der
    Rechte der betroffenen Personen sowie eine Verschärfung der Verpflichtungen für diejenigen, die personenbezogene
    Daten verarbeiten und darüber entscheiden, ebenso wie — in den Mitgliedstaaten — gleiche Befugnisse bei der
    Überwachung und Gewährleistung der Einhaltung der Vorschriften zum Schutz personenbezogener Daten sowie gleiche
    Sanktionen im Falle ihrer Verletzung.%
   \label{itm:eg-11}
   
   \crossrefReasonToArticle{11}

   \item Artikel 16 Absatz 2 AEUV ermächtigt das Europäische Parlament und den Rat, Vorschriften über den Schutz
    natürlicher Personen bei der Verarbeitung personenbezogener Daten und zum freien Verkehr solcher Daten zu
    erlassen.%
   \label{itm:eg-12}
   
   \crossrefReasonToArticle{12}

   \item Damit in der Union ein gleichmäßiges Datenschutzniveau für natürliche Personen gewährleistet ist und
    Unterschiede, die den freien Verkehr personenbezogener Daten im Binnenmarkt behindern könnten, beseitigt werden,
    ist eine Verordnung erforderlich, die für die Wirtschaftsteilnehmer einschließlich Kleinstunternehmen sowie kleiner
    und mittlerer Unternehmen Rechtssicherheit und Transparenz schafft, natürliche Personen in allen Mitgliedstaaten
    mit demselben Niveau an durchsetzbaren Rechten ausstattet, dieselben Pflichten und Zuständigkeiten für die
    Verantwortlichen und Auftragsverarbeiter vorsieht und eine gleichmäßige Kontrolle der Verarbeitung
    personenbezogener Daten und gleichwertige Sanktionen in allen Mitgliedstaaten sowie eine wirksame Zusammenarbeit
    zwischen den Aufsichtsbehörden der einzelnen Mitgliedstaaten gewährleistet. Das reibungslose Funktionieren des
    Binnenmarkts erfordert, dass der freie Verkehr personenbezogener Daten in der Union nicht aus Gründen des Schutzes
    natürlicher Personen bei der Verarbeitung personenbezogener Daten eingeschränkt oder verboten wird. Um der
    besonderen Situation der Kleinstunternehmen sowie der kleinen und mittleren Unternehmen Rechnung zu tragen, enthält
    diese Verordnung eine abweichende Regelung hinsichtlich des Führens eines Verzeichnisses für Einrichtungen, die
    weniger als 250 Mitarbeiter beschäftigen. Außerdem werden die Organe und Einrichtungen der Union sowie die
    Mitgliedstaaten und deren Aufsichtsbehörden dazu angehalten, bei der Anwendung dieser Verordnung die besonderen
    Bedürfnisse von Kleinstunternehmen sowie von kleinen und mittleren Unternehmen zu berücksichtigen. Für die
    Definition des Begriffs „Kleinstunternehmen sowie kleine und mittlere Unternehmen“ sollte Artikel 2 des Anhangs zur
    Empfehlung 2003/361/EG der Kommission\comment{Empfehlung der Kommission vom 6. Mai 2003 betreffend die Definition
    der Kleinstunternehmen sowie der kleinen und mittleren Unternehmen (C (2003) 1422) siehe \cite{ek-kmu}} maßgebend
    sein.%
   \label{itm:eg-13}
   
   \crossrefReasonToArticle{13}

   \item Der durch diese Verordnung gewährte Schutz sollte für die Verarbeitung der personenbezogenen Daten natürlicher
    Personen ungeachtet ihrer Staatsangehörigkeit oder ihres Aufenthaltsorts gelten. Diese Verordnung gilt nicht für
    die Verarbeitung personenbezogener Daten juristischer Personen und insbesondere als juristische Person gegründeter
    Unternehmen, einschließlich Name, Rechtsform oder Kontaktdaten der juristischen Person.%
   \label{itm:eg-14}
   
   \crossrefReasonToArticle{14}

   \item Um ein ernsthaftes Risiko einer Umgehung der Vorschriften zu vermeiden, sollte der Schutz natürlicher Personen
    technologieneutral sein und nicht von den verwendeten Techniken abhängen. Der Schutz natürlicher Personen sollte
    für die automatisierte Verarbeitung personenbezogener Daten ebenso gelten wie für die manuelle Verarbeitung von
    personenbezogenen Daten, wenn die personenbezogenen Daten in einem Dateisystem gespeichert sind oder gespeichert
    werden sollen. Akten oder Aktensammlungen sowie ihre Deckblätter, die nicht nach bestimmten Kriterien geordnet
    sind, sollten nicht in den Anwendungsbereich dieser Verordnung fallen.%
   \label{itm:eg-15}
   
   \crossrefReasonToArticle{15}

   \item Diese Verordnung gilt nicht für Fragen des Schutzes von Grundrechten und Grundfreiheiten und des freien
    Verkehrs personenbezogener Daten im Zusammenhang mit Tätigkeiten, die nicht in den Anwendungsbereich des
    Unionsrechts fallen, wie etwa die nationale Sicherheit betreffende Tätigkeiten. Diese Verordnung gilt nicht für die
    von den Mitgliedstaaten im Rahmen der Gemeinsamen Außen- und Sicherheitspolitik der Union durchgeführte
    Verarbeitung personenbezogener Daten.%
   \label{itm:eg-16}
   
   \crossrefReasonToArticle{16}

   \item Die Verordnung (EG) Nr. 45/2001 des Europäischen Parlaments und des Rates\comment{Verordnung (EG) Nr. 45/2001
    des Europäischen Parlaments und des Rates vom 18. Dezember 2000 zum Schutz natürlicher Personen bei der
    Verarbeitung personenbezogener Daten durch die Organe und Einrichtungen der Gemeinschaft und zum freien
    Datenverkehr siehe \cite{vo-ds-fdv}} gilt für die Verarbeitung personenbezogener Daten durch die Organe,
    Einrichtungen, Ämter und Agenturen der Union. Die Verordnung (EG) Nr. 45/2001 und sonstige Rechtsakte der Union,
    die diese Verarbeitung personenbezogener Daten regeln, sollten an die Grundsätze und Vorschriften der vorliegenden
    Verordnung angepasst und im Lichte der vorliegenden Verordnung angewandt werden. Um einen soliden und kohärenten
    Rechtsrahmen im Bereich des Datenschutzes in der Union zu gewährleisten, sollten die erforderlichen Anpassungen der
    Verordnung (EG) Nr. 45/2001 im Anschluss an den Erlass der vorliegenden Verordnung vorgenommen werden, damit sie
    gleichzeitig mit der vorliegenden Verordnung angewandt werden können.%
   \label{itm:eg-17}
   
   \crossrefReasonToArticle{17}

   \item Diese Verordnung gilt nicht für die Verarbeitung von personenbezogenen Daten, die von einer natürlichen Person
    zur Ausübung ausschließlich persönlicher oder familiärer Tätigkeiten und somit ohne Bezug zu einer beruflichen oder
    wirtschaftlichen Tätigkeit vorgenommen wird. Als persönliche oder familiäre Tätigkeiten könnte auch das Führen
    eines Schriftverkehrs oder von Anschriftenverzeichnissen oder die Nutzung sozialer Netze und Online"=Tätigkeiten im
    Rahmen solcher Tätigkeiten gelten. Diese Verordnung gilt jedoch für die Verantwortlichen oder Auftragsverarbeiter,
    die die Instrumente für die Verarbeitung personenbezogener Daten für solche persönlichen oder familiären
    Tätigkeiten bereitstellen.%
   \label{itm:eg-18}
   
   \crossrefReasonToArticle{18}

   \item Der Schutz natürlicher Personen bei der Verarbeitung personenbezogener Daten durch die zuständigen Behörden zum
    Zwecke der Verhütung, Ermittlung, Aufdeckung oder Verfolgung von Straftaten oder der Strafvollstreckung,
    einschließlich des Schutzes vor und der Abwehr von Gefahren für die öffentliche Sicherheit, sowie der freie Verkehr
    dieser Daten sind in einem eigenen Unionsrechtsakt geregelt. Deshalb sollte diese Verordnung auf
    Verarbeitungstätigkeiten dieser Art keine Anwendung finden. Personenbezogene Daten, die von Behörden nach dieser
    Verordnung verarbeitet werden, sollten jedoch, wenn sie zu den vorstehenden Zwecken verwendet werden, einem
    spezifischeren Unionsrechtsakt, nämlich der Richtlinie (EU) 2016/680 des Europäischen Parlaments und des
    Rates\comment{Richtlinie (EU) 2016/680 des Europäischen Parlaments und des Rates vom 27. April 2016 zum Schutz
    natürlicher Personen bei der Verarbeitung personenbezogener Daten durch die zuständigen Behörden zum Zwecke der
    Verhütung, Aufdeckung, Untersuchung oder Verfolgung von Straftaten oder der Strafvollstreckung sowie zum freien
    Datenverkehr und zur Aufhebung des Rahmenbeschlusses 2000/383/JI des Rates, konsolidierte Fassung siehe 
    \cite{ril-ds-strafverfolgung}} unterliegen. Die Mitgliedstaaten können die zuständigen Behörden im Sinne der
     Richtlinie (EU) 2016/680 mit Aufgaben betrauen, die nicht zwangsläufig für die Zwecke der Verhütung, Ermittlung,
     Aufdeckung oder Verfolgung von Straftaten oder der Strafvollstreckung, einschließlich des Schutzes vor und der
     Abwehr von Gefahren für die öffentliche Sicherheit, ausgeführt werden, so dass die Verarbeitung von
     personenbezogenen Daten für diese anderen Zwecke insoweit in den Anwendungsbereich dieser Verordnung fällt, als
     sie in den Anwendungsbereich des Unionsrechts fällt. In Bezug auf die Verarbeitung personenbezogener Daten durch
     diese Behörden für Zwecke, die in den Anwendungsbereich dieser Verordnung fallen, sollten die Mitgliedstaaten
     spezifischere Bestimmungen beibehalten oder einführen können, um die Anwendung der Vorschriften dieser Verordnung
     anzupassen. In den betreffenden Bestimmungen können die Auflagen für die Verarbeitung personenbezogener Daten
     durch diese zuständigen Behörden für jene anderen Zwecke präziser festgelegt werden, wobei der verfassungsmäßigen,
     organisatorischen und administrativen Struktur des betreffenden Mitgliedstaats Rechnung zu tragen ist. Soweit
     diese Verordnung für die Verarbeitung personenbezogener Daten durch private Stellen gilt, sollte sie vorsehen,
     dass die Mitgliedstaaten einige Pflichten und Rechte unter bestimmten Voraussetzungen mittels Rechtsvorschriften
     beschränken können, wenn diese Beschränkung in einer demokratischen Gesellschaft eine notwendige und
     verhältnismäßige Maßnahme zum Schutz bestimmter wichtiger Interessen darstellt, wozu auch die öffentliche
     Sicherheit und die Verhütung, Ermittlung, Aufdeckung und Verfolgung von Straftaten oder die Strafvollstreckung
     zählen, einschließlich des Schutzes vor und der Abwehr von Gefahren für die öffentliche Sicherheit. Dies ist
     beispielsweise im Rahmen der Bekämpfung der Geldwäsche oder der Arbeit kriminaltechnischer Labors von Bedeutung.%
   \label{itm:eg-19}
   
   \crossrefReasonToArticle{19}

   \item Diese Verordnung gilt zwar unter anderem für die Tätigkeiten der Gerichte und anderer Justizbehörden, doch
    könnte im Unionsrecht oder im Recht der Mitgliedstaaten festgelegt werden, wie die Verarbeitungsvorgänge und
    Verarbeitungsverfahren bei der Verarbeitung personenbezogener Daten durch Gerichte und andere Justizbehörden im
    Einzelnen auszusehen haben. Damit die Unabhängigkeit der Justiz bei der Ausübung ihrer gerichtlichen Aufgaben
    einschließlich ihrer Beschlussfassung unangetastet bleibt, sollten die Aufsichtsbehörden nicht für die Verarbeitung
    personenbezogener Daten durch Gerichte im Rahmen ihrer justiziellen Tätigkeit zuständig sein. Mit der Aufsicht über
    diese Datenverarbeitungsvorgänge sollten besondere Stellen im Justizsystem des Mitgliedstaats betraut werden
    können, die insbesondere die Einhaltung der Vorschriften dieser Verordnung sicherstellen, Richter und Staatsanwälte
    besser für ihre Pflichten aus dieser Verordnung sensibilisieren und Beschwerden in Bezug auf derartige
    Datenverarbeitungsvorgänge bearbeiten sollten.%
   \label{itm:eg-20}
   
   \crossrefReasonToArticle{20}

   \item Die vorliegende Verordnung berührt nicht die Anwendung der Richtlinie 2000/31/EG des Europäischen Parlaments
    und des Rates\comment{Richtlinie 2000/31/EG des Europäischen Parlaments und des Rates vom 8. Juni 2000 über
    bestimmte rechtliche Aspekte der Dienste der Informationsgesellschaft, insbesondere des elektronischen
    Geschäftsverkehrs, im Binnenmarkt („Richtlinie über den elektronischen Geschäftsverkehr“) siehe \cite{ril-egv}} und
    insbesondere die der Vorschriften der Artikel 12 bis 15 jener Richtlinie zur Verantwortlichkeit von Anbietern
    reiner Vermittlungsdienste. Die genannte Richtlinie soll dazu beitragen, dass der Binnenmarkt einwandfrei
    funktioniert, indem sie den freien Verkehr von Diensten der Informationsgesellschaft zwischen den Mitgliedstaaten
    sicherstellt.%
   \label{itm:eg-21}
   
   \crossrefReasonToArticle{21}

   \item Jede Verarbeitung personenbezogener Daten im Rahmen der Tätigkeiten einer Niederlassung eines Verantwortlichen
    oder eines Auftragsverarbeiters in der Union sollte gemäß dieser Verordnung erfolgen, gleich, ob die Verarbeitung
    in oder außerhalb der Union stattfindet. Eine Niederlassung setzt die effektive und tatsächliche Ausübung einer
    Tätigkeit durch eine feste Einrichtung voraus. Die Rechtsform einer solchen Einrichtung, gleich, ob es sich um eine
    Zweigstelle oder eine Tochtergesellschaft mit eigener Rechtspersönlichkeit handelt, ist dabei nicht
    ausschlaggebend.%
   \label{itm:eg-22}
   
   \crossrefReasonToArticle{22}

   \item Damit einer natürlichen Person der gemäß dieser Verordnung gewährleistete Schutz nicht vorenthalten wird,
    sollte die Verarbeitung personenbezogener Daten von betroffenen Personen, die sich in der Union befinden, durch
    einen nicht in der Union niedergelassenen Verantwortlichen oder Auftragsverarbeiter dieser Verordnung unterliegen,
    wenn die Verarbeitung dazu dient, diesen betroffenen Personen gegen Entgelt oder unentgeltlich Waren oder
    Dienstleistungen anzubieten. Um festzustellen, ob dieser Verantwortliche oder Auftragsverarbeiter betroffenen
    Personen, die sich in der Union befinden, Waren oder Dienstleistungen anbietet, sollte festgestellt werden, ob der
    Verantwortliche oder Auftragsverarbeiter offensichtlich beabsichtigt, betroffenen Personen in einem oder mehreren
    Mitgliedstaaten der Union Dienstleistungen anzubieten. Während die bloße Zugänglichkeit der Website des
    Verantwortlichen, des Auftragsverarbeiters oder eines Vermittlers in der Union, einer E"=Mail"=Adresse oder anderer
    Kontaktdaten oder die Verwendung einer Sprache, die in dem Drittland, in dem der Verantwortliche niedergelassen
    ist, allgemein gebräuchlich ist, hierfür kein ausreichender Anhaltspunkt ist, können andere Faktoren wie die
    Verwendung einer Sprache oder Währung, die in einem oder mehreren Mitgliedstaaten gebräuchlich ist, in Verbindung
    mit der Möglichkeit, Waren und Dienstleistungen in dieser anderen Sprache zu bestellen, oder die Erwähnung von
    Kunden oder Nutzern, die sich in der Union befinden, darauf hindeuten, dass der Verantwortliche beabsichtigt, den
    Personen in der Union Waren oder Dienstleistungen anzubieten.%
   \label{itm:eg-23}
   
   \crossrefReasonToArticle{23}

   \item Die Verarbeitung personenbezogener Daten von betroffenen Personen, die sich in der Union befinden, durch einen
    nicht in der Union niedergelassenen Verantwortlichen oder Auftragsverarbeiter sollte auch dann dieser Verordnung
    unterliegen, wenn sie dazu dient, das Verhalten dieser betroffenen Personen zu beobachten, soweit ihr Verhalten in
    der Union erfolgt. Ob eine Verarbeitungstätigkeit der Beobachtung des Verhaltens von betroffenen Personen gilt,
    sollte daran festgemacht werden, ob ihre Internetaktivitäten nachvollzogen werden, einschließlich der möglichen
    nachfolgenden Verwendung von Techniken zur Verarbeitung personenbezogener Daten, durch die von einer natürlichen
    Person ein Profil erstellt wird, das insbesondere die Grundlage für sie betreffende Entscheidungen bildet oder
    anhand dessen ihre persönlichen Vorlieben, Verhaltensweisen oder Gepflogenheiten analysiert oder vorausgesagt
    werden sollen.%
   \label{itm:eg-24}
   
   \crossrefReasonToArticle{24}

   \item Ist nach Völkerrecht das Recht eines Mitgliedstaats anwendbar, z. B. in einer diplomatischen oder
    konsularischen Vertretung eines Mitgliedstaats, so sollte die Verordnung auch auf einen nicht in der Union
    niedergelassenen Verantwortlichen Anwendung finden.%
   \label{itm:eg-25}
   
   \crossrefReasonToArticle{25}

   \item Die Grundsätze des Datenschutzes sollten für alle Informationen gelten, die sich auf eine identifizierte oder
    identifizierbare natürliche Person beziehen. Einer Pseudonymisierung unterzogene personenbezogene Daten, die durch
    Heranziehung zusätzlicher Informationen einer natürlichen Person zugeordnet werden könnten, sollten als
    Informationen über eine identifizierbare natürliche Person betrachtet werden. Um festzustellen, ob eine natürliche
    Person identifizierbar ist, sollten alle Mittel berücksichtigt werden, die von dem Verantwortlichen oder einer
    anderen Person nach allgemeinem Ermessen wahrscheinlich genutzt werden, um die natürliche Person direkt oder
    indirekt zu identifizieren, wie beispielsweise das Aussondern. Bei der Feststellung, ob Mittel nach allgemeinem
    Ermessen wahrscheinlich zur Identifizierung der natürlichen Person genutzt werden, sollten alle objektiven
    Faktoren, wie die Kosten der Identifizierung und der dafür erforderliche Zeitaufwand, herangezogen werden, wobei
    die zum Zeitpunkt der Verarbeitung verfügbare Technologie und technologische Entwicklungen zu berücksichtigen sind.
    Die Grundsätze des Datenschutzes sollten daher nicht für anonyme Informationen gelten, d.h. für Informationen, die
    sich nicht auf eine identifizierte oder identifizierbare natürliche Person beziehen, oder personenbezogene Daten,
    die in einer Weise anonymisiert worden sind, dass die betroffene Person nicht oder nicht mehr identifiziert werden
    kann. Diese Verordnung betrifft somit nicht die Verarbeitung solcher anonymer Daten, auch für statistische oder für
    Forschungszwecke.%
   \label{itm:eg-26}
   
   \crossrefReasonToArticle{26}

   \item Diese Verordnung gilt nicht für die personenbezogenen Daten Verstorbener. Die Mitgliedstaaten können
    Vorschriften für die Verarbeitung der personenbezogenen Daten Verstorbener vorsehen.%
   \label{itm:eg-27}
   
   \crossrefReasonToArticle{27}

   \item Die Anwendung der Pseudonymisierung auf personenbezogene Daten kann die Risiken für die betroffenen Personen
    senken und die Verantwortlichen und die Auftragsverarbeiter bei der Einhaltung ihrer Datenschutzpflichten
    unterstützen. Durch die ausdrückliche Einführung der „Pseudonymisierung“ in dieser Verordnung ist nicht
    beabsichtigt, andere Datenschutzmaßnahmen auszuschließen.%
   \label{itm:eg-28}
   
   \crossrefReasonToArticle{28}

   \item Um Anreize für die Anwendung der Pseudonymisierung bei der Verarbeitung personenbezogener Daten zu schaffen,
    sollten Pseudonymisierungsmaßnahmen, die jedoch eine allgemeine Analyse zulassen, bei demselben Verantwortlichen
    möglich sein, wenn dieser die erforderlichen technischen und organisatorischen Maßnahmen getroffen hat, um — für
    die jeweilige Verarbeitung — die Umsetzung dieser Verordnung zu gewährleisten, wobei sicherzustellen ist, dass
    zusätzliche Informationen, mit denen die personenbezogenen Daten einer speziellen betroffenen Person zugeordnet
    werden können, gesondert aufbewahrt werden. Der für die Verarbeitung der personenbezogenen Daten Verantwortliche,
    sollte die befugten Personen bei diesem Verantwortlichen angeben.%
   \label{itm:eg-29}
   
   \crossrefReasonToArticle{29}

   \item Natürlichen Personen werden unter Umständen Online"=Kennungen wie IP"=Adressen und Cookie"=Kennungen, die sein
    Gerät oder Software"=Anwendungen und -Tools oder Protokolle liefern, oder sonstige Kennungen wie
    Funkfrequenzkennzeichnungen zugeordnet. Dies kann Spuren hinterlassen, die insbesondere in Kombination mit
    eindeutigen Kennungen und anderen beim Server eingehenden Informationen dazu benutzt werden können, um Profile der
    natürlichen Personen zu erstellen und sie zu identifizieren.%
   \label{itm:eg-30}
   
   \crossrefReasonToArticle{30}

   \item Behörden, gegenüber denen personenbezogene Daten aufgrund einer rechtlichen Verpflichtung für die Ausübung
    ihres offiziellen Auftrags offengelegt werden, wie Steuer- und Zollbehörden, Finanzermittlungsstellen, unabhängige
    Verwaltungsbehörden oder Finanzmarktbehörden, die für die Regulierung und Aufsicht von Wertpapiermärkten zuständig
    sind, sollten nicht als Empfänger gelten, wenn sie personenbezogene Daten erhalten, die für die Durchführung —
    gemäß dem Unionsrecht oder dem Recht der Mitgliedstaaten — eines einzelnen Untersuchungsauftrags im Interesse der
    Allgemeinheit erforderlich sind. Anträge auf Offenlegung, die von Behörden ausgehen, sollten immer schriftlich
    erfolgen, mit Gründen versehen sein und gelegentlichen Charakter haben, und sie sollten nicht vollständige
    Dateisysteme betreffen oder zur Verknüpfung von Dateisystemen führen. Die Verarbeitung personenbezogener Daten
    durch die genannten Behörden sollte den für die Zwecke der Verarbeitung geltenden Datenschutzvorschriften
    entsprechen.%
   \label{itm:eg-31}
   
   \crossrefReasonToArticle{31}

   \item Die Einwilligung sollte durch eine eindeutige bestätigende Handlung erfolgen, mit der freiwillig, für den
    konkreten Fall, in informierter Weise und unmissverständlich bekundet wird, dass die betroffene Person mit der
    Verarbeitung der sie betreffenden personenbezogenen Daten einverstanden ist, etwa in Form einer schriftlichen
    Erklärung, die auch elektronisch erfolgen kann, oder einer mündlichen Erklärung. Dies könnte etwa durch Anklicken
    eines Kästchens beim Besuch einer Internetseite, durch die Auswahl technischer Einstellungen für Dienste der
    Informationsgesellschaft oder durch eine andere Erklärung oder Verhaltensweise geschehen, mit der die betroffene
    Person in dem jeweiligen Kontext eindeutig ihr Einverständnis mit der beabsichtigten Verarbeitung ihrer
    personenbezogenen Daten signalisiert. Stillschweigen, bereits angekreuzte Kästchen oder Untätigkeit der betroffenen
    Person sollten daher keine Einwilligung darstellen. Die Einwilligung sollte sich auf alle zu demselben Zweck oder
    denselben Zwecken vorgenommenen Verarbeitungsvorgänge beziehen. Wenn die Verarbeitung mehreren Zwecken dient,
    sollte für alle diese Verarbeitungszwecke eine Einwilligung gegeben werden. Wird die betroffene Person auf
    elektronischem Weg zur Einwilligung aufgefordert, so muss die Aufforderung in klarer und knapper Form und ohne
    unnötige Unterbrechung des Dienstes, für den die Einwilligung gegeben wird, erfolgen.%
   \label{itm:eg-32}
   
   \crossrefReasonToArticle{32}

   \item Oftmals kann der Zweck der Verarbeitung personenbezogener Daten für Zwecke der wissenschaftlichen Forschung zum
    Zeitpunkt der Erhebung der personenbezogenen Daten nicht vollständig angegeben werden. Daher sollte es betroffenen
    Personen erlaubt sein, ihre Einwilligung für bestimmte Bereiche wissenschaftlicher Forschung zu geben, wenn dies
    unter Einhaltung der anerkannten ethischen Standards der wissenschaftlichen Forschung geschieht. Die betroffenen
    Personen sollten Gelegenheit erhalten, ihre Einwilligung nur für bestimme Forschungsbereiche oder Teile von
    Forschungsprojekten in dem vom verfolgten Zweck zugelassenen Maße zu erteilen.%
   \label{itm:eg-33}
   
   \crossrefReasonToArticle{33}

   \item Genetische Daten sollten als personenbezogene Daten über die ererbten oder erworbenen genetischen Eigenschaften
    einer natürlichen Person definiert werden, die aus der Analyse einer biologischen Probe der betreffenden
    natürlichen Person, insbesondere durch eine Chromosomen, Desoxyribonukleinsäure (DNS)- oder Ribonukleinsäure
    (RNS)-Analyse oder der Analyse eines anderen Elements, durch die gleichwertige Informationen erlangt werden können,
    gewonnen werden.%
   \label{itm:eg-34}
   
   \crossrefReasonToArticle{34}

   \item Zu den personenbezogenen Gesundheitsdaten sollten alle Daten zählen, die sich auf den Gesundheitszustand einer
    betroffenen Person beziehen und aus denen Informationen über den früheren, gegenwärtigen und künftigen körperlichen
    oder geistigen Gesundheitszustand der betroffenen Person hervorgehen. Dazu gehören auch Informationen über die
    natürliche Person, die im Zuge der Anmeldung für sowie der Erbringung von Gesundheitsdienstleistungen im Sinne der
    Richtlinie 2011/24/EU des Europäischen Parlaments und des Rates\comment {Richtlinie 2011/24/EU des Europäischen
    Parlaments und des Rates vom 9. März 2011 über die Ausübung der Patientenrechte in der grenzüberschreitenden
    Gesundheitsversorgung, konsolidierte Fassung siehe \cite{ril-patrechte}} für die natürliche Person erhoben werden,
    Nummern, Symbole oder Kennzeichen, die einer natürlichen Person zugeteilt wurden, um diese natürliche Person für
    gesundheitliche Zwecke eindeutig zu identifizieren, Informationen, die von der Prüfung oder Untersuchung eines
    Körperteils oder einer körpereigenen Substanz, auch aus genetischen Daten und biologischen Proben, abgeleitet
    wurden, und Informationen etwa über Krankheiten, Behinderungen, Krankheitsrisiken, Vorerkrankungen, klinische
    Behandlungen oder den physiologischen oder biomedizinischen Zustand der betroffenen Person unabhängig von der
    Herkunft der Daten, ob sie nun von einem Arzt oder sonstigem Angehörigen eines Gesundheitsberufes, einem
    Krankenhaus, einem Medizinprodukt oder einem In"=Vitro"=Diagnostikum stammen.%
   \label{itm:eg-35}
   
   \crossrefReasonToArticle{35}

   \item Die Hauptniederlassung des Verantwortlichen in der Union sollte der Ort seiner Hauptverwaltung in der Union
    sein, es sei denn, dass Entscheidungen über die Zwecke und Mittel der Verarbeitung personenbezogener Daten in einer
    anderen Niederlassung des Verantwortlichen in der Union getroffen werden; in diesem Fall sollte die letztgenannte
    als Hauptniederlassung gelten. Zur Bestimmung der Hauptniederlassung eines Verantwortlichen in der Union sollten
    objektive Kriterien herangezogen werden; ein Kriterium sollte dabei die effektive und tatsächliche Ausübung von
    Managementtätigkeiten durch eine feste Einrichtung sein, in deren Rahmen die Grundsatzentscheidungen zur Festlegung
    der Zwecke und Mittel der Verarbeitung getroffen werden. Dabei sollte nicht ausschlaggebend sein, ob die
    Verarbeitung der personenbezogenen Daten tatsächlich an diesem Ort ausgeführt wird. Das Vorhandensein und die
    Verwendung technischer Mittel und Verfahren zur Verarbeitung personenbezogener Daten oder Verarbeitungstätigkeiten
    begründen an sich noch keine Hauptniederlassung und sind daher kein ausschlaggebender Faktor für das Bestehen einer
    Hauptniederlassung. Die Hauptniederlassung des Auftragsverarbeiters sollte der Ort sein, an dem der
    Auftragsverarbeiter seine Hauptverwaltung in der Union hat, oder — wenn er keine Hauptverwaltung in der Union hat —
    der Ort, an dem die wesentlichen Verarbeitungstätigkeiten in der Union stattfinden. Sind sowohl der Verantwortliche
    als auch der Auftragsverarbeiter betroffen, so sollte die Aufsichtsbehörde des Mitgliedstaats, in dem der
    Verantwortliche seine Hauptniederlassung hat, die zuständige federführende Aufsichtsbehörde bleiben, doch sollte
    die Aufsichtsbehörde des Auftragsverarbeiters als betroffene Aufsichtsbehörde betrachtet werden und diese
    Aufsichtsbehörde sollte sich an dem in dieser Verordnung vorgesehenen Verfahren der Zusammenarbeit beteiligen. Auf
    jeden Fall sollten die Aufsichtsbehörden des Mitgliedstaats oder der Mitgliedstaaten, in dem bzw. denen der
    Auftragsverarbeiter eine oder mehrere Niederlassungen hat, nicht als betroffene Aufsichtsbehörden betrachtet
    werden, wenn sich der Beschlussentwurf nur auf den Verantwortlichen bezieht. Wird die Verarbeitung durch eine
    Unternehmensgruppe vorgenommen, so sollte die Hauptniederlassung des herrschenden Unternehmens als
    Hauptniederlassung der Unternehmensgruppe gelten, es sei denn, die Zwecke und Mittel der Verarbeitung werden von
    einem anderen Unternehmen festgelegt.%
   \label{itm:eg-36}
   
   \crossrefReasonToArticle{36}

   \item Eine Unternehmensgruppe sollte aus einem herrschenden Unternehmen und den von diesem abhängigen Unternehmen
    bestehen, wobei das herrschende Unternehmen dasjenige sein sollte, das zum Beispiel aufgrund der
    Eigentumsverhältnisse, der finanziellen Beteiligung oder der für das Unternehmen geltenden Vorschriften oder der
    Befugnis, Datenschutzvorschriften umsetzen zu lassen, einen beherrschenden Einfluss auf die übrigen Unternehmen
    ausüben kann. Ein Unternehmen, das die Verarbeitung personenbezogener Daten in ihm angeschlossenen Unternehmen
    kontrolliert, sollte zusammen mit diesen als eine „Unternehmensgruppe“ betrachtet werden.%
   \label{itm:eg-37}
   
   \crossrefReasonToArticle{37}

   \item Kinder verdienen bei ihren personenbezogenen Daten besonderen Schutz, da Kinder sich der betreffenden Risiken,
    Folgen und Garantien und ihrer Rechte bei der Verarbeitung personenbezogener Daten möglicherweise weniger bewusst
    sind. Ein solcher besonderer Schutz sollte insbesondere die Verwendung personenbezogener Daten von Kindern für
    Werbezwecke oder für die Erstellung von Persönlichkeits- oder Nutzerprofilen und die Erhebung von personenbezogenen
    Daten von Kindern bei der Nutzung von Diensten, die Kindern direkt angeboten werden, betreffen. Die Einwilligung
    des Trägers der elterlichen Verantwortung sollte im Zusammenhang mit Präventions- oder Beratungsdiensten, die
    unmittelbar einem Kind angeboten werden, nicht erforderlich sein.%
   \label{itm:eg-38}
   
   \crossrefReasonToArticle{38}

   \item Jede Verarbeitung personenbezogener Daten sollte rechtmäßig und nach Treu und Glauben erfolgen. Für natürliche
    Personen sollte Transparenz dahingehend bestehen, dass sie betreffende personenbezogene Daten erhoben, verwendet,
    eingesehen oder anderweitig verarbeitet werden und in welchem Umfang die personenbezogenen Daten verarbeitet werden
    und künftig noch verarbeitet werden. Der Grundsatz der Transparenz setzt voraus, dass alle Informationen und
    Mitteilungen zur Verarbeitung dieser personenbezogenen Daten leicht zugänglich und verständlich und in klarer und
    einfacher Sprache abgefasst sind. Dieser Grundsatz betrifft insbesondere die Informationen über die Identität des
    Verantwortlichen und die Zwecke der Verarbeitung und sonstige Informationen, die eine faire und transparente
    Verarbeitung im Hinblick auf die betroffenen natürlichen Personen gewährleisten, sowie deren Recht, eine
    Bestätigung und Auskunft darüber zu erhalten, welche sie betreffende personenbezogene Daten verarbeitet werden.
    Natürliche Personen sollten über die Risiken, Vorschriften, Garantien und Rechte im Zusammenhang mit der
    Verarbeitung personenbezogener Daten informiert und darüber aufgeklärt werden, wie sie ihre diesbezüglichen Rechte
    geltend machen können. Insbesondere sollten die bestimmten Zwecke, zu denen die personenbezogenen Daten verarbeitet
    werden, eindeutig und rechtmäßig sein und zum Zeitpunkt der Erhebung der personenbezogenen Daten feststehen. Die
    personenbezogenen Daten sollten für die Zwecke, zu denen sie verarbeitet werden, angemessen und erheblich sowie auf
    das für die Zwecke ihrer Verarbeitung notwendige Maß beschränkt sein. Dies erfordert insbesondere, dass die
    Speicherfrist für personenbezogene Daten auf das unbedingt erforderliche Mindestmaß beschränkt bleibt.
    Personenbezogene Daten sollten nur verarbeitet werden dürfen, wenn der Zweck der Verarbeitung nicht in zumutbarer
    Weise durch andere Mittel erreicht werden kann. Um sicherzustellen, dass die personenbezogenen Daten nicht länger
    als nötig gespeichert werden, sollte der Verantwortliche Fristen für ihre Löschung oder regelmäßige Überprüfung
    vorsehen. Es sollten alle vertretbaren Schritte unternommen werden, damit unrichtige personenbezogene Daten
    gelöscht oder berichtigt werden. Personenbezogene Daten sollten so verarbeitet werden, dass ihre Sicherheit und
    Vertraulichkeit hinreichend gewährleistet ist, wozu auch gehört, dass Unbefugte keinen Zugang zu den Daten haben
    und weder die Daten noch die Geräte, mit denen diese verarbeitet werden, benutzen können.%
   \label{itm:eg-39}
   
   \crossrefReasonToArticle{39}

   \item Damit die Verarbeitung rechtmäßig ist, müssen personenbezogene Daten mit Einwilligung der betroffenen Person
    oder auf einer sonstigen zulässigen Rechtsgrundlage verarbeitet werden, die sich aus dieser Verordnung oder — wann
    immer in dieser Verordnung darauf Bezug genommen wird — aus dem sonstigen Unionsrecht oder dem Recht der
    Mitgliedstaaten ergibt, so unter anderem auf der Grundlage, dass sie zur Erfüllung der rechtlichen Verpflichtung,
    der der Verantwortliche unterliegt, oder zur Erfüllung eines Vertrags, dessen Vertragspartei die betroffene Person
    ist, oder für die Durchführung vorvertraglicher Maßnahmen, die auf Anfrage der betroffenen Person erfolgen,
    erforderlich ist.%
   \label{itm:eg-40}
   
   \crossrefReasonToArticle{40}

   \item Wenn in dieser Verordnung auf eine Rechtsgrundlage oder eine Gesetzgebungsmaßnahme Bezug genommen wird,
    erfordert dies nicht notwendigerweise einen von einem Parlament angenommenen Gesetzgebungsakt; davon unberührt
    bleiben Anforderungen gemäß der Verfassungsordnung des betreffenden Mitgliedstaats. Die entsprechende
    Rechtsgrundlage oder Gesetzgebungsmaßnahme sollte jedoch klar und präzise sein und ihre Anwendung sollte für die
    Rechtsunterworfenen gemäß der Rechtsprechung des Gerichtshofs der Europäischen Union (im Folgenden „Gerichtshof“)
    und des Europäischen Gerichtshofs für Menschenrechte vorhersehbar sein.%
   \label{itm:eg-41}
   
   \crossrefReasonToArticle{41}

   \item Erfolgt die Verarbeitung mit Einwilligung der betroffenen Person, sollte der Verantwortliche nachweisen können,
    dass die betroffene Person ihre Einwilligung zu dem Verarbeitungsvorgang gegeben hat. Insbesondere bei Abgabe einer
    schriftlichen Erklärung in anderer Sache sollten Garantien sicherstellen, dass die betroffene Person weiß, dass und
    in welchem Umfang sie ihre Einwilligung erteilt. Gemäß der Richtlinie 93/13/EWG des Rates\comment
    {Richtlinie 93/13/EWG des Rates vom 5. April 1993 über missbräuchliche Klauseln in Verbraucherverträgen,
    konsolidierte Fassung siehe \cite{ril-mk-vv}} sollte eine vom Verantwortlichen vorformulierte
    Einwilligungserklärung in verständlicher und leicht zugänglicher Form in einer klaren und einfachen Sprache zur
    Verfügung gestellt werden, und sie sollte keine missbräuchlichen Klauseln beinhalten. Damit sie in Kenntnis der
    Sachlage ihre Einwilligung geben kann, sollte die betroffene Person mindestens wissen, wer der Verantwortliche ist
    und für welche Zwecke ihre personenbezogenen Daten verarbeitet werden sollen. Es sollte nur dann davon ausgegangen
    werden, dass sie ihre Einwilligung freiwillig gegeben hat, wenn sie eine echte oder freie Wahl hat und somit in der
    Lage ist, die Einwilligung zu verweigern oder zurückzuziehen, ohne Nachteile zu erleiden.%
   \label{itm:eg-42}
   
   \crossrefReasonToArticle{42}

   \item Um sicherzustellen, dass die Einwilligung freiwillig erfolgt ist, sollte diese in besonderen Fällen, wenn
    zwischen der betroffenen Person und dem Verantwortlichen ein klares Ungleichgewicht besteht, insbesondere wenn es
    sich bei dem Verantwortlichen um eine Behörde handelt, und es deshalb in Anbetracht aller Umstände in dem
    speziellen Fall unwahrscheinlich ist, dass die Einwilligung freiwillig gegeben wurde, keine gültige Rechtsgrundlage
    liefern. Die Einwilligung gilt nicht als freiwillig erteilt, wenn zu verschiedenen Verarbeitungsvorgängen von
    personenbezogenen Daten nicht gesondert eine Einwilligung erteilt werden kann, obwohl dies im Einzelfall angebracht
    ist, oder wenn die Erfüllung eines Vertrags, einschließlich der Erbringung einer Dienstleistung, von der
    Einwilligung abhängig ist, obwohl diese Einwilligung für die Erfüllung nicht erforderlich ist.%
   \label{itm:eg-43}
   
   \crossrefReasonToArticle{43}

   \item Die Verarbeitung von Daten sollte als rechtmäßig gelten, wenn sie für die Erfüllung oder den geplanten
    Abschluss eines Vertrags erforderlich ist.%
   \label{itm:eg-44}
   
   \crossrefReasonToArticle{44}

   \item Erfolgt die Verarbeitung durch den Verantwortlichen aufgrund einer ihm obliegenden rechtlichen Verpflichtung
    oder ist die Verarbeitung zur Wahrnehmung einer Aufgabe im öffentlichen Interesse oder in Ausübung öffentlicher
    Gewalt erforderlich, muss hierfür eine Grundlage im Unionsrecht oder im Recht eines Mitgliedstaats bestehen. Mit
    dieser Verordnung wird nicht für jede einzelne Verarbeitung ein spezifisches Gesetz verlangt. Ein Gesetz als
    Grundlage für mehrere Verarbeitungsvorgänge kann ausreichend sein, wenn die Verarbeitung aufgrund einer dem
    Verantwortlichen obliegenden rechtlichen Verpflichtung erfolgt oder wenn die Verarbeitung zur Wahrnehmung einer
    Aufgabe im öffentlichen Interesse oder in Ausübung öffentlicher Gewalt erforderlich ist. Desgleichen sollte im
    Unionsrecht oder im Recht der Mitgliedstaaten geregelt werden, für welche Zwecke die Daten verarbeitet werden
    dürfen. Ferner könnten in diesem Recht die allgemeinen Bedingungen dieser Verordnung zur Regelung der
    Rechtmäßigkeit der Verarbeitung personenbezogener Daten präzisiert und es könnte darin festgelegt werden, wie der
    Verantwortliche zu bestimmen ist, welche Art von personenbezogenen Daten verarbeitet werden, welche Personen
    betroffen sind, welchen Einrichtungen die personenbezogenen Daten offengelegt, für welche Zwecke und wie lange sie
    gespeichert werden dürfen und welche anderen Maßnahmen ergriffen werden, um zu gewährleisten, dass die Verarbeitung
    rechtmäßig und nach Treu und Glauben erfolgt. Desgleichen sollte im Unionsrecht oder im Recht der Mitgliedstaaten
    geregelt werden, ob es sich bei dem Verantwortlichen, der eine Aufgabe wahrnimmt, die im öffentlichen Interesse
    liegt oder in Ausübung öffentlicher Gewalt erfolgt, um eine Behörde oder um eine andere unter das öffentliche Recht
    fallende natürliche oder juristische Person oder, sofern dies durch das öffentliche Interesse einschließlich
    gesundheitlicher Zwecke, wie die öffentliche Gesundheit oder die soziale Sicherheit oder die Verwaltung von
    Leistungen der Gesundheitsfürsorge, gerechtfertigt ist, eine natürliche oder juristische Person des Privatrechts,
    wie beispielsweise eine Berufsvereinigung, handeln sollte.%
   \label{itm:eg-45}
   
   \crossrefReasonToArticle{45}

   \item Die Verarbeitung personenbezogener Daten sollte ebenfalls als rechtmäßig angesehen werden, wenn sie
    erforderlich ist, um ein lebenswichtiges Interesse der betroffenen Person oder einer anderen natürlichen Person zu
    schützen. Personenbezogene Daten sollten grundsätzlich nur dann aufgrund eines lebenswichtigen Interesses einer
    anderen natürlichen Person verarbeitet werden, wenn die Verarbeitung offensichtlich nicht auf eine andere
    Rechtsgrundlage gestützt werden kann. Einige Arten der Verarbeitung können sowohl wichtigen Gründen des
    öffentlichen Interesses als auch lebenswichtigen Interessen der betroffenen Person dienen; so kann beispielsweise
    die Verarbeitung für humanitäre Zwecke einschließlich der Überwachung von Epidemien und deren Ausbreitung oder in
    humanitären Notfällen insbesondere bei Naturkatastrophen oder vom Menschen verursachten Katastrophen erforderlich
    sein.%
   \label{itm:eg-46}
   
   \crossrefReasonToArticle{46}

   \item Die Rechtmäßigkeit der Verarbeitung kann durch die berechtigten Interessen eines Verantwortlichen, auch eines
    Verantwortlichen, dem die personenbezogenen Daten offengelegt werden dürfen, oder eines Dritten begründet sein,
    sofern die Interessen oder die Grundrechte und Grundfreiheiten der betroffenen Person nicht überwiegen; dabei sind
    die vernünftigen Erwartungen der betroffenen Person, die auf ihrer Beziehung zu dem Verantwortlichen beruhen, zu
    berücksichtigen. Ein berechtigtes Interesse könnte beispielsweise vorliegen, wenn eine maßgebliche und angemessene
    Beziehung zwischen der betroffenen Person und dem Verantwortlichen besteht, z. B. wenn die betroffene Person ein
    Kunde des Verantwortlichen ist oder in seinen Diensten steht. Auf jeden Fall wäre das Bestehen eines berechtigten
    Interesses besonders sorgfältig abzuwägen, wobei auch zu prüfen ist, ob eine betroffene Person zum Zeitpunkt der
    Erhebung der personenbezogenen Daten und angesichts der Umstände, unter denen sie erfolgt, vernünftigerweise
    absehen kann, dass möglicherweise eine Verarbeitung für diesen Zweck erfolgen wird. Insbesondere dann, wenn
    personenbezogene Daten in Situationen verarbeitet werden, in denen eine betroffene Person vernünftigerweise nicht
    mit einer weiteren Verarbeitung rechnen muss, könnten die Interessen und Grundrechte der betroffenen Person das
    Interesse des Verantwortlichen überwiegen. Da es dem Gesetzgeber obliegt, per Rechtsvorschrift die Rechtsgrundlage
    für die Verarbeitung personenbezogener Daten durch die Behörden zu schaffen, sollte diese Rechtsgrundlage nicht für
    Verarbeitungen durch Behörden gelten, die diese in Erfüllung ihrer Aufgaben vornehmen. Die Verarbeitung
    personenbezogener Daten im für die Verhinderung von Betrug unbedingt erforderlichen Umfang stellt ebenfalls ein
    berechtigtes Interesse des jeweiligen Verantwortlichen dar. Die Verarbeitung personenbezogener Daten zum Zwecke der
    Direktwerbung kann als eine einem berechtigten Interesse dienende Verarbeitung betrachtet werden.%
   \label{itm:eg-47}
   
   \crossrefReasonToArticle{47}

   \item Verantwortliche, die Teil einer Unternehmensgruppe oder einer Gruppe von Einrichtungen sind, die einer
    zentralen Stelle zugeordnet sind können ein berechtigtes Interesse haben, personenbezogene Daten innerhalb der
    Unternehmensgruppe für interne Verwaltungszwecke, einschließlich der Verarbeitung personenbezogener Daten von
    Kunden und Beschäftigten, zu übermitteln. Die Grundprinzipien für die Übermittlung personenbezogener Daten
    innerhalb von Unternehmensgruppen an ein Unternehmen in einem Drittland bleiben unberührt.%
   \label{itm:eg-48}
   
   \crossrefReasonToArticle{48}

   \item Die Verarbeitung von personenbezogenen Daten durch Behörden, Computer"=Notdienste (Computer Emergency Response
    Teams — CERT, beziehungsweise Computer Security Incident Response Teams — CSIRT), Betreiber von elektronischen
    Kommunikationsnetzen und -diensten sowie durch Anbieter von Sicherheitstechnologien und -diensten stellt in dem
    Maße ein berechtigtes Interesse des jeweiligen Verantwortlichen dar, wie dies für die Gewährleistung der Netz- und
    Informationssicherheit unbedingt notwendig und verhältnismäßig ist, d.h. soweit dadurch die Fähigkeit eines Netzes
    oder Informationssystems gewährleistet wird, mit einem vorgegebenen Grad der Zuverlässigkeit Störungen oder
    widerrechtliche oder mutwillige Eingriffe abzuwehren, die die Verfügbarkeit, Authentizität, Vollständigkeit und
    Vertraulichkeit von gespeicherten oder übermittelten personenbezogenen Daten sowie die Sicherheit damit
    zusammenhängender Dienste, die über diese Netze oder Informationssysteme angeboten werden bzw. zugänglich sind,
    beeinträchtigen. Ein solches berechtigtes Interesse könnte beispielsweise darin bestehen, den Zugang Unbefugter zu
    elektronischen Kommunikationsnetzen und die Verbreitung schädlicher Programmcodes zu verhindern sowie Angriffe in
    Form der gezielten Überlastung von Servern („Denial of service“-Angriffe) und Schädigungen von Computer- und
    elektronischen Kommunikationssystemen abzuwehren.%
   \label{itm:eg-49}
   
   \crossrefReasonToArticle{49}

   \item Die Verarbeitung personenbezogener Daten für andere Zwecke als die, für die die personenbezogenen Daten
    ursprünglich erhoben wurden, sollte nur zulässig sein, wenn die Verarbeitung mit den Zwecken, für die die
    personenbezogenen Daten ursprünglich erhoben wurden, vereinbar ist. In diesem Fall ist keine andere gesonderte
    Rechtsgrundlage erforderlich als diejenige für die Erhebung der personenbezogenen Daten. Ist die Verarbeitung für
    die Wahrnehmung einer Aufgabe erforderlich, die im öffentlichen Interesse liegt oder in Ausübung öffentlicher
    Gewalt erfolgt, die dem Verantwortlichen übertragen wurde, so können im Unionsrecht oder im Recht der
    Mitgliedstaaten die Aufgaben und Zwecke bestimmt und konkretisiert werden, für die eine Weiterverarbeitung als
    vereinbar und rechtmäßig erachtet wird. Die Weiterverarbeitung für im öffentlichen Interesse liegende Archivzwecke,
    für wissenschaftliche oder historische Forschungszwecke oder für statistische Zwecke sollte als vereinbarer und
    rechtmäßiger Verarbeitungsvorgang gelten. Die im Unionsrecht oder im Recht der Mitgliedstaaten vorgesehene
    Rechtsgrundlage für die Verarbeitung personenbezogener Daten kann auch als Rechtsgrundlage für eine
    Weiterverarbeitung dienen. Um festzustellen, ob ein Zweck der Weiterverarbeitung mit dem Zweck, für den die
    personenbezogenen Daten ursprünglich erhoben wurden, vereinbar ist, sollte der Verantwortliche nach Einhaltung
    aller Anforderungen für die Rechtmäßigkeit der ursprünglichen Verarbeitung unter anderem prüfen, ob ein
    Zusammenhang zwischen den Zwecken, für die die personenbezogenen Daten erhoben wurden, und den Zwecken der
    beabsichtigten Weiterverarbeitung besteht, in welchem Kontext die Daten erhoben wurden, insbesondere die
    vernünftigen Erwartungen der betroffenen Person, die auf ihrer Beziehung zu dem Verantwortlichen beruhen, in Bezug
    auf die weitere Verwendung dieser Daten, um welche Art von personenbezogenen Daten es sich handelt, welche Folgen
    die beabsichtigte Weiterverarbeitung für die betroffenen Personen hat und ob sowohl beim ursprünglichen als auch
    beim beabsichtigten Weiterverarbeitungsvorgang geeignete Garantien bestehen. 

    Hat die betroffene Person ihre Einwilligung erteilt oder beruht die Verarbeitung auf Unionsrecht oder dem Recht der
    Mitgliedstaaten, was in einer demokratischen Gesellschaft eine notwendige und verhältnismäßige Maßnahme zum Schutz
    insbesondere wichtiger Ziele des allgemeinen öffentlichen Interesses darstellt, so sollte der Verantwortliche die
    personenbezogenen Daten ungeachtet der Vereinbarkeit der Zwecke weiterverarbeiten dürfen. In jedem Fall sollte
    gewährleistet sein, dass die in dieser Verordnung niedergelegten Grundsätze angewandt werden und insbesondere die
    betroffene Person über diese anderen Zwecke und über ihre Rechte einschließlich des Widerspruchsrechts unterrichtet
    wird. Der Hinweis des Verantwortlichen auf mögliche Straftaten oder Bedrohungen der öffentlichen Sicherheit und die
    Übermittlung der maßgeblichen personenbezogenen Daten in Einzelfällen oder in mehreren Fällen, die im Zusammenhang
    mit derselben Straftat oder derselben Bedrohung der öffentlichen Sicherheit stehen, an eine zuständige Behörde
    sollten als berechtigtes Interesse des Verantwortlichen gelten. Eine derartige Übermittlung personenbezogener Daten
    im berechtigten Interesse des Verantwortlichen oder deren Weiterverarbeitung sollte jedoch unzulässig sein, wenn
    die Verarbeitung mit einer rechtlichen, beruflichen oder sonstigen verbindlichen Pflicht zur Geheimhaltung
    unvereinbar ist.%
   \label{itm:eg-50}
   
   \crossrefReasonToArticle{50}

   \item Personenbezogene Daten, die ihrem Wesen nach hinsichtlich der Grundrechte und Grundfreiheiten besonders
    sensibel sind, verdienen einen besonderen Schutz, da im Zusammenhang mit ihrer Verarbeitung erhebliche Risiken für
    die Grundrechte und Grundfreiheiten auftreten können. Diese personenbezogenen Daten sollten personenbezogene Daten
    umfassen, aus denen die rassische oder ethnische Herkunft hervorgeht, wobei die Verwendung des Begriffs „rassische
    Herkunft“ in dieser Verordnung nicht bedeutet, dass die Union Theorien, mit denen versucht wird, die Existenz
    verschiedener menschlicher Rassen zu belegen, gutheißt. Die Verarbeitung von Lichtbildern sollte nicht
    grundsätzlich als Verarbeitung besonderer Kategorien von personenbezogenen Daten angesehen werden, da Lichtbilder
    nur dann von der Definition des Begriffs „biometrische Daten“ erfasst werden, wenn sie mit speziellen technischen
    Mitteln verarbeitet werden, die die eindeutige Identifizierung oder Authentifizierung einer natürlichen Person
    ermöglichen. Derartige personenbezogene Daten sollten nicht verarbeitet werden, es sei denn, die Verarbeitung ist
    in den in dieser Verordnung dargelegten besonderen Fällen zulässig, wobei zu berücksichtigen ist, dass im Recht der
    Mitgliedstaaten besondere Datenschutzbestimmungen festgelegt sein können, um die Anwendung der Bestimmungen dieser
    Verordnung anzupassen, damit die Einhaltung einer rechtlichen Verpflichtung oder die Wahrnehmung einer Aufgabe im
    öffentlichen Interesse oder die Ausübung öffentlicher Gewalt, die dem Verantwortlichen übertragen wurde, möglich
    ist. Zusätzlich zu den speziellen Anforderungen an eine derartige Verarbeitung sollten die allgemeinen Grundsätze
    und andere Bestimmungen dieser Verordnung, insbesondere hinsichtlich der Bedingungen für eine rechtmäßige
    Verarbeitung, gelten. Ausnahmen von dem allgemeinen Verbot der Verarbeitung dieser besonderen Kategorien
    personenbezogener Daten sollten ausdrücklich vorgesehen werden, unter anderem bei ausdrücklicher Einwilligung der
    betroffenen Person oder bei bestimmten Notwendigkeiten, insbesondere wenn die Verarbeitung im Rahmen rechtmäßiger
    Tätigkeiten bestimmter Vereinigungen oder Stiftungen vorgenommen wird, die sich für die Ausübung von
    Grundfreiheiten einsetzen.%
   \label{itm:eg-51}
   
   \crossrefReasonToArticle{51}

   \item Ausnahmen vom Verbot der Verarbeitung besonderer Kategorien von personenbezogenen Daten sollten auch erlaubt
    sein, wenn sie im Unionsrecht oder dem Recht der Mitgliedstaaten vorgesehen sind, und — vorbehaltlich angemessener
    Garantien zum Schutz der personenbezogenen Daten und anderer Grundrechte — wenn dies durch das öffentliche
    Interesse gerechtfertigt ist, insbesondere für die Verarbeitung von personenbezogenen Daten auf dem Gebiet des
    Arbeitsrechts und des Rechts der sozialen Sicherheit einschließlich Renten und zwecks Sicherstellung und
    Überwachung der Gesundheit und Gesundheitswarnungen, Prävention oder Kontrolle ansteckender Krankheiten und anderer
    schwerwiegender Gesundheitsgefahren. Eine solche Ausnahme kann zu gesundheitlichen Zwecken gemacht werden, wie der
    Gewährleistung der öffentlichen Gesundheit und der Verwaltung von Leistungen der Gesundheitsversorgung,
    insbesondere wenn dadurch die Qualität und Wirtschaftlichkeit der Verfahren zur Abrechnung von Leistungen in den
    sozialen Krankenversicherungssystemen sichergestellt werden soll, oder wenn die Verarbeitung im öffentlichen
    Interesse liegenden Archivzwecken, wissenschaftlichen oder historischen Forschungszwecken oder statistischen
    Zwecken dient. Die Verarbeitung solcher personenbezogener Daten sollte zudem ausnahmsweise erlaubt sein, wenn sie
    erforderlich ist, um rechtliche Ansprüche, sei es in einem Gerichtsverfahren oder in einem Verwaltungsverfahren
    oder einem außergerichtlichen Verfahren, geltend zu machen, auszuüben oder zu verteidigen.%
   \label{itm:eg-52}
   
   \crossrefReasonToArticle{52}

   \item Besondere Kategorien personenbezogener Daten, die eines höheren Schutzes verdienen, sollten nur dann für
    gesundheitsbezogene Zwecke verarbeitet werden, wenn dies für das Erreichen dieser Zwecke im Interesse einzelner
    natürlicher Personen und der Gesellschaft insgesamt erforderlich ist, insbesondere im Zusammenhang mit der
    Verwaltung der Dienste und Systeme des Gesundheits- oder Sozialbereichs, einschließlich der Verarbeitung dieser
    Daten durch die Verwaltung und die zentralen nationalen Gesundheitsbehörden zwecks Qualitätskontrolle,
    Verwaltungsinformationen und der allgemeinen nationalen und lokalen Überwachung des Gesundheitssystems oder des
    Sozialsystems und zwecks Gewährleistung der Kontinuität der Gesundheits- und Sozialfürsorge und der
    grenzüberschreitenden Gesundheitsversorgung oder Sicherstellung und Überwachung der Gesundheit und
    Gesundheitswarnungen oder für im öffentlichen Interesse liegende Archivzwecke, zu wissenschaftlichen oder
    historischen Forschungszwecken oder statistischen Zwecken, die auf Rechtsvorschriften der Union oder der
    Mitgliedstaaten beruhen, die einem im öffentlichen Interesse liegenden Ziel dienen müssen, sowie für Studien, die
    im öffentlichen Interesse im Bereich der öffentlichen Gesundheit durchgeführt werden. Diese Verordnung sollte daher
    harmonisierte Bedingungen für die Verarbeitung besonderer Kategorien personenbezogener Gesundheitsdaten im Hinblick
    auf bestimmte Erfordernisse harmonisieren, insbesondere wenn die Verarbeitung dieser Daten für gesundheitsbezogene
    Zwecke von Personen durchgeführt wird, die gemäß einer rechtlichen Verpflichtung dem Berufsgeheimnis unterliegen.
    Im Recht der Union oder der Mitgliedstaaten sollten besondere und angemessene Maßnahmen zum Schutz der Grundrechte
    und der personenbezogenen Daten natürlicher Personen vorgesehen werden. Den Mitgliedstaaten sollte gestattet
    werden, weitere Bedingungen — einschließlich Beschränkungen — in Bezug auf die Verarbeitung von genetischen Daten,
    biometrischen Daten oder Gesundheitsdaten beizubehalten oder einzuführen. Dies sollte jedoch den freien Verkehr
    personenbezogener Daten innerhalb der Union nicht beeinträchtigen, falls die betreffenden Bedingungen für die
    grenzüberschreitende Verarbeitung solcher Daten gelten.%
   \label{itm:eg-53}
   
   \crossrefReasonToArticle{53}

   \item Aus Gründen des öffentlichen Interesses in Bereichen der öffentlichen Gesundheit kann es notwendig sein,
    besondere Kategorien personenbezogener Daten auch ohne Einwilligung der betroffenen Person zu verarbeiten. Diese
    Verarbeitung sollte angemessenen und besonderen Maßnahmen zum Schutz der Rechte und Freiheiten natürlicher Personen
    unterliegen. In diesem Zusammenhang sollte der Begriff „öffentliche Gesundheit“ im Sinne der Verordnung (EG) Nr.
    1338/2008 des Europäischen Parlaments und des Rates\comment{Verordnung (EG) Nr. 1338/2008 des Europäischen
    Parlaments und des Rates vom 16. Dezember 2008 zu Gemeinschaftsstatistiken über öffentliche Gesundheit und über
    Gesundheitsschutz und Sicherheit am Arbeitsplatz, konsolidierte Fassung siehe \cite{vo-stat-ges}} ausgelegt werden
    und alle Elemente im Zusammenhang mit der Gesundheit wie den Gesundheitszustand einschließlich Morbidität und
    Behinderung, die sich auf diesen Gesundheitszustand auswirkenden Determinanten, den Bedarf an
    Gesundheitsversorgung, die der Gesundheitsversorgung zugewiesenen Mittel, die Bereitstellung von
    Gesundheitsversorgungsleistungen und den allgemeinen Zugang zu diesen Leistungen sowie die entsprechenden Ausgaben
    und die Finanzierung und schließlich die Ursachen der Mortalität einschließen. Eine solche Verarbeitung von
    Gesundheitsdaten aus Gründen des öffentlichen Interesses darf nicht dazu führen, dass Dritte, unter anderem
    Arbeitgeber oder Versicherungs- und Finanzunternehmen, solche personenbezogene Daten zu anderen Zwecken
    verarbeiten.%
   \label{itm:eg-54}
   
   \crossrefReasonToArticle{54}

   \item Auch die Verarbeitung personenbezogener Daten durch staatliche Stellen zu verfassungsrechtlich oder
    völkerrechtlich verankerten Zielen von staatlich anerkannten Religionsgemeinschaften erfolgt aus Gründen des
    öffentlichen Interesses.%
   \label{itm:eg-55}
   
   \crossrefReasonToArticle{55}

   \item Wenn es in einem Mitgliedstaat das Funktionieren des demokratischen Systems erfordert, dass die politischen
    Parteien im Zusammenhang mit Wahlen personenbezogene Daten über die politische Einstellung von Personen sammeln,
    kann die Verarbeitung derartiger Daten aus Gründen des öffentlichen Interesses zugelassen werden, sofern geeignete
    Garantien vorgesehen werden.%
   \label{itm:eg-56}
   
   \crossrefReasonToArticle{56}

   \item Kann der Verantwortliche anhand der von ihm verarbeiteten personenbezogenen Daten eine natürliche Person nicht
    identifizieren, so sollte er nicht verpflichtet sein, zur bloßen Einhaltung einer Vorschrift dieser Verordnung
    zusätzliche Daten einzuholen, um die betroffene Person zu identifizieren. Allerdings sollte er sich nicht weigern,
    zusätzliche Informationen entgegenzunehmen, die von der betroffenen Person beigebracht werden, um ihre Rechte
    geltend zu machen. Die Identifizierung sollte die digitale Identifizierung einer betroffenen Person —
    beispielsweise durch Authentifizierungsverfahren etwa mit denselben Berechtigungsnachweisen, wie sie die betroffene
    Person verwendet, um sich bei dem von dem Verantwortlichen bereitgestellten Online"=Dienst anzumelden —
    einschließen.%
   \label{itm:eg-57}
   
   \crossrefReasonToArticle{57}

   \item Der Grundsatz der Transparenz setzt voraus, dass eine für die Öffentlichkeit oder die betroffene Person
    bestimmte Information präzise, leicht zugänglich und verständlich sowie in klarer und einfacher Sprache abgefasst
    ist und gegebenenfalls zusätzlich visuelle Elemente verwendet werden. Diese Information könnte in elektronischer
    Form bereitgestellt werden, beispielsweise auf einer Website, wenn sie für die Öffentlichkeit bestimmt ist. Dies
    gilt insbesondere für Situationen, wo die große Zahl der Beteiligten und die Komplexität der dazu benötigten
    Technik es der betroffenen Person schwer machen, zu erkennen und nachzuvollziehen, ob, von wem und zu welchem Zweck
    sie betreffende personenbezogene Daten erfasst werden, wie etwa bei der Werbung im Internet. Wenn sich die
    Verarbeitung an Kinder richtet, sollten aufgrund der besonderen Schutzwürdigkeit von Kindern Informationen und
    Hinweise in einer dergestalt klaren und einfachen Sprache erfolgen, dass ein Kind sie verstehen kann.%
   \label{itm:eg-58}
   
   \crossrefReasonToArticle{58}

   \item Es sollten Modalitäten festgelegt werden, die einer betroffenen Person die Ausübung der Rechte, die ihr nach
    dieser Verordnung zustehen, erleichtern, darunter auch Mechanismen, die dafür sorgen, dass sie unentgeltlich
    insbesondere Zugang zu personenbezogenen Daten und deren Berichtigung oder Löschung beantragen und gegebenenfalls
    erhalten oder von ihrem Widerspruchsrecht Gebrauch machen kann. So sollte der Verantwortliche auch dafür sorgen,
    dass Anträge elektronisch gestellt werden können, insbesondere wenn die personenbezogenen Daten elektronisch
    verarbeitet werden. Der Verantwortliche sollte verpflichtet werden, den Antrag der betroffenen Person unverzüglich,
    spätestens aber innerhalb eines Monats zu beantworten und gegebenenfalls zu begründen, warum er den Antrag
    ablehnt.%
   \label{itm:eg-59}
   
   \crossrefReasonToArticle{59}

   \item Die Grundsätze einer fairen und transparenten Verarbeitung machen es erforderlich, dass die betroffene Person
    über die Existenz des Verarbeitungsvorgangs und seine Zwecke unterrichtet wird. Der Verantwortliche sollte der
    betroffenen Person alle weiteren Informationen zur Verfügung stellen, die unter Berücksichtigung der besonderen
    Umstände und Rahmenbedingungen, unter denen die personenbezogenen Daten verarbeitet werden, notwendig sind, um eine
    faire und transparente Verarbeitung zu gewährleisten. Darüber hinaus sollte er die betroffene Person darauf
    hinweisen, dass Profiling stattfindet und welche Folgen dies hat. Werden die personenbezogenen Daten bei der
    betroffenen Person erhoben, so sollte dieser darüber hinaus mitgeteilt werden, ob sie verpflichtet ist, die
    personenbezogenen Daten bereitzustellen, und welche Folgen eine Zurückhaltung der Daten nach sich ziehen würde. Die
    betreffenden Informationen können in Kombination mit standardisierten Bildsymbolen bereitgestellt werden, um in
    leicht wahrnehmbarer, verständlicher und klar nachvollziehbarer Form einen aussagekräftigen Überblick über die
    beabsichtigte Verarbeitung zu vermitteln. Werden die Bildsymbole in elektronischer Form dargestellt, so sollten sie
    maschinenlesbar sein.%
   \label{itm:eg-60}
   
   \crossrefReasonToArticle{60}

   \item Dass sie betreffende personenbezogene Daten verarbeitet werden, sollte der betroffenen Person zum Zeitpunkt der
    Erhebung mitgeteilt werden oder, falls die Daten nicht von ihr, sondern aus einer anderen Quelle erlangt werden,
    innerhalb einer angemessenen Frist, die sich nach dem konkreten Einzelfall richtet. Wenn die personenbezogenen
    Daten rechtmäßig einem anderen Empfänger offengelegt werden dürfen, sollte die betroffene Person bei der
    erstmaligen Offenlegung der personenbezogenen Daten für diesen Empfänger darüber aufgeklärt werden. Beabsichtigt
    der Verantwortliche, die personenbezogenen Daten für einen anderen Zweck zu verarbeiten als den, für den die Daten
    erhoben wurden, so sollte er der betroffenen Person vor dieser Weiterverarbeitung Informationen über diesen anderen
    Zweck und andere erforderliche Informationen zur Verfügung stellen. Konnte der betroffenen Person nicht mitgeteilt
    werden, woher die personenbezogenen Daten stammen, weil verschiedene Quellen benutzt wurden, so sollte die
    Unterrichtung allgemein gehalten werden.%
   \label{itm:eg-61}
   
   \crossrefReasonToArticle{61}

   \item Die Pflicht, Informationen zur Verfügung zu stellen, erübrigt sich jedoch, wenn die betroffene Person die
    Information bereits hat, wenn die Speicherung oder Offenlegung der personenbezogenen Daten ausdrücklich durch
    Rechtsvorschriften geregelt ist oder wenn sich die Unterrichtung der betroffenen Person als unmöglich erweist oder
    mit unverhältnismäßig hohem Aufwand verbunden ist. Letzteres könnte insbesondere bei Verarbeitungen für im
    öffentlichen Interesse liegende Archivzwecke, zu wissenschaftlichen oder historischen Forschungszwecken oder zu
    statistischen Zwecken der Fall sein. Als Anhaltspunkte sollten dabei die Zahl der betroffenen Personen, das Alter
    der Daten oder etwaige geeignete Garantien in Betracht gezogen werden.%
   \label{itm:eg-62}
   
   \crossrefReasonToArticle{62}

   \item Eine betroffene Person sollte ein Auskunftsrecht hinsichtlich der sie betreffenden personenbezogenen Daten, die
    erhoben worden sind, besitzen und dieses Recht problemlos und in angemessenen Abständen wahrnehmen können, um sich
    der Verarbeitung bewusst zu sein und deren Rechtmäßigkeit überprüfen zu können. Dies schließt das Recht betroffene
    Personen auf Auskunft über ihre eigenen gesundheitsbezogenen Daten ein, etwa Daten in ihren Patientenakten, die
    Informationen wie beispielsweise Diagnosen, Untersuchungsergebnisse, Befunde der behandelnden Ärzte und Angaben zu
    Behandlungen oder Eingriffen enthalten. Jede betroffene Person sollte daher ein Anrecht darauf haben zu wissen und
    zu erfahren, insbesondere zu welchen Zwecken die personenbezogenen Daten verarbeitet werden und, wenn möglich, wie
    lange sie gespeichert werden, wer die Empfänger der personenbezogenen Daten sind, nach welcher Logik die
    automatische Verarbeitung personenbezogener Daten erfolgt und welche Folgen eine solche Verarbeitung haben kann,
    zumindest in Fällen, in denen die Verarbeitung auf Profiling beruht. Nach Möglichkeit sollte der Verantwortliche
    den Fernzugang zu einem sicheren System bereitstellen können, der der betroffenen Person direkten Zugang zu ihren
    personenbezogenen Daten ermöglichen würde. Dieses Recht sollte die Rechte und Freiheiten anderer Personen, etwa
    Geschäftsgeheimnisse oder Rechte des geistigen Eigentums und insbesondere das Urheberrecht an Software, nicht
    beeinträchtigen. Dies darf jedoch nicht dazu führen, dass der betroffenen Person jegliche Auskunft verweigert wird.
    Verarbeitet der Verantwortliche eine große Menge von Informationen über die betroffene Person, so sollte er
    verlangen können, dass die betroffene Person präzisiert, auf welche Information oder welche Verarbeitungsvorgänge
    sich ihr Auskunftsersuchen bezieht, bevor er ihr Auskunft erteilt.%
   \label{itm:eg-63}
   
   \crossrefReasonToArticle{63}

   \item Der Verantwortliche sollte alle vertretbaren Mittel nutzen, um die Identität einer Auskunft suchenden
    betroffenen Person zu überprüfen, insbesondere im Rahmen von Online"=Diensten und im Fall von Online"=Kennungen.
    Ein Verantwortlicher sollte personenbezogene Daten nicht allein zu dem Zweck speichern, auf mögliche
    Auskunftsersuchen reagieren zu können.%
   \label{itm:eg-64}
   
   \crossrefReasonToArticle{64}

   \item Eine betroffene Person sollte ein Recht auf Berichtigung der sie betreffenden personenbezogenen Daten besitzen
    sowie ein „Recht auf Vergessenwerden“, wenn die Speicherung ihrer Daten gegen diese Verordnung oder gegen das
    Unionsrecht oder das Recht der Mitgliedstaaten, dem der Verantwortliche unterliegt, verstößt. Insbesondere sollten
    betroffene Personen Anspruch darauf haben, dass ihre personenbezogenen Daten gelöscht und nicht mehr verarbeitet
    werden, wenn die personenbezogenen Daten hinsichtlich der Zwecke, für die sie erhoben bzw. anderweitig verarbeitet
    wurden, nicht mehr benötigt werden, wenn die betroffenen Personen ihre Einwilligung in die Verarbeitung widerrufen
    oder Widerspruch gegen die Verarbeitung der sie betreffenden personenbezogenen Daten eingelegt haben oder wenn die
    Verarbeitung ihrer personenbezogenen Daten aus anderen Gründen gegen diese Verordnung verstößt. Dieses Recht ist
    insbesondere wichtig in Fällen, in denen die betroffene Person ihre Einwilligung noch im Kindesalter gegeben hat
    und insofern die mit der Verarbeitung verbundenen Gefahren nicht in vollem Umfang absehen konnte und die
    personenbezogenen Daten — insbesondere die im Internet gespeicherten — später löschen möchte. Die betroffene Person
    sollte dieses Recht auch dann ausüben können, wenn sie kein Kind mehr ist. Die weitere Speicherung der
    personenbezogenen Daten sollte jedoch rechtmäßig sein, wenn dies für die Ausübung des Rechts auf freie
    Meinungsäußerung und Information, zur Erfüllung einer rechtlichen Verpflichtung, für die Wahrnehmung einer Aufgabe,
    die im öffentlichen Interesse liegt oder in Ausübung öffentlicher Gewalt erfolgt, die dem Verantwortlichen
    übertragen wurde, aus Gründen des öffentlichen Interesses im Bereich der öffentlichen Gesundheit, für im
    öffentlichen Interesse liegende Archivzwecke, zu wissenschaftlichen oder historischen Forschungszwecken oder zu
    statistischen Zwecken oder zur Geltendmachung, Ausübung oder Verteidigung von Rechtsansprüchen erforderlich ist.%
   \label{itm:eg-65}
   
   \crossrefReasonToArticle{65}

   \item Um dem „Recht auf Vergessenwerden“ im Netz mehr Geltung zu verschaffen, sollte das Recht auf Löschung
    ausgeweitet werden, indem ein Verantwortlicher, der die personenbezogenen Daten öffentlich gemacht hat,
    verpflichtet wird, den Verantwortlichen, die diese personenbezogenen Daten verarbeiten, mitzuteilen, alle Links zu
    diesen personenbezogenen Daten oder Kopien oder Replikationen der personenbezogenen Daten zu löschen. Dabei sollte
    der Verantwortliche, unter Berücksichtigung der verfügbaren Technologien und der ihm zur Verfügung stehenden
    Mittel, angemessene Maßnahmen — auch technischer Art — treffen, um die Verantwortlichen, die diese
    personenbezogenen Daten verarbeiten, über den Antrag der betroffenen Person zu informieren.%
   \label{itm:eg-66}
   
   \crossrefReasonToArticle{66}

   \item Methoden zur Beschränkung der Verarbeitung personenbezogener Daten könnten unter anderem darin bestehen, dass
    ausgewählte personenbezogenen Daten vorübergehend auf ein anderes Verarbeitungssystem übertragen werden, dass sie
    für Nutzer gesperrt werden oder dass veröffentliche Daten vorübergehend von einer Website entfernt werden. In
    automatisierten Dateisystemen sollte die Einschränkung der Verarbeitung grundsätzlich durch technische Mittel so
    erfolgen, dass die personenbezogenen Daten in keiner Weise weiterverarbeitet werden und nicht verändert werden
    können. Auf die Tatsache, dass die Verarbeitung der personenbezogenen Daten beschränkt wurde, sollte in dem System
    unmissverständlich hingewiesen werden.%
   \label{itm:eg-67}
   
   \crossrefReasonToArticle{67}

   \item Um im Fall der Verarbeitung personenbezogener Daten mit automatischen Mitteln eine bessere Kontrolle über die
    eigenen Daten zu haben, sollte die betroffene Person außerdem berechtigt sein, die sie betreffenden
    personenbezogenen Daten, die sie einem Verantwortlichen bereitgestellt hat, in einem strukturierten, gängigen,
    maschinenlesbaren und interoperablen Format zu erhalten und sie einem anderen Verantwortlichen zu übermitteln. Die
    Verantwortlichen sollten dazu aufgefordert werden, interoperable Formate zu entwickeln, die die
    Datenübertragbarkeit ermöglichen. Dieses Recht sollte dann gelten, wenn die betroffene Person die personenbezogenen
    Daten mit ihrer Einwilligung zur Verfügung gestellt hat oder die Verarbeitung zur Erfüllung eines Vertrags
    erforderlich ist. Es sollte nicht gelten, wenn die Verarbeitung auf einer anderen Rechtsgrundlage als ihrer
    Einwilligung oder eines Vertrags erfolgt. Dieses Recht sollte naturgemäß nicht gegen Verantwortliche ausgeübt
    werden, die personenbezogenen Daten in Erfüllung ihrer öffentlichen Aufgaben verarbeiten. Es sollte daher nicht
    gelten, wenn die Verarbeitung der personenbezogenen Daten zur Erfüllung einer rechtlichen Verpflichtung, der der
    Verantwortliche unterliegt, oder für die Wahrnehmung einer ihm übertragenen Aufgabe, die im öffentlichen Interesse
    liegt oder in Ausübung einer ihm übertragenen öffentlichen Gewalt erfolgt, erforderlich ist. Das Recht der
    betroffenen Person, sie betreffende personenbezogene Daten zu übermitteln oder zu empfangen, sollte für den
    Verantwortlichen nicht die Pflicht begründen, technisch kompatible Datenverarbeitungssysteme zu übernehmen oder
    beizubehalten. Ist im Fall eines bestimmten Satzes personenbezogener Daten mehr als eine betroffene Person
    tangiert, so sollte das Recht auf Empfang der Daten die Grundrechte und Grundfreiheiten anderer betroffener
    Personen nach dieser Verordnung unberührt lassen. Dieses Recht sollte zudem das Recht der betroffenen Person auf
    Löschung ihrer personenbezogenen Daten und die Beschränkungen dieses Rechts gemäß dieser Verordnung nicht berühren
    und insbesondere nicht bedeuten, dass die Daten, die sich auf die betroffene Person beziehen und von ihr zur
    Erfüllung eines Vertrags zur Verfügung gestellt worden sind, gelöscht werden, soweit und solange diese
    personenbezogenen Daten für die Erfüllung des Vertrags notwendig sind. Soweit technisch machbar, sollte die
    betroffene Person das Recht haben, zu erwirken, dass die personenbezogenen Daten direkt von einem Verantwortlichen
    einem anderen Verantwortlichen übermittelt werden.%
   \label{itm:eg-68}
   
   \crossrefReasonToArticle{68}

   \item Dürfen die personenbezogenen Daten möglicherweise rechtmäßig verarbeitet werden, weil die Verarbeitung für die
    Wahrnehmung einer Aufgabe, die im öffentlichen Interesse liegt oder in Ausübung öffentlicher Gewalt — die dem
    Verantwortlichen übertragen wurde, — oder aufgrund des berechtigten Interesses des Verantwortlichen oder eines
    Dritten erforderlich ist, sollte jede betroffene Person trotzdem das Recht haben, Widerspruch gegen die
    Verarbeitung der sich aus ihrer besonderen Situation ergebenden personenbezogenen Daten einzulegen. Der für die
    Verarbeitung Verantwortliche sollte darlegen müssen, dass seine zwingenden berechtigten Interessen Vorrang vor den
    Interessen oder Grundrechten und Grundfreiheiten der betroffenen Person haben.%
   \label{itm:eg-69}
   
   \crossrefReasonToArticle{69}

   \item Werden personenbezogene Daten verarbeitet, um Direktwerbung zu betreiben, so sollte die betroffene Person
    jederzeit unentgeltlich insoweit Widerspruch gegen eine solche — ursprüngliche oder spätere — Verarbeitung
    einschließlich des Profilings einlegen können, als sie mit dieser Direktwerbung zusammenhängt. Die betroffene
    Person sollte ausdrücklich auf dieses Recht hingewiesen werden; dieser Hinweis sollte in einer verständlichen und
    von anderen Informationen getrennten Form erfolgen.%
   \label{itm:eg-70}
   
   \crossrefReasonToArticle{70}

   \item Die betroffene Person sollte das Recht haben, keiner Entscheidung — was eine Maßnahme einschließen kann — zur
    Bewertung von sie betreffenden persönlichen Aspekten unterworfen zu werden, die ausschließlich auf einer
    automatisierten Verarbeitung beruht und die rechtliche Wirkung für die betroffene Person entfaltet oder sie in
    ähnlicher Weise erheblich beeinträchtigt, wie die automatische Ablehnung eines Online"=Kreditantrags oder
    Online"=Einstellungsverfahren ohne jegliches menschliche Eingreifen. Zu einer derartigen Verarbeitung zählt auch
    das „Profiling“, das in jeglicher Form automatisierter Verarbeitung personenbezogener Daten unter Bewertung der
    persönlichen Aspekte in Bezug auf eine natürliche Person besteht, insbesondere zur Analyse oder Prognose von
    Aspekten bezüglich Arbeitsleistung, wirtschaftliche Lage, Gesundheit, persönliche Vorlieben oder Interessen,
    Zuverlässigkeit oder Verhalten, Aufenthaltsort oder Ortswechsel der betroffenen Person, soweit dies rechtliche
    Wirkung für die betroffene Person entfaltet oder sie in ähnlicher Weise erheblich beeinträchtigt. Eine auf einer
    derartigen Verarbeitung, einschließlich des Profilings, beruhende Entscheidungsfindung sollte allerdings erlaubt
    sein, wenn dies nach dem Unionsrecht oder dem Recht der Mitgliedstaaten, dem der für die Verarbeitung
    Verantwortliche unterliegt, ausdrücklich zulässig ist, auch um im Einklang mit den Vorschriften, Standards und
    Empfehlungen der Institutionen der Union oder der nationalen Aufsichtsgremien Betrug und Steuerhinterziehung zu
    überwachen und zu verhindern und die Sicherheit und Zuverlässigkeit eines von dem Verantwortlichen bereitgestellten
    Dienstes zu gewährleisten, oder wenn dies für den Abschluss oder die Erfüllung eines Vertrags zwischen der
    betroffenen Person und einem Verantwortlichen erforderlich ist oder wenn die betroffene Person ihre ausdrückliche
    Einwilligung hierzu erteilt hat. In jedem Fall sollte eine solche Verarbeitung mit angemessenen Garantien verbunden
    sein, einschließlich der spezifischen Unterrichtung der betroffenen Person und des Anspruchs auf direktes
    Eingreifen einer Person, auf Darlegung des eigenen Standpunkts, auf Erläuterung der nach einer entsprechenden
    Bewertung getroffenen Entscheidung sowie des Rechts auf Anfechtung der Entscheidung. Diese Maßnahme sollte kein
    Kind betreffen. 

    Um unter Berücksichtigung der besonderen Umstände und Rahmenbedingungen, unter denen die personenbezogenen Daten
    verarbeitet werden, der betroffenen Person gegenüber eine faire und transparente Verarbeitung zu gewährleisten,
    sollte der für die Verarbeitung Verantwortliche geeignete mathematische oder statistische Verfahren für das
    Profiling verwenden, technische und organisatorische Maßnahmen treffen, mit denen in geeigneter Weise insbesondere
    sichergestellt wird, dass Faktoren, die zu unrichtigen personenbezogenen Daten führen, korrigiert werden und das
    Risiko von Fehlern minimiert wird, und personenbezogene Daten in einer Weise sichern, dass den potenziellen
    Bedrohungen für die Interessen und Rechte der betroffenen Person Rechnung getragen wird und mit denen verhindert
    wird, dass es gegenüber natürlichen Personen aufgrund von Rasse, ethnischer Herkunft, politischer Meinung, Religion
    oder Weltanschauung, Gewerkschaftszugehörigkeit, genetischer Anlagen oder Gesundheitszustand sowie sexueller
    Orientierung zu diskriminierenden Wirkungen oder zu Maßnahmen kommt, die eine solche Wirkung haben. Automatisierte
    Entscheidungsfindung und Profiling auf der Grundlage besonderer Kategorien von personenbezogenen Daten sollten nur
    unter bestimmten Bedingungen erlaubt sein.%
   \label{itm:eg-71}
   
   \crossrefReasonToArticle{71}

   \item Das Profiling unterliegt den Vorschriften dieser Verordnung für die Verarbeitung personenbezogener Daten, wie
    etwa die Rechtsgrundlage für die Verarbeitung oder die Datenschutzgrundsätze. Der durch diese Verordnung
    eingerichtete Europäische Datenschutzausschuss (im Folgenden „Ausschuss“) sollte, diesbezüglich Leitlinien
    herausgeben können.%
   \label{itm:eg-72}
   
   \crossrefReasonToArticle{72}

   \item Im Recht der Union oder der Mitgliedstaaten können Beschränkungen hinsichtlich bestimmter Grundsätze und
    hinsichtlich des Rechts auf Unterrichtung, Auskunft zu und Berichtigung oder Löschung personenbezogener Daten, des
    Rechts auf Datenübertragbarkeit und Widerspruch, Entscheidungen, die auf der Erstellung von Profilen beruhen, sowie
    Mitteilungen über eine Verletzung des Schutzes personenbezogener Daten an eine betroffene Person und bestimmten
    damit zusammenhängenden Pflichten der Verantwortlichen vorgesehen werden, soweit dies in einer demokratischen
    Gesellschaft notwendig und verhältnismäßig ist, um die öffentliche Sicherheit aufrechtzuerhalten, wozu unter
    anderem der Schutz von Menschenleben insbesondere bei Naturkatastrophen oder vom Menschen verursachten
    Katastrophen, die Verhütung, Aufdeckung und Verfolgung von Straftaten oder die Strafvollstreckung — was auch den
    Schutz vor und die Abwehr von Gefahren für die öffentliche Sicherheit einschließt — oder die Verhütung, Aufdeckung
    und Verfolgung von Verstößen gegen Berufsstandsregeln bei reglementierten Berufen, das Führen öffentlicher Register
    aus Gründen des allgemeinen öffentlichen Interesses sowie die Weiterverarbeitung von archivierten personenbezogenen
    Daten zur Bereitstellung spezifischer Informationen im Zusammenhang mit dem politischen Verhalten unter ehemaligen
    totalitären Regimen gehört, und zum Schutz sonstiger wichtiger Ziele des allgemeinen öffentlichen Interesses der
    Union oder eines Mitgliedstaats, etwa wichtige wirtschaftliche oder finanzielle Interessen, oder die betroffene
    Person und die Rechte und Freiheiten anderer Personen, einschließlich in den Bereichen soziale Sicherheit,
    öffentliche Gesundheit und humanitäre Hilfe, zu schützen. Diese Beschränkungen sollten mit der Charta und mit der
    Europäischen Konvention zum Schutz der Menschenrechte und Grundfreiheiten im Einklang stehen.%
   \label{itm:eg-73}
   
   \crossrefReasonToArticle{73}

   \item Die Verantwortung und Haftung des Verantwortlichen für jedwede Verarbeitung personenbezogener Daten, die durch
    ihn oder in seinem Namen erfolgt, sollte geregelt werden. Insbesondere sollte der Verantwortliche geeignete und
    wirksame Maßnahmen treffen müssen und nachweisen können, dass die Verarbeitungstätigkeiten im Einklang mit dieser
    Verordnung stehen und die Maßnahmen auch wirksam sind. Dabei sollte er die Art, den Umfang, die Umstände und die
    Zwecke der Verarbeitung und das Risiko für die Rechte und Freiheiten natürlicher Personen berücksichtigen.%
   \label{itm:eg-74}
   
   \crossrefReasonToArticle{74}

   \item Die Risiken für die Rechte und Freiheiten natürlicher Personen — mit unterschiedlicher
    Eintrittswahrscheinlichkeit und Schwere — können aus einer Verarbeitung personenbezogener Daten hervorgehen, die zu
    einem physischen, materiellen oder immateriellen Schaden führen könnte, insbesondere wenn die Verarbeitung zu einer
    Diskriminierung, einem Identitätsdiebstahl oder -betrug, einem finanziellen Verlust, einer Rufschädigung, einem
    Verlust der Vertraulichkeit von dem Berufsgeheimnis unterliegenden personenbezogenen Daten, der unbefugten
    Aufhebung der Pseudonymisierung oder anderen erheblichen wirtschaftlichen oder gesellschaftlichen Nachteilen führen
    kann, wenn die betroffenen Personen um ihre Rechte und Freiheiten gebracht oder daran gehindert werden, die sie
    betreffenden personenbezogenen Daten zu kontrollieren, wenn personenbezogene Daten, aus denen die rassische oder
    ethnische Herkunft, politische Meinungen, religiöse oder weltanschauliche Überzeugungen oder die Zugehörigkeit zu
    einer Gewerkschaft hervorgehen, und genetische Daten, Gesundheitsdaten oder das Sexualleben oder strafrechtliche
    Verurteilungen und Straftaten oder damit zusammenhängende Sicherungsmaßregeln betreffende Daten verarbeitet werden,
    wenn persönliche Aspekte bewertet werden, insbesondere wenn Aspekte, die die Arbeitsleistung, wirtschaftliche Lage,
    Gesundheit, persönliche Vorlieben oder Interessen, die Zuverlässigkeit oder das Verhalten, den Aufenthaltsort oder
    Ortswechsel betreffen, analysiert oder prognostiziert werden, um persönliche Profile zu erstellen oder zu nutzen,
    wenn personenbezogene Daten schutzbedürftiger natürlicher Personen, insbesondere Daten von Kindern, verarbeitet
    werden oder wenn die Verarbeitung eine große Menge personenbezogener Daten und eine große Anzahl von betroffenen
    Personen betrifft.%
   \label{itm:eg-75}
   
   \crossrefReasonToArticle{75}

   \item Eintrittswahrscheinlichkeit und Schwere des Risikos für die Rechte und Freiheiten der betroffenen Person
    sollten in Bezug auf die Art, den Umfang, die Umstände und die Zwecke der Verarbeitung bestimmt werden. Das Risiko
    sollte anhand einer objektiven Bewertung beurteilt werden, bei der festgestellt wird, ob die Datenverarbeitung ein
    Risiko oder ein hohes Risiko birgt.%
   \label{itm:eg-76}
   
   \crossrefReasonToArticle{76}

   \item Anleitungen, wie der Verantwortliche oder Auftragsverarbeiter geeignete Maßnahmen durchzuführen hat und wie die
    Einhaltung der Anforderungen nachzuweisen ist, insbesondere was die Ermittlung des mit der Verarbeitung verbundenen
    Risikos, dessen Abschätzung in Bezug auf Ursache, Art, Eintrittswahrscheinlichkeit und Schwere und die Festlegung
    bewährter Verfahren für dessen Eindämmung betrifft, könnten insbesondere in Form von genehmigten Verhaltensregeln,
    genehmigten Zertifizierungsverfahren, Leitlinien des Ausschusses oder Hinweisen eines Datenschutzbeauftragten
    gegeben werden. Der Ausschuss kann ferner Leitlinien für Verarbeitungsvorgänge ausgeben, bei denen davon auszugehen
    ist, dass sie kein hohes Risiko für die Rechte und Freiheiten natürlicher Personen mit sich bringen, und angeben,
    welche Abhilfemaßnahmen in diesen Fällen ausreichend sein können.%
   \label{itm:eg-77}
   
   \crossrefReasonToArticle{77}

   \item Zum Schutz der in Bezug auf die Verarbeitung personenbezogener Daten bestehenden Rechte und Freiheiten
    natürlicher Personen ist es erforderlich, dass geeignete technische und organisatorische Maßnahmen getroffen
    werden, damit die Anforderungen dieser Verordnung erfüllt werden. Um die Einhaltung dieser Verordnung nachweisen zu
    können, sollte der Verantwortliche interne Strategien festlegen und Maßnahmen ergreifen, die insbesondere den
    Grundsätzen des Datenschutzes durch Technik (data protection by design) und durch datenschutzfreundliche
    Voreinstellungen (data protection by default) Genüge tun. Solche Maßnahmen könnten unter anderem darin bestehen,
    dass die Verarbeitung personenbezogener Daten minimiert wird, personenbezogene Daten so schnell wie möglich
    pseudonymisiert werden, Transparenz in Bezug auf die Funktionen und die Verarbeitung personenbezogener Daten
    hergestellt wird, der betroffenen Person ermöglicht wird, die Verarbeitung personenbezogener Daten zu überwachen,
    und der Verantwortliche in die Lage versetzt wird, Sicherheitsfunktionen zu schaffen und zu verbessern. In Bezug
    auf Entwicklung, Gestaltung, Auswahl und Nutzung von Anwendungen, Diensten und Produkten, die entweder auf der
    Verarbeitung von personenbezogenen Daten beruhen oder zur Erfüllung ihrer Aufgaben personenbezogene Daten
    verarbeiten, sollten die Hersteller der Produkte, Dienste und Anwendungen ermutigt werden, das Recht auf
    Datenschutz bei der Entwicklung und Gestaltung der Produkte, Dienste und Anwendungen zu berücksichtigen und unter
    gebührender Berücksichtigung des Stands der Technik sicherzustellen, dass die Verantwortlichen und die Verarbeiter
    in der Lage sind, ihren Datenschutzpflichten nachzukommen. Den Grundsätzen des Datenschutzes durch Technik und
    durch datenschutzfreundliche Voreinstellungen sollte auch bei öffentlichen Ausschreibungen Rechnung getragen
    werden.%
   \label{itm:eg-78}
   
   \crossrefReasonToArticle{78}

   \item Zum Schutz der Rechte und Freiheiten der betroffenen Personen sowie bezüglich der Verantwortung und Haftung der
    Verantwortlichen und der Auftragsverarbeiter bedarf es — auch mit Blick auf die Überwachungs- und sonstigen
    Maßnahmen von Aufsichtsbehörden — einer klaren Zuteilung der Verantwortlichkeiten durch diese Verordnung,
    einschließlich der Fälle, in denen ein Verantwortlicher die Verarbeitungszwecke und -mittel gemeinsam mit anderen
    Verantwortlichen festlegt oder ein Verarbeitungsvorgang im Auftrag eines Verantwortlichen durchgeführt wird.%
   \label{itm:eg-79}
   
   \crossrefReasonToArticle{79}

   \item Jeder Verantwortliche oder Auftragsverarbeiter ohne Niederlassung in der Union, dessen Verarbeitungstätigkeiten
    sich auf betroffene Personen beziehen, die sich in der Union aufhalten, und dazu dienen, diesen Personen in der
    Union Waren oder Dienstleistungen anzubieten — unabhängig davon, ob von der betroffenen Person eine Zahlung
    verlangt wird — oder deren Verhalten, soweit dieses innerhalb der Union erfolgt, zu beobachten, sollte einen
    Vertreter benennen müssen, es sei denn, die Verarbeitung erfolgt gelegentlich, schließt nicht die umfangreiche
    Verarbeitung besonderer Kategorien personenbezogener Daten oder die Verarbeitung von personenbezogenen Daten über
    strafrechtliche Verurteilungen und Straftaten ein und bringt unter Berücksichtigung ihrer Art, ihrer Umstände,
    ihres Umfangs und ihrer Zwecke wahrscheinlich kein Risiko für die Rechte und Freiheiten natürlicher Personen mit
    sich oder bei dem Verantwortlichen handelt es sich um eine Behörde oder öffentliche Stelle. Der Vertreter sollte im
    Namen des Verantwortlichen oder des Auftragsverarbeiters tätig werden und den Aufsichtsbehörden als Anlaufstelle
    dienen. Der Verantwortliche oder der Auftragsverarbeiter sollte den Vertreter ausdrücklich bestellen und
    schriftlich beauftragen, in Bezug auf die ihm nach dieser Verordnung obliegenden Verpflichtungen an seiner Stelle
    zu handeln. Die Benennung eines solchen Vertreters berührt nicht die Verantwortung oder Haftung des
    Verantwortlichen oder des Auftragsverarbeiters nach Maßgabe dieser Verordnung. Ein solcher Vertreter sollte seine
    Aufgaben entsprechend dem Mandat des Verantwortlichen oder Auftragsverarbeiters ausführen und insbesondere mit den
    zuständigen Aufsichtsbehörden in Bezug auf Maßnahmen, die die Einhaltung dieser Verordnung sicherstellen sollen,
    zusammenarbeiten. Bei Verstößen des Verantwortlichen oder Auftragsverarbeiters sollte der bestellte Vertreter
    Durchsetzungsverfahren unterworfen werden.%
   \label{itm:eg-80}
   
   \crossrefReasonToArticle{80}

   \item Damit die Anforderungen dieser Verordnung in Bezug auf die vom Auftragsverarbeiter im Namen des
    Verantwortlichen vorzunehmende Verarbeitung eingehalten werden, sollte ein Verantwortlicher, der einen
    Auftragsverarbeiter mit Verarbeitungstätigkeiten betrauen will, nur Auftragsverarbeiter heranziehen, die —
    insbesondere im Hinblick auf Fachwissen, Zuverlässigkeit und Ressourcen — hinreichende Garantien dafür bieten, dass
    technische und organisatorische Maßnahmen — auch für die Sicherheit der Verarbeitung — getroffen werden, die den
    Anforderungen dieser Verordnung genügen. Die Einhaltung genehmigter Verhaltensregeln oder eines genehmigten
    Zertifizierungsverfahrens durch einen Auftragsverarbeiter kann als Faktor herangezogen werden, um die Erfüllung der
    Pflichten des Verantwortlichen nachzuweisen. Die Durchführung einer Verarbeitung durch einen Auftragsverarbeiter
    sollte auf Grundlage eines Vertrags oder eines anderen Rechtsinstruments nach dem Recht der Union oder der
    Mitgliedstaaten erfolgen, der bzw. das den Auftragsverarbeiter an den Verantwortlichen bindet und in dem Gegenstand
    und Dauer der Verarbeitung, Art und Zwecke der Verarbeitung, die Art der personenbezogenen Daten und die Kategorien
    von betroffenen Personen festgelegt sind, wobei die besonderen Aufgaben und Pflichten des Auftragsverarbeiters bei
    der geplanten Verarbeitung und das Risiko für die Rechte und Freiheiten der betroffenen Person zu berücksichtigen
    sind. Der Verantwortliche und der Auftragsverarbeiter können entscheiden, ob sie einen individuellen Vertrag oder
    Standardvertragsklauseln verwenden, die entweder unmittelbar von der Kommission erlassen oder aber nach dem
    Kohärenzverfahren von einer Aufsichtsbehörde angenommen und dann von der Kommission erlassen wurden. Nach
    Beendigung der Verarbeitung im Namen des Verantwortlichen sollte der Auftragsverarbeiter die personenbezogenen
    Daten nach Wahl des Verantwortlichen entweder zurückgeben oder löschen, sofern nicht nach dem Recht der Union oder
    der Mitgliedstaaten, dem der Auftragsverarbeiter unterliegt, eine Verpflichtung zur Speicherung der
    personenbezogenen Daten besteht.%
   \label{itm:eg-81}
   
   \crossrefReasonToArticle{81}

   \item Zum Nachweis der Einhaltung dieser Verordnung sollte der Verantwortliche oder der Auftragsverarbeiter ein
    Verzeichnis der Verarbeitungstätigkeiten, die seiner Zuständigkeit unterliegen, führen. Jeder Verantwortliche und
    jeder Auftragsverarbeiter sollte verpflichtet sein, mit der Aufsichtsbehörde zusammenzuarbeiten und dieser auf
    Anfrage das entsprechende Verzeichnis vorzulegen, damit die betreffenden Verarbeitungsvorgänge anhand dieser
    Verzeichnisse kontrolliert werden können.%
   \label{itm:eg-82}
   
   \crossrefReasonToArticle{82}

   \item Zur Aufrechterhaltung der Sicherheit und zur Vorbeugung gegen eine gegen diese Verordnung verstoßende
    Verarbeitung sollte der Verantwortliche oder der Auftragsverarbeiter die mit der Verarbeitung verbundenen Risiken
    ermitteln und Maßnahmen zu ihrer Eindämmung, wie etwa eine Verschlüsselung, treffen. Diese Maßnahmen sollten unter
    Berücksichtigung des Stands der Technik und der Implementierungskosten ein Schutzniveau — auch hinsichtlich der
    Vertraulichkeit — gewährleisten, das den von der Verarbeitung ausgehenden Risiken und der Art der zu schützenden
    personenbezogenen Daten angemessen ist. Bei der Bewertung der Datensicherheitsrisiken sollten die mit der
    Verarbeitung personenbezogener Daten verbundenen Risiken berücksichtigt werden, wie etwa — ob unbeabsichtigt oder
    unrechtmäßig — Vernichtung, Verlust, Veränderung oder unbefugte Offenlegung von oder unbefugter Zugang zu
    personenbezogenen Daten, die übermittelt, gespeichert oder auf sonstige Weise verarbeitet wurden, insbesondere wenn
    dies zu einem physischen, materiellen oder immateriellen Schaden führen könnte.%
   \label{itm:eg-83}
   
   \crossrefReasonToArticle{83}

   \item Damit diese Verordnung in Fällen, in denen die Verarbeitungsvorgänge wahrscheinlich ein hohes Risiko für die
    Rechte und Freiheiten natürlicher Personen mit sich bringen, besser eingehalten wird, sollte der Verantwortliche
    für die Durchführung einer Datenschutz"=Folgenabschätzung, mit der insbesondere die Ursache, Art, Besonderheit und
    Schwere dieses Risikos evaluiert werden, verantwortlich sein. Die Ergebnisse der Abschätzung sollten berücksichtigt
    werden, wenn darüber entschieden wird, welche geeigneten Maßnahmen ergriffen werden müssen, um nachzuweisen, dass
    die Verarbeitung der personenbezogenen Daten mit dieser Verordnung in Einklang steht. Geht aus einer
    Datenschutz"=Folgenabschätzung hervor, dass Verarbeitungsvorgänge ein hohes Risiko bergen, das der Verantwortliche
    nicht durch geeignete Maßnahmen in Bezug auf verfügbare Technik und Implementierungskosten eindämmen kann, so
    sollte die Aufsichtsbehörde vor der Verarbeitung konsultiert werden.%
   \label{itm:eg-84}
   
   \crossrefReasonToArticle{84}

   \item Eine Verletzung des Schutzes personenbezogener Daten kann — wenn nicht rechtzeitig und angemessen reagiert
    wird — einen physischen, materiellen oder immateriellen Schaden für natürliche Personen nach sich ziehen, wie etwa
    Verlust der Kontrolle über ihre personenbezogenen Daten oder Einschränkung ihrer Rechte, Diskriminierung,
    Identitätsdiebstahl oder -betrug, finanzielle Verluste, unbefugte Aufhebung der Pseudonymisierung, Rufschädigung,
    Verlust der Vertraulichkeit von dem Berufsgeheimnis unterliegenden Daten oder andere erhebliche wirtschaftliche
    oder gesellschaftliche Nachteile für die betroffene natürliche Person. Deshalb sollte der Verantwortliche, sobald
    ihm eine Verletzung des Schutzes personenbezogener Daten bekannt wird, die Aufsichtsbehörde von der Verletzung des
    Schutzes personenbezogener Daten unverzüglich und, falls möglich, binnen höchstens 72 Stunden, nachdem ihm die
    Verletzung bekannt wurde, unterrichten, es sei denn, der Verantwortliche kann im Einklang mit dem Grundsatz der
    Rechenschaftspflicht nachweisen, dass die Verletzung des Schutzes personenbezogener Daten voraussichtlich nicht zu
    einem Risiko für die persönlichen Rechte und Freiheiten natürlicher Personen führt. Falls diese Benachrichtigung
    nicht binnen 72 Stunden erfolgen kann, sollten in ihr die Gründe für die Verzögerung angegeben werden müssen, und
    die Informationen können schrittweise ohne unangemessene weitere Verzögerung bereitgestellt werden.%
   \label{itm:eg-85}
   
   \crossrefReasonToArticle{85}

   \item Der für die Verarbeitung Verantwortliche sollte die betroffene Person unverzüglich von der Verletzung des
    Schutzes personenbezogener Daten benachrichtigen, wenn diese Verletzung des Schutzes personenbezogener Daten
    voraussichtlich zu einem hohen Risiko für die persönlichen Rechte und Freiheiten natürlicher Personen führt, damit
    diese die erforderlichen Vorkehrungen treffen können. Die Benachrichtigung sollte eine Beschreibung der Art der
    Verletzung des Schutzes personenbezogener Daten sowie an die betroffene natürliche Person gerichtete Empfehlungen
    zur Minderung etwaiger nachteiliger Auswirkungen dieser Verletzung enthalten. Solche Benachrichtigungen der
    betroffenen Person sollten stets so rasch wie nach allgemeinem Ermessen möglich, in enger Absprache mit der
    Aufsichtsbehörde und nach Maßgabe der von dieser oder von anderen zuständigen Behörden wie beispielsweise
    Strafverfolgungsbehörden erteilten Weisungen erfolgen. Um beispielsweise das Risiko eines unmittelbaren Schadens
    mindern zu können, müssten betroffene Personen sofort benachrichtigt werden, wohingegen eine längere
    Benachrichtigungsfrist gerechtfertigt sein kann, wenn es darum geht, geeignete Maßnahmen gegen fortlaufende oder
    vergleichbare Verletzungen des Schutzes personenbezogener Daten zu treffen.%
   \label{itm:eg-86}
   
   \crossrefReasonToArticle{86}

   \item Es sollte festgestellt werden, ob alle geeigneten technischen Schutz- sowie organisatorischen Maßnahmen
    getroffen wurden, um sofort feststellen zu können, ob eine Verletzung des Schutzes personenbezogener Daten
    aufgetreten ist, und um die Aufsichtsbehörde und die betroffene Person umgehend unterrichten zu können. Bei der
    Feststellung, ob die Meldung unverzüglich erfolgt ist, sollten die Art und Schwere der Verletzung des Schutzes
    personenbezogener Daten sowie deren Folgen und nachteilige Auswirkungen für die betroffene Person berücksichtigt
    werden. Die entsprechende Meldung kann zu einem Tätigwerden der Aufsichtsbehörde im Einklang mit ihren in dieser
    Verordnung festgelegten Aufgaben und Befugnissen führen.%
   \label{itm:eg-87}
   
   \crossrefReasonToArticle{87}

   \item Bei der detaillierten Regelung des Formats und der Verfahren für die Meldung von Verletzungen des Schutzes
    personenbezogener Daten sollten die Umstände der Verletzung hinreichend berücksichtigt werden, beispielsweise ob
    personenbezogene Daten durch geeignete technische Sicherheitsvorkehrungen geschützt waren, die die
    Wahrscheinlichkeit eines Identitätsbetrugs oder anderer Formen des Datenmissbrauchs wirksam verringern. Überdies
    sollten solche Regeln und Verfahren den berechtigten Interessen der Strafverfolgungsbehörden in Fällen Rechnung
    tragen, in denen die Untersuchung der Umstände einer Verletzung des Schutzes personenbezogener Daten durch eine
    frühzeitige Offenlegung in unnötiger Weise behindert würde.%
   \label{itm:eg-88}
   
   \crossrefReasonToArticle{88}

   \item Gemäß der Richtlinie 95/46/EG waren Verarbeitungen personenbezogener Daten bei den Aufsichtsbehörden generell
    meldepflichtig. Diese Meldepflicht ist mit einem bürokratischen und finanziellen Aufwand verbunden und hat dennoch
    nicht in allen Fällen zu einem besseren Schutz personenbezogener Daten geführt. Diese unterschiedslosen allgemeinen
    Meldepflichten sollten daher abgeschafft und durch wirksame Verfahren und Mechanismen ersetzt werden, die sich
    stattdessen vorrangig mit denjenigen Arten von Verarbeitungsvorgängen befassen, die aufgrund ihrer Art, ihres
    Umfangs, ihrer Umstände und ihrer Zwecke wahrscheinlich ein hohes Risiko für die Rechte und Freiheiten natürlicher
    Personen mit sich bringen. Zu solchen Arten von Verarbeitungsvorgängen gehören insbesondere solche, bei denen neue
    Technologien eingesetzt werden oder die neuartig sind und bei denen der Verantwortliche noch keine
    Datenschutz"=Folgenabschätzung durchgeführt hat bzw. bei denen aufgrund der seit der ursprünglichen Verarbeitung
    vergangenen Zeit eine Datenschutz"=Folgenabschätzung notwendig geworden ist.%
   \label{itm:eg-89}
   
   \crossrefReasonToArticle{89}

   \item In derartigen Fällen sollte der Verantwortliche vor der Verarbeitung eine Datenschutz"=Folgenabschätzung
    durchführen, mit der die spezifische Eintrittswahrscheinlichkeit und die Schwere dieses hohen Risikos unter
    Berücksichtigung der Art, des Umfangs, der Umstände und der Zwecke der Verarbeitung und der Ursachen des Risikos
    bewertet werden. Diese Folgenabschätzung sollte sich insbesondere mit den Maßnahmen, Garantien und Verfahren
    befassen, durch die dieses Risiko eingedämmt, der Schutz personenbezogener Daten sichergestellt und die Einhaltung
    der Bestimmungen dieser Verordnung nachgewiesen werden soll.%
   \label{itm:eg-90}
   
   \crossrefReasonToArticle{90}

   \item Dies sollte insbesondere für umfangreiche Verarbeitungsvorgänge gelten, die dazu dienen, große Mengen
    personenbezogener Daten auf regionaler, nationaler oder supranationaler Ebene zu verarbeiten, eine große Zahl von
    Personen betreffen könnten und — beispielsweise aufgrund ihrer Sensibilität — wahrscheinlich ein hohes Risiko mit
    sich bringen und bei denen entsprechend dem jeweils aktuellen Stand der Technik in großem Umfang eine neue
    Technologie eingesetzt wird, sowie für andere Verarbeitungsvorgänge, die ein hohes Risiko für die Rechte und
    Freiheiten der betroffenen Personen mit sich bringen, insbesondere dann, wenn diese Verarbeitungsvorgänge den
    betroffenen Personen die Ausübung ihrer Rechte erschweren. Eine Datenschutz"=Folgenabschätzung sollte auch
    durchgeführt werden, wenn die personenbezogenen Daten für das Treffen von Entscheidungen in Bezug auf bestimmte
    natürliche Personen im Anschluss an eine systematische und eingehende Bewertung persönlicher Aspekte natürlicher
    Personen auf der Grundlage eines Profilings dieser Daten oder im Anschluss an die Verarbeitung besonderer
    Kategorien von personenbezogenen Daten, biometrischen Daten oder von Daten über strafrechtliche Verurteilungen und
    Straftaten sowie damit zusammenhängende Sicherungsmaßregeln verarbeitet werden. Gleichermaßen erforderlich ist eine
    Datenschutz"=Folgenabschätzung für die weiträumige Überwachung öffentlich zugänglicher Bereiche, insbesondere
    mittels optoelektronischer Vorrichtungen, oder für alle anderen Vorgänge, bei denen nach Auffassung der zuständigen
    Aufsichtsbehörde die Verarbeitung wahrscheinlich ein hohes Risiko für die Rechte und Freiheiten der betroffenen
    Personen mit sich bringt, insbesondere weil sie die betroffenen Personen an der Ausübung eines Rechts oder der
    Nutzung einer Dienstleistung bzw. Durchführung eines Vertrags hindern oder weil sie systematisch in großem Umfang
    erfolgen. Die Verarbeitung personenbezogener Daten sollte nicht als umfangreich gelten, wenn die Verarbeitung
    personenbezogene Daten von Patienten oder von Mandanten betrifft und durch einen einzelnen Arzt, sonstigen
    Angehörigen eines Gesundheitsberufes oder Rechtsanwalt erfolgt. In diesen Fällen sollte eine
    Datenschutz"=Folgenabschätzung nicht zwingend vorgeschrieben sein.%
   \label{itm:eg-91}
   
   \crossrefReasonToArticle{91}

   \item Unter bestimmten Umständen kann es vernünftig und unter ökonomischen Gesichtspunkten zweckmäßig sein, eine
    Datenschutz"=Folgenabschätzung nicht lediglich auf ein bestimmtes Projekt zu beziehen, sondern sie thematisch
    breiter anzulegen — beispielsweise wenn Behörden oder öffentliche Stellen eine gemeinsame Anwendung oder
    Verarbeitungsplattform schaffen möchten oder wenn mehrere Verantwortliche eine gemeinsame Anwendung oder
    Verarbeitungsumgebung für einen gesamten Wirtschaftssektor, für ein bestimmtes Marktsegment oder für eine weit
    verbreitete horizontale Tätigkeit einführen möchten.%
   \label{itm:eg-92}
   
   \crossrefReasonToArticle{92}

   \item Anlässlich des Erlasses des Gesetzes des Mitgliedstaats, auf dessen Grundlage die Behörde oder öffentliche
    Stelle ihre Aufgaben wahrnimmt und das den fraglichen Verarbeitungsvorgang oder die fraglichen Arten von
    Verarbeitungsvorgängen regelt, können die Mitgliedstaaten es für erforderlich erachten, solche Folgeabschätzungen
    vor den Verarbeitungsvorgängen durchzuführen.%
   \label{itm:eg-93}
   
   \crossrefReasonToArticle{93}

   \item Geht aus einer Datenschutz"=Folgenabschätzung hervor, dass die Verarbeitung bei Fehlen von Garantien,
    Sicherheitsvorkehrungen und Mechanismen zur Minderung des Risikos ein hohes Risiko für die Rechte und Freiheiten
    natürlicher Personen mit sich bringen würde, und ist der Verantwortliche der Auffassung, dass das Risiko nicht
    durch in Bezug auf verfügbare Technologien und Implementierungskosten vertretbare Mittel eingedämmt werden kann, so
    sollte die Aufsichtsbehörde vor Beginn der Verarbeitungstätigkeiten konsultiert werden. Ein solches hohes Risiko
    ist wahrscheinlich mit bestimmten Arten der Verarbeitung und dem Umfang und der Häufigkeit der Verarbeitung
    verbunden, die für natürliche Personen auch eine Schädigung oder eine Beeinträchtigung der persönlichen Rechte und
    Freiheiten mit sich bringen können. Die Aufsichtsbehörde sollte das Beratungsersuchen innerhalb einer bestimmten
    Frist beantworten. Allerdings kann sie, auch wenn sie nicht innerhalb dieser Frist reagiert hat, entsprechend ihren
    in dieser Verordnung festgelegten Aufgaben und Befugnissen eingreifen, was die Befugnis einschließt,
    Verarbeitungsvorgänge zu untersagen. Im Rahmen dieses Konsultationsprozesses kann das Ergebnis einer im Hinblick
    auf die betreffende Verarbeitung personenbezogener Daten durchgeführten Datenschutz"=Folgenabschätzung der
    Aufsichtsbehörde unterbreitet werden; dies gilt insbesondere für die zur Eindämmung des Risikos für die Rechte und
    Freiheiten natürlicher Personen geplanten Maßnahmen.%
   \label{itm:eg-94}
   
   \crossrefReasonToArticle{94}

   \item Der Auftragsverarbeiter sollte erforderlichenfalls den Verantwortlichen auf Anfrage bei der Gewährleistung der
    Einhaltung der sich aus der Durchführung der Datenschutz"=Folgenabschätzung und der vorherigen Konsultation der
    Aufsichtsbehörde ergebenden Auflagen unterstützen.%
   \label{itm:eg-95}
   
   \crossrefReasonToArticle{95}

   \item Eine Konsultation der Aufsichtsbehörde sollte auch während der Ausarbeitung von Gesetzes- oder
    Regelungsvorschriften, in denen eine Verarbeitung personenbezogener Daten vorgesehen ist, erfolgen, um die
    Vereinbarkeit der geplanten Verarbeitung mit dieser Verordnung sicherzustellen und insbesondere das mit ihr für die
    betroffene Person verbundene Risiko einzudämmen.%
   \label{itm:eg-96}
   
   \crossrefReasonToArticle{96}

   \item In Fällen, in denen die Verarbeitung durch eine Behörde — mit Ausnahmen von Gerichten oder unabhängigen
    Justizbehörden, die im Rahmen ihrer justiziellen Tätigkeit handeln –, im privaten Sektor durch einen
    Verantwortlichen erfolgt, dessen Kerntätigkeit in Verarbeitungsvorgängen besteht, die eine regelmäßige und
    systematische Überwachung der betroffenen Personen in großem Umfang erfordern, oder wenn die Kerntätigkeit des
    Verantwortlichen oder des Auftragsverarbeiters in der umfangreichen Verarbeitung besonderer Kategorien von
    personenbezogenen Daten oder von Daten über strafrechtliche Verurteilungen und Straftaten besteht, sollte der
    Verantwortliche oder der Auftragsverarbeiter bei der Überwachung der internen Einhaltung der Bestimmungen dieser
    Verordnung von einer weiteren Person, die über Fachwissen auf dem Gebiet des Datenschutzrechts und der
    Datenschutzverfahren verfügt, unterstützt werden Im privaten Sektor bezieht sich die Kerntätigkeit eines
    Verantwortlichen auf seine Haupttätigkeiten und nicht auf die Verarbeitung personenbezogener Daten als
    Nebentätigkeit. Das erforderliche Niveau des Fachwissens sollte sich insbesondere nach den durchgeführten
    Datenverarbeitungsvorgängen und dem erforderlichen Schutz für die von dem Verantwortlichen oder dem
    Auftragsverarbeiter verarbeiteten personenbezogenen Daten richten. Derartige Datenschutzbeauftragte sollten
    unabhängig davon, ob es sich bei ihnen um Beschäftigte des Verantwortlichen handelt oder nicht, ihre Pflichten und
    Aufgaben in vollständiger Unabhängigkeit ausüben können.%
   \label{itm:eg-97}
   
   \crossrefReasonToArticle{97}

   \item Verbände oder andere Vereinigungen, die bestimmte Kategorien von Verantwortlichen oder Auftragsverarbeitern
    vertreten, sollten ermutigt werden, in den Grenzen dieser Verordnung Verhaltensregeln auszuarbeiten, um eine
    wirksame Anwendung dieser Verordnung zu erleichtern, wobei den Besonderheiten der in bestimmten Sektoren
    erfolgenden Verarbeitungen und den besonderen Bedürfnissen der Kleinstunternehmen sowie der kleinen und mittleren
    Unternehmen Rechnung zu tragen ist. Insbesondere könnten in diesen Verhaltensregeln — unter Berücksichtigung des
    mit der Verarbeitung wahrscheinlich einhergehenden Risikos für die Rechte und Freiheiten natürlicher Personen — die
    Pflichten der Verantwortlichen und der Auftragsverarbeiter bestimmt werden.%
   \label{itm:eg-98}
   
   \crossrefReasonToArticle{98}

   \item Bei der Ausarbeitung oder bei der Änderung oder Erweiterung solcher Verhaltensregeln sollten Verbände und oder
    andere Vereinigungen, die bestimmte Kategorien von Verantwortlichen oder Auftragsverarbeitern vertreten, die
    maßgeblichen Interessenträger, möglichst auch die betroffenen Personen, konsultieren und die Eingaben und
    Stellungnahmen, die sie dabei erhalten, berücksichtigen.%
   \label{itm:eg-99}
   
   \crossrefReasonToArticle{99}

   \item Um die Transparenz zu erhöhen und die Einhaltung dieser Verordnung zu verbessern, sollte angeregt werden, dass
    Zertifizierungsverfahren sowie Datenschutzsiegel und -prüfzeichen eingeführt werden, die den betroffenen Personen
    einen raschen Überblick über das Datenschutzniveau einschlägiger Produkte und Dienstleistungen ermöglichen.%
   \label{itm:eg-100}
   
   \crossrefReasonToArticle{100}

   \item Der Fluss personenbezogener Daten aus Drittländern und internationalen Organisationen und in Drittländer und
    internationale Organisationen ist für die Ausweitung des internationalen Handels und der internationalen
    Zusammenarbeit notwendig. Durch die Zunahme dieser Datenströme sind neue Herausforderungen und Anforderungen in
    Bezug auf den Schutz personenbezogener Daten entstanden. Das durch diese Verordnung unionsweit gewährleistete
    Schutzniveau für natürliche Personen sollte jedoch bei der Übermittlung personenbezogener Daten aus der Union an
    Verantwortliche, Auftragsverarbeiter oder andere Empfänger in Drittländern oder an internationale Organisationen
    nicht untergraben werden, und zwar auch dann nicht, wenn aus einem Drittland oder von einer internationalen
    Organisation personenbezogene Daten an Verantwortliche oder Auftragsverarbeiter in demselben oder einem anderen
    Drittland oder an dieselbe oder eine andere internationale Organisation weiterübermittelt werden. In jedem Fall
    sind derartige Datenübermittlungen an Drittländer und internationale Organisationen nur unter strikter Einhaltung
    dieser Verordnung zulässig. Eine Datenübermittlung könnte nur stattfinden, wenn die in dieser Verordnung
    festgelegten Bedingungen zur Übermittlung personenbezogener Daten an Drittländer oder internationale Organisationen
    vorbehaltlich der übrigen Bestimmungen dieser Verordnung von dem Verantwortlichen oder dem Auftragsverarbeiter
    erfüllt werden.%
   \label{itm:eg-101}
   
   \crossrefReasonToArticle{101}

   \item Internationale Abkommen zwischen der Union und Drittländern über die Übermittlung von personenbezogenen Daten
    einschließlich geeigneter Garantien für die betroffenen Personen werden von dieser Verordnung nicht berührt. Die
    Mitgliedstaaten können völkerrechtliche Übereinkünfte schließen, die die Übermittlung personenbezogener Daten an
    Drittländer oder internationale Organisationen beinhalten, sofern sich diese Übereinkünfte weder auf diese
    Verordnung noch auf andere Bestimmungen des Unionsrechts auswirken und ein angemessenes Schutzniveau für die
    Grundrechte der betroffenen Personen umfassen.%
   \label{itm:eg-102}
   
   \crossrefReasonToArticle{102}

   \item Die Kommission darf mit Wirkung für die gesamte Union beschließen, dass ein bestimmtes Drittland, ein Gebiet
    oder ein bestimmter Sektor eines Drittlands oder eine internationale Organisation ein angemessenes
    Datenschutzniveau bietet, und auf diese Weise in Bezug auf das Drittland oder die internationale Organisation, das
    bzw. die für fähig gehalten wird, ein solches Schutzniveau zu bieten, in der gesamten Union Rechtssicherheit
    schaffen und eine einheitliche Rechtsanwendung sicherstellen. In derartigen Fällen dürfen personenbezogene Daten
    ohne weitere Genehmigung an dieses Land oder diese internationale Organisation übermittelt werden. Die Kommission
    kann, nach Abgabe einer ausführlichen Erklärung, in der dem Drittland oder der internationalen Organisation eine
    Begründung gegeben wird, auch entscheiden, eine solche Feststellung zu widerrufen.%
   \label{itm:eg-103}
   
   \crossrefReasonToArticle{103}

   \item In Übereinstimmung mit den Grundwerten der Union, zu denen insbesondere der Schutz der Menschenrechte zählt,
    sollte die Kommission bei der Bewertung des Drittlands oder eines Gebiets oder eines bestimmten Sektors eines
    Drittlands berücksichtigen, inwieweit dort die Rechtsstaatlichkeit gewahrt ist, der Rechtsweg gewährleistet ist und
    die internationalen Menschenrechtsnormen und -standards eingehalten werden und welche allgemeinen und
    sektorspezifischen Vorschriften, wozu auch die Vorschriften über die öffentliche Sicherheit, die Landesverteidigung
    und die nationale Sicherheit sowie die öffentliche Ordnung und das Strafrecht zählen, dort gelten. Die Annahme
    eines Angemessenheitsbeschlusses in Bezug auf ein Gebiet oder einen bestimmten Sektor eines Drittlands sollte unter
    Berücksichtigung eindeutiger und objektiver Kriterien wie bestimmter Verarbeitungsvorgänge und des
    Anwendungsbereichs anwendbarer Rechtsnormen und geltender Rechtsvorschriften in dem Drittland erfolgen. Das
    Drittland sollte Garantien für ein angemessenes Schutzniveau bieten, das dem innerhalb der Union gewährleisteten
    Schutzniveau der Sache nach gleichwertig ist, insbesondere in Fällen, in denen personenbezogene Daten in einem oder
    mehreren spezifischen Sektoren verarbeitet werden. Das Drittland sollte insbesondere eine wirksame unabhängige
    Überwachung des Datenschutzes gewährleisten und Mechanismen für eine Zusammenarbeit mit den Datenschutzbehörden der
    Mitgliedstaaten vorsehen, und den betroffenen Personen sollten wirksame und durchsetzbare Rechte sowie wirksame
    verwaltungsrechtliche und gerichtliche Rechtsbehelfe eingeräumt werden.%
   \label{itm:eg-104}
   
   \crossrefReasonToArticle{104}

   \item Die Kommission sollte neben den internationalen Verpflichtungen, die das Drittland oder die internationale
    Organisation eingegangen ist, die Verpflichtungen, die sich aus der Teilnahme des Drittlands oder der
    internationalen Organisation an multilateralen oder regionalen Systemen insbesondere im Hinblick auf den Schutz
    personenbezogener Daten ergeben, sowie die Umsetzung dieser Verpflichtungen berücksichtigen. Insbesondere sollte
    der Beitritt des Drittlands zum Übereinkommen des Europarates vom 28. Januar 1981 zum Schutz des Menschen bei der
    automatischen Verarbeitung personenbezogener Daten und dem dazugehörigen Zusatzprotokoll berücksichtigt werden. Die
    Kommission sollte den Ausschuss konsultieren, wenn sie das Schutzniveau in Drittländern oder internationalen
    Organisationen bewertet.%
   \label{itm:eg-105}
   
   \crossrefReasonToArticle{105}

   \item Die Kommission sollte die Wirkungsweise von Feststellungen zum Schutzniveau in einem Drittland, einem Gebiet
    oder einem bestimmten Sektor eines Drittlands oder einer internationalen Organisation überwachen; sie sollte auch
    die Wirkungsweise der Feststellungen, die auf der Grundlage des Artikels 25 Absatz 6 oder des Artikels 26 Absatz 4
    der Richtlinie 95/46/EG erlassen werden, überwachen. In ihren Angemessenheitsbeschlüssen sollte die Kommission
    einen Mechanismus für die regelmäßige Überprüfung von deren Wirkungsweise vorsehen. Diese regelmäßige Überprüfung
    sollte in Konsultation mit dem betreffenden Drittland oder der betreffenden internationalen Organisation erfolgen
    und allen maßgeblichen Entwicklungen in dem Drittland oder der internationalen Organisation Rechnung tragen. Für
    die Zwecke der Überwachung und der Durchführung der regelmäßigen Überprüfungen sollte die Kommission die
    Standpunkte und Feststellungen des Europäischen Parlaments und des Rates sowie der anderen einschlägigen Stellen
    und Quellen berücksichtigen. Die Kommission sollte innerhalb einer angemessenen Frist die Wirkungsweise der
    letztgenannten Beschlüsse bewerten und dem durch diese Verordnung eingesetzten Ausschuss im Sinne der Verordnung
    (EU) Nr. 182/2011 des Europäischen Parlaments und des Rates\comment{Verordnung (EU) Nr. 182/2011 des Europäischen
    Parlaments und des Rates vom 16. Februar 2011 zur Festlegung der allgemeinen Regeln und Grundsätze, nach denen die
    Mitgliedstaaten die Wahrnehmung der Durchführungsbefugnisse durch die Kommission kontrollieren siehe 
    \cite{vo-kontr-komm}} sowie dem Europäischen Parlament und dem Rat über alle maßgeblichen Feststellungen Bericht
     erstatten.%
   \label{itm:eg-106}
   
   \crossrefReasonToArticle{106}

   \item Die Kommission kann feststellen, dass ein Drittland, ein Gebiet oder ein bestimmter Sektor eines Drittlands
    oder eine internationale Organisation kein angemessenes Datenschutzniveau mehr bietet. Die Übermittlung
    personenbezogener Daten an dieses Drittland oder an diese internationale Organisation sollte daraufhin verboten
    werden, es sei denn, die Anforderungen dieser Verordnung in Bezug auf die Datenübermittlung vorbehaltlich
    geeigneter Garantien, einschließlich verbindlicher interner Datenschutzvorschriften und auf Ausnahmen für bestimmte
    Fälle werden erfüllt. In diesem Falle sollten Konsultationen zwischen der Kommission und den betreffenden
    Drittländern oder internationalen Organisationen vorgesehen werden. Die Kommission sollte dem Drittland oder der
    internationalen Organisation frühzeitig die Gründe mitteilen und Konsultationen aufnehmen, um Abhilfe für die
    Situation zu schaffen.%
   \label{itm:eg-107}
   
   \crossrefReasonToArticle{107}

   \item Bei Fehlen eines Angemessenheitsbeschlusses sollte der Verantwortliche oder der Auftragsverarbeiter als
    Ausgleich für den in einem Drittland bestehenden Mangel an Datenschutz geeignete Garantien für den Schutz der
    betroffenen Person vorsehen. Diese geeigneten Garantien können darin bestehen, dass auf verbindliche interne
    Datenschutzvorschriften, von der Kommission oder von einer Aufsichtsbehörde angenommene Standarddatenschutzklauseln
    oder von einer Aufsichtsbehörde genehmigte Vertragsklauseln zurückgegriffen wird. Diese Garantien sollten
    sicherstellen, dass die Datenschutzvorschriften und die Rechte der betroffenen Personen auf eine der Verarbeitung
    innerhalb der Union angemessene Art und Weise beachtet werden; dies gilt auch hinsichtlich der Verfügbarkeit von
    durchsetzbaren Rechten der betroffenen Person und von wirksamen Rechtsbehelfen einschließlich des Rechts auf
    wirksame verwaltungsrechtliche oder gerichtliche Rechtsbehelfe sowie des Rechts auf Geltendmachung von
    Schadenersatzansprüchen in der Union oder in einem Drittland. Sie sollten sich insbesondere auf die Einhaltung der
    allgemeinen Grundsätze für die Verarbeitung personenbezogener Daten, die Grundsätze des Datenschutzes durch Technik
    und durch datenschutzfreundliche Voreinstellungen beziehen. Datenübermittlungen dürfen auch von Behörden oder
    öffentlichen Stellen an Behörden oder öffentliche Stellen in Drittländern oder an internationale Organisationen mit
    entsprechenden Pflichten oder Aufgaben vorgenommen werden, auch auf der Grundlage von Bestimmungen, die in
    Verwaltungsvereinbarungen — wie beispielsweise einer gemeinsamen Absichtserklärung –, mit denen den betroffenen
    Personen durchsetzbare und wirksame Rechte eingeräumt werden, aufzunehmen sind. Die Genehmigung der zuständigen
    Aufsichtsbehörde sollte erlangt werden, wenn die Garantien in nicht rechtsverbindlichen Verwaltungsvereinbarungen
    vorgesehen sind.%
   \label{itm:eg-108}
   
   \crossrefReasonToArticle{108}

   \item Die dem Verantwortlichen oder dem Auftragsverarbeiter offenstehende Möglichkeit, auf die von der Kommission
    oder einer Aufsichtsbehörde festgelegten Standard"=Datenschutzklauseln zurückzugreifen, sollte den Verantwortlichen
    oder den Auftragsverarbeiter weder daran hindern, die Standard"=Datenschutzklauseln auch in umfangreicheren
    Verträgen, wie zum Beispiel Verträgen zwischen dem Auftragsverarbeiter und einem anderen Auftragsverarbeiter, zu
    verwenden, noch ihn daran hindern, ihnen weitere Klauseln oder zusätzliche Garantien hinzuzufügen, solange diese
    weder mittelbar noch unmittelbar im Widerspruch zu den von der Kommission oder einer Aufsichtsbehörde erlassenen
    Standard"=Datenschutzklauseln stehen oder die Grundrechte und Grundfreiheiten der betroffenen Personen beschneiden.
    Die Verantwortlichen und die Auftragsverarbeiter sollten ermutigt werden, mit vertraglichen Verpflichtungen, die
    die Standard"=Schutzklauseln ergänzen, zusätzliche Garantien zu bieten.%
   \label{itm:eg-109}
   
   \crossrefReasonToArticle{109}

   \item Jede Unternehmensgruppe oder jede Gruppe von Unternehmen, die eine gemeinsame Wirtschaftstätigkeit ausüben,
    sollte für ihre internationalen Datenübermittlungen aus der Union an Organisationen derselben Unternehmensgruppe
    oder derselben Gruppe von Unternehmen, die eine gemeinsame Wirtschaftstätigkeit ausüben, genehmigte verbindliche
    interne Datenschutzvorschriften anwenden dürfen, sofern diese sämtliche Grundprinzipien und durchsetzbaren Rechte
    enthalten, die geeignete Garantien für die Übermittlungen beziehungsweise Kategorien von Übermittlungen
    personenbezogener Daten bieten.%
   \label{itm:eg-110}
   
   \crossrefReasonToArticle{110}

   \item Datenübermittlungen sollten unter bestimmten Voraussetzungen zulässig sein, nämlich wenn die betroffene Person
    ihre ausdrückliche Einwilligung erteilt hat, wenn die Übermittlung gelegentlich erfolgt und im Rahmen eines
    Vertrags oder zur Geltendmachung von Rechtsansprüchen, sei es vor Gericht oder auf dem Verwaltungswege oder in
    außergerichtlichen Verfahren, wozu auch Verfahren vor Regulierungsbehörden zählen, erforderlich ist. Die
    Übermittlung sollte zudem möglich sein, wenn sie zur Wahrung eines im Unionsrecht oder im Recht eines
    Mitgliedstaats festgelegten wichtigen öffentlichen Interesses erforderlich ist oder wenn sie aus einem durch
    Rechtsvorschriften vorgesehenen Register erfolgt, das von der Öffentlichkeit oder Personen mit berechtigtem
    Interesse eingesehen werden kann. In letzterem Fall sollte sich eine solche Übermittlung nicht auf die Gesamtheit
    oder ganze Kategorien der im Register enthaltenen personenbezogenen Daten erstrecken dürfen. Ist das betreffende
    Register zur Einsichtnahme durch Personen mit berechtigtem Interesse bestimmt, sollte die Übermittlung nur auf
    Anfrage dieser Personen oder nur dann erfolgen, wenn diese Personen die Adressaten der Übermittlung sind, wobei den
    Interessen und Grundrechten der betroffenen Person in vollem Umfang Rechnung zu tragen ist.%
   \label{itm:eg-111}
   
   \crossrefReasonToArticle{111}

   \item Diese Ausnahmen sollten insbesondere für Datenübermittlungen gelten, die aus wichtigen Gründen des öffentlichen
    Interesses erforderlich sind, beispielsweise für den internationalen Datenaustausch zwischen Wettbewerbs-, Steuer-
    oder Zollbehörden, zwischen Finanzaufsichtsbehörden oder zwischen für Angelegenheiten der sozialen Sicherheit oder
    für die öffentliche Gesundheit zuständigen Diensten, beispielsweise im Falle der Umgebungsuntersuchung bei
    ansteckenden Krankheiten oder zur Verringerung und/oder Beseitigung des Dopings im Sport. Die Übermittlung
    personenbezogener Daten sollte ebenfalls als rechtmäßig angesehen werden, wenn sie erforderlich ist, um ein
    Interesse, das für die lebenswichtigen Interessen — einschließlich der körperlichen Unversehrtheit oder des
    Lebens — der betroffenen Person oder einer anderen Person wesentlich ist, zu schützen und die betroffene Person
    außerstande ist, ihre Einwilligung zu geben. Liegt kein Angemessenheitsbeschluss vor, so können im Unionsrecht oder
    im Recht der Mitgliedstaaten aus wichtigen Gründen des öffentlichen Interesses ausdrücklich Beschränkungen der
    Übermittlung bestimmter Kategorien von Daten an Drittländer oder internationale Organisationen vorgesehen werden.
    Die Mitgliedstaaten sollten solche Bestimmungen der Kommission mitteilen. Jede Übermittlung personenbezogener Daten
    einer betroffenen Person, die aus physischen oder rechtlichen Gründen außerstande ist, ihre Einwilligung zu
    erteilen, an eine internationale humanitäre Organisation, die erfolgt, um eine nach den Genfer Konventionen
    obliegende Aufgabe auszuführen oder um dem in bewaffneten Konflikten anwendbaren humanitären Völkerrecht
    nachzukommen, könnte als aus einem wichtigen Grund im öffentlichen Interesse notwendig oder als im lebenswichtigen
    Interesse der betroffenen Person liegend erachtet werden.%
   \label{itm:eg-112}
   
   \crossrefReasonToArticle{112}

   \item Übermittlungen, die als nicht wiederholt erfolgend gelten können und nur eine begrenzte Zahl von betroffenen
    Personen betreffen, könnten auch zur Wahrung der zwingenden berechtigten Interessen des Verantwortlichen möglich
    sein, sofern die Interessen oder Rechte und Freiheiten der betroffenen Person nicht überwiegen und der
    Verantwortliche sämtliche Umstände der Datenübermittlung geprüft hat. Der Verantwortliche sollte insbesondere die
    Art der personenbezogenen Daten, den Zweck und die Dauer der vorgesehenen Verarbeitung, die Situation im
    Herkunftsland, in dem betreffenden Drittland und im Endbestimmungsland berücksichtigen und angemessene Garantien
    zum Schutz der Grundrechte und Grundfreiheiten natürlicher Personen in Bezug auf die Verarbeitung ihrer
    personenbezogener Daten vorsehen. Diese Übermittlungen sollten nur in den verbleibenden Fällen möglich sein, in
    denen keiner der anderen Gründe für die Übermittlung anwendbar ist. Bei wissenschaftlichen oder historischen
    Forschungszwecken oder bei statistischen Zwecken sollten die legitimen gesellschaftlichen Erwartungen in Bezug auf
    einen Wissenszuwachs berücksichtigt werden. Der Verantwortliche sollte die Aufsichtsbehörde und die betroffene
    Person von der Übermittlung in Kenntnis setzen.%
   \label{itm:eg-113}
   
   \crossrefReasonToArticle{113}

   \item In allen Fällen, in denen kein Kommissionsbeschluss zur Angemessenheit des in einem Drittland bestehenden
    Datenschutzniveaus vorliegt, sollte der Verantwortliche oder der Auftragsverarbeiter auf Lösungen zurückgreifen,
    mit denen den betroffenen Personen durchsetzbare und wirksame Rechte in Bezug auf die Verarbeitung ihrer
    personenbezogenen Daten in der Union nach der Übermittlung dieser Daten eingeräumt werden, damit sie weiterhin die
    Grundrechte und Garantien genießen können.%
   \label{itm:eg-114}
   
   \crossrefReasonToArticle{114}

   \item Manche Drittländer erlassen Gesetze, Vorschriften und sonstige Rechtsakte, die vorgeben, die
    Verarbeitungstätigkeiten natürlicher und juristischer Personen, die der Rechtsprechung der Mitgliedstaaten
    unterliegen, unmittelbar zu regeln. Dies kann Urteile von Gerichten und Entscheidungen von Verwaltungsbehörden in
    Drittländern umfassen, mit denen von einem Verantwortlichen oder einem Auftragsverarbeiter die Übermittlung oder
    Offenlegung personenbezogener Daten verlangt wird und die nicht auf eine in Kraft befindliche internationale
    Übereinkunft wie etwa ein Rechtshilfeabkommen zwischen dem ersuchenden Drittland und der Union oder einem
    Mitgliedstaat gestützt sind. Die Anwendung dieser Gesetze, Verordnungen und sonstigen Rechtsakte außerhalb des
    Hoheitsgebiets der betreffenden Drittländer kann gegen internationales Recht verstoßen und dem durch diese
    Verordnung in der Union gewährleisteten Schutz natürlicher Personen zuwiderlaufen. Datenübermittlungen sollten
    daher nur zulässig sein, wenn die Bedingungen dieser Verordnung für Datenübermittlungen an Drittländer eingehalten
    werden. Dies kann unter anderem der Fall sein, wenn die Offenlegung aus einem wichtigen öffentlichen Interesse
    erforderlich ist, das im Unionsrecht oder im Recht des Mitgliedstaats, dem der Verantwortliche unterliegt,
    anerkannt ist.%
   \label{itm:eg-115}
   
   \crossrefReasonToArticle{115}

   \item Wenn personenbezogene Daten in ein anderes Land außerhalb der Union übermittelt werden, besteht eine erhöhte
    Gefahr, dass natürliche Personen ihre Datenschutzrechte nicht wahrnehmen können und sich insbesondere gegen die
    unrechtmäßige Nutzung oder Offenlegung dieser Informationen zu schützen. Ebenso kann es vorkommen, dass
    Aufsichtsbehörden Beschwerden nicht nachgehen oder Untersuchungen nicht durchführen können, die einen Bezug zu
    Tätigkeiten außerhalb der Grenzen ihres Mitgliedstaats haben. Ihre Bemühungen um grenzüberschreitende
    Zusammenarbeit können auch durch unzureichende Präventiv- und Abhilfebefugnisse, widersprüchliche Rechtsordnungen
    und praktische Hindernisse wie Ressourcenknappheit behindert werden. Die Zusammenarbeit zwischen den
    Datenschutzaufsichtsbehörden muss daher gefördert werden, damit sie Informationen austauschen und mit den
    Aufsichtsbehörden in anderen Ländern Untersuchungen durchführen können. Um Mechanismen der internationalen
    Zusammenarbeit zu entwickeln, die die internationale Amtshilfe bei der Durchsetzung von Rechtsvorschriften zum
    Schutz personenbezogener Daten erleichtern und sicherstellen, sollten die Kommission und die Aufsichtsbehörden
    Informationen austauschen und bei Tätigkeiten, die mit der Ausübung ihrer Befugnisse in Zusammenhang stehen, mit
    den zuständigen Behörden der Drittländer nach dem Grundsatz der Gegenseitigkeit und gemäß dieser Verordnung
    zusammenarbeiten.%
   \label{itm:eg-116}
   
   \crossrefReasonToArticle{116}

   \item Die Errichtung von Aufsichtsbehörden in den Mitgliedstaaten, die befugt sind, ihre Aufgaben und Befugnisse
    völlig unabhängig wahrzunehmen, ist ein wesentlicher Bestandteil des Schutzes natürlicher Personen bei der
    Verarbeitung personenbezogener Daten. Die Mitgliedstaaten sollten mehr als eine Aufsichtsbehörde errichten können,
    wenn dies ihrer verfassungsmäßigen, organisatorischen und administrativen Struktur entspricht.%
   \label{itm:eg-117}
   
   \crossrefReasonToArticle{117}

   \item Die Tatsache, dass die Aufsichtsbehörden unabhängig sind, sollte nicht bedeuten, dass sie hinsichtlich ihrer
    Ausgaben keinem Kontroll- oder Überwachungsmechanismus unterworfen werden bzw. sie keiner gerichtlichen Überprüfung
    unterzogen werden können.%
   \label{itm:eg-118}
   
   \crossrefReasonToArticle{118}

   \item Errichtet ein Mitgliedstaat mehrere Aufsichtsbehörden, so sollte er mittels Rechtsvorschriften sicherstellen,
    dass diese Aufsichtsbehörden am Kohärenzverfahren wirksam beteiligt werden. Insbesondere sollte dieser
    Mitgliedstaat eine Aufsichtsbehörde bestimmen, die als zentrale Anlaufstelle für eine wirksame Beteiligung dieser
    Behörden an dem Verfahren fungiert und eine rasche und reibungslose Zusammenarbeit mit anderen Aufsichtsbehörden,
    dem Ausschuss und der Kommission gewährleistet.%
   \label{itm:eg-119}
   
   \crossrefReasonToArticle{119}

   \item Jede Aufsichtsbehörde sollte mit Finanzmitteln, Personal, Räumlichkeiten und einer Infrastruktur ausgestattet
    werden, wie sie für die wirksame Wahrnehmung ihrer Aufgaben, einschließlich derer im Zusammenhang mit der Amtshilfe
    und Zusammenarbeit mit anderen Aufsichtsbehörden in der gesamten Union, notwendig sind. Jede Aufsichtsbehörde
    sollte über einen eigenen, öffentlichen, jährlichen Haushaltsplan verfügen, der Teil des gesamten Staatshaushalts
    oder nationalen Haushalts sein kann.%
   \label{itm:eg-120}
   
   \crossrefReasonToArticle{120}

   \item Die allgemeinen Anforderungen an das Mitglied oder die Mitglieder der Aufsichtsbehörde sollten durch
    Rechtsvorschriften von jedem Mitgliedstaat geregelt werden und insbesondere vorsehen, dass diese Mitglieder im Wege
    eines transparenten Verfahrens entweder — auf Vorschlag der Regierung, eines Mitglieds der Regierung, des
    Parlaments oder einer Parlamentskammer — vom Parlament, der Regierung oder dem Staatsoberhaupt des Mitgliedstaats
    oder von einer unabhängigen Stelle ernannt werden, die nach dem Recht des Mitgliedstaats mit der Ernennung betraut
    wird. Um die Unabhängigkeit der Aufsichtsbehörde zu gewährleisten, sollten ihre Mitglieder ihr Amt integer ausüben,
    von allen mit den Aufgaben ihres Amts nicht zu vereinbarenden Handlungen absehen und während ihrer Amtszeit keine
    andere mit ihrem Amt nicht zu vereinbarende entgeltliche oder unentgeltliche Tätigkeit ausüben. Die
    Aufsichtsbehörde sollte über eigenes Personal verfügen, das sie selbst oder eine nach dem Recht des Mitgliedstaats
    eingerichtete unabhängige Stelle auswählt und das ausschließlich der Leitung des Mitglieds oder der Mitglieder der
    Aufsichtsbehörde unterstehen sollte.%
   \label{itm:eg-121}
   
   \crossrefReasonToArticle{121}

   \item Jede Aufsichtsbehörde sollte dafür zuständig sein, im Hoheitsgebiet ihres Mitgliedstaats die Befugnisse
    auszuüben und die Aufgaben zu erfüllen, die ihr mit dieser Verordnung übertragen wurden. Dies sollte insbesondere
    für Folgendes gelten: die Verarbeitung im Rahmen der Tätigkeiten einer Niederlassung des Verantwortlichen oder
    Auftragsverarbeiters im Hoheitsgebiet ihres Mitgliedstaats, die Verarbeitung personenbezogener Daten durch Behörden
    oder private Stellen, die im öffentlichen Interesse handeln, Verarbeitungstätigkeiten, die Auswirkungen auf
    betroffene Personen in ihrem Hoheitsgebiet haben, oder Verarbeitungstätigkeiten eines Verantwortlichen oder
    Auftragsverarbeiters ohne Niederlassung in der Union, sofern sie auf betroffene Personen mit Wohnsitz in ihrem
    Hoheitsgebiet ausgerichtet sind. Dies sollte auch die Bearbeitung von Beschwerden einer betroffenen Person, die
    Durchführung von Untersuchungen über die Anwendung dieser Verordnung sowie die Förderung der Information der
    Öffentlichkeit über Risiken, Vorschriften, Garantien und Rechte im Zusammenhang mit der Verarbeitung
    personenbezogener Daten einschließen.%
   \label{itm:eg-122}
   
   \crossrefReasonToArticle{122}

   \item Die Aufsichtsbehörden sollten die Anwendung der Bestimmungen dieser Verordnung überwachen und zu ihrer
    einheitlichen Anwendung in der gesamten Union beitragen, um natürliche Personen im Hinblick auf die Verarbeitung
    ihrer Daten zu schützen und den freien Verkehr personenbezogener Daten im Binnenmarkt zu erleichtern. Zu diesem
    Zweck sollten die Aufsichtsbehörden untereinander und mit der Kommission zusammenarbeiten, ohne dass eine
    Vereinbarung zwischen den Mitgliedstaaten über die Leistung von Amtshilfe oder über eine derartige Zusammenarbeit
    erforderlich wäre.%
   \label{itm:eg-123}
   
   \crossrefReasonToArticle{123}

   \item Findet die Verarbeitung personenbezogener Daten im Zusammenhang mit der Tätigkeit einer Niederlassung eines
    Verantwortlichen oder eines Auftragsverarbeiters in der Union statt und hat der Verantwortliche oder der
    Auftragsverarbeiter Niederlassungen in mehr als einem Mitgliedstaat oder hat die Verarbeitungstätigkeit im
    Zusammenhang mit der Tätigkeit einer einzigen Niederlassung eines Verantwortlichen oder Auftragsverarbeiters in der
    Union erhebliche Auswirkungen auf betroffene Personen in mehr als einem Mitgliedstaat bzw. wird sie voraussichtlich
    solche Auswirkungen haben, so sollte die Aufsichtsbehörde für die Hauptniederlassung des Verantwortlichen oder
    Auftragsverarbeiters oder für die einzige Niederlassung des Verantwortlichen oder Auftragsverarbeiters als
    federführende Behörde fungieren. Sie sollte mit den anderen Behörden zusammenarbeiten, die betroffen sind, weil der
    Verantwortliche oder Auftragsverarbeiter eine Niederlassung im Hoheitsgebiet ihres Mitgliedstaats hat, weil die
    Verarbeitung erhebliche Auswirkungen auf betroffene Personen mit Wohnsitz in ihrem Hoheitsgebiet hat oder weil bei
    ihnen eine Beschwerde eingelegt wurde. Auch wenn eine betroffene Person ohne Wohnsitz in dem betreffenden
    Mitgliedstaat eine Beschwerde eingelegt hat, sollte die Aufsichtsbehörde, bei der Beschwerde eingelegt wurde, auch
    eine betroffene Aufsichtsbehörde sein. Der Ausschuss sollte — im Rahmen seiner Aufgaben in Bezug auf die Herausgabe
    von Leitlinien zu allen Fragen im Zusammenhang mit der Anwendung dieser Verordnung — insbesondere Leitlinien zu den
    Kriterien ausgeben können, die bei der Feststellung zu berücksichtigen sind, ob die fragliche Verarbeitung
    erhebliche Auswirkungen auf betroffene Personen in mehr als einem Mitgliedstaat hat und was einen maßgeblichen und
    begründeten Einspruch darstellt.%
   \label{itm:eg-124}
   
   \crossrefReasonToArticle{124}

   \item Die federführende Behörde sollte berechtigt sein, verbindliche Beschlüsse über Maßnahmen zu erlassen, mit denen
    die ihr gemäß dieser Verordnung übertragenen Befugnisse ausgeübt werden. In ihrer Eigenschaft als federführende
    Behörde sollte diese Aufsichtsbehörde für die enge Einbindung und Koordinierung der betroffenen Aufsichtsbehörden
    im Entscheidungsprozess sorgen. Wird beschlossen, die Beschwerde der betroffenen Person vollständig oder teilweise
    abzuweisen, so sollte dieser Beschluss von der Aufsichtsbehörde angenommen werden, bei der die Beschwerde eingelegt
    wurde.%
   \label{itm:eg-125}
   
   \crossrefReasonToArticle{125}

   \item Der Beschluss sollte von der federführenden Aufsichtsbehörde und den betroffenen Aufsichtsbehörden gemeinsam
    vereinbart werden und an die Hauptniederlassung oder die einzige Niederlassung des Verantwortlichen oder
    Auftragsverarbeiters gerichtet sein und für den Verantwortlichen und den Auftragsverarbeiter verbindlich sein. Der
    Verantwortliche oder Auftragsverarbeiter sollte die erforderlichen Maßnahmen treffen, um die Einhaltung dieser
    Verordnung und die Umsetzung des Beschlusses zu gewährleisten, der der Hauptniederlassung des Verantwortlichen oder
    Auftragsverarbeiters im Hinblick auf die Verarbeitungstätigkeiten in der Union von der federführenden
    Aufsichtsbehörde mitgeteilt wurde.%
   \label{itm:eg-126}
   
   \crossrefReasonToArticle{126}

   \item Jede Aufsichtsbehörde, die nicht als federführende Aufsichtsbehörde fungiert, sollte in örtlichen Fällen
    zuständig sein, wenn der Verantwortliche oder Auftragsverarbeiter Niederlassungen in mehr als einem Mitgliedstaat
    hat, der Gegenstand der spezifischen Verarbeitung aber nur die Verarbeitungstätigkeiten in einem einzigen
    Mitgliedstaat und nur betroffene Personen in diesem einen Mitgliedstaat betrifft, beispielsweise wenn es um die
    Verarbeitung von personenbezogenen Daten von Arbeitnehmern im spezifischen Beschäftigungskontext eines
    Mitgliedstaats geht. In solchen Fällen sollte die Aufsichtsbehörde unverzüglich die federführende Aufsichtsbehörde
    über diese Angelegenheit unterrichten. Nach ihrer Unterrichtung sollte die federführende Aufsichtsbehörde
    entscheiden, ob sie den Fall nach den Bestimmungen zur Zusammenarbeit zwischen der federführenden Aufsichtsbehörde
    und anderen betroffenen Aufsichtsbehörden gemäß der Vorschrift zur Zusammenarbeit zwischen der federführenden
    Aufsichtsbehörde und anderen betroffenen Aufsichtsbehörden (im Folgenden „Verfahren der Zusammenarbeit und
    Kohärenz“) regelt oder ob die Aufsichtsbehörde, die sie unterrichtet hat, den Fall auf örtlicher Ebene regeln
    sollte. Dabei sollte die federführende Aufsichtsbehörde berücksichtigen, ob der Verantwortliche oder der
    Auftragsverarbeiter in dem Mitgliedstaat, dessen Aufsichtsbehörde sie unterrichtet hat, eine Niederlassung hat,
    damit Beschlüsse gegenüber dem Verantwortlichen oder dem Auftragsverarbeiter wirksam durchgesetzt werden.
    Entscheidet die federführende Aufsichtsbehörde, den Fall selbst zu regeln, sollte die Aufsichtsbehörde, die sie
    unterrichtet hat, die Möglichkeit haben, einen Beschlussentwurf vorzulegen, dem die federführende Aufsichtsbehörde
    bei der Ausarbeitung ihres Beschlussentwurfs im Rahmen dieses Verfahrens der Zusammenarbeit und Kohärenz
    weitestgehend Rechnung tragen sollte.%
   \label{itm:eg-127}
   
   \crossrefReasonToArticle{127}

   \item Die Vorschriften über die federführende Behörde und das Verfahren der Zusammenarbeit und Kohärenz sollten keine
    Anwendung finden, wenn die Verarbeitung durch Behörden oder private Stellen im öffentlichen Interesse erfolgt. In
    diesen Fällen sollte die Aufsichtsbehörde des Mitgliedstaats, in dem die Behörde oder private Einrichtung ihren
    Sitz hat, die einzige Aufsichtsbehörde sein, die dafür zuständig ist, die Befugnisse auszuüben, die ihr mit dieser
    Verordnung übertragen wurden.%
   \label{itm:eg-128}
   
   \crossrefReasonToArticle{128}

   \item Um die einheitliche Überwachung und Durchsetzung dieser Verordnung in der gesamten Union sicherzustellen,
    sollten die Aufsichtsbehörden in jedem Mitgliedstaat dieselben Aufgaben und wirksamen Befugnisse haben, darunter,
    insbesondere im Fall von Beschwerden natürlicher Personen, Untersuchungsbefugnisse, Abhilfebefugnisse und
    Sanktionsbefugnisse und Genehmigungsbefugnisse und beratende Befugnisse, sowie — unbeschadet der Befugnisse der
    Strafverfolgungsbehörden nach dem Recht der Mitgliedstaaten — die Befugnis, Verstöße gegen diese Verordnung den
    Justizbehörden zur Kenntnis zu bringen und Gerichtsverfahren anzustrengen. Dazu sollte auch die Befugnis zählen,
    eine vorübergehende oder endgültige Beschränkung der Verarbeitung, einschließlich eines Verbots, zu verhängen. Die
    Mitgliedstaaten können andere Aufgaben im Zusammenhang mit dem Schutz personenbezogener Daten im Rahmen dieser
    Verordnung festlegen. Die Befugnisse der Aufsichtsbehörden sollten in Übereinstimmung mit den geeigneten
    Verfahrensgarantien nach dem Unionsrecht und dem Recht der Mitgliedstaaten unparteiisch, gerecht und innerhalb
    einer angemessenen Frist ausgeübt werden. Insbesondere sollte jede Maßnahme im Hinblick auf die Gewährleistung der
    Einhaltung dieser Verordnung geeignet, erforderlich und verhältnismäßig sein, wobei die Umstände des jeweiligen
    Einzelfalls zu berücksichtigen sind, das Recht einer jeden Person, gehört zu werden, bevor eine individuelle
    Maßnahme getroffen wird, die nachteilige Auswirkungen auf diese Person hätte, zu achten ist und überflüssige Kosten
    und übermäßige Unannehmlichkeiten für die Betroffenen zu vermeiden sind. Untersuchungsbefugnisse im Hinblick auf
    den Zugang zu Räumlichkeiten sollten im Einklang mit besonderen Anforderungen im Verfahrensrecht der
    Mitgliedstaaten ausgeübt werden, wie etwa dem Erfordernis einer vorherigen richterlichen Genehmigung. Jede
    rechtsverbindliche Maßnahme der Aufsichtsbehörde sollte schriftlich erlassen werden und sie sollte klar und
    eindeutig sein; die Aufsichtsbehörde, die die Maßnahme erlassen hat, und das Datum, an dem die Maßnahme erlassen
    wurde, sollten angegeben werden und die Maßnahme sollte vom Leiter oder von einem von ihm bevollmächtigen Mitglied
    der Aufsichtsbehörde unterschrieben sein und eine Begründung für die Maßnahme sowie einen Hinweis auf das Recht auf
    einen wirksamen Rechtsbehelf enthalten. Dies sollte zusätzliche Anforderungen nach dem Verfahrensrecht der
    Mitgliedstaaten nicht ausschließen. Der Erlass eines rechtsverbindlichen Beschlusses setzt voraus, dass er in dem
    Mitgliedstaat der Aufsichtsbehörde, die den Beschluss erlassen hat, gerichtlich überprüft werden kann.%
   \label{itm:eg-129}
   
   \crossrefReasonToArticle{129}

   \item Ist die Aufsichtsbehörde, bei der die Beschwerde eingereicht wurde, nicht die federführende Aufsichtsbehörde,
    so sollte die federführende Aufsichtsbehörde gemäß den Bestimmungen dieser Verordnung über Zusammenarbeit und
    Kohärenz eng mit der Aufsichtsbehörde zusammenarbeiten, bei der die Beschwerde eingereicht wurde. In solchen Fällen
    sollte die federführende Aufsichtsbehörde bei Maßnahmen, die rechtliche Wirkungen entfalten sollen, unter anderem
    bei der Verhängung von Geldbußen, den Standpunkt der Aufsichtsbehörde, bei der die Beschwerde eingereicht wurde und
    die weiterhin befugt sein sollte, in Abstimmung mit der zuständigen Aufsichtsbehörde Untersuchungen im
    Hoheitsgebiet ihres eigenen Mitgliedstaats durchzuführen, weitestgehend berücksichtigen.%
   \label{itm:eg-130}
   
   \crossrefReasonToArticle{130}

   \item Wenn eine andere Aufsichtsbehörde als federführende Aufsichtsbehörde für die Verarbeitungstätigkeiten des
    Verantwortlichen oder des Auftragsverarbeiters fungieren sollte, der konkrete Gegenstand einer Beschwerde oder der
    mögliche Verstoß jedoch nur die Verarbeitungstätigkeiten des Verantwortlichen oder des Auftragsverarbeiters in dem
    Mitgliedstaat betrifft, in dem die Beschwerde eingereicht wurde oder der mögliche Verstoß aufgedeckt wurde, und die
    Angelegenheit keine erheblichen Auswirkungen auf betroffene Personen in anderen Mitgliedstaaten hat oder haben
    dürfte, sollte die Aufsichtsbehörde, bei der eine Beschwerde eingereicht wurde oder die Situationen, die mögliche
    Verstöße gegen diese Verordnung darstellen, aufgedeckt hat bzw. auf andere Weise darüber informiert wurde,
    versuchen, eine gütliche Einigung mit dem Verantwortlichen zu erzielen; falls sich dies als nicht erfolgreich
    erweist, sollte sie die gesamte Bandbreite ihrer Befugnisse wahrnehmen. Dies sollte auch Folgendes umfassen: die
    spezifische Verarbeitung im Hoheitsgebiet des Mitgliedstaats der Aufsichtsbehörde oder im Hinblick auf betroffene
    Personen im Hoheitsgebiet dieses Mitgliedstaats; die Verarbeitung im Rahmen eines Angebots von Waren oder
    Dienstleistungen, das speziell auf betroffene Personen im Hoheitsgebiet des Mitgliedstaats der Aufsichtsbehörde
    ausgerichtet ist; oder eine Verarbeitung, die unter Berücksichtigung der einschlägigen rechtlichen Verpflichtungen
    nach dem Recht der Mitgliedstaaten bewertet werden muss.%
   \label{itm:eg-131}
   
   \crossrefReasonToArticle{131}

   \item Auf die Öffentlichkeit ausgerichtete Sensibilisierungsmaßnahmen der Aufsichtsbehörden sollten spezifische
    Maßnahmen einschließen, die sich an die Verantwortlichen und die Auftragsverarbeiter, einschließlich
    Kleinstunternehmen sowie kleiner und mittlerer Unternehmen, und an natürliche Personen, insbesondere im
    Bildungsbereich, richten.%
   \label{itm:eg-132}
   
   \crossrefReasonToArticle{132}

   \item Die Aufsichtsbehörden sollten sich gegenseitig bei der Erfüllung ihrer Aufgaben unterstützen und Amtshilfe
    leisten, damit eine einheitliche Anwendung und Durchsetzung dieser Verordnung im Binnenmarkt gewährleistet ist.
    Eine Aufsichtsbehörde, die um Amtshilfe ersucht hat, kann eine einstweilige Maßnahme erlassen, wenn sie nicht
    binnen eines Monats nach Eingang des Amtshilfeersuchens bei der ersuchten Aufsichtsbehörde eine Antwort von dieser
    erhalten hat.%
   \label{itm:eg-133}
   
   \crossrefReasonToArticle{133}

   \item Jede Aufsichtsbehörde sollte gegebenenfalls an gemeinsamen Maßnahmen von anderen Aufsichtsbehörden teilnehmen.
    Die ersuchte Aufsichtsbehörde sollte auf das Ersuchen binnen einer bestimmten Frist antworten müssen.%
   \label{itm:eg-134}
   
   \crossrefReasonToArticle{134}

   \item Um die einheitliche Anwendung dieser Verordnung in der gesamten Union sicherzustellen, sollte ein Verfahren zur
    Gewährleistung einer einheitlichen Rechtsanwendung (Kohärenzverfahren) für die Zusammenarbeit zwischen den
    Aufsichtsbehörden eingeführt werden. Dieses Verfahren sollte insbesondere dann angewendet werden, wenn eine
    Aufsichtsbehörde beabsichtigt, eine Maßnahme zu erlassen, die rechtliche Wirkungen in Bezug auf
    Verarbeitungsvorgänge entfalten soll, die für eine bedeutende Zahl betroffener Personen in mehreren Mitgliedstaaten
    erhebliche Auswirkungen haben. Ferner sollte es zur Anwendung kommen, wenn eine betroffene Aufsichtsbehörde oder
    die Kommission beantragt, dass die Angelegenheit im Rahmen des Kohärenzverfahrens behandelt wird. Dieses Verfahren
    sollte andere Maßnahmen, die die Kommission möglicherweise in Ausübung ihrer Befugnisse nach den Verträgen trifft,
    unberührt lassen.%
   \label{itm:eg-135}
   
   \crossrefReasonToArticle{135}

   \item Bei Anwendung des Kohärenzverfahrens sollte der Ausschuss, falls von der Mehrheit seiner Mitglieder so
    entschieden wird oder falls eine andere betroffene Aufsichtsbehörde oder die Kommission darum ersuchen, binnen
    einer festgelegten Frist eine Stellungnahme abgeben. Dem Ausschuss sollte auch die Befugnis übertragen werden, bei
    Streitigkeiten zwischen Aufsichtsbehörden rechtsverbindliche Beschlüsse zu erlassen. Zu diesem Zweck sollte er in
    klar bestimmten Fällen, in denen die Aufsichtsbehörden insbesondere im Rahmen des Verfahrens der Zusammenarbeit
    zwischen der federführenden Aufsichtsbehörde und den betroffenen Aufsichtsbehörden widersprüchliche Standpunkte zu
    dem Sachverhalt, vor allem in der Frage, ob ein Verstoß gegen diese Verordnung vorliegt, vertreten, grundsätzlich
    mit einer Mehrheit von zwei Dritteln seiner Mitglieder rechtsverbindliche Beschlüsse erlassen.%
   \label{itm:eg-136}
   
   \crossrefReasonToArticle{136}

   \item Es kann dringender Handlungsbedarf zum Schutz der Rechte und Freiheiten von betroffenen Personen bestehen,
    insbesondere wenn eine erhebliche Behinderung der Durchsetzung des Rechts einer betroffenen Person droht. Eine
    Aufsichtsbehörde sollte daher hinreichend begründete einstweilige Maßnahmen in ihrem Hoheitsgebiet mit einer
    festgelegten Geltungsdauer von höchstens drei Monaten erlassen können.%
   \label{itm:eg-137}
   
   \crossrefReasonToArticle{137}

   \item Die Anwendung dieses Verfahrens sollte in den Fällen, in denen sie verbindlich vorgeschrieben ist, eine
    Bedingung für die Rechtmäßigkeit einer Maßnahme einer Aufsichtsbehörde sein, die rechtliche Wirkungen entfalten
    soll. In anderen Fällen von grenzüberschreitender Relevanz sollte das Verfahren der Zusammenarbeit zwischen der
    federführenden Aufsichtsbehörde und den betroffenen Aufsichtsbehörden zur Anwendung gelangen, und die betroffenen
    Aufsichtsbehörden können auf bilateraler oder multilateraler Ebene Amtshilfe leisten und gemeinsame Maßnahmen
    durchführen, ohne auf das Kohärenzverfahren zurückzugreifen.%
   \label{itm:eg-138}
   
   \crossrefReasonToArticle{138}

   \item Zur Förderung der einheitlichen Anwendung dieser Verordnung sollte der Ausschuss als unabhängige Einrichtung
    der Union eingesetzt werden. Damit der Ausschuss seine Ziele erreichen kann, sollte er Rechtspersönlichkeit
    besitzen. Der Ausschuss sollte von seinem Vorsitz vertreten werden. Er sollte die mit der Richtlinie 95/46/EG
    eingesetzte Arbeitsgruppe für den Schutz der Rechte von Personen bei der Verarbeitung personenbezogener Daten
    ersetzen. Er sollte aus dem Leiter einer Aufsichtsbehörde jedes Mitgliedstaats und dem Europäischen
    Datenschutzbeauftragten oder deren jeweiligen Vertretern gebildet werden. An den Beratungen des Ausschusses sollte
    die Kommission ohne Stimmrecht teilnehmen und der Europäische Datenschutzbeauftragte sollte spezifische Stimmrechte
    haben. Der Ausschuss sollte zur einheitlichen Anwendung der Verordnung in der gesamten Union beitragen, die
    Kommission insbesondere im Hinblick auf das Schutzniveau in Drittländern oder internationalen Organisationen
    beraten und die Zusammenarbeit der Aufsichtsbehörden in der Union fördern. Der Ausschuss sollte bei der Erfüllung
    seiner Aufgaben unabhängig handeln.%
   \label{itm:eg-139}
   
   \crossrefReasonToArticle{139}

   \item Der Ausschusssollte von einem Sekretariat unterstützt werden, das von dem Europäischen Datenschutzbeauftragten
    bereitgestellt wird. Das Personal des Europäischen Datenschutzbeauftragten, das an der Wahrnehmung der dem
    Ausschuss gemäß dieser Verordnung übertragenen Aufgaben beteiligt ist, sollte diese Aufgaben ausschließlich gemäß
    den Anweisungen des Vorsitzes des Ausschusses durchführen und diesem Bericht erstatten.%
   \label{itm:eg-140}
   
   \crossrefReasonToArticle{140}

   \item Jede betroffene Person sollte das Recht haben, bei einer einzigen Aufsichtsbehörde insbesondere in dem
    Mitgliedstaat ihres gewöhnlichen Aufenthalts eine Beschwerde einzureichen und gemäß Artikel 47 der Charta einen
    wirksamen gerichtlichen Rechtsbehelf einzulegen, wenn sie sich in ihren Rechten gemäß dieser Verordnung verletzt
    sieht oder wenn die Aufsichtsbehörde auf eine Beschwerde hin nicht tätig wird, eine Beschwerde teilweise oder ganz
    abweist oder ablehnt oder nicht tätig wird, obwohl dies zum Schutz der Rechte der betroffenen Person notwendig ist.
    Die auf eine Beschwerde folgende Untersuchung sollte vorbehaltlich gerichtlicher Überprüfung so weit gehen, wie
    dies im Einzelfall angemessen ist. Die Aufsichtsbehörde sollte die betroffene Person innerhalb eines angemessenen
    Zeitraums über den Fortgang und die Ergebnisse der Beschwerde unterrichten. Sollten weitere Untersuchungen oder die
    Abstimmung mit einer anderen Aufsichtsbehörde erforderlich sein, sollte die betroffene Person über den
    Zwischenstand informiert werden. Jede Aufsichtsbehörde sollte Maßnahmen zur Erleichterung der Einreichung von
    Beschwerden treffen, wie etwa die Bereitstellung eines Beschwerdeformulars, das auch elektronisch ausgefüllt werden
    kann, ohne dass andere Kommunikationsmittel ausgeschlossen werden.%
   \label{itm:eg-141}
   
   \crossrefReasonToArticle{141}

   \item Betroffene Personen, die sich in ihren Rechten gemäß dieser Verordnung verletzt sehen, sollten das Recht haben,
    nach dem Recht eines Mitgliedstaats gegründete Einrichtungen, Organisationen oder Verbände ohne
    Gewinnerzielungsabsicht, deren satzungsmäßige Ziele im öffentlichem Interesse liegen und die im Bereich des
    Schutzes personenbezogener Daten tätig sind, zu beauftragen, in ihrem Namen Beschwerde bei einer Aufsichtsbehörde
    oder einen gerichtlichen Rechtsbehelf einzulegen oder das Recht auf Schadensersatz in Anspruch zu nehmen, sofern
    dieses im Recht der Mitgliedstaaten vorgesehen ist. Die Mitgliedstaaten können vorsehen, dass diese Einrichtungen,
    Organisationen oder Verbände das Recht haben, unabhängig vom Auftrag einer betroffenen Person in dem betreffenden
    Mitgliedstaat eine eigene Beschwerde einzulegen, und das Recht auf einen wirksamen gerichtlichen Rechtsbehelf haben
    sollten, wenn sie Grund zu der Annahme haben, dass die Rechte der betroffenen Person infolge einer nicht im
    Einklang mit dieser Verordnung stehenden Verarbeitung verletzt worden sind. Diesen Einrichtungen, Organisationen
    oder Verbänden kann unabhängig vom Auftrag einer betroffenen Person nicht gestattet werden, im Namen einer
    betroffenen Person Schadenersatz zu verlangen.%
   \label{itm:eg-142}
   
   \crossrefReasonToArticle{142}

   \item Jede natürliche oder juristische Person hat das Recht, unter den in Artikel 263 AEUV genannten Voraussetzungen
    beim Gerichtshof eine Klage auf Nichtigerklärung eines Beschlusses des Ausschusses zu erheben. Als Adressaten
    solcher Beschlüsse müssen die betroffenen Aufsichtsbehörden, die diese Beschlüsse anfechten möchten, binnen zwei
    Monaten nach deren Übermittlung gemäß Artikel 263 AEUV Klage erheben. Sofern Beschlüsse des Ausschusses einen
    Verantwortlichen, einen Auftragsverarbeiter oder den Beschwerdeführer unmittelbar und individuell betreffen, so
    können diese Personen binnen zwei Monaten nach Veröffentlichung der betreffenden Beschlüsse auf der Website des
    Ausschusses im Einklang mit Artikel 263 AEUV eine Klage auf Nichtigerklärung erheben. Unbeschadet dieses Rechts
    nach Artikel 263 AEUV sollte jede natürliche oder juristische Person das Recht auf einen wirksamen gerichtlichen
    Rechtsbehelf bei dem zuständigen einzelstaatlichen Gericht gegen einen Beschluss einer Aufsichtsbehörde haben, der
    gegenüber dieser Person Rechtswirkungen entfaltet. Ein derartiger Beschluss betrifft insbesondere die Ausübung von
    Untersuchungs-, Abhilfe- und Genehmigungsbefugnissen durch die Aufsichtsbehörde oder die Ablehnung oder Abweisung
    von Beschwerden. Das Recht auf einen wirksamen gerichtlichen Rechtsbehelf umfasst jedoch nicht rechtlich nicht
    bindende Maßnahmen der Aufsichtsbehörden wie von ihr abgegebene Stellungnahmen oder Empfehlungen. Verfahren gegen
    eine Aufsichtsbehörde sollten bei den Gerichten des Mitgliedstaats angestrengt werden, in dem die Aufsichtsbehörde
    ihren Sitz hat, und sollten im Einklang mit dem Verfahrensrecht dieses Mitgliedstaats durchgeführt werden. Diese
    Gerichte sollten eine uneingeschränkte Zuständigkeit besitzen, was die Zuständigkeit, sämtliche für den bei ihnen
    anhängigen Rechtsstreit maßgebliche Sach- und Rechtsfragen zu prüfen, einschließt. Wurde eine Beschwerde von einer
    Aufsichtsbehörde abgelehnt oder abgewiesen, kann der Beschwerdeführer Klage bei den Gerichten desselben
    Mitgliedstaats erheben. 

    Im Zusammenhang mit gerichtlichen Rechtsbehelfen in Bezug auf die Anwendung dieser Verordnung können
    einzelstaatliche Gerichte, die eine Entscheidung über diese Frage für erforderlich halten, um ihr Urteil erlassen
    zu können, bzw. müssen einzelstaatliche Gerichte in den Fällen nach Artikel 267 AEUV den Gerichtshof um eine
    Vorabentscheidung zur Auslegung des Unionsrechts — das auch diese Verordnung einschließt — ersuchen. Wird darüber
    hinaus der Beschluss einer Aufsichtsbehörde zur Umsetzung eines Beschlusses des Ausschusses vor einem
    einzelstaatlichen Gericht angefochten und wird die Gültigkeit des Beschlusses des Ausschusses in Frage gestellt, so
    hat dieses einzelstaatliche Gericht nicht die Befugnis, den Beschluss des Ausschusses für nichtig zu erklären,
    sondern es muss im Einklang mit Artikel 267 AEUV in der Auslegung des Gerichtshofs den Gerichtshof mit der Frage
    der Gültigkeit befassen, wenn es den Beschluss für nichtig hält. Allerdings darf ein einzelstaatliches Gericht den
    Gerichtshof nicht auf Anfrage einer natürlichen oder juristischen Person mit Fragen der Gültigkeit des Beschlusses
    des Ausschusses befassen, wenn diese Person Gelegenheit hatte, eine Klage auf Nichtigerklärung dieses Beschlusses
    zu erheben — insbesondere wenn sie unmittelbar und individuell von dem Beschluss betroffen war –, diese Gelegenheit
    jedoch nicht innerhalb der Frist gemäß Artikel 263 AEUV genutzt hat.%
   \label{itm:eg-143}
   
   \crossrefReasonToArticle{143}

   \item Hat ein mit einem Verfahren gegen die Entscheidung einer Aufsichtsbehörde befasstes Gericht Anlass zu der
    Vermutung, dass ein dieselbe Verarbeitung betreffendes Verfahren — etwa zu demselben Gegenstand in Bezug auf die
    Verarbeitung durch denselben Verantwortlichen oder Auftragsverarbeiter oder wegen desselben Anspruchs — vor einem
    zuständigen Gericht in einem anderen Mitgliedstaat anhängig ist, so sollte es mit diesem Gericht Kontakt aufnehmen,
    um sich zu vergewissern, dass ein solches verwandtes Verfahren existiert. Sind verwandte Verfahren vor einem
    Gericht in einem anderen Mitgliedstaat anhängig, so kann jedes später angerufene Gericht das Verfahren aussetzen
    oder sich auf Anfrage einer Partei auch zugunsten des zuerst angerufenen Gerichts für unzuständig erklären, wenn
    dieses später angerufene Gericht für die betreffenden Verfahren zuständig ist und die Verbindung von solchen
    verwandten Verfahren nach seinem Recht zulässig ist. Verfahren gelten als miteinander verwandt, wenn zwischen ihnen
    eine so enge Beziehung gegeben ist, dass eine gemeinsame Verhandlung und Entscheidung geboten erscheint, um zu
    vermeiden, dass in getrennten Verfahren einander widersprechende Entscheidungen ergehen.%
   \label{itm:eg-144}
   
   \crossrefReasonToArticle{144}

   \item Bei Verfahren gegen Verantwortliche oder Auftragsverarbeiter sollte es dem Kläger überlassen bleiben, ob er die
    Gerichte des Mitgliedstaats anruft, in dem der Verantwortliche oder der Auftragsverarbeiter eine Niederlassung hat,
    oder des Mitgliedstaats, in dem die betroffene Person ihren Aufenthaltsort hat; dies gilt nicht, wenn es sich bei
    dem Verantwortlichen um eine Behörde eines Mitgliedstaats handelt, die in Ausübung ihrer hoheitlichen Befugnisse
    tätig geworden ist.%
   \label{itm:eg-145}
   
   \crossrefReasonToArticle{145}

   \item Der Verantwortliche oder der Auftragsverarbeiter sollte Schäden, die einer Person aufgrund einer Verarbeitung
    entstehen, die mit dieser Verordnung nicht im Einklang steht, ersetzen. Der Verantwortliche oder der
    Auftragsverarbeiter sollte von seiner Haftung befreit werden, wenn er nachweist, dass er in keiner Weise für den
    Schaden verantwortlich ist. Der Begriff des Schadens sollte im Lichte der Rechtsprechung des Gerichtshofs weit auf
    eine Art und Weise ausgelegt werden, die den Zielen dieser Verordnung in vollem Umfang entspricht. Dies gilt
    unbeschadet von Schadenersatzforderungen aufgrund von Verstößen gegen andere Vorschriften des Unionsrechts oder des
    Rechts der Mitgliedstaaten. Zu einer Verarbeitung, die mit der vorliegenden Verordnung nicht im Einklang steht,
    zählt auch eine Verarbeitung, die nicht mit den nach Maßgabe der vorliegenden Verordnung erlassenen delegierten
    Rechtsakten und Durchführungsrechtsakten und Rechtsvorschriften der Mitgliedstaaten zur Präzisierung von
    Bestimmungen der vorliegenden Verordnung im Einklang steht. Die betroffenen Personen sollten einen vollständigen
    und wirksamen Schadenersatz für den erlittenen Schaden erhalten. Sind Verantwortliche oder Auftragsverarbeiter an
    derselben Verarbeitung beteiligt, so sollte jeder Verantwortliche oder Auftragsverarbeiter für den gesamten Schaden
    haftbar gemacht werden. Werden sie jedoch nach Maßgabe des Rechts der Mitgliedstaaten zu demselben Verfahren
    hinzugezogen, so können sie im Verhältnis zu der Verantwortung anteilmäßig haftbar gemacht werden, die jeder
    Verantwortliche oder Auftragsverarbeiter für den durch die Verarbeitung entstandenen Schaden zu tragen hat, sofern
    sichergestellt ist, dass die betroffene Person einen vollständigen und wirksamen Schadenersatz für den erlittenen
    Schaden erhält. Jeder Verantwortliche oder Auftragsverarbeiter, der den vollen Schadenersatz geleistet hat, kann
    anschließend ein Rückgriffsverfahren gegen andere an derselben Verarbeitung beteiligte Verantwortliche oder
    Auftragsverarbeiter anstrengen.%
   \label{itm:eg-146}
   
   \crossrefReasonToArticle{146}

   \item Soweit in dieser Verordnung spezifische Vorschriften über die Gerichtsbarkeit — insbesondere in Bezug auf
    Verfahren im Hinblick auf einen gerichtlichen Rechtsbehelf einschließlich Schadenersatz gegen einen
    Verantwortlichen oder Auftragsverarbeiter — enthalten sind, sollten die allgemeinen Vorschriften über die
    Gerichtsbarkeit, wie sie etwa in der Verordnung (EU) Nr. 1215/2012 des Europäischen Parlaments und des
    Rates\comment{Verordnung (EU) Nr. 1215/2012 des Europäischen Parlaments und des Rates vom 12. Dezember 2012 über
    die gerichtliche Zuständigkeit und die Anerkennung und Vollstreckung von Entscheidungen in Zivil- und
    Handelssachen, konsolidierte Fassung siehe \cite{vo-gz}} enthalten sind, der Anwendung dieser spezifischen
    Vorschriften nicht entgegenstehen.%
   \label{itm:eg-147}
   
   \crossrefReasonToArticle{147}

   \item Im Interesse einer konsequenteren Durchsetzung der Vorschriften dieser Verordnung sollten bei Verstößen gegen
    diese Verordnung zusätzlich zu den geeigneten Maßnahmen, die die Aufsichtsbehörde gemäß dieser Verordnung verhängt,
    oder an Stelle solcher Maßnahmen Sanktionen einschließlich Geldbußen verhängt werden. Im Falle eines
    geringfügigeren Verstoßes oder falls voraussichtlich zu verhängende Geldbuße eine unverhältnismäßige Belastung für
    eine natürliche Person bewirken würde, kann anstelle einer Geldbuße eine Verwarnung erteilt werden. Folgendem
    sollte jedoch gebührend Rechnung getragen werden: der Art, Schwere und Dauer des Verstoßes, dem vorsätzlichen
    Charakter des Verstoßes, den Maßnahmen zur Minderung des entstandenen Schadens, dem Grad der Verantwortlichkeit
    oder jeglichem früheren Verstoß, der Art und Weise, wie der Verstoß der Aufsichtsbehörde bekannt wurde, der
    Einhaltung der gegen den Verantwortlichen oder Auftragsverarbeiter angeordneten Maßnahmen, der Einhaltung von
    Verhaltensregeln und jedem anderen erschwerenden oder mildernden Umstand. Für die Verhängung von Sanktionen
    einschließlich Geldbußen sollte es angemessene Verfahrensgarantien geben, die den allgemeinen Grundsätzen des
    Unionsrechts und der Charta, einschließlich des Rechts auf wirksamen Rechtsschutz und ein faires Verfahren,
    entsprechen.%
   \label{itm:eg-148}
   
   \crossrefReasonToArticle{148}

   \item Die Mitgliedstaaten sollten die strafrechtlichen Sanktionen für Verstöße gegen diese Verordnung, auch für
    Verstöße gegen auf der Grundlage und in den Grenzen dieser Verordnung erlassene nationale Vorschriften, festlegen
    können. Diese strafrechtlichen Sanktionen können auch die Einziehung der durch die Verstöße gegen diese Verordnung
    erzielten Gewinne ermöglichen. Die Verhängung von strafrechtlichen Sanktionen für Verstöße gegen solche nationalen
    Vorschriften und von verwaltungsrechtlichen Sanktionen sollte jedoch nicht zu einer Verletzung des Grundsatzes „ne
    bis in idem“, wie er vom Gerichtshof ausgelegt worden ist, führen.%
   \label{itm:eg-149}
   
   \crossrefReasonToArticle{149}

   \item Um die verwaltungsrechtlichen Sanktionen bei Verstößen gegen diese Verordnung zu vereinheitlichen und ihnen
    mehr Wirkung zu verleihen, sollte jede Aufsichtsbehörde befugt sein, Geldbußen zu verhängen. In dieser Verordnung
    sollten die Verstöße sowie die Obergrenze der entsprechenden Geldbußen und die Kriterien für ihre Festsetzung
    genannt werden, wobei diese Geldbußen von der zuständigen Aufsichtsbehörde in jedem Einzelfall unter
    Berücksichtigung aller besonderen Umstände und insbesondere der Art, Schwere und Dauer des Verstoßes und seiner
    Folgen sowie der Maßnahmen, die ergriffen worden sind, um die Einhaltung der aus dieser Verordnung erwachsenden
    Verpflichtungen zu gewährleisten und die Folgen des Verstoßes abzuwenden oder abzumildern, festzusetzen sind.
    Werden Geldbußen Unternehmen auferlegt, sollte zu diesem Zweck der Begriff „Unternehmen“ im Sinne der Artikel 101
    und 102 AEUV verstanden werden. Werden Geldbußen Personen auferlegt, bei denen es sich nicht um Unternehmen
    handelt, so sollte die Aufsichtsbehörde bei der Erwägung des angemessenen Betrags für die Geldbuße dem allgemeinen
    Einkommensniveau in dem betreffenden Mitgliedstaat und der wirtschaftlichen Lage der Personen Rechnung tragen. Das
    Kohärenzverfahren kann auch genutzt werden, um eine kohärente Anwendung von Geldbußen zu fördern. Die
    Mitgliedstaaten sollten bestimmen können, ob und inwieweit gegen Behörden Geldbußen verhängt werden können. Auch
    wenn die Aufsichtsbehörden bereits Geldbußen verhängt oder eine Verwarnung erteilt haben, können sie ihre anderen
    Befugnisse ausüben oder andere Sanktionen nach Maßgabe dieser Verordnung verhängen.%
   \label{itm:eg-150}
   
   \crossrefReasonToArticle{150}

   \item Nach den Rechtsordnungen Dänemarks und Estlands sind die in dieser Verordnung vorgesehenen Geldbußen nicht
    zulässig. Die Vorschriften über die Geldbußen können so angewandt werden, dass die Geldbuße in Dänemark durch die
    zuständigen nationalen Gerichte als Strafe und in Estland durch die Aufsichtsbehörde im Rahmen eines Verfahrens bei
    Vergehen verhängt wird, sofern eine solche Anwendung der Vorschriften in diesen Mitgliedstaaten die gleiche Wirkung
    wie die von den Aufsichtsbehörden verhängten Geldbußen hat. Daher sollten die zuständigen nationalen Gerichte die
    Empfehlung der Aufsichtsbehörde, die die Geldbuße in die Wege geleitet hat, berücksichtigen. In jeden Fall sollten
    die verhängten Geldbußen wirksam, verhältnismäßig und abschreckend sein.%
   \label{itm:eg-151}
   
   \crossrefReasonToArticle{151}

   \item Soweit diese Verordnung verwaltungsrechtliche Sanktionen nicht harmonisiert oder wenn es in anderen Fällen —
    beispielsweise bei schweren Verstößen gegen diese Verordnung — erforderlich ist, sollten die Mitgliedstaaten eine
    Regelung anwenden, die wirksame, verhältnismäßige und abschreckende Sanktionen vorsieht. Es sollte im Recht der
    Mitgliedstaaten geregelt werden, ob diese Sanktionen strafrechtlicher oder verwaltungsrechtlicher Art sind.%
   \label{itm:eg-152}
   
   \crossrefReasonToArticle{152}

   \item Im Recht der Mitgliedstaaten sollten die Vorschriften über die freie Meinungsäußerung und Informationsfreiheit,
    auch von Journalisten, Wissenschaftlern, Künstlern und/oder Schriftstellern, mit dem Recht auf Schutz der
    personenbezogenen Daten gemäß dieser Verordnung in Einklang gebracht werden. Für die Verarbeitung personenbezogener
    Daten ausschließlich zu journalistischen Zwecken oder zu wissenschaftlichen, künstlerischen oder literarischen
    Zwecken sollten Abweichungen und Ausnahmen von bestimmten Vorschriften dieser Verordnung gelten, wenn dies
    erforderlich ist, um das Recht auf Schutz der personenbezogenen Daten mit dem Recht auf Freiheit der
    Meinungsäußerung und Informationsfreiheit, wie es in Artikel 11 der Charta garantiert ist, in Einklang zu bringen.
    Dies sollte insbesondere für die Verarbeitung personenbezogener Daten im audiovisuellen Bereich sowie in
    Nachrichten- und Pressearchiven gelten. Die Mitgliedstaaten sollten daher Gesetzgebungsmaßnahmen zur Regelung der
    Abweichungen und Ausnahmen erlassen, die zum Zwecke der Abwägung zwischen diesen Grundrechten notwendig sind. Die
    Mitgliedstaaten sollten solche Abweichungen und Ausnahmen in Bezug auf die allgemeinen Grundsätze, die Rechte der
    betroffenen Person, den Verantwortlichen und den Auftragsverarbeiter, die Übermittlung von personenbezogenen Daten
    an Drittländer oder an internationale Organisationen, die unabhängigen Aufsichtsbehörden, die Zusammenarbeit und
    Kohärenz und besondere Datenverarbeitungssituationen erlassen. Sollten diese Abweichungen oder Ausnahmen von
    Mitgliedstaat zu Mitgliedstaat unterschiedlich sein, sollte das Recht des Mitgliedstaats angewendet werden, dem der
    Verantwortliche unterliegt. Um der Bedeutung des Rechts auf freie Meinungsäußerung in einer demokratischen
    Gesellschaft Rechnung zu tragen, müssen Begriffe wie Journalismus, die sich auf diese Freiheit beziehen, weit
    ausgelegt werden.%
   \label{itm:eg-153}
   
   \crossrefReasonToArticle{153}

   \item Diese Verordnung ermöglicht es, dass bei ihrer Anwendung der Grundsatz des Zugangs der Öffentlichkeit zu
    amtlichen Dokumenten berücksichtigt wird. Der Zugang der Öffentlichkeit zu amtlichen Dokumenten kann als
    öffentliches Interesse betrachtet werden. Personenbezogene Daten in Dokumenten, die sich im Besitz einer Behörde
    oder einer öffentlichen Stelle befinden, sollten von dieser Behörde oder Stelle öffentlich offengelegt werden
    können, sofern dies im Unionsrecht oder im Recht der Mitgliedstaaten, denen sie unterliegt, vorgesehen ist. Diese
    Rechtsvorschriften sollten den Zugang der Öffentlichkeit zu amtlichen Dokumenten und die Weiterverwendung von
    Informationen des öffentlichen Sektors mit dem Recht auf Schutz personenbezogener Daten in Einklang bringen und
    können daher die notwendige Übereinstimmung mit dem Recht auf Schutz personenbezogener Daten gemäß dieser
    Verordnung regeln. Die Bezugnahme auf Behörden und öffentliche Stellen sollte in diesem Kontext sämtliche Behörden
    oder sonstigen Stellen beinhalten, die vom Recht des jeweiligen Mitgliedstaats über den Zugang der Öffentlichkeit
    zu Dokumenten erfasst werden. Die Richtlinie 2003/98/EG des Europäischen Parlaments und des Rates\comment
    {Richtlinie 2003/98/EG des Europäischen Parlaments und des Rates vom 17. November 2003 über die Weiterverwendung
    von Informationen des öffentlichen Sektors, außer Kraft, letzte konsolidierte Fassung unter \cite
    {ril-wvi-os}} lässt das Schutzniveau für natürliche Personen in Bezug auf die Verarbeitung personenbezogener Daten
    gemäß den Bestimmungen des Unionsrechts und des Rechts der Mitgliedstaaten unberührt und beeinträchtigt diesen in
    keiner Weise, und sie bewirkt insbesondere keine Änderung der in dieser Verordnung dargelegten Rechte und
    Pflichten. Insbesondere sollte die genannte Richtlinie nicht für Dokumente gelten, die nach den Zugangsregelungen
    der Mitgliedstaaten aus Gründen des Schutzes personenbezogener Daten nicht oder nur eingeschränkt zugänglich sind,
    oder für Teile von Dokumenten, die nach diesen Regelungen zugänglich sind, wenn sie personenbezogene Daten
    enthalten, bei denen Rechtsvorschriften vorsehen, dass ihre Weiterverwendung nicht mit dem Recht über den Schutz
    natürlicher Personen in Bezug auf die Verarbeitung personenbezogener Daten vereinbar ist.%
   \label{itm:eg-154}
   
   \crossrefReasonToArticle{154}

   \item Im Recht der Mitgliedstaaten oder in Kollektivvereinbarungen (einschließlich ’Betriebsvereinbarungen’) können
    spezifische Vorschriften für die Verarbeitung personenbezogener Beschäftigtendaten im Beschäftigungskontext
    vorgesehen werden, und zwar insbesondere Vorschriften über die Bedingungen, unter denen personenbezogene Daten im
    Beschäftigungskontext auf der Grundlage der Einwilligung des Beschäftigten verarbeitet werden dürfen, über die
    Verarbeitung dieser Daten für Zwecke der Einstellung, der Erfüllung des Arbeitsvertrags einschließlich der
    Erfüllung von durch Rechtsvorschriften oder durch Kollektivvereinbarungen festgelegten Pflichten, des Managements,
    der Planung und der Organisation der Arbeit, der Gleichheit und Diversität am Arbeitsplatz, der Gesundheit und
    Sicherheit am Arbeitsplatz sowie für Zwecke der Inanspruchnahme der mit der Beschäftigung zusammenhängenden
    individuellen oder kollektiven Rechte und Leistungen und für Zwecke der Beendigung des
    Beschäftigungsverhältnisses.%
   \label{itm:eg-155}
   
   \crossrefReasonToArticle{155}

   \item Die Verarbeitung personenbezogener Daten für im öffentlichen Interesse liegende Archivzwecke, zu
    wissenschaftlichen oder historischen Forschungszwecken oder zu statistischen Zwecken sollte geeigneten Garantien
    für die Rechte und Freiheiten der betroffenen Person gemäß dieser Verordnung unterliegen. Mit diesen Garantien
    sollte sichergestellt werden, dass technische und organisatorische Maßnahmen bestehen, mit denen insbesondere der
    Grundsatz der Datenminimierung gewährleistet wird. Die Weiterverarbeitung personenbezogener Daten zu im
    öffentlichen Interesse liegende Archivzwecken, zu wissenschaftlichen oder historischen Forschungszwecken oder zu
    statistischen Zwecken erfolgt erst dann, wenn der Verantwortliche geprüft hat, ob es möglich ist, diese Zwecke
    durch die Verarbeitung von personenbezogenen Daten, bei der die Identifizierung von betroffenen Personen nicht oder
    nicht mehr möglich ist, zu erfüllen, sofern geeignete Garantien bestehen (wie z. B. die Pseudonymisierung von
    personenbezogenen Daten). Die Mitgliedstaaten sollten geeignete Garantien in Bezug auf die Verarbeitung
    personenbezogener Daten für im öffentlichen Interesse liegende Archivzwecke, zu wissenschaftlichen oder
    historischen Forschungszwecken oder zu statistischen Zwecken vorsehen. Es sollte den Mitgliedstaaten erlaubt sein,
    unter bestimmten Bedingungen und vorbehaltlich geeigneter Garantien für die betroffenen Personen Präzisierungen und
    Ausnahmen in Bezug auf die Informationsanforderungen sowie der Rechte auf Berichtigung, Löschung, Vergessenwerden,
    zur Einschränkung der Verarbeitung, auf Datenübertragbarkeit sowie auf Widerspruch bei der Verarbeitung
    personenbezogener Daten zu im öffentlichen Interesse liegende Archivzwecken, zu wissenschaftlichen oder
    historischen Forschungszwecken oder zu statistischen Zwecken vorzusehen. Im Rahmen der betreffenden Bedingungen und
    Garantien können spezifische Verfahren für die Ausübung dieser Rechte durch die betroffenen Personen vorgesehen
    sein — sofern dies angesichts der mit der spezifischen Verarbeitung verfolgten Zwecke angemessen ist — sowie
    technische und organisatorische Maßnahmen zur Minimierung der Verarbeitung personenbezogener Daten im Hinblick auf
    die Grundsätze der Verhältnismäßigkeit und der Notwendigkeit. Die Verarbeitung personenbezogener Daten zu
    wissenschaftlichen Zwecken sollte auch anderen einschlägigen Rechtsvorschriften, beispielsweise für klinische
    Prüfungen, genügen.%
   \label{itm:eg-156}
   
   \crossrefReasonToArticle{156}

   \item Durch die Verknüpfung von Informationen aus Registern können Forscher neue Erkenntnisse von großem Wert in
    Bezug auf weit verbreiteten Krankheiten wie Herz"=Kreislauferkrankungen, Krebs und Depression erhalten. Durch die
    Verwendung von Registern können bessere Forschungsergebnisse erzielt werden, da sie auf einen größeren
    Bevölkerungsanteil gestützt sind. Im Bereich der Sozialwissenschaften ermöglicht die Forschung anhand von Registern
    es den Forschern, entscheidende Erkenntnisse über den langfristigen Zusammenhang einer Reihe sozialer Umstände zu
    erlangen, wie Arbeitslosigkeit und Bildung mit anderen Lebensumständen. Durch Register erhaltene
    Forschungsergebnisse bieten solide, hochwertige Erkenntnisse, die die Basis für die Erarbeitung und Umsetzung
    wissensgestützter politischer Maßnahmen darstellen, die Lebensqualität zahlreicher Menschen verbessern und die
    Effizienz der Sozialdienste verbessern können. Zur Erleichterung der wissenschaftlichen Forschung können daher
    personenbezogene Daten zu wissenschaftlichen Forschungszwecken verarbeitet werden, wobei sie angemessenen
    Bedingungen und Garantien unterliegen, die im Unionsrecht oder im Recht der Mitgliedstaaten festgelegt sind.%
   \label{itm:eg-157}
   
   \crossrefReasonToArticle{157}

   \item Diese Verordnung sollte auch für die Verarbeitung personenbezogener Daten zu Archivzwecken gelten, wobei darauf
    hinzuweisen ist, dass die Verordnung nicht für verstorbene Personen gelten sollte. Behörden oder öffentliche oder
    private Stellen, die Aufzeichnungen von öffentlichem Interesse führen, sollten gemäß dem Unionsrecht oder dem Recht
    der Mitgliedstaaten rechtlich verpflichtet sein, Aufzeichnungen von bleibendem Wert für das allgemeine öffentliche
    Interesse zu erwerben, zu erhalten, zu bewerten, aufzubereiten, zu beschreiben, mitzuteilen, zu fördern, zu
    verbreiten sowie Zugang dazu bereitzustellen. Es sollte den Mitgliedstaaten ferner erlaubt sein vorzusehen, dass
    personenbezogene Daten zu Archivzwecken weiterverarbeitet werden, beispielsweise im Hinblick auf die Bereitstellung
    spezifischer Informationen im Zusammenhang mit dem politischen Verhalten unter ehemaligen totalitären Regimen,
    Völkermord, Verbrechen gegen die Menschlichkeit, insbesondere dem Holocaust, und Kriegsverbrechen.%
   \label{itm:eg-158}
   
   \crossrefReasonToArticle{158}

   \item Diese Verordnung sollte auch für die Verarbeitung personenbezogener Daten zu wissenschaftlichen
    Forschungszwecken gelten. Die Verarbeitung personenbezogener Daten zu wissenschaftlichen Forschungszwecken im Sinne
    dieser Verordnung sollte weit ausgelegt werden und die Verarbeitung für beispielsweise die technologische
    Entwicklung und die Demonstration, die Grundlagenforschung, die angewandte Forschung und die privat finanzierte
    Forschung einschließen. Darüber hinaus sollte sie dem in Artikel 179 Absatz 1 AEUV festgeschriebenen Ziel, einen
    europäischen Raum der Forschung zu schaffen, Rechnung tragen. Die wissenschaftlichen Forschungszwecke sollten auch
    Studien umfassen, die im öffentlichen Interesse im Bereich der öffentlichen Gesundheit durchgeführt werden. Um den
    Besonderheiten der Verarbeitung personenbezogener Daten zu wissenschaftlichen Forschungszwecken zu genügen, sollten
    spezifische Bedingungen insbesondere hinsichtlich der Veröffentlichung oder sonstigen Offenlegung personenbezogener
    Daten im Kontext wissenschaftlicher Zwecke gelten. Geben die Ergebnisse wissenschaftlicher Forschung insbesondere
    im Gesundheitsbereich Anlass zu weiteren Maßnahmen im Interesse der betroffenen Person, sollten die allgemeinen
    Vorschriften dieser Verordnung für diese Maßnahmen gelten.%
   \label{itm:eg-159}
   
   \crossrefReasonToArticle{159}

   \item Diese Verordnung sollte auch für die Verarbeitung personenbezogener Daten zu historischen Forschungszwecken
    gelten. Dazu sollte auch historische Forschung und Forschung im Bereich der Genealogie zählen, wobei darauf
    hinzuweisen ist, dass diese Verordnung nicht für verstorbene Personen gelten sollte.%
   \label{itm:eg-160}
   
   \crossrefReasonToArticle{160}

   \item Für die Zwecke der Einwilligung in die Teilnahme an wissenschaftlichen Forschungstätigkeiten im Rahmen
    klinischer Prüfungen sollten die einschlägigen Bestimmungen der Verordnung (EU) Nr. 536/2014 des Europäischen
    Parlaments und des Rates\comment{Verordnung (EU) Nr. 536/2014 des Europäischen Parlaments und des Rates vom 16.
    April 2014 über klinische Prüfungen mit Humanarzneimitteln und zur Aufhebung der Richtlinie 2001/20/EG Text von
    Bedeutung für den EWR, konsolidierte Fassung siehe \cite{vo-klin-pr-ham}} gelten.%
   \label{itm:eg-161}
   
   \crossrefReasonToArticle{161}

   \item Diese Verordnung sollte auch für die Verarbeitung personenbezogener Daten zu statistischen Zwecken gelten. Das
    Unionsrecht oder das Recht der Mitgliedstaaten sollte in den Grenzen dieser Verordnung den statistischen Inhalt,
    die Zugangskontrolle, die Spezifikationen für die Verarbeitung personenbezogener Daten zu statistischen Zwecken und
    geeignete Maßnahmen zur Sicherung der Rechte und Freiheiten der betroffenen Personen und zur Sicherstellung der
    statistischen Geheimhaltung bestimmen. Unter dem Begriff „statistische Zwecke“ ist jeder für die Durchführung
    statistischer Untersuchungen und die Erstellung statistischer Ergebnisse erforderliche Vorgang der Erhebung und
    Verarbeitung personenbezogener Daten zu verstehen. Diese statistischen Ergebnisse können für verschiedene Zwecke,
    so auch für wissenschaftliche Forschungszwecke, weiterverwendet werden. Im Zusammenhang mit den statistischen
    Zwecken wird vorausgesetzt, dass die Ergebnisse der Verarbeitung zu statistischen Zwecken keine personenbezogenen
    Daten, sondern aggregierte Daten sind und diese Ergebnisse oder personenbezogenen Daten nicht für Maßnahmen oder
    Entscheidungen gegenüber einzelnen natürlichen Personen verwendet werden.%
   \label{itm:eg-162}
   
   \crossrefReasonToArticle{162}

   \item Die vertraulichen Informationen, die die statistischen Behörden der Union und der Mitgliedstaaten zur
    Erstellung der amtlichen europäischen und der amtlichen nationalen Statistiken erheben, sollten geschützt werden.
    Die europäischen Statistiken sollten im Einklang mit den in Artikel 338 Absatz 2 AEUV dargelegten statistischen
    Grundsätzen entwickelt, erstellt und verbreitet werden, wobei die nationalen Statistiken auch mit dem Recht der
    Mitgliedstaaten übereinstimmen müssen. Die Verordnung (EG) Nr. 223/2009 des Europäischen Parlaments und des
    Rates\comment{Verordnung (EG) Nr. 223/2009 des Europäischen Parlaments und des Rates vom 11. März 2009 über
    europäische Statistiken und zur Aufhebung der Verordnung (EG, Euratom) Nr. 1101/2008 des Europäischen Parlaments
    und des Rates über die Übermittlung von unter die Geheimhaltungspflicht fallenden Informationen an das Statistische
    Amt der Europäischen Gemeinschaften, der Verordnung (EG) Nr. 322/97 des Rates über die Gemeinschaftsstatistiken und
    des Beschlusses 89/382/EWG, Euratom des Rates zur Einsetzung eines Ausschusses für das Statistische Programm der
    Europäischen Gemeinschaften, konsolidierte Fassung siehe \cite{vo-eustat-euratom}} enthält genauere Bestimmungen
    zur Vertraulichkeit europäischer Statistiken.%
   \label{itm:eg-163}
   
   \crossrefReasonToArticle{163}

   \item Hinsichtlich der Befugnisse der Aufsichtsbehörden, von dem Verantwortlichen oder vom Auftragsverarbeiter Zugang
    zu personenbezogenen Daten oder zu seinen Räumlichkeiten zu erlangen, können die Mitgliedstaaten in den Grenzen
    dieser Verordnung den Schutz des Berufsgeheimnisses oder anderer gleichwertiger Geheimhaltungspflichten durch
    Rechtsvorschriften regeln, soweit dies notwendig ist, um das Recht auf Schutz der personenbezogenen Daten mit einer
    Pflicht zur Wahrung des Berufsgeheimnisses in Einklang zu bringen. Dies berührt nicht die bestehenden
    Verpflichtungen der Mitgliedstaaten zum Erlass von Vorschriften über das Berufsgeheimnis, wenn dies aufgrund des
    Unionsrechts erforderlich ist.%
   \label{itm:eg-164}
   
   \crossrefReasonToArticle{164}

   \item Im Einklang mit Artikel 17 AEUV achtet diese Verordnung den Status, den Kirchen und religiöse Vereinigungen
    oder Gemeinschaften in den Mitgliedstaaten nach deren bestehenden verfassungsrechtlichen Vorschriften genießen, und
    beeinträchtigt ihn nicht.%
   \label{itm:eg-165}
   
   \crossrefReasonToArticle{165}

   \item Um die Zielvorgaben dieser Verordnung zu erfüllen, d. h. die Grundrechte und Grundfreiheiten natürlicher
    Personen und insbesondere ihr Recht auf Schutz ihrer personenbezogenen Daten zu schützen und den freien Verkehr
    personenbezogener Daten innerhalb der Union zu gewährleisten, sollte der Kommission die Befugnis übertragen werden,
    gemäß Artikel 290 AEUV Rechtsakte zu erlassen. Delegierte Rechtsakte sollten insbesondere in Bezug auf die für
    Zertifizierungsverfahren geltenden Kriterien und Anforderungen, die durch standardisierte Bildsymbole
    darzustellenden Informationen und die Verfahren für die Bereitstellung dieser Bildsymbole erlassen werden. Es ist
    von besonderer Bedeutung, dass die Kommission im Zuge ihrer Vorbereitungsarbeit angemessene Konsultationen, auch
    auf der Ebene von Sachverständigen, durchführt. Bei der Vorbereitung und Ausarbeitung delegierter Rechtsakte sollte
    die Kommission gewährleisten, dass die einschlägigen Dokumente dem Europäischen Parlament und dem Rat gleichzeitig,
    rechtzeitig und auf angemessene Weise übermittelt werden.%
   \label{itm:eg-166}
   
   \crossrefReasonToArticle{166}

   \item Zur Gewährleistung einheitlicher Bedingungen für die Durchführung dieser Verordnung sollten der Kommission
    Durchführungsbefugnisse übertragen werden, wenn dies in dieser Verordnung vorgesehen ist. Diese Befugnisse sollten
    nach Maßgabe der Verordnung (EU) Nr. 182/2011 des Europäischen Parlaments und des Rates ausgeübt werden. In diesem
    Zusammenhang sollte die Kommission besondere Maßnahmen für Kleinstunternehmen sowie kleine und mittlere Unternehmen
    erwägen.%
   \label{itm:eg-167}
   
   \crossrefReasonToArticle{167}

   \item Für den Erlass von Durchführungsrechtsakten bezüglich Standardvertragsklauseln für Verträge zwischen
    Verantwortlichen und Auftragsverarbeitern sowie zwischen Auftragsverarbeitern; Verhaltensregeln; technische
    Standards und Verfahren für die Zertifizierung; Anforderungen an die Angemessenheit des Datenschutzniveaus in einem
    Drittland, einem Gebiet oder bestimmten Sektor dieses Drittlands oder in einer internationalen Organisation;
    Standardschutzklauseln; Formate und Verfahren für den Informationsaustausch zwischen Verantwortlichen,
    Auftragsverarbeitern und Aufsichtsbehörden im Hinblick auf verbindliche interne Datenschutzvorschriften; Amtshilfe;
    sowie Vorkehrungen für den elektronischen Informationsaustausch zwischen Aufsichtsbehörden und zwischen
    Aufsichtsbehörden und dem Ausschuss sollte das Prüfverfahren angewandt werden.%
   \label{itm:eg-168}
   
   \crossrefReasonToArticle{168}

   \item Die Kommission sollte sofort geltende Durchführungsrechtsakte erlassen, wenn anhand vorliegender Beweise
    festgestellt wird, dass ein Drittland, ein Gebiet oder ein bestimmter Sektor in diesem Drittland oder eine
    internationale Organisation kein angemessenes Schutzniveau gewährleistet, und dies aus Gründen äußerster
    Dringlichkeit erforderlich ist.%
   \label{itm:eg-169}
   
   \crossrefReasonToArticle{169}

   \item Da das Ziel dieser Verordnung, nämlich die Gewährleistung eines gleichwertigen Datenschutzniveaus für
    natürliche Personen und des freien Verkehrs personenbezogener Daten in der Union, von den Mitgliedstaaten nicht
    ausreichend verwirklicht werden kann, sondern vielmehr wegen des Umfangs oder der Wirkungen der Maßnahme auf
    Unionsebene besser zu verwirklichen ist, kann die Union im Einklang mit dem in Artikel 5 des Vertrags über die
    Europäische Union (EUV) verankerten Subsidiaritätsprinzip tätig werden. Entsprechend dem in demselben Artikel
    genannten Grundsatz der Verhältnismäßigkeit geht diese Verordnung nicht über das für die Verwirklichung dieses
    Ziels erforderliche Maß hinaus.%
   \label{itm:eg-170}
   
   \crossrefReasonToArticle{170}

   \item Die Richtlinie 95/46/EG sollte durch diese Verordnung aufgehoben werden. Verarbeitungen, die zum Zeitpunkt der
    Anwendung dieser Verordnung bereits begonnen haben, sollten innerhalb von zwei Jahren nach dem Inkrafttreten dieser
    Verordnung mit ihr in Einklang gebracht werden. Beruhen die Verarbeitungen auf einer Einwilligung gemäß der
    Richtlinie 95/46/EG, so ist es nicht erforderlich, dass die betroffene Person erneut ihre Einwilligung dazu
    erteilt, wenn die Art der bereits erteilten Einwilligung den Bedingungen dieser Verordnung entspricht, so dass der
    Verantwortliche die Verarbeitung nach dem Zeitpunkt der Anwendung der vorliegenden Verordnung fortsetzen kann. Auf
    der Richtlinie 95/46/EG beruhende Entscheidungen bzw. Beschlüsse der Kommission und Genehmigungen der
    Aufsichtsbehörden bleiben in Kraft, bis sie geändert, ersetzt oder aufgehoben werden.%
   \label{itm:eg-171}
   
   \crossrefReasonToArticle{171}

   \item Der Europäische Datenschutzbeauftragte wurde gemäß Artikel 28 Absatz 2 der Verordnung (EG) Nr. 45/2001
    konsultiert und hat am 7. März 2012\comment{siehe \cite{sn-dsref}} eine Stellungnahme abgegeben.%
   \label{itm:eg-172}
   
   \crossrefReasonToArticle{172}

   \item Diese Verordnung sollte auf alle Fragen des Schutzes der Grundrechte und Grundfreiheiten bei der Verarbeitung
    personenbezogener Daten Anwendung finden, die nicht den in der Richtlinie 2002/58/EG des Europäischen Parlaments
    und des Rates\comment{Richtlinie 2002/58/EG des Europäischen Parlaments und des Rates vom 12. Juli 2002 über die
    Verarbeitung personenbezogener Daten und den Schutz der Privatsphäre in der elektronischen Kommunikation
    (Datenschutzrichtlinie für elektronische Kommunikation) siehe \cite{ril-ds-ek}} bestimmte Pflichten, die dasselbe
    Ziel verfolgen, unterliegen, einschließlich der Pflichten des Verantwortlichen und der Rechte natürlicher Personen.
    Um das Verhältnis zwischen der vorliegenden Verordnung und der Richtlinie 2002/58/EG klarzustellen, sollte die
    Richtlinie entsprechend geändert werden. Sobald diese Verordnung angenommen ist, sollte die Richtlinie 2002/58/EG
    einer Überprüfung unterzogen werden, um insbesondere die Kohärenz mit dieser Verordnung zu gewährleisten —%
   \label{itm:eg-173}
   
   \crossrefReasonToArticle{173}

\end{enumerate}